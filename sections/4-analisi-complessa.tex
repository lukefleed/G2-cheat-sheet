\chapter{Analisi Complessa}

\begin{multicols*}{2}


\section{Richiami di calcolo in pi\`u variabili}
\begin{definition}[Differenziabilit\`a e differenziale]
Sia $U\subseteq \R^m$ un aperto e sia $f:U\to \R^n$ una funzione. $f$ \`e  \textbf{differenziabile} in $x_0\in U$ se esiste una funzione lineare $L:\R^m\to \R^n$ tale che
\[f(x_0+v)=f(x_0)+L(v)+o(|v|).\]
In tal caso $L$ \`e unica e si dice \textbf{differenziale} di $f$ in $x_0$ e si denota $df_{x_0}:\R^m\to \R^n$.\footnote{La notazione usata nel corso di analisi 2 \`e $\Dc f(x_0)$ al posto di $df_{x_0}$.}
\end{definition}
\begin{remark}
Se $f$ \`e $\R-$lineare allora coincide con il proprio differenziale.
\end{remark}
\begin{remark}
Il differenziale di una composizione \`e la composizione dei differenziali.
\end{remark}
\begin{definition}[Derivate parziali]
Se $\{e_i\}$ \`e la base canonica di $\R^m$ allora definiamo la \textbf{$i-$esima derivata parziale} di $f$ come
\[df_{x_0}(e_i)=\pdpi if(x_0).\]
\end{definition}

\noindent
Per il resto del capitolo identifichiamo $\R^2$ con $\C$ tramite l'isomorfismo $(x,y)\leftrightarrow x+iy$.

\begin{remark}
$\C$ \`e sia un $\C-$spazio vettoriale (di dimensione $1$ con base $\{1\}$) che un $\R-$spazio vettoriale (di dimensione $2$ con base $\{1,i\}$).
\end{remark}

\begin{definition}[Coniugio]
La seguente mappa
\[\funcDef\C\C {x+iy}{x-iy}\]
\`e detta \textbf{coniugio}. Se $x\in \C$ spesso indichiamo la sua immagine tramite la mappa di coniugio con $\ol z$ e chiamiamo questo il \textbf{coniugato (complesso)} di $z$.
\end{definition}
\begin{remark}
Il coniugio \`e $\R-$lineare ma non $\C-$lineare ($-i=\ol{i}\neq i\ol{1}=i$).
\end{remark}

\begin{proposition}[Il differenziale \`e $\C$-lineare]
Il differenziale \`e $\C$-lineare, cio\`e
\[d((a+bi)f)_P=(a+bi)df_P.\]
\end{proposition}


\begin{proposition}[Propriet\`a del differenziale]
Siano $f_1,f_2:U\to\C$ con $U\subseteq\R^m$ aperto mappe differenziabili in $P$, allora
\begin{enumerate}[noitemsep]
\item $(d\ol f_1)_P=(\ol{df_1})_P$
\item $d(f_1f_2)_P=(df_1)_Pf_2(P)+f_1(P)(df_2)_P$
\item Se $f_1(P)\neq 0$ allora $\frac 1{f_1}$ \`e definita in un intorno di $P$ e
\[d\pa{\frac1{f_1}}_P=-\frac{(df_1)_P}{f_1(P)^2}.\]
\end{enumerate}
\end{proposition}

\begin{notation}
Per semplicit\`a notazionale a volte scriveremo $z$ al posto di $id_\C$, $\ol z$ al posto della mappa coniugio, $x$ per $a+bi\mapsto a=\Real(a+bi)$ e $y$ per $a+bi\mapsto b=\Imag(a+bi)$. Cercher\`o di rendere la notazione meno ambigua possibile quando possibile.
\end{notation}

\begin{remark}
$z=x+iy$ e $\ol z=x-iy$ come funzioni. Osserviamo inoltre che
\[dz=dx+idy,\quad d\ol z=dx-idy\]
per linearit\`a. Nonostante $z=dz$ e $\ol z=d\ol z$ manterremo le $d$ quando questo render\`a pi\`u chiari i ragionamenti (vale a dire nel contesto delle 1-forme).\\
Osserviamo che con queste notazioni $dx=x$ e $dy=y$.
\end{remark}

\begin{remark}
Se $f:U\to\C$ \`e differenziabile in $P\in U$ si ha che
\[df_P=\pdp[x]f(P)dx+\pdp[y]f(P)dy.\]
\end{remark}
\begin{remark}
Dato che $dz=dx+idy$ e $d\ol z=dx-idy$ si ha che
\[dx=\frac{dz+d\ol z}2,\qquad dy=\frac{dz-d\ol z}{2i}.\]
Da questo segue che
\begin{align*}
df=&\pdp[x]fdx+\pdp[y]fdy =\\
=&\pdp[x]f\frac{dz+d\ol z}2+\pdp[y]f\frac{dz-d\ol z}{2i}=\\
=&\frac12\pa{\pdp[x]f-i\pdp[y]f}dz+\frac12\pa{\pdp[x]f+i\pdp[y]f}d\ol z.
\end{align*}
\end{remark}
\noindent
Motivati dall'espressione sopra per $df$ diamo le seguenti definizioni:
\begin{notation}
Poniamo
\[\pdp[z]f=\frac12\pa{\pdp[x]f-i\pdp[y]f},\qquad \pdp[\ol z]f=\frac12\pa{\pdp[x]f+i\pdp[y]f},\]
da cui
\[df=\pdp[z]fdz+\pdp[\ol z]fd\ol z.\]
\end{notation}
\begin{remark}
Si ha che $\pdp[z]fdz$ \`e $\C-$lineare e $\pdp[\ol z]fd\ol z$ \`e $\C-$antilineare.
\end{remark}

Abbiamo dunque decomposto $f$ in una componente $\C-$lineare e una componente $\C-$antilineare. \\
\begin{proposition}[Le $\R$ lineari sono la somma diretta delle $\C$ lineari e $\C$ antilineari]\label{RlinSonoSommaDirettaClinECantilin}

\footnotesize{$\cpa{f:\C\to \C\sep \R-lin.}=\cpa{f:\C\to \C\sep \C-lin.}\oplus\cpa{f:\C\to \C\sep \C-antilin.}$}
\end{proposition}

\section{Funzioni olomorfe}
\begin{definition}[Funzione olomorfa]
Siano $U\subseteq \C$ un aperto, $z_0\in U$ e $f:U\to \C$. La funzione $f$ si dice \textbf{olomorfa} (o \textbf{derivabile in senso complesso}) in $z_0$ se il seguente limite esiste
\[\lim_{z\to z_0}\frac{f(z)-f(z_0)}{z-z_0}=f'(z_0)\in\C.\]
$f$ \`e \textbf{olomorfa} se lo \`e in $z_0$ per ogni $z_0\in U$.\\
Poniamo
\[\Oc(U)=\cpa{f:U\to\C\sep f\text{ olomorfa}}\]
\end{definition}
\noindent
Osserviamo che essere olomorfa ed essere differenziabile hanno definizioni formalmente molto simili, ma sono propriet\`a diverse come mostra la seguente proposizione. La differenza sostanziale sta nel fatto che $z-z_0$ nella definizione \`e un numero complesso, non una distanza. Pi\`u esplicitamente, la differenziabilit\`a richiede solo l'esistenza di \[\lim_{z\to z_0}\frac{f(z)-f(z_0)}{|z-z_0|},\] che \`e una condizione pi\`u debole.

\begin{lemma}
Se $f:U\to \C$ \`e olomorfa in $z_0\in U$ e $f'(z_0)$ denota il limite della definizione di olomorfa allora
\[df_{z_0}(w)=f'(z_0)w.\]
\end{lemma}

\begin{proposition}[Caratterizzazioni delle funzioni olomorfe]\label{CaratterizzazioneOlomorfaInPunto}
Sia $f:U\to \C$ con $U$ aperto e $z_0\in U$. Le seguenti affermazioni sono equivalenti:
\begin{enumerate}[noitemsep]
\item $f$ \`e olomorfa in $z_0$
\item $f$ \`e differenziabile in $z_0$ e $df_{z_0}$ \`e $\C-$lineare
\item $f$ \`e differenziabile in $z_0$ e $\displaystyle \pdp[\ol z]f(z_0)=0$
\item $f$ \`e differenziabile e valgono le \textbf{equazioni di Cauchy-Riemann}, cio\`e
\[\pdp[x]{\Real(f)}=\pdp[y]{\Imag(f)},\quad \pdp[y]{\Real(f)}=-\pdp[x]{\Imag(f)}.\]
\item $df_{z_0}(w)=f'(z_0)w$.
\end{enumerate}
\end{proposition}

\begin{theorem}[Propriet\`a aritmetiche delle olomorfe]\label{ProprietaAritmeticheOlomorfe}
Siano $f:U\to \C$ e $g:V\to \C$ olomorfe. Allora
\begin{enumerate}[noitemsep]
\item se $U=V$ allora $f+g:U\to \C$ \`e olomorfa
\item se $U=V$ allora $fg:U\to \C$ \`e olomorfa e
\[(fg)'=f'g+fg'\]
\item se $f(U)\subseteq V$ allora $g\circ f:U\to \C$ \`e olomorfa e
\[(g\circ f)'(z_0)=g'(f(z_0))f'(z_0)\]
\item se $f'(z_0)\neq 0$ e $f\in C^1$ allora esistono un intorno $W$ di $f(z_0)$ in $\C$ e un intorno $Z$ di $z_0$ in $U$ tali che $f(Z)=W$, $f\res Z\to W$ \`e bigettiva e $(f\res Z)\ii :W\to Z$ \`e olomorfa con
\[((f\res Z)\ii)'(f(z_0))=\frac1{f'(z_0)}\]
\end{enumerate}
\end{theorem}



\section{Analitiche}
Studiamo una classe di funzioni apparentemente pi\`u piccola delle funzioni olomorfe. Mostreremo successivamente che le funzioni analitiche da $U\subseteq \C$ aperto a $\C$ sono esattamente le funzioni olomorfe. Per il momento limitiamoci a studiarle in quanto tali.

\subsection{Serie di potenze}
\begin{definition}[Serie di potenze]
Una serie della forma
\[\sum_{n=0}^\infty a_n(z-z_0)^n\]
\`e detta \textbf{serie di potenze} di \textbf{centro} $z_0$.
\end{definition}
\begin{remark}
Una serie di potenze ammette un \textbf{raggio di convergenza}, cio\`e esiste $R>0$ tale che la serie converge assolutamente in $B_R(z_0)$ e diverge in $\C\bs \ol{B_R(z_0)}$.
\end{remark}

\begin{definition}[Funzione analitica]
Una funzione $f:U\to \C$ \`e \textbf{analitica} se per ogni $z_0\in U$ esiste $R>0$ tale che $f$ sia esprimibile come serie di potenze centrata in $z_0$ per ogni $z\in B_R(z_0)$.
\end{definition}
\begin{proposition}[Le analitiche sono continue]\label{AnaliticaImplicaContinua}
Una funzione $f:U\to \C$ analitica \`e continua.
\end{proposition}

\begin{theorem}[Serie di potenze sono analitiche]\label{SeriePotenzeSonoAnalitiche}
Supponiamo che
\[f(z)=\sum_{n=0}^\infty a_n(z-z_0)^n\]
per ogni $z\in B_R(z_0)$ con $R>0$. Allora $f$ \`e analitica su $B_R(z_0)$.
\end{theorem}
\begin{remark}
Il teorema potrebbe sembrare banale ma ricordiamo che analitica significa che la funzione si esprime come serie di potenze centrata in un \textit{qualsiasi} punto del dominio. La scrittura con centro in $z_0$ non \`e sufficiente.
\end{remark}

\begin{theorem}[Serie derivata]\label{DerivataSerieDiPotenze}
Se $f=\sum a_n(z-z_0)^n$ su $B_R(z_0)$ allora $f$ \`e olomorfa su $B_R(z_0)$ e
\[f'(z)=\under{\text{\textbf{serie derivata}}}{\sum_{n=1}^\infty a_nn(z-z_0)^{n-1}}\quad \text{per ogni $z\in B_R(z_0)$}\]
\end{theorem}
\begin{corollary}\label{AnaliticaImplicaOlomorfa}
Una funzione analitica complessa \`e $C^\infty$ in senso complesso.
\end{corollary}
\begin{corollary}[Serie di Taylor]\label{TaylorSerieDiPotenze}
Se $f(z)=\sum_{n=0}^\infty a_n(z-z_0)^n$ in un intorno di $z_0$ allora
\[f^{(n)}(z_0)=n!\ a_n,\]
in particolare l'espressione di $f$ come serie di potenze \`e unica:
\[f(z)=\sum_{n=0}^\infty \frac{f^{(n)}(z_0)}{n!}(z-z_0)^n\]
\end{corollary}

\begin{remark}[Composizione di analitiche \`e analitica]\label{ComposizioneAnaliticheEAnalitica}
Siano $f:U\to V$ e $g:V\to \C$ analitiche, allora $g\circ f$ \`e analitica.
\end{remark}



\subsection{Ordine di annullamento}
Studiamo come si comportano gli zeri delle funzioni analitiche. Questo si riveler\`a molto utile nel dare principi di rigidit\`a per le olomorfe.

\begin{definition}[Ordine di annullamento]
Sia $f:U\to \C$ olomorfa. L'\textbf{ordine di annullamento} (o \textbf{di svanimento} o \textbf{di zero}) di $f$ in $z_0$ \`e dato da
\[\begin{cases}
\min\cpa{n\in\N\sep
f^{(n)}(z_0)\neq 0} & \text{se }\exists n\in\N\ t.c.\ f^{(n)}(z_0)\neq 0\\
\infty &\text{altrimenti.}
\end{cases}\]
\end{definition}
\begin{remark}
$f$ ha ordine di annullamento $0$ se $f(z_0)\neq 0$ e ha ordine $\geq 1$ se $f(z_0)=0$.
\end{remark}

\begin{lemma}[Caratterizzazione dell'ordine di annullamento]\label{CaratterizzazioneOrdineDiAnnullamento}
Sia $f$ analitica (e quindi olomorfa per (\ref{AnaliticaImplicaOlomorfa})). Le seguenti proposizioni sono equivalenti:
\begin{enumerate}[noitemsep]
\item $z_0$ ha ordine di annullamento $n_0$ per $f$
\item $f(z)=(z-z_0)^{n_0}g(z)$ in un intorno di $z_0$ con $g(z_0)\neq 0$ e $g$ analitica.
\end{enumerate}
\end{lemma}

\begin{remark}
Se $f$ \`e analitica, $z_0$ ha ordine di annullamento $\infty$ per $f$ se e solo se $f=0$ in un intorno di $z_0$.
\end{remark}


\begin{lemma}\label{ZeriAnaliticaSonoIsolatiOIntornoNullo}
Sia $f:U\to \C$ analitica e sia $Z=\{z_0\in U\mid f(z_0)=0\}$. Sia $z_0\in Z$. Si ha che $z_0$ \`e un punto isolato di $Z$ o $z_0\in int(Z)$.
\end{lemma}

\begin{theorem}[Zeri di analitica sono isolati o coprono la comp. connessa]\label{ZeriDiAnaliticaSonoIsolatiOCopronoLaComponenteConnessa}
Sia $f:U\to\C$ analitica con $U$ connesso. Supponiamo che $Z=f\ii(0)\subseteq U$ abbia un punto di accumulazione in $U$. Allora $f=0$ come funzione su $U$.
\end{theorem}
\noindent
Possiamo riformulare questo risultato nel potente
\begin{corollary}[Principio di identit\`a per analitiche]\label{LuogoDoveAnaliticheCoincidonoSonoIsolatiOCopronoLaComponenteConnessa}
Se $U$ \`e connesso e $f,g:U\to \C$ sono analitiche e $W\subseteq U$ contiene un punto non isolato allora $f=g$ su $U$ se e solo se $f=g$ su $W$.
\end{corollary}

\section{Esponenziale e logaritmo complessi}
\subsection{Esponenziale complesso}
\begin{definition}[Esponenziale complesso]
Definiamo
\[e^z=\sum_{n=0}^\infty \frac{z^n}{n!}.\]
\end{definition}
\begin{remark}
L'esponenziale ha raggio di convergenza infinito. In particolare \`e sempre assolutamente convergente.
\end{remark}

\begin{proposition}[Propriet\`a dell'esponenziale complesso]\label{ProprietaEsponenzialeComplesso}
Valgono i seguenti fatti
\begin{enumerate}[noitemsep]
\item $e^{z+w}=e^ze^w$ per ogni $z,w\in \C$. In particolare $(e^{z})\ii=e^{-z}$ quindi $exp:\C\to \C^\times$ \`e ben definita.
\item Se $z=x+iy$ con $x,y\in \R$ allora
\[e^z=e^xe^{iy}=e^x(\cos y+i\sin y).\]
\item $\abs{e^{iy}}=\abs{\cos y+i\sin y}=1$, in particolare
\[\abs{e^{x+iy}}=\abs{e^x}=e^{\Real(z)}.\]
\item $exp:\C\to \C^\times$ \`e surgettiva
\item Essendo analitica, $\exp$ \`e olomorfa ed evidentemente $\exp'(z)=\exp(z)$.
\item $e^z=e^w$ se e solo se $z-w=2k\pi i$
\end{enumerate}
\end{proposition}


\subsection{Logaritmo complesso}
Sappiamo che $\exp:\C\to \C^\times$ \`e surgettiva. Questa mappa non \`e iniettiva, dunque non ha senso cercare una inversa globale, ma possiamo provare perlomeno a cercare $L:\C^\times\to C$ tale che $\exp L=id_{\C^\times}$.\\
Insiemisticamente una tale mappa esiste\footnote{(Curiosit\`a) \`e una formulazione equivalente dell'assioma della scelta}, ma vorremmo cercarla continua. Purtroppo mostreremo che una tale funzione non esiste.

\begin{proposition}[L'esponenziale complesso \`e un rivestimento]\label{EsponenzialeComplessoERivestimento}
La mappa $\exp:\C\to \C^\times$ \`e un rivestimento.
\end{proposition}

\begin{corollary}
Non esiste $L:\C^\times\to \C$ continua tale che $e^{L(z)}=z$ per ogni $z\in\C^\times$.
\end{corollary}

\noindent
Cerchiamo allora di invertire l'esponenziale su un sottodominio di $\C^\times$ in modo continuo

\begin{theorem}[Branche del logaritmo]\label{BrancheDelLogaritmo}
Sia $U\subseteq \C^\times$ aperto connesso tale che l'inclusione $U\inj \C^\times$ induca l'omomorfismo banale $\pi_1(U)\to \pi_1(\C^\times)$\footnote{Per esempio possiamo prendere $U$ semplicemente connesso. L'idea geometrica \`e che non possiamo considerare un insieme che contenga lacci che si attorcigliano attorno l'origine.}. Allora esiste una mappa detta \textbf{branca del logaritmo} continua
\[L:U\to\C\quad t.c.\quad e^{L(z)}=z\quad \forall z\in U.\]
Due tali branche che coincidono in un punto coincidono su tutto $U$.
\end{theorem}

\noindent
Definiamo allora un logaritmo standard:
\begin{definition}[Branca principale del logaritmo complesso]
Sia \[U=\C\bs\{z\in\C\mid \Imag(z)=0\ e\ \Real(z)\leq 0\}.\]
Chiaramente $U\subseteq\C^\times$ \`e semplicemente connesso quindi il teorema (\ref{BrancheDelLogaritmo}) garantisce l'esistenza di un logaritmo $L:U\to \C$, il quale \`e unico per l'unicit\`a dei sollevamenti (\ref{UnicitaSollevamenti}) se fissiamo $L(1)=0$ (scelta lecita perch\'e $\exp\ii(1)=\{2k\pi\}_{k\in\Z}$).
\end{definition}
\begin{proposition}[Formula esplicita per la branca principale del logaritmo]
Se $L$ \`e la branca principale del logaritmo si ha che
\[L(z)=\log|z|+i\arg(z),\]
dove $\log$ \`e il logaritmo reale e $\arg(z)$ \`e l'argomento principale di $z$, definito per esempio da
\[\arg(x+iy)=\begin{cases}
\arccos\pa{\frac x{\sqrt{x^2+y^2}}} & y\geq 0\\
-\arccos\pa{\frac x{\sqrt{x^2+y^2}}} & y\leq 0
\end{cases}\]
\end{proposition}



\begin{proposition}[Le branche del logaritmo sono olomorfe]\label{BrancheDelLogaritmoSonoOlomorfe}
Le branche del logaritmo complesso sono olomorfe.
\end{proposition}

\begin{theorem}[Espansione in serie del logaritmo]\label{EspansioneInSerieDelLogaritmo}
Sia $L:\C\bs\{z\in\R\subseteq\C\mid z\leq 0\}\to \C$ la branca principale del logaritmo. Allora per ogni $z\in B_1(0)$ abbiamo
\[L(1+z)=\sum_{n=1}^\infty(-1)^{n+1}\frac{z^n}{n}\]
\end{theorem}

\section{1-Forme complesse}
\begin{definition}[1-forma continua]
Sia $U\subseteq\C$ un aperto. Una \textbf{1-Forma} continua su $U$\footnote{sottointenderemo ``continua" quasi sempre} \`e una funzione $\omega:U\times \C\to \C$ continua e $\R-$lineare nel secondo argomento.\\
Di solito al posto di $\omega(z_0,v)$ si scrive $\omega(z_0)(v)$ o $\omega_{z_0}(v)$ per rimarcare il fatto che $\omega(z_0,\cdot)$ \`e una funzione $\R-$lineare.
\end{definition}
\begin{remark}
Se $f:U\to \C$ \`e $C^1$ allora $df$ \`e una 1-forma, infatti
\[df(z,w)=df_z(w),\]
dove la continuit\`a rispetto a $z$ \`e la definizione di $C^1$.
\end{remark}

\begin{definition}[Forme esatte e chiuse]
Una 1-forma $\omega$ si dice \textbf{esatta} se esiste $f:U\to \C$ di classe $C^1$ tale che $\omega=df$. In tal caso $f$ si chiama \textbf{primitiva} di $\omega=df$.\\
Una 1-forma $\omega$ su $U$ si dice \textbf{chiusa} se \`e localmente esatta, cio\`e se per ogni $z_0\in U$ esiste un intorno $V$ di $z_0$ con $z_0\in V\subseteq U$ tale che $\omega\res V$ \`e esatta, cio\`e esiste $f:V\to \C$ di classe $C^1$ con $\omega\res V\doteqdot \omega\res{V\times \C}=df$.
\end{definition}
\begin{remark}
Per definizione ogni forma esatta \`e chiusa.
\end{remark}

\begin{remark}[1-forme in coordinate]
Poich\'e $\Hom_\R(\C,\C)$ \`e un $\C-$spazio vettoriale e sia $\{dx,dy\}$ che $\{dz,d\ol z\}$ ne sono basi, possiamo scrivere ogni 1-forma $\omega$ su $U$ nelle forme
\[\omega(x+iy)=a(x,y)dx+b(x,y)dy,\]
dove $a,b:U\to \C$ sono continue oppure
\[\omega(z)=f(z)dz+g(z)d\ol z\]
dove $f,g:U\to \C$ sono continue.
\end{remark}

\subsection{Integrazione di 1-forme}
\begin{definition}[Integrale di funzione da intervallo reale a $\C$]
Se $f(t)=g(t)+ih(t)$ con $g,h:[a,b]\to \R$ allora
\[\int_a^bf(t)dt=\int_a^bg(t)dt+i\int_a^bh(t)dt.\]
\end{definition}

\begin{definition}[Integrale lungo una curva $C^1$]
Sia $\omega$ una 1-forma su $U\subseteq \C$ e sia $\gamma:[a,b]\to U$ un cammino $C^1$. \textbf{L'integrale di $\omega$ lungo $\gamma$} \`e dato da
\[\int_\gamma\omega=\int_a^b\omega_{\gamma(t)}(\gamma'(t))dt.\]
\end{definition}

\begin{definition}[Curva $C^1$ a tratti]
Una curva $\gamma:[a,b]\to \C$ \`e \textbf{$C^1$ a tratti} se esiste una partizione $\{t_0,\cdots, t_n\}$\footnote{Ricordiamo da Analisi 1 che una partizione ha la forma
\[a=t_0<t_1<\cdots<t_{n-1}<t_n=b\]
} di $[a,b]$ tale che $\gamma\res{\spa{t_i,t_{i+1}}}$ \`e $C^1$ per ogni $i$.
\end{definition}

\begin{remark}[Caso per curve $C^1$ a tratti]
Se $\gamma:[a,b]\to \C$ \`e $C^1$ a tratti poniamo
\[\int_\gamma\omega=\sum_{i=0}^{n-1}\int_{t_i}^{t_{i+1}}\omega_{\gamma(t)}(\gamma'(t))dt\]
\end{remark}

\begin{proposition}[Invarianza dell'integrale per riparametrizzazione]\label{InvarianzaIntegralePerRiparametrizzazione}
Sia $\delta:[c,d]\to U$ una riparametrizzazione di $\gamma$, cio\`e $\delta=\gamma\circ \vp$ con $\vp:[c,d]\to [a,b]$ di classe $C^1$ con $\vp(c)=a$ e $\vp(d)=b$. Allora
\[\int_\delta\omega=\int_\gamma\omega\]
\end{proposition}
\begin{proposition}[Integrazione di 1-forme esatte]\label{IntegrazioneDi1formeEsatte}
Se $\omega$ \`e esatta con primitiva $f:U\to \C$ allora
\[\int_\gamma\omega=f(\gamma(b))-f(\gamma(a)).\]
\end{proposition}

\begin{corollary}
Se $\gamma:[a,b]\to U$ \`e una curva $C^1$ a tratti chiusa e $\omega$ \`e una forma esatta su $U$ allora $\int_\gamma\omega=0$.
\end{corollary}

\begin{example}[Forma chiusa non esatta]
Consideriamo la forma $\omega=\frac1zdz$ definita su $\C^\times$.\\
\ul{\textit{Chiusa}}) Segue dal fatto che il logaritmo \`e localmente olomorfo e la sua derivata \`e \[L'(z)=\frac1z.\]
\ul{\textit{Non Esatta}}) Sia $\gamma:[0,1]\to \C^\times$ data da $\gamma(t)=e^{2\pi it}$. Allora
\[\int_\gamma\omega=\int_0^1\frac1{\gamma(t)}\gamma'(t)dt=\int_0^1e^{-2\pi it}2\pi ie^{2\pi it}dt=2\pi i\neq 0.\]
\end{example}

\noindent
Poich\'e diremo spesso ``aperto connesso"diamo la seguente
\begin{definition}[Dominio]
Un insieme $D\subseteq \C$ \`e un \textbf{dominio} se \`e aperto e connesso.
\end{definition}
\begin{remark}
Poich\'e $\C$ \`e localmente connesso per archi si ha che un dominio   \`e anche connesso per archi e localmente connesso per archi (\ref{ApertoInLocalmenteConnessoPerArchiEConnessoPerArchi}).
\end{remark}



\begin{proposition}[Caratterizzazione esattezza con integrali su lacci]\label{CaratterizzazioneEsatezzaInTerminiDiIntegraliSuLacci}
Sia $D\subseteq \C$ dominio e $\omega$ una 1-forma su $D$. Allora $\omega$ \`e esatta se e solo se $\int_\gamma\omega=0$ per ogni $\gamma$ laccio su $D$ di classe $C^1$ a tratti.
\end{proposition}

\begin{corollary}[Caratterizzazione di forme esatte in disco]\label{CaratterizzazioneEsattezzaInDiscoConIntegraliSuBordiRettangolari}
Se $D$ \`e un disco allora $\omega$ \`e esatta in $D$ se e solo se $\int_\gamma\omega=0$ per ogni curva chiusa $\gamma$ bordo di un rettangolo parallelo agli assi.
\end{corollary}

\begin{corollary}
Se $D$ \`e un disco allora $\omega$ \`e esatta in $D$ se e solo se $\omega$ \`e chiusa in $D$.
\end{corollary}


\subsection{Primitive lungo curve e lungo omotopie}
\begin{lemma}[Numero di Lebesgue per forme chiuse]\label{NumeroLebesgueFormeChiuse}
Siano $\gamma:[a,b]\to D$ continua e $\omega$ una 1-forma chiusa su $D$. Allora esiste $\e>0$ tale che se $[c,d]\subseteq [a,b]$ e $d-c<\e$ allora esiste un disco $U\subseteq D$ tale che $\omega$ \`e esatta su $U$ e $\gamma([c,d])\subseteq U$. Per brevit\`a diremo che $\e$ \`e un \textbf{numero di Lebesgue} per $\omega$ sulla curva $\gamma$\footnote{Questa definizione NON \`e standard e NON \`e stata data durante il corso. L'ho data io per rendere pi\`u scorrevole il testo quando useremo questo lemma.}.
\end{lemma}

\begin{definition}[Primitiva lungo una curva]
Sia $\gamma:[a,b]\to D$ continua. Sia $\omega=Pdx+Qdy$ una 1-forma chiusa su $D$. Una \textbf{primitiva} per $\omega$ \textbf{lungo $\gamma$} \`e una funzione continua $f:[a,b]\to \C$ tale che per ogni $t_0\in[a,b]$ esistono $U\subseteq D$ intorno aperto di $\gamma(t_0)$ e $F:U\to \C$ primitiva di $\omega$ su $U$ tale che $F(\gamma(t))=f(t)$ per ogni $t\in\gamma\ii(U)$.\footnote{Ricordo che la notazione che uso in queste dispense assegna il simbolo $\ol \gamma$ al cammino inverso. Con $\gamma\ii(U)$ intendo la preimmagine di $U$ tramite la mappa $\gamma:[a,b]\to D$}
\end{definition}

\begin{theorem}[Esistenza e quasi unicit\`a delle primitive lungo curve]\label{EsistenzaQuasiUnicitaPrimitiveLungoCurve}
Per ogni $\gamma:[a,b]\to D$ continua e per ogni 1-forma $\omega$ chiusa su $D$ esiste una primitiva di $\omega$ lungo $\gamma$, che \`e unica a meno di costanti additive.
\end{theorem}


\begin{definition}[Primitiva lungo una omotopia]
Siano $H:[0,1]\times[a,b]\to D$ una omotopia fra $\gamma_0$ e $\gamma_1$ e $\omega$ una 1-forma chiusa su $D$. Una \textbf{primitiva di $\omega$ lungo $H$} \`e $f:[0,1]\times[a,b]\to D$ tale che per ogni $(s_0,t_0)\in [0,1]\times[a,b]$ esistono $U\subseteq D$ intorno di $H(s_0,t_0)$ e $F:U\to\C$ primitiva di $\omega$ tale che $F(H(s,t))=f(s,t)$ per ogni $(s,t)\in H\ii(U)$.
\end{definition}
\begin{remark}
La restrizione di $f$ primitiva lungo una omotopia ai segmenti orizzontali o verticali restituisce una primitiva di $\omega$ lungo la curva corrispondente.
\end{remark}

\begin{theorem}[Esistenza e quasi unicit\`a delle primitive lungo omotopie]\label{EsistenzaEQuasiUnicitaPrimitiveLungoOmotopie}
Se $\omega$ \`e una 1-forma chiusa allora ammette una primitiva lungo qualsiasi omotopia $H:[0,1]\times[a,b]\to D$.
\end{theorem}

\begin{corollary}[Integrazione di forme chiuse tramite primitiva lungo curve]\label{IntegrazioneFormeChiuseTramitePrimitivaLungoCurve}
Se $\omega$ \`e chiusa, $\gamma$ \`e $C^1$ a tratti e $f$ \`e primitiva di $\omega$ lungo $\gamma$ allora
\[\int_\gamma\omega=f(b)-f(a).\]
\end{corollary}

\noindent Se $\gamma$ \`e continua ma non $C^1$ a tratti diamo come definizione di integrale quella che mantiene vero questo risultato
\begin{definition}[Integrale di 1-forme chiuse su cammini continui]
Data $\omega$ una forma chiusa su $U$ e $\gamma:[a,b]\to U$, allora, detta $f$ una primitiva di $\omega$ lungo $\gamma$, definiamo
\[\int_\gamma\omega=f(b)-f(a).\]
\end{definition}


\begin{theorem}[Invarianza dell'integrale per cammini omotopi]\label{InvarianzaIntegraleCamminiOmotopi}
Siano $D\subseteq\C$ dominio, $\gamma_0,\gamma_1:[a,b]\to D$ curve omotope a estremi fissi e $\omega$ chiusa su $D$. Allora
\[\int_{\gamma_0}\omega=\int_{\gamma_1}\omega.\]
\end{theorem}


\subsection{Forme chiuse da funzioni olomorfe}
Da questo momento in poi nel corso la definizione di ``funzione olomorfa" diventer\`a implicitamente ``funzione olomorfa e $C^1$". Il motivo verr\`a reso chiaro tra qualche risultato.
\bigskip

\noindent Citiamo il
\begin{theorem}[Gauss-Green]\label{TeoremaGaussGreen}
Sia $\omega=Pdx+Qdy$ una 1-forma $C^1$ nell'intorno di un rettangolo $\ol R$ con lati paralleli agli assi. Allora
\[\iint_R\pa{\pdp Q-\pdp[y]P}dxdy=\int_{\partial R}\omega.\]
\end{theorem}
\begin{remark}
Se $\omega=Adz+Bd\ol z$ allora
\[\iint_R\pa{\pdp[z] B-\pdp[\ol z]A}dzd\ol z=\int_{\partial R}\omega.\]
\end{remark}
\begin{theorem}[Caratterizzazione delle forme chiuse $C^1$ con derivate parziali]\label{CaratterizzazioneFormeChiuseC1TramiteDerivateParziali}
Sia $\omega=Pdx+Qdy$ una 1-forma $C^1$ su $D$ dominio. Allora \[\omega  \text{ chiusa} \coimplies \pdp[y]P=\pdp Q.\]
\end{theorem}

\begin{corollary}[$fdz$ chiusa se e solo se $f$ olomorfa]\label{fdzchiusaSeESoloSefOlomorfaInC1}
Sia $f\in C^1(D)$. Allora $\omega=fdz$ \`e chiusa se e solo se $f$ \`e olomorfa.
\end{corollary}

\begin{corollary}
La forma $fdz$ \`e esatta in $D$ se e solo se esiste $F:D\to\C$ olomorfa tale che $F'=f$.
\end{corollary}

\noindent In realt\`a una implicazione della caratterizzazione (\ref{fdzchiusaSeESoloSefOlomorfaInC1}) vale anche senza supporre $f$ di classe $C^1$.

\begin{theorem}[Cauchy]\label{OlomorfaDefinisceFormaChiusa}
Se $f$ \`e olomorfa allora $fdz$ \`e chiusa.
\end{theorem}

\noindent Poich\'e l'implicazione (\ref{OlomorfaDefinisceFormaChiusa}) non \`e stata dimostrata, la trattazione che segue \`e rigorosa solo se sostituiamo la parola ``olomorfa" con ``olomorfa $C^1$". Per rendere pi\`u generali le dispense proceder\`o invocando quando necessario il teorema di Cauchy (\ref{OlomorfaDefinisceFormaChiusa}) quando il corso ha usato l'analogo per le $C^1$. Se preferite attenervi solo a ci\`o che \`e stato dimostrato considerate solo funzioni olomorfe di classe $C^1$ e sostituite ogni riferimento a (\ref{OlomorfaDefinisceFormaChiusa}) con un riferimento a (\ref{fdzchiusaSeESoloSefOlomorfaInC1}).
\bigskip

\noindent
Concludiamo la sezione fornendo un utile risultato sull'estensione dell'olomorfia.


\begin{theorem}[Continua olomorfa fuori un segmento d\`a forma chiusa]\label{ContinuaOlomorfaEccettoInSegmentoDaFormaChiusa}
Sia $D\subseteq \C$ aperto, $f$ continua in $D$ e olomorfa fuori da un segmento $L\subseteq D$. Allora $fdz$ \`e chiusa in $D$.
\end{theorem}
\begin{corollary}\label{ContinuaOlomorfaEccettoUnPuntoEOlomorfa}
Sia $D\subseteq \C$ aperto, $f$ continua in $D$ e olomorfa $C^1$ fuori da un segmento $L\subseteq D$. Allora $f$ \`e olomorfa in $D$.
\end{corollary}

\section{Indice di avvolgimento e Formula di Cauchy}
\begin{definition}[Indice di avvolgimento]
Sia $\gamma$ una curva chiusa in $\C$ e sia $z_0$ un punto che NON appartiene al supporto di $\gamma$ ($z_0\in \C\bs \gamma([0,1])$). Allora l'\textbf{indice di avvolgimento} di $\gamma$ attorno $z_0$ \`e dato da
\[\Ind(\gamma,z_0)=\frac1{2\pi i}\int_{\gamma}\frac{dz}{z-z_0}.\]
\end{definition}
\begin{remark}
La forma $\frac{dz}{z-z_0}$ \`e chiusa in $\C\bs\{z_0\}$, quindi ammette primitiva lungo $\gamma$ anche se $\gamma$ \`e continua ma non $C^1$ (\ref{EsistenzaQuasiUnicitaPrimitiveLungoCurve}). L'integrale nella definizione di $\Ind(\gamma,z_0)$ \`e quindi ben definito anche se $\gamma$ non \`e $C^1$ a tratti.
\end{remark}

\begin{proposition}[Indice di avvolgimento \`e intero]\label{IndiceAvvolgimentoEIntero}
Per ogni $z_0$ e per ogni curva continua $\gamma$ che non passa per $z_0$ si ha che
\[\Ind(\gamma,z_0)\in \Z.\]
\end{proposition}
\begin{remark}
Intuitivamente $\Ind(\gamma,z_0)$ conta quante volte $\gamma$ ``gira attorno a $z_0$".
\end{remark}





\begin{theorem}[Formula integrale di Cauchy]\label{FormulaIntegraleCauchy}
Siano $D\subseteq\C$ aperto, $f\in\Oc(D)$ e $\gamma:[0,1]\to D$ una curva chiusa omotopa ad una curva costante. Sia $z\in D\bs\gamma([0,1])$. Allora\footnote{Qui e in futuro useremo $\zeta$ quando $z$ \`e gi\`a occupato. Quindi per $d\zeta$ intendiamo comunque $dx+idy$.}
\[\Ind(\gamma,z)f(z)=\frac1{2\pi i}\int_\gamma \frac{f(\zeta)}{\zeta-z}d\zeta.\]
\end{theorem}
\begin{remark}
Dal teorema segue immediatamente che una funzione olomorfa definita su un disco \`e completamente determinata dal suo valore sul bordo del disco.
\end{remark}

\subsection{Olomorfa implica analitica}
\begin{theorem}[Continue sul bordo di un disco definiscono olomorfa nel disco]\label{ContinueSuBordoDiDiscoDefinisconoOlomorfaInDisco}
Sia $B=B(z_0,R)$ e $h:\partial B\to \C$ continua. Sia $\gamma:[0,2\pi]\to \C$ data da $\gamma(t)=z_0+Re^{it}$ e poniamo
\[f(z)=\frac1{2\pi i}\int_{\gamma}\frac{h(\zeta)}{\zeta-z}d\zeta.\]
Allora abbiamo che
\begin{enumerate}[noitemsep]
\item $f(z)=\ser na_n(z-z_0)^n$ serie di potenze con raggio di convergenza maggiore o uguale a $R$.
\item $a_n=\displaystyle\frac1{2\pi i}\int_\gamma\frac{h(\zeta)}{(\zeta-z_0)^{n+1}}d\zeta$.
\end{enumerate}
In particolare $f\in \Oc(B)$ e $a_n=\frac{f^{(n)}(z_0)}{n!}$.
\end{theorem}


\begin{corollary}[Una funzione olomorfa \`e analitica]\label{OlomorfaImplicaAnalitica}
Dati $D\subseteq \C$ aperto, $f\in \Oc(D)$ e $\ol B=\ol{B(z_0,R)}\subseteq D$ si ha che
\begin{enumerate}[noitemsep]
\item $\displaystyle f(z)=\ser na_n(z-z_0)^n$ con raggio di convergenza maggiore o uguale a $R$
\item Posto $\gamma(t)=z_0+Re^{it}$ si ha che \[a_n=\frac{f^{(n)}(z_0)}{n!}=\frac1{2\pi i}\int_{\gamma}\frac{f(\zeta)}{(\zeta-z_0)^{n+1}}d\zeta.\]
\end{enumerate}
\end{corollary}

\subsubsection{Propriet\`a delle olomorfe ereditate delle analitiche}
\begin{corollary}[Propriet\`a delle olomorfe ereditate dalle analitiche]\label{ProprietaCheOlomorfeEreditanoDaAnalitiche}
Sia $D\subseteq \C$ aperto, $f\in \Oc(D)$ e $\ol B=\ol{B(z_0,R)}\subseteq D$, allora
\begin{enumerate}[noitemsep]
\item $f\in C^\infty(D)$
\item Per ogni $z\in B$ e per ogni $n\in \N$
\[f^{(n)}(z)=n!\cauchyint[n+1]<-z>{\gamma},\]
dove $\gamma(t)=z_0+Re^{it}$.
\item Per ogni $n\in\N$ abbiamo $f^{(n)}\in \Oc(D)$
\end{enumerate}
\end{corollary}


\noindent Data l'importanza della seguente propriet\`a ereditata dalle analitiche la isoliamo in una proposizione:
\begin{proposition}[Principio di identit\`a]\label{PrincipioDiIdentita}
Sia $D$ un dominio e siano $f,g:D\to \C$ olomorfe. Se $f$ e $g$ coincidono su un aperto allora coincidono su tutto $D$.
\end{proposition}

\section{Applicazioni}
\subsection{Disuguaglianze di Cauchy e Teorema di Liouville}
\begin{notation}
Siano $D\subseteq \C$ aperto, $z_0\in D$, $\ol B=\ol{B(z_0,R)}\subseteq D$ e $f$ olomorfa su $D$. Per ogni $r\leq R$ poniamo
\[M(r)=\max\cpa{|f(z)|\sep \abs{z-z_0}=r}=\max_{z\in \partial B_r(z_0)}\{|f(z)|\}.\]
\end{notation}
\begin{proposition}[Disuguaglianze di Cauchy]\label{DisuguaglianzeCauchy}
Siano $D\subseteq \C$ aperto, $z_0\in D$, $\ol B=\ol{B(z_0,r)}\subseteq D$ e $f$ olomorfa su $D$.
Allora
\[\abs{f^{(n)}(z_0)}\leq \frac{n!M(r)}{r^n}.\]
\end{proposition}
\begin{remark}
Le disuguaglianze di Cauchy danno una maggiorazione di $\abs{f^{(n)}(z_0)}$ in termini di $n,\ r$ e il massimo che $f$ assume sul BORDO del disco di raggio $r$ centrato in $z_0$.
\end{remark}

\begin{corollary}[Teorema di Liouville]\label{TeoremaLiouville}
Ogni funzione limitata olomorfa su $\C$\footnote{Le funzioni olomorfe su $\C$ si dicono \textbf{intere}.} \`e costante.
\end{corollary}

\begin{corollary}[Teorema fondamentale dell'algebra]\label{TeoremaFondamentaleAlgebra}
Ogni polinomio $p\in\C[z]$ di grado $n\geq 1$ ha esattamente $n$ radici distinte contate con molteplicit\`a.
\end{corollary}

\subsection{Principio della media}
\begin{definition}[Propriet\`a della media]
Una funzione $f:D\to \C$ soddisfa la \textbf{propriet\`a della media} se per ogni $z_0\in D$ esiste $r_0>0$ tale che $\ol{B_{r_0}(z_0)}\subseteq D$ e per ogni $r\in(0,r_0)$ si ha che
\[f(z_0)=\frac1{2\pi}\int_0^{2\pi}f(z_0+re^{it})dt.\]
\end{definition}
\begin{proposition}[Propriet\`a della media implica continua]\label{ProprietaDellaMediaImplicaContinua}
Se $f:D\to \C$ soddisfa la propriet\`a della media allora \`e continua.
\end{proposition}
\begin{theorem}[Principio della media]\label{PrincipioDellaMedia}
Siano $D\subseteq\C$ aperto, $f\in\Oc(D)$ e $\ol B=\ol{B(z_0,r)}\subseteq D$. Allora
\[f(z_0)=\frac1{2\pi}\int_0^{2\pi} f(z_0+re^{it})dt,\]
in particolare $f$ soddisfa la propriet\`a della media.
\end{theorem}
\begin{remark}
Nelle ipotesi del teorema, anche $\Real(f)$ e $\Imag(f)$ soddisfano la propriet\`a della media.
\end{remark}
\begin{remark}[Propriet\`a della media su area]
Nelle ipotesi del teorema vale
\[f(z_0)=\frac1{\pi r^2}\int_Bf(z)dxdy.\]
\end{remark}



\subsection{Principio del massimo}
\begin{theorem}[Principio del massimo 1]\label{PrincipioMassimo1}
Sia $f:D\to \C$ con la propriet\`a della media. Supponiamo che esista $z_0\in D$ massimo locale per $|f|$ (oppure per $f$ se $f:D\to \R$). Allora $f$ \`e costante in un intorno di $z_0$.
\end{theorem}

\begin{corollary}[Olomorfa con massimo locale in modulo o per componente \`e costante]\label{OlomorfaConMassimoLocalInModuloOPerComponenteECostante}
Se $D\subseteq \C$ dominio e $f\in\Oc(D)$ \`e tale che esiste $z_0\in D$ punto di massimo locale per $|f|$ (oppure $\Real(f)$ oppure $\Imag(f)$) allora $f$ \`e costante.
\end{corollary}


\begin{corollary}[Principio del massimo 2]\label{PrincipioMassimo2}
Sia $D\subseteq \C$ un dominio limitato. Sia $f\in \Oc(D)\cap C^0\pa{\ol D}$. Poniamo
\[M=\max_{x\in \partial D}|f(z)|.\]
Allora
\begin{enumerate}[noitemsep]
\item $|f(z)|\leq M$ per ogni $z\in D$
\item se esiste $z_0\in D$ tale che $|f(z_0)|=M$ allora $f$ \`e costante.
\end{enumerate}
Valgono anche risultati analoghi per $\Real(f)$ e $\Imag(f)$ al posto di $|f|$.
\end{corollary}
\begin{remark}
L'ipotesi di limitatezza \`e necessaria. Consideriamo per esempio
\[D=\cpa{z\in\C\sep\Real(z)>0}\quad\text{e}\quad f(z)=e^z.\]
Evidentemente $e^z$ \`e olomorfa e continua su $D$ perch\'e restrizione di olomorfa su $\C$. Osserviamo che $|f|\res{\partial D}=1$ ma $\displaystyle\sup_{z\in D}|f(z)|=+\infty$.\\
Il problema \`e il comportamento nel punto all'infinito.
\end{remark}
\begin{remark}
Nel caso $\Real(f)$ e $\Imag(f)$ valgono anche dei principi del minimo. Basta applicare il principio del massimo a $-\Real(f)$ e $-\Imag(f)$.
\end{remark}

\begin{theorem}[Teorema dell'applicazione aperta]\label{TeoremaApplicazioneAperta}
Ogni $f\in \Oc(D)$ non costante \`e una mappa aperta.
\end{theorem}

\section{Singolarit\`a}
\begin{definition}[Singolarit\`a]
Se $f\in \Oc(B_r(z_0)\bs\{z_0\})$ allora $z_0$ \`e una \textbf{singolarit\`a} per $f$.
\end{definition}


\begin{definition}[Corona]
Dati $0\leq r<R\leq+\infty$, l'\textbf{anello} (o \textbf{corona}) definito da $r$ e $R$ \`e l'insieme
\[A(r,R)=\cpa{z\in\C\sep r<|z|<R}.\]
Se $r>R$ poniamo $A(r,R)=\emptyset$.
\end{definition}
\begin{remark}
Esempi particolari di anelli sono $A(0,R)=B(0,R)\nz$, $A(r,+\infty)=\C\bs\ol{B(0,r)}$ e $A(0,+\infty)=\C\nz$
\end{remark}
\begin{lemma}\label{IntegraleDiUnGiroAttornoAnelloNonDipendeDallaDistanzaDalCentro}
Se $f\in \Oc(A(r,R))$ allora dato $\rho\in(r,R)$, $\int_{\rho e^{2\pi it}}fdz$ non dipende da $\rho$.
\end{lemma}
\begin{lemma}\label{ValoriOlomorfaSuAnelloComeSottrazioneDiIntegrali}
Sia $f\in \Oc(A(r,R))$ e $z_0\in A(r,R)$. Siano $\rho_1<\rho_2$ e $r<\rho_1<|z_0|<\rho_2<R$. Allora
\[f(z_0)=\cauchyint{\rho_2e^{2\pi it}}-\cauchyint{\rho_1e^{2\pi it}}\]
\end{lemma}

\subsection{Serie di Laurent}
Poich\'e le serie di potenze convergono in un intero disco non possono essere lo strumento adatto per catturare le singolarit\`a. Consideriamo allora una generalizzazione delle serie di potenze che ammette potenze negative, ovvero le serie di Laurent
\begin{definition}[Serie di Laurent]
Una \textbf{serie di Laurent} centrata in $z_0$ \`e un'espressione della forma
\[f=\sum_{n=-\infty}^{+\infty}a_n(z-z_0)^n,\]
con $a_n\in\C$.\\
Diremo che $f$ \`e \textbf{convergente} se
\[f_+=\ser n a_n(z-z_0)^n\quad \text{e}\quad f_-=\ser m a_{-m}\pa{\frac1{z-z_0}}^m\]
sono convergenti.
\end{definition}
\begin{remark}[Le serie di Laurent convergono in anello]
Data $f$ serie di Laurent proviamo a capire dove converge assolutamente. Per ci\`o che sappiamo sulle serie di potenze si ha che $f_+$ converge assolutamente in un disco. Sia $R$ il raggio di questo disco. Possiamo interpretare $f_-$ come una serie di potenze nelle $\frac1{z-z_0}$ e questo restituisce un raggio di convergenza assoluta $\frac1r$, cio\`e $f_-$ converge assolutamente se
\[\frac1{|z-z_0|}<\frac1r\coimplies |z-z_0|>r.\]
Mettendo insieme queste informazioni abbiamo che $f$ converge assolutamente nell'anello
\[z_0+A(r,R).\]
Osserviamo che se $r>R$ allora $f$ non converge perch\'e in ogni punto una tra $f_+$ e $f_-$ non converge.
\end{remark}


\begin{proposition}[propriet\`a delle serie di Laurent]\label{ProprietaLaurent}
Se $f(z)=\sum_{n\in\Z}a_n(z-z_0)^n$ converge in $\Omega=z_0+A(r,R)$ allora
\begin{enumerate}[noitemsep]
\item $f$ \`e olomorfa in $\Omega$
\item per ogni $r<\rho<R$ vale $\displaystyle a_n=\cauchyint[n+1]{z_0+\rho e^{2\pi it}}$.
\end{enumerate}
\end{proposition}

\begin{theorem}[Olomorfa su Anello \`e serie di Laurent]\label{OlomorfaSuAnelloESerieDiLaurent}
Siano $0\leq r<R\leq +\infty$, $z_0\in \C$ e $\Omega=z_0+A(r,R)$. Se $f\in\Oc(\Omega)$ allora $f$ si esprime tramite un'unica serie di Laurent in $z_0$ convergente in $\Omega$.
\end{theorem}


\subsection{Tipi di singolarit\`a}
\begin{definition}[Tipi di singolarit\`a]
Sia $f\in \Oc(B_R(z_0)\bs\{z_0\})=\Oc(z_0+A(0,R))$ e scriviamo $f$ in serie di Laurent come segue
\[f(z)=\sum_{n\in\Z}a_n(z-z_0)^n.\]
Si dice che $z_0$ \`e
\begin{itemize}[noitemsep]
\item una \textbf{singolarit\`a eliminabile} se $a_n=0$ per ogni $n<0$.
\item un \textbf{polo} di \textbf{ordine} $k>0$ se $a_{-k}\neq 0$ e $a_n=0$ per $n<-k$.\\
Un polo si dice \textbf{semplice} se ha ordine $1$.
\item una \textbf{singolarit\`a essenziale} altrimenti.
\end{itemize}
\end{definition}
\begin{remark}[Singolarit\`a eliminabili possono essere ignorate]
Se $z_0$ \`e una singolarit\`a eliminabile di $f=\sum a_k(z-z_0)^k$ allora ponendo $f(z_0)=a_0$ si ha che $f\in\Oc(B_R(z_0))$.
\end{remark}
\begin{remark}[I poli si possono fattorizzare]
Se $z_0$ \`e un polo di ordine $k$ per $f=\sum_n a_n$ allora
\[f(z)=\frac1{(z-z_0)^k}h(z)\]
dove $h$ \`e una funzione olomorfa e $h(z_0)\neq 0$.
\end{remark}

\begin{theorem}[Estensione di Riemann]\label{TeoremaEstensioneRiemann}
Sia $f\in \Oc(B^\ast)$, dove $B^\ast=B_R(z_0)\bs\{z_0\}$. Allora se $|f|$ \`e limitato in $B^\ast$ si ha che $z_0$ \`e una singolarit\`a eliminabile.
\end{theorem}

\begin{theorem}[Casorati-Weierstass]\label{TeoremaCasoratiWeierstrass}
Sia $f\in \Oc(B^\ast)$, dove $B^\ast=B_R(z_0)\bs\{z_0\}$. Se $z_0$ \`e un singolarit\`a essenziale allora $f(B^\ast)$ \`e denso in $\C$.
\end{theorem}


\section{Funzioni meromorfe e Residui}
\begin{definition}[Funzione meromorfa]
Sia $U\subseteq \C$ aperto. Una funzione \textbf{meromorfa} su $U$ \`e una funzione olomorfa $f:U\bs S\to \C$ dove $S\subseteq U$ \`e discreto e per ogni $z_0\in S$ si ha che $z_0$ \`e un polo.\footnote{Intuitivamente una funzione \`e meromorfa se \`e olomorfa ovunque eccetto in qualche polo isolato.}
\end{definition}

\begin{remark}
Se $f,g:U\to\C$ sono olomorfe con $U$ connesso e $g$ non costantemente nulla, allora $\frac fg$ \`e meromorfa.
\end{remark}

\begin{remark}[Inversione mantiene l'ordine scambiando zeri e poli]
Se $f:U\to \C$ \`e olomorfa non costantemente nulla con $U$ dominio e $f(z_0)=0$ con ordine di annullamento $n_0$ allora $\frac1{f}$ ha un polo di ordine $n_0$ in $z_0$.
\end{remark}

\begin{definition}[Residuo]
Siano $U\subseteq \C$ aperto, $z_0\in U$ e $f:U\bs \{z_0\}\to \C$ olomorfa. Definiamo il \textbf{residuo} di $f$ in $z_0$ come
\[\Res(f,z_0)=a_{-1}=\cauchyint[-1+1]{\al}=\frac1{2\pi i}\int_\al f(z)dz,\]
dove $\sum_{n\in\Z}a_n(z-z_0)^n$ \`e lo sviluppo di Laurent di $f$ su $B(z_0,\e)\bs\{z_0\}$ per $\e$ abbastanza piccolo e $\al(t)=z_0+\e e^{it}$.
\end{definition}
\begin{remark}
Il residuo \`e una misura di   ``quanto la forma $fdz$ non \`e esatta".
\end{remark}



\begin{proposition}[Formula del residuo per rapporto di olomorfe]\label{FormulaResiduoPerRapportoDiOlomorfe}
Se $g,h:U\to \C$ olomorfe con $z_0$ uno zero di ordine $1$ per $h$ e $g(z_0)\neq 0$ allora posta $f=\frac gh$ abbiamo che $z_0$ \`e un polo di ordine $1$ per $f$ e
\[\Res(f,z_0)=\lim_{z\to z_0}(z-z_0)\frac{g(z)}{h(z)}=\frac{g(z_0)}{h'(z_0)}.\]
\end{proposition}

\begin{theorem}[Teorema dei Residui]\label{TeoremaResidui}
Sia $U\subseteq \C$ aperto, $S\subseteq U$ discreto chiuso\footnote{Per esempio: una successione che tende a un punto in $U$ non \`e ammessa, invece una che accumula a un punto sul bordo di $U$ va bene perch\'e come sottoinsieme di $U$ \`e chiusa.}, $f:U\bs S\to \C$ olomorfa. Sia $K\subseteq U$ una regione compatta omeomorfa a $D^2$. Sia $\al:[0,1]\to\partial K$ una parametrizzazione di $\partial K$ in senso antiorario. Assumiamo anche $S\cap \partial K=\emptyset$. Allora valgono le seguenti affermazioni:
\begin{enumerate}[noitemsep]
\item $S\cap K$ \`e finito
\item $\displaystyle\int_\al f(z)dz=2\pi i\sum_{z_0\in S\cap K}\Res(f,z_0)$.
\end{enumerate}
\end{theorem}

\subsection{Derivata Logaritmica}
\begin{definition}[Derivata logaritmica]
Data una funzione derivabile $f$ definiamo la sua \textbf{derivata logaritmica} nei punti dove non si annulla come
\[\frac{f'}f.\]
\end{definition}
\begin{remark}[La derivata logaritmica trasforma poli e zeri in poli semplici]
Se $z_0$ \`e uno zero o polo di $f$ tale che
\[f(z)=(z-z_0)^{n_0}g(z)\]
con $g(z_0)\neq 0$, $g(z)$ olomorfa vicina a $z_0$ e $n_0\neq 0$ allora vicino a $z_0$ si ha che
\begin{align*}
\frac{f'(z)}{f(z)}=&\frac{n_0(z-z_0)^{n_0-1}g(z)+(z-z_0)^{n_0}g'(z)}{(z-z_0)^{n_0}g(z)}=\\
=&\frac{n_0}{z-z_0}+\frac{g'(z)}{g(z)}.
\end{align*}
Osserviamo inoltre che $\frac{g'(z)}{g(z)}$ \`e olomorfa vicino a $z_0$ perch\'e $g(z_0)\neq 0$.
\end{remark}
\begin{remark}
Simbolicamente si ha che
\[\frac{f'}{f}=\log(f)',\]
da cui il nome.
\end{remark}


\begin{proposition}[Residuo in polo della derivata logaritmica \`e l'ordine del polo/zero]\label{ResiduoDellaDerivataLogaritmicaEOrdine}
Se $f:U\to \C$ \`e meromorfa anche $\frac{f'}{f}$ lo \`e e i poli di $\frac{f'}{f}$ coincidono esattamente con i poli e gli zeri di $f$, inoltre
\[\Res\pa{\frac{f'}f,z_0}=n_0\]
dove $n_0$ \`e l'ordine di $z_0$ come zero di $f$ se positivo o $-n_0$ \`e l'ordine di $z_0$ come polo di $f$.
\end{proposition}

\begin{theorem}[Teorema di Derivata logaritmica]\label{TeoremaDerivataLogaritmica}
Sia $f:U\to \C$ meromorfa e $K\subseteq U$ compatto omeomorfo a $D^2$. Se $\al$ \`e una parametrizzazione in senso antiorario di $\partial K$ e $\partial K$ non contiene n\'e zeri n\'e poli di $f$ allora
\[\int_\al\frac{f'(z)}{f(z)}dz=2\pi i(Z-P)\]
dove $Z$ \`e la somma degli ordini di tutti gli zeri di $f$ contenuti in $K$ e $P$ \`e la somma degli ordini di tutti i poli di $f$ in $K$.
\end{theorem}

\begin{corollary}[Teorema di Rouch\'e]\label{TeoremaDiRouche}
Siano $f,g:U\to\C$ olomorfe e  $K\subseteq U$ un compatto omeomorfo a $D^2$. Supponiamo che $|g(z)|<|f(z)|$ per ogni $z\in\partial K$ (in particolare $f(z)\neq 0$ per ogni $z\in\partial K$). Allora il numero di zeri contati con molteplicit\`a di $f$ e di $f-g$ in $K$ coincidono.
\end{corollary}

\end{multicols*}
