\documentclass[10pt]{report}
\usepackage[top=1.5cm,bottom=1.5cm,left=1.5cm,right=1.5cm]{geometry}
\usepackage[utf8]{inputenc}
\usepackage[italian]{babel}
\usepackage{amsmath,amssymb,amsfonts,amsthm,stmaryrd}
\usepackage{mathrsfs} % per mathscr
\usepackage{graphicx}% ruota freccia per le azioni
\usepackage{marvosym}% per il \Lightning
\usepackage{array}
\usepackage{faktor} % per gli insiemi quoziente
\usepackage{hyperref}
\usepackage{xparse} % Per nuovi comandi con tanti input opzionali
\usepackage{relsize} % per \mathlarger
\usepackage{tikz-cd}
\usepackage{multicol}
\usepackage{multirow}
\usepackage{cancel}
\usepackage{enumitem}
\usepackage{fourier}

\newtheoremstyle{customth}
{\topsep}{\topsep}
{\itshape}{}{\bfseries}{.}{\newline}{}

\newtheoremstyle{customdef}
{\topsep}{\topsep}
{\normalfont}{}{\bfseries}{.}{\newline}{}

\newtheoremstyle{customrem}
{\topsep}{\topsep}
{\normalfont}{}{\itshape}{.}{\newline}{}

\theoremstyle{customth}
\newtheorem{theorem}{Teorema}[chapter]
\newtheorem{lemma}[theorem]{Lemma}
\newtheorem{corollary}[theorem]{Corollario}
\newtheorem{proposition}[theorem]{Proposizione}
\newtheorem{fact}[theorem]{Fatto}
\newtheorem{application}[theorem]{Applicazione}
\theoremstyle{customrem}
\newtheorem{remark}
[theorem]{Osservazione}
\theoremstyle{customdef}
\newtheorem{definition}[theorem]{Definizione}
\newtheorem{notation}[theorem]{Notazione}
\newtheorem{example}[theorem]{Esempio}


\makeatletter
\renewenvironment{proof}[1][\proofname]{\par
  \pushQED{\qed}%
  \normalfont \topsep6\p@\@plus6\p@\relax
  \trivlist
  \item[\hskip\labelsep
        \itshape
    #1\@addpunct{.}]\mbox{}\\*
}{%
  \popQED\endtrivlist\@endpefalse
}
\makeatother

%============ Simboli standard =================
%----------------- Lettere ---------------------
\newcommand{\A}{\mathbb{A}}
\newcommand{\B}{\mathbb{B}}
\newcommand{\C}{\mathbb{C}}
\newcommand{\D}{\mathbb{D}}
\newcommand{\E}{\mathbb{E}}
\newcommand{\F}{\mathbb{F}}
\newcommand{\G}{\mathbb{G}}
\newcommand{\Hb}{\mathbb{H}}
\newcommand{\I}{\mathbb{I}}
\newcommand{\J}{\mathbb{J}}
\newcommand{\K}{\mathbb{K}}
\newcommand{\Lb}{\mathbb{L}}
\newcommand{\M}{\mathbb{M}}
\newcommand{\N}{\mathbb{N}}
\newcommand{\Ob}{\mathbb{O}}
\newcommand{\Pj}{\mathbb{P}}
\newcommand{\Q}{\mathbb{Q}}
\newcommand{\R}{\mathbb{R}}
\newcommand{\Sb}{\mathbb{S}}
\newcommand{\T}{\mathbb{T}}
\newcommand{\U}{\mathbb{U}}
\newcommand{\V}{\mathbb{V}}
\newcommand{\W}{\mathbb{W}}
\newcommand{\X}{\mathbb{X}}
\newcommand{\Y}{\mathbb{Y}}
\newcommand{\Z}{\mathbb{Z}}

\newcommand{\Ac}{\mathcal{A}}
\newcommand{\Bc}{\mathcal{B}}
\newcommand{\Cc}{\mathcal{C}}
\newcommand{\Dc}{\mathcal{D}}
\newcommand{\Ec}{\mathcal{E}}
\newcommand{\Fc}{\mathcal{F}}
\newcommand{\Gc}{\mathcal{G}}
\newcommand{\Hc}{\mathcal{H}}
\newcommand{\Ic}{\mathcal{I}}
\newcommand{\Jc}{\mathcal{J}}
\newcommand{\Kc}{\mathcal{K}}
\newcommand{\Lc}{\mathcal{L}}
\newcommand{\Mc}{\mathcal{M}}
\newcommand{\Nc}{\mathcal{N}}
\newcommand{\Oc}{\mathcal{O}}
\newcommand{\Pc}{\mathcal{P}}
\newcommand{\Qc}{\mathcal{Q}}
\newcommand{\Rc}{\mathcal{R}}
\newcommand{\Sc}{\mathcal{S}}
\newcommand{\Tc}{\mathcal{T}}
\newcommand{\Uc}{\mathcal{U}}
\newcommand{\Vc}{\mathcal{V}}
\newcommand{\Wc}{\mathcal{W}}
\newcommand{\Xc}{\mathcal{X}}
\newcommand{\Yc}{\mathcal{Y}}
\newcommand{\Zc}{\mathcal{Z}}

\newcommand{\Af}{\mathfrak{A}}
\newcommand{\Bf}{\mathfrak{B}}
\newcommand{\Cf}{\mathfrak{C}}
\newcommand{\Df}{\mathfrak{D}}
\newcommand{\Ef}{\mathfrak{E}}
\newcommand{\Ff}{\mathfrak{F}}
\newcommand{\Gf}{\mathfrak{G}}
\newcommand{\Hf}{\mathfrak{H}}
\newcommand{\If}{\mathfrak{I}}
\newcommand{\Jf}{\mathfrak{J}}
\newcommand{\Kf}{\mathfrak{K}}
\newcommand{\Lf}{\mathfrak{L}}
\newcommand{\Mf}{\mathfrak{M}}
\newcommand{\Nf}{\mathfrak{N}}
\newcommand{\Of}{\mathfrak{O}}
\newcommand{\Pf}{\mathfrak{P}}
\newcommand{\Qf}{\mathfrak{Q}}
\newcommand{\Rf}{\mathfrak{R}}
\newcommand{\Sf}{\mathfrak{S}}
\newcommand{\Tf}{\mathfrak{T}}
\newcommand{\Uf}{\mathfrak{U}}
\newcommand{\Vf}{\mathfrak{V}}
\newcommand{\Wf}{\mathfrak{W}}
\newcommand{\Xf}{\mathfrak{X}}
\newcommand{\Yf}{\mathfrak{Y}}
\newcommand{\Zf}{\mathfrak{Z}}

\newcommand{\As}{\mathscr{A}}
\newcommand{\Bs}{\mathscr{B}}
\newcommand{\Cs}{\mathscr{C}}
\newcommand{\Ds}{\mathscr{D}}
\newcommand{\Es}{\mathscr{E}}
\newcommand{\Fs}{\mathscr{F}}
\newcommand{\Gs}{\mathscr{G}}
\newcommand{\Hs}{\mathscr{H}}
\newcommand{\Is}{\mathscr{I}}
\newcommand{\Js}{\mathscr{J}}
\newcommand{\Ks}{\mathscr{K}}
\newcommand{\Ls}{\mathscr{L}}
\newcommand{\Ms}{\mathscr{M}}
\newcommand{\Ns}{\mathscr{N}}
\newcommand{\Os}{\mathscr{O}}
\newcommand{\Ps}{\mathscr{P}}
\newcommand{\Qs}{\mathscr{Q}}
\newcommand{\Rs}{\mathscr{R}}
\newcommand{\Ss}{\mathscr{S}}
\newcommand{\Ts}{\mathscr{T}}
\newcommand{\Us}{\mathscr{U}}
\newcommand{\Vs}{\mathscr{V}}
\newcommand{\Ws}{\mathscr{W}}
\newcommand{\Xs}{\mathscr{X}}
\newcommand{\Ys}{\mathscr{Y}}
\newcommand{\Zs}{\mathscr{Z}}

\newcommand{\ula}{{\underline{a}}}
\newcommand{\ulb}{{\underline{b}}}
\newcommand{\ulc}{{\underline{c}}}
\newcommand{\uld}{{\underline{d}}}
\newcommand{\ule}{{\underline{e}}}
\newcommand{\ulf}{{\underline{f}}}
\newcommand{\ulg}{{\underline{g}}}
\newcommand{\ulh}{{\underline{h}}}
\newcommand{\uli}{{\underline{i}}}
\newcommand{\ulj}{{\underline{j}}}
\newcommand{\ulk}{{\underline{k}}}
\newcommand{\ull}{{\underline{l}}}
\newcommand{\ulm}{{\underline{m}}}
\newcommand{\uln}{{\underline{n}}}
\newcommand{\ulo}{{\underline{o}}}
\newcommand{\ulp}{{\underline{p}}}
\newcommand{\ulq}{{\underline{q}}}
\newcommand{\ulr}{{\underline{r}}}
\newcommand{\uls}{{\underline{s}}}
\newcommand{\ult}{{\underline{t}}}
\newcommand{\ulu}{{\underline{u}}}
\newcommand{\ulv}{{\underline{v}}}
\newcommand{\ulw}{{\underline{w}}}
\newcommand{\ulx}{{\underline{x}}}
\newcommand{\uly}{{\underline{y}}}
\newcommand{\ulz}{{\underline{z}}}

%---------- Funzioni standard ------------------
\newcommand{\Ann}{\mathrm{Ann}\,}
\newcommand{\Arg}{\mathrm{Arg}\,}
\newcommand{\Aut}{\mathrm{Aut}\,}
\newcommand{\adj}{\mathrm{adj}\,}
\newcommand{\cha}{\mathrm{char}\,}
\newcommand{\comb}{\mathrm{Comb}\,}
\newcommand{\End}{\mathrm{End}\,}
\newcommand{\Fix}{\mathrm{Fix}\,}
\newcommand{\Hom}{\mathrm{Hom}\,}
\newcommand{\imm}{\mathrm{Imm}\,}
\newcommand{\mcd}{\mathrm{mcd}\,}
\newcommand{\mcm}{\mathrm{mcm}\,}
\newcommand{\orb}{\mathrm{orb}\,}
\newcommand{\ord}{\mathrm{ord}\,}
\newcommand{\rnk}{\mathrm{rnk}\,}
\newcommand{\sgn}{\mathrm{sgn}\,}
\newcommand{\Span}{\mathrm{Span}\,}
\newcommand{\stab}{\mathrm{stab}\,}
\newcommand{\supp}{\mathrm{supp}\,}
\newcommand{\tr}{\mathrm{tr}\,}

\newcommand{\Real}{\,\Re\mathfrak{e}}
\newcommand{\Imag}{\,\Im\mathfrak{m}}

%-------------- Frecce -------------------------
\newcommand{\coimplies}{\Longleftrightarrow}
\newcommand{\inj}{\hookrightarrow}
\newcommand{\onto}{\twoheadrightarrow}
\newcommand{\ot}{\leftarrow}
\newcommand{\acts}{\curvearrowright}

%----------- Lettere greche -------------------
\newcommand{\al}{\alpha}
\newcommand{\de}{\delta}
\newcommand{\e}{\varepsilon}
%\newcommand{\th}{\theta}
\newcommand{\la}{\lambda}
\newcommand{\vp}{\varphi}

%----------- Modifica testo -------------------
\newcommand{\ul}[1]{\underline{#1}}
\newcommand{\ol}[1]{\overline{#1}}
\newcommand{\wt}[1]{\widetilde{#1}}
\newcommand{\wh}[1]{\widehat{#1}}
\newcommand{\td}[1]{\Tilde{#1}}
\newcommand{\rg}[1]{{\mathring {#1}}}
\newcommand{\under}[2]{\underset{#1}{\underbrace{#2}}}

%-------------- Derivate ----------------------
\newcommand{\ddx}[2][x]{\frac{d#2}{d#1}}
\newcommand{\pdpi}[2]{{\frac{\partial #2}{\partial x_{#1}}}}
\newcommand{\pdp}[2][x]{\frac{\partial #2}{\partial #1}}

%-------------- Parentesi ---------------------
\newcommand{\pa}[1]{\left(#1\right)}
\newcommand{\spa}[1]{\left[#1\right]}
\newcommand{\cpa}[1]{\left\{#1\right\}}
\newcommand{\abs}[1]{\left|#1\right|}
\newcommand{\norm}[1]{\left\Vert#1\right\Vert}
\newcommand{\ps}[1]{\left\langle #1\right\rangle}

%--------------- Matrici ----------------------
\newcommand{\mat}[1]{\begin{pmatrix}#1\end{pmatrix}}
\newcommand{\dmat}[1]{\begin{vmatrix}#1\end{vmatrix}}
\newcommand{\smat}[1]{\begin{smallmatrix}#1\end{smallmatrix}}
\newcommand{\BIG}[1]{\mathlarger{\mathlarger{\mathlarger{\mathlarger{#1}}}}}

%--------------- Funzioni ---------------------
\newcommand{\funcDef}[4]{
\begin{array}{ccc}
{#1} & \longrightarrow & {#2}\\
{#3} & \longmapsto & {#4}
\end{array}}

%---------------- Altro -----------------------
\newcommand{\bs}{\setminus}
\newcommand{\res}[1]{\raisebox{-.5ex}{$|$}_{#1}}
\newcommand{\quot}[2]{\faktor{#1}{#2}}
\newcommand{\sep}{\,\middle|\,}

\newcommand{\ii}{^{-1}}
\newcommand{\nz}{\bs\{0\}}

\newcommand{\powerset}{\mathscr{P}}
\newcommand{\0}{{\underline{0}}}
\newcommand{\normal}{\triangleleft}

\DeclareMathOperator*{\bigast}{{\mathlarger{\mathlarger{\ast}}}}

\NewDocumentCommand{\Kon}{G{x} G{n} G{\mathbb{K}}}{
  #3[#1_1,\cdots, #1_{#2}]
}


\NeedsTeXFormat{LaTeX2e}
%\ProvidesPackage{quiver}[2021/01/11 quiver]

% `tikz-cd` is necessary to draw commutative diagrams.
\RequirePackage{tikz-cd}
% `amssymb` is necessary for `\lrcorner` and `\ulcorner`.
\RequirePackage{amssymb}
% `calc` is necessary to draw curved arrows.
\usetikzlibrary{calc}
% `pathmorphing` is necessary to draw squiggly arrows.
\usetikzlibrary{decorations.pathmorphing}

% A TikZ style for curved arrows of a fixed height, due to AndréC.
\tikzset{curve/.style={settings={#1},to path={(\tikztostart)
    .. controls ($(\tikztostart)!\pv{pos}!(\tikztotarget)!\pv{height}!270:(\tikztotarget)$)
    and ($(\tikztostart)!1-\pv{pos}!(\tikztotarget)!\pv{height}!270:(\tikztotarget)$)
    .. (\tikztotarget)\tikztonodes}},
    settings/.code={\tikzset{quiver/.cd,#1}
        \def\pv##1{\pgfkeysvalueof{/tikz/quiver/##1}}},
    quiver/.cd,pos/.initial=0.35,height/.initial=0}

% TikZ arrowhead/tail styles.
\tikzset{tail reversed/.code={\pgfsetarrowsstart{tikzcd to}}}
\tikzset{2tail/.code={\pgfsetarrowsstart{Implies[reversed]}}}
\tikzset{2tail reversed/.code={\pgfsetarrowsstart{Implies}}}
% TikZ arrow styles.
\tikzset{no body/.style={/tikz/dash pattern=on 0 off 1mm}}

%PER CAMBIARE I MARGINI
%\usepackage[margin=2cm]{geometry}

\newcommand{\znz}[1]{{\Z/#1\Z}}

%----------- Setup stilistico ----------------
\definecolor{DarkRed}{HTML}{B6321C}
\hypersetup{
    colorlinks=true,
    linkcolor=DarkRed,
    filecolor=blue,
    citecolor = black,
    urlcolor=cyan,
}
\renewcommand\thefootnote{\textcolor{blue}{\arabic{footnote}}}

%\AddToHook{cmd/section/before}{\clearpage}
%--------- Comandi dattilografici ------------
\newcommand{\alg}[1]{\begin{align*}#1\end{align*}}
\newcommand{\pasg}[3]{\overset{\hyperref[#3]{\text{#2}}}{#1}}
\newcommand{\pasgnl}[2]{\overset{\text{#2}}{#1}}
\newcommand{\pasgnlmath}[2]{\overset{#2}{#1}}

\newcommand{\PV}{{\mathbb{P}(V)}}
\newcommand{\PW}{{\mathbb{P}(W)}}
\newcommand{\ser}[1]{\sum_{#1=0}^\infty}
\newcommand{\Ind}{\mathrm{Ind}}
\newcommand{\Res}{\mathrm{Res}}
\newcommand{\gf}[2]{\pi_1(#1,#2)}


\NewDocumentCommand{\PGL}{o m}{
    \IfNoValueTF{#1}
        {{\mathbb{P}GL(#2)}}
    {{\mathbb{P}GL_{#1}(#2)}}
}
\NewDocumentCommand{\GL}{o m}{
    \IfNoValueTF{#1}
        {{GL(#2)}}
    {{GL_{#1}(#2)}}
}


% power of denominator / center point / function / path
% use small o for no parameter
\NewDocumentCommand{\cauchyint}{oD<>{-z_0} O{f} m}{
    \IfNoValueTF{#1}
        {{\frac1{2\pi i}\int_{{#4}}\frac{{#3}(\zeta)}{\zeta{#2}}d\zeta}}
    {{\frac1{2\pi i}\int_{{#4}}\frac{{#3}(\zeta)}{(\zeta{#2})^{#1}}d\zeta}}
}
% =====================================================
\title{\Huge{Cheat Sheet di Geometria 2}}
\date{A.A 2022-2023}
\author{}

\begin{document}
\maketitle

\begin{multicols*}{2}
    \tableofcontents
\end{multicols*}

\newpage
%--------------------------------------------------------------------
\chapter{Geometria proiettiva}
\setlength{\parindent}{2pt}

\begin{multicols*}{2}
    \section{Spazi proiettivi e nozioni introduttive}
    \subsection{Spazio proiettivo}
    Definiamo l'oggetto centrale dei nostri studi.
    \begin{definition}[Spazio Proiettivo]
    Dato $V$ spazio vettoriale su $\K$, definiamo il suo \textbf{spazio proiettivo associato} come
    \[\PV=\quot{V\nz}{\sim},\]
    dove $v\sim w\coimplies \exists \la\in\K\nz$ t.c. $w=\la v$.
    \end{definition}
    \noindent
    Intuitivamente la relazione collassa tutti i vettori appartenenti alla stessa retta in un elemento. Possiamo quindi pensare allo spazio proiettivo come l'insieme delle rette o direzioni in $V$.

    Troviamo in effetti la seguente bigezione naturale:
    \begin{center}
    \begin{tabular}{ccc}
        $\PV$ & $\longleftrightarrow$ & rette di $V$ \\
        $[v]$ & $\longmapsto$ & $\Span(v)$\\
        $[v_r]$ & $\longmapsfrom$ & $r$
    \end{tabular}
    \end{center}
    dove $v_r$ \`e un qualsiasi vettore in $r\nz$.

    \begin{example}[Proiettivi dello spazio banale e di una retta]
    Osserviamo che $\Pj(\{0\})=\quot{\emptyset}{\sim}=\emptyset$, mentre per $v\neq0$
    \[\Pj(\Span(v))=\quot{\{\la v\mid \la\in\K\nz\}}{\sim}=\{[v]\},\]
    ovvero lo spazio proiettivo associato ad una retta contiene un solo elemento.
    \end{example}

    \begin{definition}[Dimensione di uno spazio proiettivo]
        Definiamo la \textbf{dimensione} dello spazio proiettivo $\PV$ come
        \[\dim_\K\PV=\dim_\K V -1.\]
    \end{definition}

    \begin{remark}
    Gli spazi proiettivi non sono spazi vettoriali. Come vedremo sono in un certo senso una estensione degli spazi affini.
    \end{remark}

    \begin{definition}[Punti, Rette e piani proiettivi]
    Definiamo i seguenti termini:
    \begin{itemize}[noitemsep]
    \item un \textbf{punto proiettivo} \`e uno spazio proiettivo di dimensione $0$,
    \item una \textbf{retta proiettiva} \`e uno spazio proiettivo di dimensione $1$,
    \item un \textbf{piano proiettivo} \`e uno spazio proiettivo di dimensione $2$.
    \end{itemize}
    \end{definition}

    \begin{definition}[Spazio proiettivo standard]
    Definiamo $\Pj(\K^{n+1})=\Pj^{n}(\K)=\K\Pj^{n}$ lo \textbf{spazio proiettivo standard} di dimensione $n$. Se il campo risulta chiaro da contesto scriveremo solo $\Pj^n$.
    \end{definition}

    \subsection{Trasformazioni Proiettive}

    Studiamo ora quali mappe preservano la struttura di spazio proiettivo.
    \begin{definition}[Trasformazione proiettiva]
    Una funzione $f:\PV\to\PW$ \`e una \textbf{trasformazione proiettiva} se $\exists \vp: V\to W$ lineare tale che
    \[f([v])=[\vp(v)].\]
    In questa notazione affermiamo che $f$ \`e \textbf{indotta} da $\vp$.
    \end{definition}
    \begin{notation}
    Se $f$ \`e la trasformazione proiettiva indotta da $\vp$ scriviamo $f=[\vp]$.
    \end{notation}

    \begin{remark}
    Se $f$ \`e una trasformazione proiettiva ben definita indotta da $\vp$ allora $\vp$ \`e iniettiva.
    \end{remark}

    \begin{remark}
    Ogni mappa lineare iniettiva $\vp:V\to W$ induce una trasformazione proiettiva $f:\PV\to\PW$ tramite $[v]\mapsto[\vp(v)]$.
    \end{remark}
    \begin{remark}
        Tutte le trasformazioni proiettive sono iniettive.
    \end{remark}

    \begin{remark}
        $id_\PV$ \`e proiettiva ed \`e indotta da $id_V$.
    \end{remark}
    \begin{proposition}
        Date $f:\PV\to\PW$ e $g:\PW\to\Pj(Z)$ proiettive abbiamo che $g\circ f:\PV\to\Pj(Z)$ \`e proiettiva.
    \end{proposition}

    \noindent Caratterizziamo ora gli isomorfismi di spazi proiettivi
    \begin{definition}[Isomorfismo proiettivo]
    Una trasformazione proiettiva surgettiva \`e chiamata \textbf{isomorfismo proiettivo}.
    \end{definition}
    La seguente proposizione ci permette di giustificare la definizione
    \begin{proposition}[Caratterizzazione degli isomorfismi proiettivi]
    Sia $f:\PV\to\PW$ proiettiva. Le affermazioni seguenti sono equivalenti
    \begin{enumerate}
    \item $f$ surgettiva,
    \item $f$ bigettiva,
    \item $\dim\PV=\dim\PW$,
    \item $f$ invertibile e $f\ii:\PW\to\PV$ \`e proiettiva.
    \end{enumerate}
    \end{proposition}

    \begin{definition}[Proiettivit\`a]
    Una trasformazione proiettiva $f:\PV\to\PV$ \`e definita \textbf{proiettivit\`a}. Denotiamo l'insieme delle proiettivit\`a con $\PGL V$.
    \end{definition}
    \begin{remark}
    Ogni proiettivit\`a \`e un isomorfismo proiettivo.
    \end{remark}
    \begin{remark}
    Le proiettivit\`a di $\PV$ munito della composizione \`e un gruppo.
    \end{remark}

    \begin{remark}[Punti fissi delle proiettivit\`a]\label{PuntiFissiProiettivita}
    Possiamo caratterizzare i punti fissi delle proiettivit\`a. Sia $f$ una proiettivit\`a indotta da $\vp$ e $[v]$ un punto fisso:
    \[[v]=f([v])=[\vp(v)],\]
    da cui $\la v=\vp(v)$, cio\`e $v$ \`e un autovettore di $\vp$. Similmente se $v$ \`e un autovettore di $\vp$ abbiamo che $[v]$ \`e un punto fisso per le stesse relazioni.
    \end{remark}

    \subsection{Sottospazi proiettivi}

    Poniamo per semplicit\`a notazionale $\pi:V\nz\to\PV$ la proiezione per la relazione definita all'inizio del capitolo.

    % \begin{definition}[Grassmanniana (NON DATA DURANTE IL CORSO)]
    % Sia $V$ uno spazio vettoriale di dimensione $n$ e sia $k\in \{0,\cdots, n\}$. La \textbf{grassmanniana} $k$ di $V$ sono l'insieme di tutti i sottospazi vettoriali di $V$ di dimensione $k$
    % \[Gr_k(V)=\cpa{W\sep V\supseteq W\text{ ssp. vett.},\ \dim W=k}.\]
    % Poniamo inoltre
    % \[Gr(k,n)=Gr_k(\K^n).\]
    % \end{definition}

    \begin{definition}[Sottospazio proiettivo]
        Un \textbf{sottospazio proiettivo} $S$ di $\PV$ \`e un sottoinsieme di $\PV$ tale che
        \[S=\pi(H\nz),\text{ per }H\text{ sottospazio vettoriale di }V.\]
    \end{definition}
    \begin{remark}
    Dalla definizione segue che un sottospazio proiettivo \`e uno spazio proiettivo. Pi\`u precisamente \[\pi(H\nz)=\Pj(H).\]
    \end{remark}
    \begin{definition}[Iperpiano proiettivo]
    Un \textbf{iperpiano} di $\PV$ \`e un sottospazio proiettivo $S$ di $\PV$ tale che $\dim S=\dim \PV-1$.
    \end{definition}

    \begin{proposition}[Corrispondenza tra sottospazi proiettivi e vettoriali]
    Se $S$ \`e un sottospazio proiettivo come sopra abbiamo che $\pi\ii(S)=H\nz$.
    \bigskip

    \noindent
    In particolare abbiamo una bigezione tra i sottospazi vettoriali di $V$ e i sottospazi proiettivi di $\PV$
    \begin{center}
    \begin{tabular}{ccc}
    $\{\text{ssp vett. di }V\}$ & $\longleftrightarrow$ & $\{\text{ssp prj. di }\PV\}$ \\
    $H$ & $\longmapsto$ & $\pi(H\nz)$\\
    $\pi\ii(S)\cup\{0\}$ & $\longmapsfrom$ & $S$
    \end{tabular}
    \end{center}
    \end{proposition}
    % \begin{remark}[Grassmaniane proiettive (NON DATA DURANTE IL CORSO)]
    % La corrispondenza appena mostrata si comporta bene con le dimensioni, cio\`e $\dim \Pj(H)=\dim H-1$ e $\dim \pi(S)\ii \cup \{0\}=\dim S+1$. Questo ci dice che
    % \[Gr_k(\PV)\cong Gr_{k+1}(V)\]
    % dove $Gr_k(\PV)$ \`e definito in modo analogo alle grassmanniane per spazi vettoriali ma considerando sottospazi proiettivi.
    % \end{remark}


    \noindent
    Consideriamo adesso l'intersezione e la somma di sottospazi proiettivi.
    \begin{proposition}[Sottospazi proiettivi sono stabili per intersezione]
    Siano $S_i,\ i\in I$ sottospazi proiettivi di $\PV$. Si ha che
    \[\bigcap_{i\in I}S_i\text{ \`e un sottospazio proiettivo di }\PV.\]
    \end{proposition}
    \begin{remark}
    $\PV\cap\PW=\Pj(V\cap W).$
    \end{remark}

    \noindent
    Come per gli spazi vettoriali, l'unione di sottospazi proiettivi non \`e in generale un sottospazio proiettivo. Definiamo allora una somma.

    \begin{definition}[Sottospazio proiettivo generato]
    Sia $A\subseteq\PV$ un sottoinsieme di $\PV$, il \textbf{sottospazio proiettivo di $\PV$ generato da $A$} \`e il pi\`u piccolo sottospazio proiettivo di $\PV$ contenente $A$ e viene indicato con $L(A)$.
    \[L(A)=\bigcap_{\smat{S\text{ ssp.prj.}\\ A\subseteq S}}S.\]
    L'intersezione non \`e vuota perch\'e $A\subseteq \PV$ e $\PV$ \`e un sottospazio proiettivo di se stesso.
    \end{definition}

    \begin{remark}
    La definizione si estende ad una somma tra sottospazi proiettivi considerando il generato dell'unione
    \[L(S_1,S_2)=L(S_1\cup S_2).\]
    \end{remark}


    \begin{proposition}[Traduzione tra somme di proiettivi e vettoriali]
    Se $S_1=\Pj(H_1)$ e $S_2=\Pj(H_2)$ per $H_1,H_2$ sottospazi di $V$ abbiamo che
    \[L(S_1,S_2)=\Pj(H_1+H_2).\]
    \end{proposition}
    \begin{proposition}[Trasformazioni proiettive rispettano i generati]
    Dato $S\subseteq\PV$ e $f$ una trasformazione proiettiva abbiamo che
    \[f(L(S))=L(f(S)).\]
    \end{proposition}

    \noindent
    Vediamo ora come generalizzare la formula di Grassmann.

    \begin{theorem}[Formula di Grassmann proiettiva]
        Dati $S_1,S_2$ sottospazi proiettivi di $\PV$ abbiamo che
        \[\dim L(S_1,S_2)=\dim S_1+\dim S_2-\dim S_1\cap S_2.\]
    \end{theorem}
    \begin{corollary}[Criterio per intersezione non vuota]
    Se $S_1,S_2$ sono sottospazi proiettivi tali che $\dim S_1+\dim S_2\geq\dim \PV$ allora $S_1\cap S_2\neq\emptyset$.
    \end{corollary}
    \begin{remark}
        Sui piani proiettivi non esistono ``rette parallele". Più precisamente, date $r_1,r_2$ sottospazi proiettivi di $\PV$ con $\dim \PV=2$ e $\dim r_1=\dim r_2=1$ abbiamo che $r_1=r_2$ o $r_1\cap r_2=\{P\}$ con $P$ punto proiettivo.
    \end{remark}

    \subsection{Riferimenti proiettivi}
    Estendiamo i parallelismi con gli spazi vettoriali cercando un equivalente per indipendenza lineare e basi.

    \begin{definition}[Punti indipendenti]
    Siano $P_1,\cdots,P_k\in\PV$, essi sono \textbf{indipendenti} se scelti $v_1,\cdots,v_k\in V$ tali che $[v_i]=P_i$ allora $v_1,\cdots,v_k$ sono linearmente indipendenti.
    \medskip

    \noindent
    La definizione \`e indipendente dai rappresentanti scelti, infatti per $\la_i\in\K\nz$ abbiamo che
    \[v_1,\cdots,v_k\text{ lin. indipendenti}\coimplies \la_1v_1,\cdots,\la_kv_k\text{ lin. indipendenti}.\]
    \end{definition}

    \begin{definition}[Posizione generale]
    Dati $P_1,\cdots,P_k$ essi sono in \textbf{posizione generale} se ogni sottoinsieme di essi costituito da $h$ punti distinti con $h\leq n+1$ \`e indipendente.
    \end{definition}
    \begin{remark}
    Se $k\leq n+1$ allora la definizione coincide con l'indipendenza. Se $k>n+1$ la definizione \`e equivalente a richiedere l'indipendenza di tutte le $(n+1)-$uple di punti nell'insieme.
    \end{remark}

    \begin{definition}[Riferimento proiettivo]
        Un \textbf{riferimento proiettivo} di $\PV$ con $\dim \PV=n$ \`e una $(n+2)-$upla di punti in posizione generale.\\
        L'ultimo punto nel riferimento viene chiamato \textbf{punto unit\`a}, mentre gli altri sono detti \textbf{punti fondamentali}.
    \end{definition}
    \begin{definition}[Base normalizzata]
    Dato $\Rc=(P_0,\cdots,P_{n+1})$ un riferimento proiettivo di $\PV$, una \textbf{base normalizzata di $V$ associata a $\Rc$} \`e una base $(v_0,\cdots,v_n)$ tali che \[\forall i\in\{0,\cdots,n\},\ [v_i]=P_i\text{ e }P_{n+1}=[v_0+\cdots+v_n].\]
    \end{definition}

    \begin{theorem}[Esistenza e unicit\`a della base normalizzata]
    Dato $\Rc=(P_0,\cdots,P_{n+1})$ un riferimento proiettivo di $\PV$ abbiamo che $\exists \Bc=\{v_0,\cdots,v_n\}$ base normalizzata. Se $\Bc'=\{u_0,\cdots,u_n\}$ \`e una base normalizzata di $\Rc$ abbiamo inoltre che $\exists \la\in\K\nz$ tale che $u_i=\la v_i$.
    \end{theorem}

    \begin{remark}
    Una differenza sostanziale tra la geometria proiettiva e l'algebra lineare \`e che non \`e possibile estendere riferimenti proiettivi di sottospazi proiettivi a riferimenti di sottospazi che li estendono. Se $\Rc=\{P_0,\cdots,P_{n+1}\}$ \`e un riferimento proiettivo di $S$ sottospazio proiettivo di $H$ osserviamo che $\Rc$ non sono punti in posizione generale letti come punti di $H$. Infatti se la dimensione aumenta anche solo di $1$ \`e necessario che le $(n+2)-$uple di punti siano indipendenti ma da come sappiamo dalla definizione di base normalizzata il punto unit\`a si scrive come somma dei punti fondamentali passando ad una base normalizzata.
    \end{remark}


    \begin{theorem}[Trasformazioni proiettive sono univocamente determinate dal valore su un riferimento]\label{TrasformazioniProiettiveSonoUnivocamenteDeterminateDalValoreSuUnRiferimento}
    Siano $f,g:\PV\to\PW$ trasformazioni proiettive indotte da $\vp$ e $\psi$ rispettivamente. Sia $\Rc$ un riferimento proiettivo di $\PV$. Le seguenti condizioni sono equivalenti:
    \begin{enumerate}
    \item $\exists \la\in\K\nz$ tale che $\vp=\la\psi$;
    \item $f=g$;
    \item $\forall P\in \Rc,\ f(P)=g(P)$.
    \end{enumerate}
    \end{theorem}
    \begin{corollary}
        \[\PGL V\cong\quot{\GL V}{N}\] dove $N=\{\la id\mid \la\in \K\nz\}\triangleleft \GL V$
    \end{corollary}

    \begin{notation}[Proiettivit\`a standard]
    Le proiettivit\`a di $\Pj^n(\K)$ formano un gruppo che denotiamo $\PGL{\K^{n+1}}=\Pj \GL[n+1]\K$.\\
    L'ultimo $n+1$ corrisponde alla taglia delle matrici che rappresenteranno le proiettivit\`a, non la dimensione dello spazio su cui agiscono.
    \end{notation}

    \noindent
    Concludiamo la sezione introducendo un teorema che ci permette di identificare trasformazioni proiettive definendole su un riferimento.
    \begin{theorem}[Teorema fondamentale delle trasformazioni proiettive]\label{TeoremaFondamentaleTrasformazioniProiettive}
    Siano $\PV$ e $\PW$ spazi proiettivi su $\K$ tali che $\dim \PV=\dim \PW=n$. Fissiamo $\Rc=(P_0,\cdots,P_{n+1})$ e $\Rc'=(P_0',\cdots,P_{n+1}')$ riferimenti proiettivi di $\PV$ e $\PW$ rispettivamente. Si ha che esiste un'unica trasformazione proiettiva $f:\PV\to\PW$ tale che $\forall i\in\{0,\cdots,n+1\},\ f(P_i)=P_i'$.
    \end{theorem}



    \subsection{Coordinate omogenee}
    Come per il caso vettoriale, \`e spesso utile ricorrere a un sistema di coordinate. Per capire come definirle studiamo il caso di $\Pj^n(K)$. Per definizione abbiamo
    \[\Pj^n(\K)=\quot{\K^{n+1}\nz}{\sim}=\{[(x_0,\cdots,x_n)],\ \text{con entrate non tutte nulle}\}.\]
    \begin{definition}[Riferimento proiettivo canonico]
    Il \textbf{riferimento standard (o canonico)} di $\Pj^n(\K)$ \`e il riferimento proiettivo che ha come base normalizzata la base canonica di $\K^{n+1}$.\medskip

    \noindent
    Affermiamo che $[(x_0,\cdots,x_n)]$ ha \textbf{coordinate omogenee} $[x_0,\cdots,x_n]$ o $[x_0:\cdots:x_n]$ rispetto al riferimento canonico di $\Pj^n(\K)$.
    \end{definition}
    \begin{remark}
    Il riferimento proiettivo standard consiste dei punti con coordinate omogenee:
    \[[1:0:\cdots:0],[0:1:0:\cdots:0],\cdots,[0:\cdots:0:1],[1:1:\cdots:1].\]
    \end{remark}

    \noindent
    Cerchiamo ora di definire le coordinate omogenee per un qualsiasi spazio proiettivo.

    \begin{definition}[Coordinate omogenee]
        Fissiamo $\Rc=\{P_0,\cdots,P_{n+1}\}$ un riferimento proiettivo di $\PV$. Dato $P\in\PV$ le sue \textbf{coordinate omogenee} rispetto a $\Rc$ sono date da una delle seguenti equivalenti definizioni:
        \begin{itemize}[noitemsep]
            \item Se $f:\PV\to\Pj^n(\K)$ \`e l'unico isomorfismo proiettivo che porta $\Rc$ nel riferimento proiettivo canonico di $\Pj^n(\K)$ allora le coordinate omogenee di $P$ sono $f(P)$.
            \item Se $\Bc=\{v_0,\cdots,v_n\}$ \`e una base normalizzata associata a $\Rc$ e $P=[v]$ consideriamo la combinazione lineare $v=\sum_{i=0}^na_iv_i$. Le coordinate omogenee di $P$ rispetto a $\Rc$ sono $[a_0:\cdots:a_n]$.
        \end{itemize}
    \end{definition}

    Come per il caso vettoriale, possiamo rappresentare le trasformazioni proiettive con matrici e sottospazi proiettivi come luoghi di zeri di equazioni.
    \begin{definition}[Matrice associata a isomorfismo proiettivo]
    Sia $f:\PV\to\PW$ un isomorfismo proiettivo e siano $\Rc,\Rc'$ riferimenti proiettivi di $\PV$ e $\PW$ rispettivamente. Siano $\Bc,\Bc'$ le relative basi normalizzate. Se $\vp$ \`e una mappa lineare che induce $f$ allora $f$ \`e \textbf{rappresentata} da $M=M^\Bc_{\Bc'}(\vp)\in M(n+1,\K)$.
    \end{definition}
    \begin{remark}[Prodotto matrice-coordinate omogenee]
    Dati $P=[v]\in \PV$, $f:\PV\to\PW$, $f=[\vp]$, $M$ una matrice che rappresenta $f$, $\Rc,\Rc'$ un riferimenti proiettivi di $\PV$ e $\PW$ rispettivamente con $\Bc$ e $\Bc'$ basi normalizzate, se indichiamo il passaggio a coordinate omogenee rispetto a $\Rc$ con $[\cdot]_\Rc$ e il passaggio a coordinate rispetto a $\Bc$ con $[\cdot]_\Bc$ (similmente per $\Rc'$ e $\Bc'$) si ha che
    \begin{align*}
    [f(P)]_{\Rc'}=&[[\vp(v)]]_{\Rc'}=[[[\cdot]_{\Bc'}\ii(M[v]_{\Bc})]]_{\Rc'}=\\
    =&[[[\cdot]_{\Rc'}\ii([M[v]_{\Bc}])]]_{\Rc'}=\\
    =& [M[v]_{\Bc}].
    \end{align*}
    \end{remark}
    \begin{notation}
    Se $M$ \`e una matrice, $P\in \PV$ con $P=[v]$, $\Rc$ \`e un riferimento proiettivo di $\PV$ e $\B$ \`e una sua base normalizzata poniamo
    \[M[P]_\Rc= [M][P]_\Rc\doteqdot [M[v]_\Bc].\]
    \end{notation}

    \begin{definition}[Equazioni cartesiane proiettive]
    Dato $S$ un sottospazio proiettivo di $\PV$ sia $W$ il sottospazio vettoriale di $V$ tale che $S=\PW$. Fissato un riferimento proiettivo su $\PV$ individuiamo univocamente una base normalizzata di $V$, quindi $W$ \`e esprimibile come luogo di zeri di $\dim V-\dim W$ equazioni. Chiamiamo queste le \textbf{equazioni cartesiane per $S$ rispetto a $\Rc$}.
    \end{definition}
    \begin{remark}
    Con le notazioni appena usate, il numero di equazioni si esprime in termini di $S$ e $\PV$ come
    \[\dim \PV-\dim S=\dim V-\dim W\]
    \end{remark}

    \section{Spazi proiettivi estendono gli spazi affini}
    Approfondiamo la geometria di $\Pj^n(\K)$ come oggetto $n-$dimensionale che "contiene" $\K^n$.
    \subsection{Carte affini}
    Consideriamo cosa succede quando fissiamo una coordinata proiettiva a $0$

    \begin{definition}[Iperpiano coordinato]
    Dato $\Pj^n(\K)$ costruiamo per ogni indice $i\in\{0,\cdots,n\}$ il seguente sottoinsieme di $\Pj^n(\K)$
    \[H_i=\{[x_0:\cdots:x_n]\in\Pj^n(\K)\mid x_i=0\}.\]
    Chiamiamo $H_i$ \textbf{l'$i-$esimo iperpiano coordinato}.
    \end{definition}
    \begin{remark}
    Considerati come spazi proiettivi
    \[H_i\cong\Pj^{n-1}(\K).\]
    \end{remark}

    \begin{definition}[Carta affine]
    Definiamo l'$i-$esima \textbf{carta affine} come
    \[U_i=\Pj^n(\K)\bs H_i.\]
    \end{definition}
    \begin{proposition}[Le carte affini sono ``isomorfe" all'affine]
    Esiste una bigezione naturale tra $U_i$ e $\K^n$ per ogni $i$.
    \end{proposition}

    \begin{definition}[Carta affine]
    La bigezione $J_i:\K^n\to U_i$ si chiama $i-$esima \textbf{carta affine}
    \end{definition}

    \noindent
    L'esistenza della carta affine ci permette di dividere gli spazi proiettivi in due parti
    \[\begin{tikzcd}
    {\mathbb{P}^n(\mathbb{K})} & {U_0} & {H_0} \\
    & {\mathbb{K}^n} & {\mathbb{P}^{n-1}(\mathbb{K})}
    \arrow["\bigsqcup", shift right=2, draw=none, from=1-2, to=1-3]
    \arrow["\cong"', shift left=2, draw=none, from=1-2, to=2-2]
    \arrow["\cong", shift right=3, draw=none, from=1-3, to=2-3]
    \arrow["{=}", shift right=1, draw=none, from=1-1, to=1-2]
    \end{tikzcd}\]
    \begin{definition}[Punti propri e impropri]
        I punti di $U_0\cong\K^n$ si chiamano \textbf{punti propri} o \textbf{affini}, mentre i punti di $H_0$ si chiamano \textbf{punti impropri} o \textbf{all'infinito}.
    \end{definition}

    \begin{proposition}[Parte affine]
    Sia $K$ un sottospazio proiettivo di $\Pj^n(\K)$ non contenuto in $H_0$. Allora $J_0\ii(K\cap U_0)\subseteq \K^n$ \`e un sottospazio affine della stessa dimensione di $K$ che chiamiamo \textbf{parte affine} di $K$.
    \end{proposition}

    \begin{proposition}[Chiusura proiettiva]
    Sia $Z\neq\emptyset$ un sottospazio affine di $\K^n$. Allora $Z$ \`e la parte affine di un unico sottospazio proiettivo $\ol Z$ di $\Pj^n(\K)$. Inoltre $\ol Z\nsubseteq H_0$ e ha la stessa dimensione di $Z$. Chiamiamo $\ol Z$ la \textbf{chiusura proiettiva} di $Z$.
    \end{proposition}

    Dalle due proposizioni precedenti vediamo che esiste una bigezione naturale tra i $k-$sottospazi affini di $\K^n$ e i $k-$sottospazi proiettivi di $\Pj^n(\K)$ non contenuti in $H_0$.

    \begin{definition}[Polinomio omogeneo]
    Un polinomio $p\in\K[x_0,\cdots,x_n]$ \`e \textbf{omogeneo} di grado $d$ se tutti i monomi a coefficienti non nulli di $p$ hanno grado $d$.
    \end{definition}
    \begin{remark}
    La mappa
    \[\funcDef{\K[x_1,\cdots,x_n]}{\K_{\deg p}[x_0,\cdots,x_1]}{p(x_1,\cdots,x_n)}{x_0^{\deg p}p(x_1/x_0,\cdots,x_n/x_0)}\]
    \`e detta \textbf{omogenizzazione} e se $p$ ha grado $d$ allora il suo omogenizzato \`e un polinomio omogeneo di grado $d$. Questa operazione corrisponde a ``mettere tante $x_0$ quanto basta affinch\'e tutti i termini abbiano lo stesso grado".
    \end{remark}
    \begin{remark}
    Se $Z$ \`e un iperpiano di $\K^n$ di equazione $a_1X_1+\cdots+a_n X_n=b$ allora la sua chiusura proiettiva $\ol Z\subset\Pj^n(\K)$ ha equazione $-bx_0+a_1x_1+\cdots+a_nx_n=0$. In generale se un sottospazio \`e dato da un sistema di equazioni, la sua chiusura \`e data dal sistema di queste stesse equazioni ma omogeneizzato.
    \end{remark}

    \begin{remark}
    Solo gli zeri di polinomi omogenei descrivono luoghi in $\Pj^n(\K)$ dato che altrimenti quando scaliamo le indeterminate otterremmo una potenza diversa del fattore per monomi di grado diverso.
    \end{remark}

    \noindent Prendere la chiusura proiettiva di spazi affini in genere aggiunge dei punti allo spazio. Diamo un nome a questi punti:

    \begin{definition}[Punti all'infinito di sottospazi vettoriali]
    Se $Z$ \`e un sottospazio affine di $\K^n$ i \textbf{punti all'infinito} di $Z$ sono i punti di $\ol Z\cap H_0$.
    \end{definition}


    \begin{center}
    Ma perch\'e li chiamiamo ``all'infinito"?
    \end{center}
    La proposizione seguente ci fornisce una intuizione.
    \begin{proposition}[Rette si incontrano all'infinito se e solo se sono parallele]
    Siano $r,s$ due rette affini in $\K^n$. Allora $r,s$ hanno lo stesso punto all'infinito se e solo se sono parallele.
    \end{proposition}

    Vediamo quindi che $H_0$ \`e in bigezione con le direzioni in $\K^n$, ovvero
    \[\Pj^n(\K)=\K^n\cup\{\text{direzioni di }\K^n\}.\]


    \section{Approfondimento sulle proiettivit\`a}
    \subsection{Prospettivit\`a}
    Andiamo ora a studiare ci\`o per cui la geometria proiettiva \`e nata, ovvero la prospettiva.

    \begin{definition}[Prospettivit\`a]
    Siano $r,s$ rette distinte di un piano proiettivo $\PV$. Sia $O\in\PV\bs(r\cup s)$ e definiamo
    \[\pi_O:\funcDef{r}{s}{p}{L(O,p)\cap s}\]
    $\pi_O$ \`e detta \textbf{prospettivit\`a} di centro $O$.
    \end{definition}

    \begin{proposition}[Le prospettivit\`a sono ben definite e sono trasformazioni proiettive]
    La prospettivit\`a di centro $O$ \`e una trasformazione proiettiva ben definita.
    \end{proposition}

    \noindent
    Proviamo a dare una caratterizzazione delle prospettivit\`a
    \begin{theorem}[Caratterizzazione delle prospettivit\`a]\label{CaratterizzazioneProspettivita}
    Siano $r,s$ rette distinte di un piano proiettivo $\PV$ e sia $A=r\cap s$. Data $f:r\to s$ trasformazione proiettiva abbiamo che $f$ \`e una prospettivit\`a se e solo se $f(A)=A$.
    \end{theorem}

    \subsection{Corrispondenza tra Affinit\`a e Proiettivit\`a}
    \begin{proposition}[Affinit\`a come proiettivit\`a]\label{AffinitaComeProiettivita}
    Il gruppo delle affinit\`a $Aff(\K^n)$ \`e isomorfo a $G<\PGL[n+1]\K$, dove \[G=\{f\in \PGL[n+1]\K\mid f(H_0)=H_0\}.\]
    \end{proposition}
    \subsection{Trasformazioni lineari fratte}
    \begin{definition}[Infinito]
    Per quanto detto sappiamo che $\Pj^1(\K)=U_0\cup H_0\cong \K\cup\{[0,1]\}$. Definiamo allora $\infty\doteqdot[0,1]$. Con questa identificazione
    \[\Pj^1(\K)=\K\cup\{\infty\}.\]
    \end{definition}
    Consideriamo le proiettivit\`a su $\Pj^1(\K)$: esse sono rappresentate dalle matrici $\PGL[2]\K$.
    \begin{definition}[Trasformazione lineare fratta]
    Una mappa della forma
    \[z\mapsto \frac{c+dz}{a+bz}\]
    con $a,b,c,d\in \K$ e $ad-bc\neq0$ \`e detta \textbf{trasformazione lineare fratta}.\\
    Nel caso in cui $\K=\C$ queste mappe si chiamano anche \textbf{trasformazioni di M\"obius}.
    \end{definition}
    \begin{remark}
    Possiamo definire come una trasformazione lineare fratta agisce su $\infty$ o restituisce $\infty$ ponendo
    \[-\frac ab\mapsto \infty,\qquad \infty\mapsto \frac db.\]
    \end{remark}
    \begin{proposition}
    Identificando $\Pj^1(\K)$ con $\K\cup\{\infty\}$ le proiettivit\`a di $\Pj^1(\K)$ corrispondono a trasformazioni lineari fratte.
    \end{proposition}


    \section{Dualit\`a}
    \begin{definition}[Spazio proiettivo duale]
    Dato $V$ spazio vettoriale su $\K$ di dimensione finita, definiamo lo \textbf{spazio proiettivo duale} di $\PV$ come
    \[\PV^*=\Pj(V^*)=\Pj(\Hom(V,\K))\]
    \end{definition}
    \begin{remark}
        Dato che $V\cong V^*$ abbiamo che $\PV\cong \PV^*$.
    \end{remark}
    \begin{proposition}
    La seguente mappa \`e una bigezione naturale tra il proiettivo duale e gli iperpiani del proiettivo
    \[\phi:\funcDef{\PV^*}{\{\text{iperpiani in }\PV\}}{[f]}{\Pj(\ker f)}.\]
    \end{proposition}

    \begin{definition}[Riferimento duale]
    Se $\Rc$ \`e un riferimento proiettivo di $\PV$ e $\Bc$ \`e una sua base normalizzata allora, definendo $\Bc^*$ la base duale di $\Bc$, troviamo un riferimento $\Rc^*$ che ha come base normalizzata $\Bc^*$. $\Rc^*$ \`e detto \textbf{riferimento duale} di $\Rc$ e fornisce delle $\textbf{coordinate omogenee duali}$ su $\PV^*$.
    \end{definition}
    \begin{remark}
    Le coordinate duali del funzionale corrispondente all'iperpiano $a_0x_0+\cdots+a_nx_0=0$ sono $[a_0:\cdots:a_n]$.
    \end{remark}

    \noindent
    Definiamo allora una mappa che dualizza tutti i sottospazi di $\PV$. Sia $S$ un sottospazio proiettivo di $\PV$ e sia $W$ tale che $S=\PW$. Poniamo $\dim S=k$ e $\dim \PV=n$. Allora per $-1\leq k\leq n$ definisco la mappa\footnote{Ricordiamo che $Ann(W)=\{f\in V^\ast\mid f(W)=0\}$ e che $\dim Ann(W)=n-\dim W$.}
    \[\delta_k:\funcDef{\left\{\begin{matrix}
    \text{ssp.prj. }S\subseteq\PV\\\dim S=k
    \end{matrix}
    \right\}}
    {\left\{\begin{matrix}
    \text{ssp.prj. }T\subseteq\PV^*\\\dim T=n-k-1
    \end{matrix}
    \right\}}
    {S}{\Pj(Ann(W))}\]
    \begin{proposition}
    $\delta_k$ \`e biunivoca per ogni $k$.
    \end{proposition}
    \begin{remark}
    Per $k=n-1$ vediamo che $\delta_{n-1}$ \`e l'inversa di
    \[\phi:\funcDef{\PV^*}{\{\text{iperpiani in }\PV\}}{[f]}{\Pj(\ker f)},\]
    invece per $k=0$ troviamo una corrispondenza tra punti di $\PV$ e iperpiani di $\PV^*$, che \`e la situazione inversa rispetto a prima. Questo fatto \`e legato all'isomorfismo naturale tra $V$ e $V^{**}$.
    \end{remark}

    \begin{definition}[Corrispondenza di dualit\`a]
    Definiamo la \textbf{corrispondenza di dualit\`a}
    \[\delta:\funcDef{\{\text{ssp.prj. di }\PV\}}{\{\text{ssp.prj. di }\PV^*\}}{S}{\delta_{\dim S}(S)}\]
    \end{definition}

    \begin{proposition}
    Se $S_1,S_2$ sono sottospazi proiettivi di $\PV$ allora
    \begin{enumerate}
    \item $S_1\subseteq S_2\implies \delta(S_2)\subseteq \delta(S_1)$
    \item $\delta(L(S_1,S_2))=\delta(S_1)\cap\delta(S_2)$
    \item $\delta(S_1\cap S_2)=L(\delta(S_1),\delta(S_2))$.
    \end{enumerate}
    \end{proposition}

    \begin{remark}
    Fissando una base, $\PV\cong\PV^*$, quindi posso pensare a $\delta$ come funzione tra sottospazi di $\PV$.
    \end{remark}

    \noindent
    Viste queste corrispondenze ci \`e permesso riformulare enunciati tramite il seguente

    \begin{theorem}[Principio di dualit\`a]
    Se $\mathcal{P}$ \`e una proposizione di oggetti di $\PV$ allora c'\`e una proposizione $\mathcal{P}^*$ con lo stesso valore di verit\`a ottenuta da $\mathcal{P}$ scambiando intersezioni con spazi generati e viceversa, invertendo i contenimenti e considerando oggetti della codimensione diminuita di $1$ (cio\`e spazi di dimensione $k$ diventano spazi di dimensione $n-k-1$).
    \end{theorem}

    \begin{proposition}
    Tramite la bigezione tra $\PV^*$ e gli iperpiani di $\PV$ si ha che per $S$ sottospazio proiettivo di $\PV$
    \[\delta(S)=\{H\subset \PV\text{ iperpiano t.c. }S\subseteq H\}\subseteq \PV^*\]
    \end{proposition}

    \begin{definition}[Sistema lineare di iperpiani]
    Dato $S$ sottospazio proiettivo di $\PV$ chiamiamo $\delta(S)$ il \textbf{sistema lineare di iperpiani} di $\PV$ di \textbf{centro} $S$.
    \end{definition}

    \begin{remark}
    Dato che $\delta$ \`e biunivoca ogni sottospazio proiettivo $T\subseteq \PV^*$ si scrive come sistema lineare di iperpiani il cui centro \`e $\delta\ii(T)$.
    \end{remark}
    \begin{definition}[Fascio di iperpiani]
    Una retta proiettiva in $\PV^*$ \`e detta \textbf{fascio} di iperpiani.
    \end{definition}

    \begin{definition}[Proiettivit\`a duale]
    Se $f:\PV\to\PW$ \`e un isomorfismo proiettivo e $f=[\vp]$ allora definiamo $f^*=[\vp^*]:\PW^*\to\PV^*$ la \textbf{proiettivit\`a duale} di $f$, dove $\vp^*:W^*\to V^*$ \`e la mappa lineare duale di $\vp$\footnote{Ricordiamo che data $\vp:V\to W$ lineare, la sua mappa duale \`e data da
    \[\vp^\ast:\funcDef{W^\ast}{V^\ast}{g}{g\circ \vp}\]}.
    \end{definition}

    \begin{remark}
    Osserviamo che $f^*(\delta(H))=\delta(f\ii(H))$ per $H$ iperpiano di $\PW$. Infatti posto $H=\Pj(Z)$ abbiamo
    \begin{align*}
    f^*(\delta(H))=&f^*(\Pj(Ann(Z)))=\{f^*([g])\mid g\in Ann(Z)\nz\}=\\
    =&\{[\vp^*(g)]\mid g\in Ann(Z)\nz\}=\{[g\circ \vp]\mid g\in Ann(Z)\nz\}=\\
    =&\{[g\circ \vp]\mid g\circ \vp\in Ann(\vp\ii(Z))\nz\}=\Pj(Ann(\vp\ii(Z)))=\\
    =&\delta(\Pj(\vp\ii(Z)))=\delta(f\ii(\Pj(Z)))=\\
    =&\delta(f\ii(H)).
    \end{align*}
    \end{remark}

    \section{Birapporto}
    Sappiamo che dati tre punti su una retta proiettiva ed altri tre punti allineati possiamo definire una proiettivit\`a che manda i primi nei secondi. Per le trasformazioni affini potevamo definire il rapporto semplice di tre punti allineati e questo forniva un invariante per affinit\`a. Purtroppo il rapporto semplice non \`e invariante per proiettivit\`a. Cerchiamo un tale invariante.
    \begin{definition}[Birapporto]
    Dati quattro punti $P_1,P_2,P_3,P_4\in\PV$ con $\dim \PV=1$ e $P_1,P_2,P_3$ distinti definiamo il loro \textbf{birapporto} come
    \[\beta(P_1,P_2,P_3,P_4)=\frac{x_1}{x_0}\in\K\cup\{\infty\}\]
    dove $[x_0,x_1]$ sono le coordinate omogenee di $P_4$ rispetto al riferimento proiettivo $(P_1,P_2,P_3)$.
    \end{definition}
    \begin{remark}
    Possiamo interpretare il birapporto come la coordinata affine di $P_4$ nella carta $U_0$ rispetto al riferimento $(P_1,P_2,P_3)$.
    \end{remark}
    \begin{remark}
    Se $P_4=P_1$ allora il birapporto \`e $0/1=0$, se $P_4=P_2$ abbiamo $1/0=\infty$ e per $P_4=P_3$ ricaviamo $1/1=1$. Pi\`u in generale la mappa
    \[\beta(P_1,P_2,P_3,\cdot):\PV\to\K\cup\{\infty\}\]
    \`e una bigezione (per definizione dato che $P_4$ \`e libero di avere qualsiasi coordinata omogenea rispetto a $(P_1,P_2,P_3)$).
    \end{remark}
    \newcommand{\lamudmat}[2]{\left|
    \begin{matrix}
    \lambda_{#1} & \lambda_{#2}\\
    \mu_{#1} & \mu_{#2}
    \end{matrix}
    \right|}
    \begin{proposition}[Calcolo del birapporto]
    Se in un riferimento proiettivo fissato abbiamo $P_i=[\la_i,\mu_i]$ per $i\in\{1,2,3,4\}$ allora
    \small{\[\beta(P_1,P_2,P_3,P_4)=\frac{\lamudmat14\lamudmat32}{\lamudmat42\lamudmat13}=\frac{(\la_1\mu_4-\la_4\mu_1)(\la_2\mu_3-\la_3\mu_2)}{(\la_2\mu_4-\la_4\mu_2)(\la_1\mu_3-\la_3\mu_1)}\]}
    \end{proposition}
    \begin{remark}
    Se $\la_i\neq0$ per ogni $i$ allora ponendo $z_i=\mu_i/\la_i$ troviamo
    \[\beta(P_1,P_2,P_3,P_4)=\frac{(z_4-z_1)(z_3-z_2)}{(z_4-z_2)(z_3-z_1)}=\dfrac{\;\frac{z_4-z_1}{z_3-z_1}\;}{\frac{z_4-z_2}{z_3-z_2}}=\frac{[P_1,P_3,P_4]}{[P_2,P_3,P_4]},\]
    ovvero il birapporto \`e il rapporto di due rapporti semplici.
    \end{remark}
    \begin{corollary}
    Se $P_4$ \`e il punto all'infinito, ponendo $z_i$ come sopra per $i=1,2,3$ si ha che
    \[\beta(P_1,P_2,P_3,\infty)=\frac{z_3-z_2}{z_3-z_1}=[P_3,P_1,P_2].\]
    \end{corollary}

    Vediamo ora che effettivamente il birapporto \`e invariante per proiettivit\`a:
    \begin{proposition}[Proiettivit\`a conservano birapporto]
    Se $f:\PV\to\PW$ \`e una proiettivit\`a tra due rette proiettive e $P_1,P_2,P_3,P_4\in \PV$ con $P_1,P_2,P_3$ distinti allora
    \[\beta(P_1,P_2,P_3,P_4)=\beta(f(P_1),f(P_2),f(P_3),f(P_4)).\]
    \end{proposition}

    Possiamo estendere la proposizione precedente come segue:
    \begin{proposition}[Criterio del birapporto per l'esistenza di proiettivit\`a]\label{CriterioEsistenzaProiettivitaConBirapporto}
    Se $\PV$ e $\PW$ sono rette proiettive, $P_1,P_2,P_3,P_4\in \PV$ con $P_1,P_2,P_3$ distinti e $Q_1,Q_2,Q_3,Q_4\in \PW$ con $Q_1,Q_2,Q_3$ distinti allora esiste una proiettivit\`a $f:\PV\to\PW$ tale che $f(P_i)=Q_i$ per tutti gli $i$ se e solo se
    \[\beta(P_1,P_2,P_3,P_4)=\beta(Q_1,Q_2,Q_3,Q_4).\]
    \end{proposition}

    \begin{proposition}
    Dati $P_1,P_2,P_3,P_4$ allineati distinti e ponendo $\beta=\beta(P_1,P_2,P_3,P_4)$ allora abbiamo che $\forall \sigma\in S_4$, il birapporto $\beta(P_{\sigma(1)},P_{\sigma(2)},P_{\sigma(3)},P_{\sigma(4)})$ pu\`o assumere solo uno tra i seguenti valori:
    \[\beta,\frac1\beta,1-\beta,\frac1{1-\beta},\frac{\beta-1}\beta,\frac\beta{\beta-1}.\]
    \end{proposition}

    Vediamo quindi che il birapporto non \`e un invariante per le quaterne non ordinate. Cerchiamo di costruire un invariante.

    \begin{proposition}
    Supponiamo $\cha\K=0$ e consideriamo la funzione razionale
    \[j(x)=\frac{(x^2-x+1)^3}{x^2(x-1)^2}.\]
    Se $\beta$ \`e il birapporto di quattro punti distinti ($\beta\neq0,1$ in particolare) allora $j(\beta)=j(\beta')$ se e solo se
    \[\beta'\in\left\{\beta,\frac1\beta,1-\beta,\frac1{1-\beta},\frac{\beta-1}\beta,\frac\beta{\beta-1}\right\}=B.\]
    \end{proposition}

    \begin{definition}[Modulo/j-invariante]
    Per $P_1,P_2,P_3,P_4$ distinti abbiamo visto che $j(\beta(P_1,P_2,P_3,P_4))$ \`e un invariante e si chiama \textbf{modulo} o \textbf{j-invariante} della quaterna. Per comodit\`a indichiamo
    $j(\beta(P_1,P_2,P_3,P_4))$ con $j(\{P_1,P_2,P_3,P_4\})$.
    \end{definition}

    \begin{theorem}
    Se $\PV,\PW$ sono rette proiettive e $\{P_1,P_2,P_3,P_4\}$, $\{Q_1,Q_2,Q_3,Q_4\}$ sono quaterne di punti distinti di $\PV$ e $\PW$ rispettivamente allora esiste $f:\PV\to\PW$ tale che
    \[f(\{P_1,P_2,P_3,P_4\})=\{Q_1,Q_2,Q_3,Q_4\}\]
    se e solo se $j(\{P_i\})=j(\{Q_i\})$.
    \end{theorem}

    \section{Coniche proiettive}
    \begin{definition}[Conica proiettiva]
    Una \textbf{conica proiettiva} di $\Pj^2(\K)$ \`e un elemento di $\Pj(\K_2[x_0,x_1,x_2])$. Definiamo il \textbf{supporto} di una conica $[p]$ come
    \[V([p])=\{[x_0,x_1,x_2]\mid p(x_0, x_1, x_2)=0\}\]
    \end{definition}
    \begin{remark}
    $\dim\K_2[x_0,x_1,x_2]=|\{x_0^2,x_1^2,x_2^2,x_0x_1,x_0x_2,x_1x_2\}|=6$, quindi lo spazio proiettivo delle coniche ha dimensione $5$.
    \end{remark}
    \begin{remark}
    Osserviamo che per $[p']=[p]$ e $[x_0,x_1,x_2]=[y_0,y_1,y_2]$
    \[p'(y_0,y_1,y_2)=\mu p(\la x_0,\la x_1, \la x_2)=\mu\la^2p(x_0,x_1,x_2).\]
    \`E quindi ben definito quando un punto del proiettivo \`e annullato da un polinomio omogeneo ma NON \`e definita la sua immagine.
    \end{remark}

    \noindent
    Come per il caso affine possiamo definire delle matrici che rappresentano coniche. Data la conica
    \[[ax_0^2+bx_1^2+cx_2^2+dx_0x_1+ex_0x_2+fx_1x_2]\]
    la possiamo scrivere come $x^\top A x$ dove
    \[x=\mat{x_0\\ x_1\\ x_2},\quad A=\mat{a & d/2 & e/2\\ d/2 & b & f/2\\ e/2 & f/2 & c}.\]
    Chiamiamo $A$ la matrice che \textbf{rappresenta} la conica.
    \begin{remark}
    $A$ \`e simmetrica e in effetti c'\`e una bigezione tra lo spazio delle coniche e $\Pj(S(3,\K))$.
    \end{remark}
    \begin{remark}
    Se pongo l'indeterminata $x_0$ uguale a $1$ la teoria che stiamo sviluppando ci restituisce ci\`o che avevamo gi\`a ricavato per le coniche affini. La mappa
    \[\funcDef{\Pj(\K_d[x_0,\cdots,x_n])}{\Pj(\K[x_1,\cdots, x_n])}{[p(x_0,\cdots,x_n)]}{[p(1,x_1,\cdots,x_n)]}\]
    \`e detta \textbf{deomogenizzazione} rispetto all'indeterminata $x_0$.
    \end{remark}


    \subsection{Equivalenza proiettiva e Classificazione delle coniche}
    Studiamo ora come mettere in relazione coniche e proiettivit\`a.

    \begin{definition}[Immagine di una conica tramite una proiettivit\`a]
    Data $f:\Pj^2\to\Pj^2$ e una conica $C=[p]$ poniamo $f(C)\doteqdot f^\ast C\doteqdot [p\circ M\ii]$ dove $M$ \`e una matrice che rappresenta $f$ nel riferimento standard. La notazione $f(C)$ non crea ambiguit\`a, infatti $f:\Pj^2\to\Pj^2$ e $f^\ast$ hanno domini diversi per esempio.
    \end{definition}
    \begin{remark}
    La definizione \`e ben posta, infattis
    \[[\la p\circ (\mu M)\ii]=[\la \mu^{-2}(p\circ M\ii)]=[p\circ M\ii].\]
    \end{remark}

    \begin{proposition}
    Sia $C$ una conica e $f$ una proiettivit\`a del piano proiettivo, allora $V(f(C))=f(V(C))$.
    \end{proposition}

    \begin{definition}[Equivalenza proiettiva]
    Due coniche $C,C'$ di $\Pj^2$ sono \textbf{proiettivamente equivalenti} se esiste una proiettivit\`a $f:\Pj^2\to\Pj^2$ tale che $f(C)=C'$.
    \end{definition}
    \begin{remark}
    $f(g(C))=(f\circ g)(C)$, infatti se $F, G$ sono matrici che rappresentano $f$ e $g$ rispettivamente abbiamo
    \[f(g(C))=f([p\circ G\ii])=[p\circ G\ii\circ F\ii]=[p\circ (FG)\ii]=(f\circ g)(C).\]
    \end{remark}

    Osserviamo che l'equivalenza proiettiva \`e una relazione di equivalenza, quindi ha senso classificare le coniche a meno di questa equivalenza.

    \begin{remark}
    Se $A$ \`e una matrice simmetrica che rappresenta la conica $[p]=C$ e $f=[M]$ \`e una proiettivit\`a allora
    \[f(C)=[(M\ii)^\top AM\ii].\]
    \end{remark}
    \begin{remark}
    $A$ e $A'$ sono matrici simmetriche che rappresentano coniche proiettivamente equivalenti se e solo se esistono $M\in \GL[3]\K$ e $\la\in \K^\times$ tali che
    \[A'=\la M^\top A M,\]
    ovvero se $A$ e $A'$ sono congruenti a meno di scalare.
    \end{remark}

    Allora classificare le coniche proiettive ci porta a classificare i prodotti scalari a meno di scalare. Questo sappiamo farlo per $\R$ e $\C$.

    \begin{theorem}[Classificazione delle coniche proiettive complesse]
    Le coniche proiettive complesse si distinguono in tre classi a meno di equivalenza proiettiva. Queste sono determinate dal rango della matrice che le rappresenta. Dei rappresentanti sono dati da
    \begin{center}
    \small{\begin{tabular}{|c|c|c|c|}
    \hline
    Rango & Rappresentante & Matrice & Nome\\\hline
    $3$ & $x_0^2+x_1^2+x_2^2$ & $\mat{1&&\\&1&\\&&1}$ & Non degenere\\
    $2$ & $x_0^2+x_1^2$ & $\mat{1&&\\&1&\\&&0}$ & Degenere\\
    $1$ & $x_0^2$ & $\mat{1&&\\&0&\\&&0}$ & Doppiamente degenere\\\hline
    \end{tabular}}
    \end{center}
    \end{theorem}
    \begin{remark}
    Nel caso di rango $2$, essendo in un campo algebricamente chiuso, vale che $x_0^2+x_1^2=(x_0+ix_1)(x_0-ix_1)$, quindi il supporto delle coniche proiettive complesse degeneri \`e dato da due rette distinte. Per il caso di rango $1$ abbiamo una retta doppia data da $x_0=0$.
    \end{remark}
    \begin{remark}
    In realt\`a questa classificazione continua a valere per un qualsiasi campo algebricamente chiuso di caratteristica diversa da $2$.
    \end{remark}

    \begin{theorem}[Classificazione delle coniche proiettive reali]
    Le coniche proiettive reali si distinguono in cinque classi a meno di equivalenza proiettiva. Queste sono determinate dalla segnatura della matrice che le rappresenta a meno di identificare $(a,b,c)$ e $(b,a,c)$. Dei rappresentanti sono dati da
    \begin{center}
    \footnotesize{\begin{tabular}{|c|c|c|c|}
    \hline
    Segnatura & Rappresentante & Degenericit\`a & Supporto\\\hline
    $(3,0,0)$ & $x_0^2+x_1^2+x_2^2$ & Non degenere & Vuoto\\
    $(2,1,0)$ & $x_0^2+x_1^2-x_2^2$ & Non degenere & Non vuoto\\
    $(2,0,1)$ & $x_0^2+x_1^2$ & Degenere & Punto\\
    $(1,1,1)$ & $x_0^2-x_1^2$ & Degenere & Due rette distinte\\
    $(1,0,2)$ & $x_0^2$ & Doppiamente degenere & Retta doppia\\\hline
    \end{tabular}}
    \end{center}
    \end{theorem}
    \begin{remark}
    Nei casi degeneri possiamo trovare facilmente il supporto per i rappresentanti:
    \begin{align*}
    &V([x_0^2+x_1^2])=[0,0,1]\\
    &V([x_0^2-x_1^2])=\{x_0-x_1=0\}\cup\{x_1+x_0=0\}\\
    &V([x_0^2])=\{x_0=0\}
    \end{align*}
    \end{remark}

    \subsection{Parte affine e chiusura proiettiva}

    \begin{definition}[Parte affine]
    Sia $C=[p]$ una conica proiettiva e supponiamo che $x_0\nmid p$ . Allora il polinomio $f(x,y)=p(1,x,y)$ (il \textbf{deomogeneizzato} di $p$ rispetto alla prima indeterminata) definisce una conica affine che definiamo \textbf{parte affine} di $C$.\\
    \noindent
    Poniamo come notazione $[f]=j_0\ii(C)$
    \end{definition}
    \begin{remark}
    La condizione $x_0\nmid p$ assicura che $\deg f=2$. Geometricamente stiamo escludendo componenti interamente all'infinito.
    \end{remark}
    \begin{remark}
    Se siamo su un campo infinito, per il principio di identit\`a dei polinomi abbiamo che $x_0\nmid p$ \`e equivalente a richiedere $\{x_0=0\}\not\subseteq V(C)$.
    \end{remark}
    \begin{proposition}
    Per $p$ e $f$ come sopra abbiamo (identificando $\K^2$ con $U_0$) $V([p])\cap U_0=V([f])$.
    \end{proposition}

    \begin{definition}[Chiusura proiettiva]
    Se $C=[f]$ \`e una conica affine chiamiamo la sua \textbf{chiusura proiettiva} la conica proiettiva di equazione
    \[p(x_0,x_1,x_2)=x_0^2f(x_1/x_0,x_2/x_0).\]
    Poniamo come notazione $[p]=j_0(C)=\ol{C}$.
    \end{definition}



    \begin{remark}
    Per come le abbiamo definite valgono
    $j_0\ii(\ol{C})=C$ e, se $x_0\nmid p$, $\ol{j_0\ii([p])}=[p]$.
    \end{remark}

    \begin{definition}[Punti all'infinito di una conica]
    Data una conica affine $C$ definiamo $V(\ol C)\cap H_0$ i \textbf{punti all'infinito} o \textbf{impropri} di $C$.
    \end{definition}

    \begin{remark}
    Se $A$ \`e la matrice che rappresenta la conica affine $f$, cio\`e
    \[f(x,y)=\mat{1 & x & y}A\mat{1\\ x\\ y}\]
    allora $A$ \`e la matrice che rappresenta anche la chiusura proiettiva
    \[p(x_0,x_1,x_2)=\mat{x_0 & x_1 & x_2}A\mat{x_0\\ x_1\\ x_2}.\]
    Segue che per $C$ conica affine, $C$ \`e non degenere se e solo se $\ol C$ \`e non degenere.
    \end{remark}
    \begin{remark}
    Osserviamo che il numero di classi di coniche non degeneri a supporto non vuoto nel caso reale non coincidono tra le coniche affini e le coniche proiettive.
    In particolare vediamo che le chiusure proiettive di ellissi, iperboli e parabole devono essere proiettivamente equivalenti. Vedremo che la differenza corrisponde a come incontrano la retta all'infinito.
    \end{remark}

    \begin{theorem}
    Sia $C$ una conica affine reale non vuota non degenere, allora
    \begin{itemize}[noitemsep]
    \item $C$ \`e un'ellisse $\coimplies$ $V(\ol C)\cap H_0=\emptyset$
    \item $C$ \`e una parabola $\coimplies$ $|V(\ol C)\cap H_0|=1$
    \item $C$ \`e un'iperbole $\coimplies$ $|V(\ol C)\cap H_0|=2$
    \end{itemize}
    \end{theorem}

    \begin{proposition}[Coniche per 5 punti]
    Dati cinque punti $P_0,\cdots, P_4\in \Pj^2$ in posizione generale esiste un'unica conica $C$ tale che $P_i\in V(C)$ per ogni $i$ e questa \`e non degenere.
    \end{proposition}



    \subsection{Tangenti}
    [Da ora in poi considereremo solo campi infiniti con caratteristica diversa da $2$. Principalmente tratteremo di $\R$ e $\C$.]

    \subsubsection{Intersezioni con rette e riducibilit\`a}
    Studiamo come si comportano le intersezioni tra una retta e una conica in modo da poter definire successivamente la tangente.

    \begin{definition}[Componenti e riducibilit\`a]
    Una retta $r$ di equazione $\ell$ \`e una \textbf{componente} della conica $C=[F]$ se $\ell\mid F$.\\
    Una conica $C=[F]$ \`e \textbf{irriducibile} se $F$ \`e irriducibile.
    \end{definition}
    \begin{remark}
    Se $r$ \`e una componente di $C$ allora $C$ non \`e irriducibile e viceversa dato che $\deg C=2$.
    \end{remark}
    \begin{remark}
    L'irriducibilit\`a \`e un invariante per equivalenza affine/proiettiva.
    \end{remark}
    \begin{remark}
    Se $r$ \`e componente di $C$ allora $r\subseteq V(C)$ e quindi $|r\cap V(C)|$ \`e infinito.
    \end{remark}


    \begin{lemma}
    Dato un polinomio $p\in \K[x_0,\cdots,x_n]$ omogeneo tale che $x_0\nmid p$, esso \`e irriducibile se e solo se $f=p(1,x_1,\cdots, x_n)$ \`e irriducibile.
    \end{lemma}
    \begin{remark}
    I fattori irriducibili di un polinomio omogeneo sono omogenei.
    \end{remark}



    \begin{proposition}[Numero di intersezioni tra coniche e rette]
    Sia $C=[F]$ una conica in $\Pj^2$ e $r$ una retta, allora $|V(C)\cap r|$ \`e finito se e solo se \`e minore o uguale a $2$. Se il campo \`e algebricamente chiuso allora $|V(C)\cap r|\geq 1$.
    \end{proposition}



    \begin{proposition}[Non degenere se e solo se irriducible]
    Se siamo su un campo algebricamente chiuso una conica $C$ \`e irriducibile se e solo se \`e non degenere.
    \end{proposition}
    \begin{remark}
    Non degenere implica sempre irriducibile, la dimostrazione proposta sopra non ha usato l'ipotesi di chiusura algebrica.
    \end{remark}

    \begin{proposition}
    Se $C$ \`e una conica tale che $|V(C)\cap r|=\infty$ per una retta $r$ di equazione $\ell$ allora $r$ \`e una componente di $C=[p]$.
    \end{proposition}

    \subsubsection{Tangenti}
    \begin{definition}[Retta tangente]
    Se $C$ \`e una conica non degenere allora $r$ \`e \textbf{tangente} a $C$ in $P=[w]\in V(C)\bs H_0$ se $t^2\mid C([w+tv])$ dove $r=\{[w+tv]\mid t\in \K\}$.
    \end{definition}
    \begin{remark}
    Per le coniche in particolare, questo corrisponde a chiedere $|V(C)\cap r|=\{P\}$.
    \end{remark}

    \begin{remark}
    Se $f:\Pj^2\to\Pj^2$ \`e una proiettivit\`a, $C$ una conica, $P\in V(C)$ e $r$ tangente a $C$ in $P$ allora $f(r)$ \`e tangente a $f(C)$ in $f(P)$.
    \end{remark}
    \begin{proposition}[Equazione della tangente]
    Sia $C$ una conica non degenere. Per ogni $P\in V(C)$ esiste un unica retta $\tau_P$ tangente a $C$ in $P$. Inoltre, se $M\in S(3,\K)$ rappresenta $C$ e $P=[v]$ allora $\tau_P$ ha equazione $(Mv)^\top x=0$.
    \end{proposition}

    \begin{remark}
    $\nabla C(v)=Mv$ quindi possiamo scrivere la retta tangente come lo spazio ortogonale al gradiente dell'equazione nel punto (ricordiamo che stiamo parlando del caso non degenere).
    \end{remark}


    \begin{definition}[Tangenti a conica degenere]
    Se $C=\ell_1\ell_2$ con $\ell_1,\ell_2$ omogenei lineari allora, posto $s_1=V(\ell_1)$ e $s_2=V(\ell_2)$, si ha che $r$ \`e tangente a $C$ in $P\in V(C)$ se
    \begin{itemize}[noitemsep]
    \item $P\in s_1\bs s_2$ e $r=s_1$
    \item $P\in s_2\bs s_1$ e $r=s_2$
    \item $P\in s_1\cap s_2$ e $r$ \`e qualsiasi.
    \end{itemize}
    \end{definition}
    \begin{remark}
    Se $\ell_1=\ell_2$ allora ogni retta \`e tangente a $C$ in $P$ per ogni $P\in C$.
    \end{remark}

    \subsection{Polarit\`a}
    \begin{definition}[Retta polare]
    Sia $C$ una conica non degenere e sia $M$ una matrice simmetrica che la rappresenta. Dato un punto $P=[v]$, la \textbf{retta polare} di $P$ rispetto a $C$ \`e la retta di equazione
    \[(Mv)^\top x=0\]
    che indichiamo con $pol(P)$.
    \end{definition}
    \begin{remark}
    Dato che $M$ \`e invertibile si ha che per ogni retta $r$ di coordinate $w$ nel duale possiamo costruire il punto $P=[M\ii w]$ tale che $r=pol(P)$ (rispetto alla conica $C$ rappresentata da $M$). Il punto $P$ si dice \textbf{polo} della retta rispetto a $C$.
    \end{remark}

    \begin{proposition}
    Siano $P,Q\in\Pj^2$ e sia $C$ una conica non degenere.
    \begin{enumerate}
    \item $P\in pol(Q)\coimplies Q\in pol(P)$
    \item Se $P\in V(C)$ allora $pol(P)$ \`e la tangente a $C$ in $P$ ($\tau_P$)
    \item $pol(P)\cap V(C)=\{Q\in V(C)\mid P\in \tau_Q\}$.
    \end{enumerate}
    \end{proposition}

    \begin{definition}[Conica duale]
    Consideriamo l'isomorfismo proiettivo
    \[\funcDef{\Pj^2}{(\Pj^2)^\ast}{P}{pol(P)},\]
    che \`e tale perch\'e trasformazione proiettiva tra spazi della stessa dimensione.
    Se restringiamo il dominio al supporto della conica in questione vediamo che l'immagine deve essere il supporto di una conica in $(\Pj^2)^\ast$, che chiamiamo \textbf{conica duale} di $C$ e indichiamo con $C^\ast$ o $pol(C)$.
    \end{definition}
    \begin{remark}
    Se $M$ rappresenta $C$ allora $C^\ast$ \`e rappresentata da $M\ii$
    \end{remark}

    \subsection{Punti reali e punti complessi}
    \begin{definition}[Complessificazione]
    Se $C=[F]$ \`e una conica su $\Pj^2\R$, la sua \textbf{complessificata} \`e la conica $C_\C$ data dalla stessa equazione ma vista a coefficienti in $\C$.
    \end{definition}

    \begin{remark}
    Se $C$ e $D$ sono coniche complesse allora
    \[C=D\coimplies V(C)=V(D)\]
    \end{remark}

    \begin{remark}
    Tramite l'inclusione $\Pj^2\R\subseteq \Pj^2\C$ si ha che $V(C)\subseteq V(C_\C)$. Pi\`u precisamente vale
    \[V(C)=V(C_\C)\cap \Pj^2\R.\]
    \end{remark}

    \begin{remark}
    Non tutte le coniche in $\Pj^2\C$ si ottengono complessificando coniche reali. Inoltre non tutti i punti di $\Pj^2\R$ (visto nell'immersione $\Pj^2\C$) hanno coordinate reali, per esempio $[1:1:2]=[i:i:2i]$.
    \end{remark}


    \subsection{Sistemi lineari di coniche}

    \begin{definition}[Sistema lineare di coniche]
    Un \textbf{sistema lineare di coniche} \`e un sottospazio proiettivo di
    \[\Pj(\K_2[x_0,x_1,x_2]).\]
    Un sistema di dimensione $1$ si dice \textbf{fascio}.
    \end{definition}

    \begin{proposition}
    Se $P_1,\cdots, P_4$ sono punti in posizione generale allora
    \[\Lambda=\{C\mid P_1,\cdots, P_4\in V(C)\}\]
    \`e un fascio di coniche.
    \end{proposition}

    \begin{corollary}
    Dati $P_1,\cdots,P_4$ come sopra, se $\ell_1,\ell_2,g_1,g_2$ sono le equazioni delle rette \[L(P_1,P_3),L(P_2,P_4),L(P_1,P_4),L(P_2,P_3)\] allora la conica generica del fascio della proposizione \`e della forma
    \[\la\ell_1\ell_2+\mu g_1g_2.\]
    \end{corollary}

    \begin{remark}
    Anche imporre la tangenza ad una data retta corrisponde ad una condizione lineare sulla conica.
    \end{remark}

    \begin{definition}[Punto base]
    Se $\Lambda$ \`e un sistema lineare di coniche, un punto $P$ \`e un \textbf{punto base} di $\Lambda$ se $P\in V(C)$ per ogni $C\in\Lambda$.
    \end{definition}

    \begin{remark}
    Se $\Lambda$ \`e un fascio e $G_1,G_2$ sono coniche distinte di $\Lambda$ allora i punti base di $\Lambda$ sono $V(G_1)\cap V(G_2)$, infatti i punti base sono inclusi in questa intersezione e ogni altra conica di $\Lambda$ \`e della forma $\la G_1+\mu G_2$, che quindi si annulla su questi punti.
    \end{remark}

    \begin{remark}
    I punti base possono essere infiniti, si consideri per esempio il fascio $\la x_0x_1+\mu x_0x_2=x_0(\la x_1+\mu x_2)$.
    \end{remark}

    \begin{remark}
    Per quattro punti in posizione generale passano esattamente tre coniche degeneri
    \end{remark}










\end{multicols*}

\chapter{Topologia generale}
\setlength{\parindent}{2pt}

\begin{multicols*}{2}

\section{Spazi Metrici}
Gli spazi metrici, come suggerisce il nome, sono insiemi dotati di una distanza o metrica. In un certo senso sono gli spazi più vicini alla nostra intuizione (ma non sempre, per alcune metriche per esempio tutti i triangoli risultano isosceli!).
\begin{definition}[Spazio metrico]
Dato un insieme $X$, una \textbf{distanza} su $X$ è una funzione $d:X\times X\to[0,+\infty)$ tale che
\begin{itemize}[noitemsep]
\item $d(x,y)=0\coimplies x=y$
\item $d(x,y)=d(y,x)$
\item $d(x,z)\leq d(x,y)+d(y,z)$
\end{itemize}
L'ultima proprietà è detta \textbf{disuguaglianza triangolare}.\\
La coppia $(X,d)$ con $d$ distanza è detta uno \textbf{spazio metrico}.
\end{definition}

\begin{definition}[Distanza punto-insieme]
Se $(X,d)$ è uno spazio metrico e $A\subseteq X$ allora definiamo la distanza tra un punto $x\in X$ e $A$ come
\[d_A(x)=\inf\{d(x,a)\mid a\in A\}.\]
\end{definition}

Un modo semplice di costruire spazi metrici è dotare spazi vettoriali di una
\begin{definition}[Norma]
Dato $V$ uno spazio vettoriale, una \textbf{norma} su $V$ è una funzione $\|\cdot\|:V\to[0,+\infty)$ tale che
\begin{itemize}[noitemsep]
\item $\|x\|=0\coimplies x=0$
\item $\|\la x\|=|\la|\|x\|$
\item $\|x+y\|\leq \|x\|+\|y\|$.
\end{itemize}
\end{definition}
\begin{remark}
la funzione $d(x,y)=\|x-y\|$ è una distanza, detta \textbf{distanza indotta} dalla norma $\|\cdot\|$.
\end{remark}
\vspace{0.5cm}

\noindent Definiamo delle distanze ricorrenti:
\begin{definition}[Distanza discreta]
Dato un qualsiasi insieme $X$, la \textbf{distanza discreta} su $X$ è data da
\[d(x,y)=\begin{cases}
0& se\ x=y\\
1& se\ x\neq y
\end{cases}.\]
Vediamo che effettivamente è una distanza: le prime due proprietà sono chiaramente rispettate e la disuguaglianza triangolare vale perché $d(x,y)+d(y,z)\geq1$ eccetto il caso dove $x=y=z$, dove la disuguaglianza triangolare coincide con affermare $0+0\geq0$.
\end{definition}
\begin{definition}[Distanze $p$]
Sia $p\geq1$ e consideriamo le seguenti norme su $\R^n$:
\[\|x\|_p=\pa{\sum_{i=1}^n|x_i|^p}^{1/p}.\]
Definiamo $d_p$ come le distanze indotte da queste norme. Si può verificare che prendendo il limite $p\to\infty$ si definisce un'ulteriore norma:
\[\|x\|_\infty=\max_i|x_i|\]
che induce la distanza $d_\infty$.\\
Chiamiamo $\|\cdot\|_2=|\cdot|$ la \textbf{norma euclidea} e $d_2$ la \textbf{distanza euclidea}.
\end{definition}

\begin{definition}[Distanze $p$ integrali]
Consideriamo lo spazio $X=C^0([0,1])$ delle funzioni continue sull'intervallo chiuso $[0,1]$ e un reale $p\geq1$. Definiamo le seguenti norme:
\[\|f\|_p=\pa{\int_0^1|f(t)|^p}^{1/p}\]
\[\|f\|_\infty=\sup_{t\in[0,1]}|f(t)|\overset{Weierstrass}{=}\max_{t\in[0,1]}|f(t)|.\]
In appendice è disponibile una dimostrazione che queste sono effettivamente norme, e quindi inducono distanze sullo spazio delle funzioni continue.
\end{definition}

\begin{definition}[Embedding isometrico]
Dati $(X,d)$ e $(Y,d')$ spazi metrici, $f:X\to Y$ è un \textbf{embedding isometrico} se \[d'(f(x),f(y))=d(x,y).\]
\end{definition}
\begin{remark}
Un embedding isometrico è sempre iniettivo, infatti
\[f(x)=f(y)\implies d'(f(x),f(y))=0=d(x,y)\implies x=y.\]
\end{remark}
\begin{remark}
La composizione di embedding isometrici è un embedding isometrico. L'identità $(X,d)\to(X,d)$ è un embedding isometrico.
\end{remark}
\begin{remark}
Gli spazi metrici con gli embedding isometrici sono una categoria.
\end{remark}

\begin{definition}[Isometria]
Una \textbf{isometria} è un embedding isometrico bigettivo.
\end{definition}
\begin{remark}
L'inversa di una isometria è una isometria e la composizione di isometrie è una isometria.
Banalmente l'identità è una isometria.
\end{remark}

\begin{notation}[Gruppo delle isometrie]
Denotiamo il \textbf{gruppo delle isometrie} di $(X,d)$ con $Isom(X)$.
\end{notation}
\vspace{0.5cm}

\noindent Diamo ora una delle definizioni che risulteranno essere tra le più importanti del corso:
\begin{definition}[Palla aperta]
Dato $(X,d)$ spazio metrico, $r\in\R,\ r>0$, e $x\in X$, definiamo \[B_r(x)=B_d(x,r)=B(x,r)=\{y\in X\mid d(x,y)<r\}\] la \textbf{palla aperta} di \textbf{raggio} $r$ e \textbf{centro} $x$.
\end{definition}
\noindent (Purtroppo la notazione che userò per le palle sarà estremamente variabile.)

\begin{definition}[Continuità in un punto]
Siano $(X,d)$ e $(Y,d')$ spazi metrici. Una funzione $f:X\to Y$ è \textbf{continua} in $x_0\in X$ se
\[\forall \e>0,\ \exists \delta>0\ t.c.\ f(B_d(x_0,\delta))\subseteq B_{d'}(f(x_0),\e).\]
La funzione è \textbf{continua} se è continua per ogni $x_0\in X$.
\end{definition}

\begin{definition}[Aperto metrico]
Se $(X,d)$ è spazio metrico, $A\subseteq X$ è \textbf{aperto} (rispetto alla metrica $d$) se $\forall x\in A,\ \exists \e>0$ tale che $B_\e(x)\subseteq A$.
\end{definition}

\begin{lemma}
Le palle aperte sono insiemi aperti nella metrica che le definisce.
\end{lemma}

\begin{theorem}[Caratterizzazione delle continue]
Data una funzione $f:X\to Y$ tra spazi metrici vediamo che essa è continua se e solo se la controimmagine di un aperto di $Y$ tramite $f$ è un aperto di $X$.
\end{theorem}
\begin{definition}[Mappa Lipschitziana]
Se $(X,d)$ e $(Y,d')$ sono spazi metrici, dato $k\geq 0$, una funzione $f:X\to Y$ è \textbf{$k-$Lipschitziana} se
\[d'(f(p),f(q))\leq k\ d(p,q),\ \forall p,q\in X.\]
\end{definition}
\begin{proposition}
Una funzione lipschitziana è continua.
\end{proposition}

\vspace{0.5cm}

\noindent Verifichiamo ora che le palle aperte degli spazi metrici rispettano le seguenti fondamentali proprietà
\begin{proposition}
Sia $(X,d)$ uno spazio metrico, allora
\begin{enumerate}[noitemsep]
\item $\emptyset$ e $X$ sono aperti.
\item Se $A,B$ sono aperti allora $A\cap B$ è aperto.
\item Se $A_i$ con $i\in I$ è una famiglia di aperti allora $\displaystyle \bigcup_{i\in I}A_i$ è aperto.
\end{enumerate}
\end{proposition}
\begin{remark}
L'intersezione arbitraria di aperti può non essere aperta, per esempio l'intersezione della famiglia data da $B_{1/n}(0)$ in $\R$.
\end{remark}
\section{Spazi topologici}
Le proprietà date sopra hanno in realtà una validità generale e ci permettono di definire gli oggetti principali che tratteremo in questo capitolo:
\begin{definition}[Spazio topologico]
Uno \textbf{spazio topologico} è una coppia $(X,\tau)$ dove $X$ è un insieme e $\tau\subseteq \powerset(X)$ per il quale valgono le seguenti proprietà:
\begin{enumerate}[noitemsep]
\item $\emptyset,X\in\tau$
\item $A,B\in\tau\implies A\cap B\in \tau$
\item $\Phi\subseteq \tau\implies \bigcup_{A\in\Phi}A\in \tau$.
\end{enumerate}
L'insieme $\tau$ è detta una \textbf{topologia} su $X$ e i suoi elementi sono detti insiemi \textbf{aperti} di $(X,\tau)$.
\end{definition}
\begin{remark}
Ogni distanza su $X$ induce una topologia su $X$.
\end{remark}

\begin{definition}[Topologie discreta e indiscreta]
Gli insiemi $\powerset(X)$ e $\{\emptyset,X\}$ sono delle topologie su $X$ e sono chiamate \textbf{topologia discreta} e \textbf{topologia indiscreta} rispettivamente.
\end{definition}
\begin{definition}[Topologia cofinita]
Dato l'insieme $X$, l'insieme $\tau=\{\emptyset\}\cup\{A\subseteq X\mid |X\bs A|\in \N\}$ è una topologia su $X$ ed è detta la \textbf{topologia cofinita}. Essa ha come chiusi gli insiemi finiti e tutto lo spazio.\\
In modo analogo si pu\`o definire la topologia \textbf{conumerabile}.
\end{definition}
\begin{remark}
Non tutte le topologie sono indotte da metriche, per esempio la topologia indiscreta non può essere descritta come topologia indotta da metrica se $|X|\geq2$. Questo deriva dal fatto che presi $x_1,x_2\in X$ distinti, $B(x_1,d(x_1,x_2)/2)$ e $B(x_2,d(x_1,x_2)/2)$ sono disgiunte e non vuote.
\end{remark}

\begin{definition}[Chiuso]
Dato $(X,\tau)$ uno spazio topologico, $C\subseteq X$ è \textbf{chiuso} se $X\bs C\in \tau$.
\end{definition}
\begin{remark}
Una topologia può essere descritta anche dai chiusi. Tramite le leggi di De Morgan troviamo la seguente caratterizzazione equivalente di una topologia: Sia $\chi=\{X\bs A\mid A\in \tau\}$ allora $\tau$ è una topologia su $X$ se e solo se
\begin{enumerate}[noitemsep]
\item $\emptyset,X\in\chi$
\item $A,B\in\chi\implies A\cup B\in \chi$
\item $\Phi\subseteq \chi\implies \bigcap_{C\in\Phi}C\in \chi$.
\end{enumerate}
\end{remark}


\begin{definition}[Finezza]
Siano $\tau_1,\tau_2$ topologie su $X$. $\tau_1$ è \textbf{meno fine} di $\tau_2$ se $\tau_1\subseteq \tau_2$.
\end{definition}
\begin{remark}
La finezza descrive un ordinamento parziale sulle topologie con massimo (la topologia discreta) e minimo (l'indiscreta).
\end{remark}
\noindent Intuitivamente una topologia è più fine se ha più aperti o più chiusi, cioè ci permette di distinguere meglio i punti.

\subsection{Equivalenza topologica di distanze e limitatezza} Consideriamo nuovamente le topologie indotte da metriche:\begin{definition}[Metriche topologicamente equivalenti]
Date due distanze $d_1,d_2$ su $X$, esse sono \textbf{topologicamente equivalenti} se inducono la stessa topologia su $X$.
\end{definition}

\begin{proposition}\label{ConfrontoTraDistanze}
Siano $d_1,d_2$ distanze su $X$ con topologie indotte $\tau_1$ e $\tau_2$ rispettivamente. Se $\exists k>0$ tale che $d_1(x,y)\leq k d_2(x,y)$ allora $\tau_2$ è più fine di $\tau_1$.
\end{proposition}
\begin{corollary}[Criterio per equivalenza topologica]\label{CriterioEquivalenzaDistanze}
Siano $d_1,d_2$ distanze su $X$ tali che $\exists k>0,h>0$ tali che
\[d_1(x,y)\leq kd_2(x,y)\qquad d_2(x,y)\leq h d_1(x,y),\]
allora $d_1$ e $d_2$ sono topologicamente equivalenti.
\end{corollary}
\begin{notation}
Distanze come nel corollario precedente si dicono \textbf{bilipschitziane} tra loro.
\end{notation}
\begin{corollary}\label{Norme12inftySonoEquivalenti}
Le distanze indotte da da $\|\cdot\|_1,\|\cdot\|_2$ e $\|\cdot\|_\infty$ sono topologicamente equivalenti su $\R^n$
\end{corollary}

\begin{remark}
Non tutte le metriche sullo stesso spazio sono topologicamente equivalenti (\ref{DistanzeNonTopEquivalenti})
\end{remark}

\begin{definition}[Limitatezza]
Se $(X,d)$ è uno spazio metrico, $Y\subseteq X$ è \textbf{limitato} se esistono $x\in X$ e $R\in \R$ tali che
\[Y\subseteq B_R(x).\]
\end{definition}
\begin{proposition}[Ogni spazio metrico ``è limitato"]\label{OgniMetricoETopologicamenteEquivalenteALimitato}
Se $(X,d)$ è uno spazio metrico allora esiste una metrica $d'$ su $X$ tale $d$ e $d'$ sono topologicamente equivalenti e $d'(x,y)\leq 1$ per ogni $x,y\in X$, in particolare $X$ è limitato in $(X,d')$.
\end{proposition}

\subsection{La categoria Top}

\begin{definition}[Funzione continua]
Una funzione $f:X\to Y$ con $(X,\tau_X),(Y,\tau_Y)$ spazi topologici è \textbf{continua} se $A\in \tau_Y\implies f\ii(A)\in \tau_X$.
\end{definition}
\begin{remark}
La definizione funziona anche chiedendo che la controimmagine di chiusi sia chiusa dato che prendere la controimmagine commuta con prendere il complementare.
\[f\ii(Y\bs A)=X\bs f\ii(A).\]
\end{remark}
\begin{definition}[Omeomorfismo]
Una funzione continua tra spazi topologici $f:X\to Y$ è un \textbf{omeomorfismo} se è biunivoca e $f\ii$ è continua.
\end{definition}

\begin{remark}
L'identità $id:(X,\tau)\to (X,\tau)$ è un omeomorfismo.
\end{remark}
\begin{remark}
Se $f:X\to Y$ e $g:Y\to Z$ sono continue allora anche $g\circ f: X\to Z$ è continua. Segue che la composizione di omeomorfismi è un omeomorfismo.
\end{remark}
\begin{remark}
Quanto detto ci permette di definire la categoria $Top$, i cui oggetti sono spazi topologici e i cui morfismi sono funzioni continue. Gli isomorfismi corrispondono agli omeomorfismi.
\end{remark}

\begin{remark}
Omeomorfismi e funzioni continue bigettive non coincidono.
\end{remark}

\subsection{Chiusura e Parte interna}
\begin{definition}[Chiusura]
Dato $X$ spazio topologico con topologia $\tau$ e un suo sottoinsieme $Y\subseteq X$, la \textbf{chiusura} di $Y$ in $X$ è il più piccolo chiuso che contiene $Y$, ovvero
\[\ol Y=\bigcap_{C\text{ chiuso, }C\supseteq Y}C.\]
Dato che l'intersezione di chiusi è chiusa e $X$ è chiuso, la chiusura è ben definita.
\end{definition}
\begin{definition}[Parte interna]
Dato $X$ spazio topologico con topologia $\tau$ e un suo sottoinsieme $Y\subseteq X$, la \textbf{parte interna} di $Y$ in $X$ è il più grande aperto contenuto in $Y$, ovvero
\[\rg Y=int(Y)=\bigcup_{A\in\tau,\ A\subseteq Y}A.\]
Dato che l'unione di aperti è aperta e $\emptyset$ è aperto, la parte interna è ben definita.
\end{definition}

\begin{remark}
Un aperto $A$ in $X$ è contenuto in $Z$ se e solo se $X\bs A$ è un chiuso che contiene $X\bs Z$, segue quindi che
\[int(Z)=X\bs\ol{(X\bs Z)}.\]
Analogamente
\[\ol Z=X\bs int(X\bs Z).\]
\end{remark}
\begin{remark}
$\rg Z=Z\coimplies Z$ aperto e $\ol Z=Z\coimplies Z$ chiuso.
\end{remark}
\begin{definition}[Frontiera]
La \textbf{frontiera (topologica)} o \textbf{bordo} di $Z\subseteq X$ è definita come
\[\partial Z=\ol Z\bs \rg Z\]
\end{definition}
\begin{definition}[Punti aderenti e di accumulazione]
Un punto $P\in X$ con $Z\subseteq X$ è
\begin{itemize}[noitemsep]
\item \textbf{aderente} a $Z$ se $P\in \ol Z$;
\item \textbf{di accumulazione} per $Z$ se $P\in \ol{Z\bs \{P\}}$.
\end{itemize}
\end{definition}
\begin{remark}
Se $Z\subseteq Y\subseteq X$ allora $\ol Z\subseteq \ol Y$ e $\rg Z\subseteq \rg Y$. In particolare i punti di accumulazione sono anche aderenti.
\end{remark}
\begin{remark}
Esistono punti aderenti che non sono di accumulazione (\ref{AderenteNonAccumulazione})
\end{remark}
\noindent
Diamo ora un criterio utile per capire se un punto è aderente. L'idea è che ogni aperto che lo contiene dovrà essere così vicino all'insieme che almeno una parte deve intersecarlo.
\begin{proposition}[Caratterizzazione della chiusura]\label{CaratterizzazioneChiusura}
Sia $X$ uno spazio topologico e $Z\subseteq X$. Allora $P\in \ol Z$ se e solo se $\forall A\subseteq X$ aperto tale che $P\in A$ abbiamo che $A\cap Z\neq \emptyset$.
\end{proposition}

\begin{proposition}
Se $P\notin Z$ e $P$ è aderente allora $P$ è di accumulazione.
\end{proposition}

\begin{remark}
$\ol{Z_1\cap Z_2}\neq \ol Z_1\cap \ol Z_2$, per esempio $\ol{\Q\cap (\R\bs \Q)}=\ol\emptyset=\emptyset$ ma $\ol \Q=\ol{\R\bs \Q}=\R$.
\end{remark}

\begin{definition}[Insieme denso]
Un sottoinsieme $Z\subseteq X$ è \textbf{denso} se $\ol Z= X$.
\end{definition}

\begin{remark}
$Z$ è denso in $X$ se e solo se $\forall A\subseteq X$ aperto non vuoto si ha $A\cap Z\neq\emptyset$.
\end{remark}

\subsection{Basi e Prebasi}
Se $X$ è un insieme e $S\subseteq \powerset(X)$ voglio trovare ``la topologia generata da $S$", cioè la più piccola topologia che contiene $S$.
\begin{lemma}
Siano $\tau_i$ delle topologie su $X$. Allora $\tau=\bigcap \tau_i\subseteq\powerset(X)$ è una topologia su $X$.
\end{lemma}
\begin{definition}[Topologia generata]
La \textbf{topologia generata} da $S\subseteq\powerset(X)$ è
\[\tau_S=\bigcap_{\tau\text{ topologia, }S\subseteq \tau}\tau.\]
Osservo che la definizione è ben posta per il lemma e perché l'intersezione non è vuota in quanto $\powerset(X)\supseteq S$.
\end{definition}

\begin{definition}[Base topologica]
Se $(X,\tau)$ è uno spazio topologico, una \textbf{base} di $\tau$ è un sottoinsieme $\Bc\subseteq \tau$ tale che \[\forall A\in \tau,\ \exists \Bc'\subseteq\Bc\ t.c.\ A=\bigcup_{B\in \Bc'}B.\]
\end{definition}

\begin{proposition}[Caratterizzazione delle basi]
L'insieme $\Bc\subseteq\powerset(X)$ è una base di qualche topologia se e solo se $X=\bigcup_{B\in \Bc}B$ e $\forall B_1,B_2\in \Bc,$ esiste $ \Bc'\subseteq\Bc$ tale che \[B_1\cap B_2=\bigcup_{B\in \Bc'}B.\]
\end{proposition}
\begin{remark}
In generale $S$ non è una base della topologia generata da $S$ perché può non rispettare le condizioni sopra.
\end{remark}

\begin{definition}[Prebase topologica]
Una \textbf{prebase} di una topologia $\tau$ su $X$ è un sottoinsieme $U\subseteq\powerset(X)$ tale che
\[\left\{\bigcap_{i=1}^k U_i\mid k\in\N,\ U_i\in U\right\}\]
è una base di $\tau$. Per evitare di riscrivere questo insieme troppe volte lo chiameremo ``le intersezioni finite di $U$".
\end{definition}

\begin{remark}
Ogni base è una prebase ma non è garantito il viceversa.
\end{remark}

\begin{theorem}[Caratterizzazione della topologia generata]
Se $X$ è un insieme e $S\subseteq\powerset(X)$ allora $\tau$ è la topologia generata da $S$ se e solo se $S\cup\{X\}$ è una prebase di $\tau$.
\end{theorem}

\begin{remark}
La topologia generata da $S$ sono le unioni arbitrarie delle intersezioni finite di $S$ a cui aggiungo $X$.
\end{remark}

\begin{proposition}[Criterio per continuità]
Se $X$ e $Y$ sono spazi topologici e $f:X\to Y$ è una funzione allora le seguenti affermazioni sono equivalenti:
\begin{enumerate}[noitemsep]
\item $f$ è continua.
\item $\exists \Bc\subseteq \powerset(Y)$ base per $Y$ tale che $\forall B\in \Bc$, $f\ii(B)$ è aperto in $X$.
\item $\exists U\subseteq \powerset(Y)$ prebase per $Y$ tale che $\forall U'\in U$, $f\ii(U')$ è aperto in $X$.
\end{enumerate}
\end{proposition}

\section{Assiomi di Numerabilità e Intorni}
Siamo pronti a definire il concetto di intorno e a dare gli assiomi di numerabilità. Questi concetti ci danno un modo per misurare quanto e come i punti del nostro spazio topologico sono vicini.


\subsection{Intorni}
\begin{definition}[Intorno]
Dato $X$ spazio topologico, un \textbf{intorno} di $x_0\in X$ è un sottoinsieme $U$ di $X$ tale che $\exists A$ aperto in $X$ tale che $x_0\in A\subseteq U$.
\end{definition}
\begin{remark}
Un intorno può non essere aperto.
\end{remark}
\begin{notation}
Indicheremo, se non altrimenti specificato, l'insieme degli intorni di un dato punto $x_0$ come $I(x_0)$ o $I_X(x_0)$ nel caso sia necessario specificare lo spazio.
\end{notation}


\begin{remark}
Per gli intorni valgono le seguenti proprietà:
\begin{enumerate}[noitemsep]
\item $U$ è un intorno di $x_0$ se e solo se $x_0\in \rg U$.
\item Se $U$ intorno di $x_0$ e $U\subseteq V$ allora anche $V$ è un intorno di $x_0$.
\item Se $U$ e $V$ sono intorni di $x_0$ allora anche $U\cap V$ è un intorno di $x_0$. Questo deriva dal fatto che $x_0\in \rg U\cap \rg V$ che essendo intersezione di aperti è ancora aperto e dunque $x_0\in \rg U\cap \rg V\subseteq int(U\cap V)$.
\end{enumerate}
\end{remark}

\begin{proposition}[Caratterizzazione di aperti/chiusi con intorni]\label{CaratterizzazioneApertiEChiusuraConIntorni}
Valgono le seguenti proposizioni:
\begin{enumerate}[noitemsep]
\item $A$ è aperto se e solo se $A$ è intorno di ogni suo punto
\item Se $Z\subseteq X$ allora la chiusura di $Z$ sono i punti $x$ tali che ogni $U$ intorno di $x$, $U\cap Z\neq\emptyset$.
\item $Z$ è chiuso se e solo se $(x\in X,\ U\in I(x),\ U\cap Z\neq\emptyset)\implies x\in Z$.
\end{enumerate}
\end{proposition}

\noindent Possiamo ora dare la caratterizzazione di continuità familiare dal contesto dell'analisi:
\begin{definition}[Continuità in un punto]
Una funzione $f:X\to Y$ è \textbf{continua in} $x_0\in X$ se $\forall U\in I(f(x_0))$ esiste $V\in I(x_0)$ tale che $f(V)\subseteq U$ (Equivalentemente se $\forall U\in I(f(x_0)),\ f\ii(U)\in I(x_0)$).
\end{definition}
\begin{proposition}[Continua equivale a continua in ogni punto]\label{ContinuaEquivaleContinuaInOgniPunto}
Una funzione $f:X\to Y$ è continua se e solo se è continua in $x_0$ per ogni $x_0$ in $X$.
\end{proposition}

\subsection{Sistemi fondamentali di intorni e I-numerabilità}
\begin{definition}[Sistema fondamentale di intorni]
Un \textbf{sistema fondamentale di intorni} (SFI) di $x_0\in X$ è un sottoinsieme $J\subseteq I(x_0)$ tale che $\forall U\in I(x_0)$ esiste $V\in J$ tale che $V\subseteq U$
\end{definition}

\begin{definition}[I-numerabilità]
Uno spazio $X$ è \textbf{primo numerabile} (I-numerabile) o soddisfa il \textbf{primo assioma di numerabilità} se per ogni $x_0\in X$ posso trovare un SFI al più numerabile.
\end{definition}

\begin{proposition}[Gli spazi metrici sono I-numerabili]\label{MetricoEINumerabile}
Se $X$ è uno spazio metrico allora $X$ è I-numerabile.
\end{proposition}

\begin{lemma}\label{LemmaINumerabile}
Se $X$ è I-numerabile e $U=\{U_i\}_{i\in \N}$ è un SFI di $x\in X$ allora posso supporre senza perdita di generalità che $U_{i+1}\subseteq U_i$ per ogni $i$.
\end{lemma}

\subsection{II-numerabilità e Separabilità}
\begin{definition}[II-numerabilità]
Uno spazio topologico è \textbf{II-numerabile} (letto ``secondo numerabile") o soddisfa il \textbf{secondo assioma di numerabilità} se esiste una base (al più) numerabile per la topologia.
\end{definition}
\begin{remark}
L'uso di ``al più" nella definizione sopra non è necessario. Nel caso di spazi finiti basta ripetere frequentemente dei termini. Preferiamo specificarlo però in quanto alcune definizioni successive richiederanno la specifica ``al più numerabile" in un contesto dove ``numerabile" da solo sarebbe scorretto.
\end{remark}
\begin{definition}[Separabilità]
Uno spazio topologico è \textbf{separabile} se contiene un sottoinsieme al più numerabile denso.
\end{definition}
\begin{theorem}[II-numerabile è separabile e in metrico coincidono]\label{IINumerabileESeparabileEInMetricoCoincidono}
Se $X$ è II-numerabile allora $X$ è separabile. Se $X$ è metrizzabile allora vale anche l'altra implicazione.
\end{theorem}
\begin{corollary}
$\R^n$ è II-numerabile.
\end{corollary}

\begin{proposition}[II-numerabile implica I-numerabile]\label{IINumerabileImplicaINumerabile}
Se $X$ è II-numerabile allora è anche I-numerabile.
\end{proposition}

\begin{remark}
I-numerabile e II-numerabile NON sono equivalenti (\ref{INumerabileNonIINumerabile}).
\end{remark}

\subsection{Successioni}
\begin{definition}[Successione]
Sia $X$ uno spazio topologico. Una \textbf{successione a valori in $X$} è una funzione $x:\N\to X$ (dove indichiamo tradizionalmente $x(n)$ con $x_n$).\\
Una \textbf{sottosuccessione} di $x_n$ è successione data da $x_{n_k}$, dove $n:\N\to \N$ è strettamente crescente.
\end{definition}
\begin{definition}[Definitivamente e Frequentemente]
Una successione $x_n$ rispetta una proprietà $P$ \textbf{definitivamente} se $\exists k_0$ tale che $\forall k\geq k_0$, $x_k$ rispetta $P$, mentre $x_n$ rispetta $P$ \textbf{frequentemente} se $\forall k_0,\ \exists k\geq k_0$ tale che $x_k$ rispetta $P$.
\end{definition}
\begin{definition}[Limite]
Affermiamo che la successione $x_n$ \textbf{tende} a $\ol x\in X$ (o che $\ol x$ è un \textbf{limite} di $x_n$) e scriviamo $x_n\to\ol x$ o $\displaystyle \lim_{n\to+\infty}x_n=\ol x$ se
\[\forall U\text{ intorno di }\ol x,\ x_n\in U\ \text{definitivamente}.\]
\end{definition}
\begin{remark}
Osserviamo che
\begin{itemize}[noitemsep]
\item i limiti possono non esistere.
\item i limiti di una data successione possono non essere unici. Per esempio nella topologia indiscreta ogni punto è limite di ogni successione.
\item se $x_n$ è una successione (definitivamente) costante allora quella costante è un limite di $x_n$.
\item Se $x_n\to\ol x$ allora ogni sottosuccessione $x_{n_k}$ è tale che $x_{n_k}\to \ol x$.
\end{itemize}
\end{remark}

\begin{definition}[Chiuso per successioni]
Un insieme $Y\subseteq X$ è \textbf{chiuso per successioni} se per ogni $x_n$ successione a valori in $Y$, se $x_n\to \ol x$ allora $\ol x\in Y$.
\end{definition}

\begin{proposition}[Chiusura e Chiusura per successioni]
Sia $Y\subseteq X$ e poniamo
\[\hat Y=\{\ol x\in X\mid \exists x_n\text{ successione }t.c.\ x_n\in Y\text{ e }x_n\to\ol x\}.\]
Allora $\hat Y\subseteq \ol Y$ e se $X$ è I-numerabile allora $\hat Y=\ol Y$.
\end{proposition}

\begin{definition}[Aperto per successioni]
Se $Y\subseteq X$ allora $Y$ è \textbf{aperto per successioni}\footnote{In realtà non ho mai visto questo termine usato, ho voluto dare questa definizione solo per aiutare i lettori ad afferrare la simmetria degli argomenti presentati in questa sezione.} se, per ogni $\ol x\in Y$, se $x_n$ è una successione in $X$ tale che $x_n\to \ol x$, allora $x_n\in Y$ definitivamente.
\end{definition}

\begin{proposition}[Parte interna e Parte interna per successioni]
Sia $Y\subseteq X$. Se
$\ol x\in\rg Y$, allora per ogni successione tale che $x_n\to \ol x$ si ha $x_n\in Y$ definitivamente.\\
Se $X$ è I-numerabile vale anche il viceversa.
\end{proposition}

\begin{definition}[Continuità per successioni]
Una funzione $f:X\to Y$ è \textbf{continua per successioni} se $\forall x_n$ tale che $x_n\to \ol x$ allora $f(x_n)\to f(\ol x)$.
\end{definition}
\begin{proposition}[Continuità e Continuità per successioni]
Sia $f:X\to Y$ tra spazi topologici. Allora se $f$ è continua è continua per successioni. Se $X$ è I-numerabile allora continuità e continuità per successioni sono equivalenti.
\end{proposition}

\noindent Possiamo riassumere questi risultati nella seguente
\begin{proposition}\label{ChiusiApertiContinuePerSuccessioni}
Se $Y\subseteq X$ e $f:X\to Z$ allora
\begin{itemize}[noitemsep]
\item se $Y$ è chiuso, è chiuso per successioni
\item se $Y$ è aperto, è aperto per successioni
\item se $f$ è continua allora è continua per successioni.
\end{itemize}
Se $X$ è I-numerabile allora le implicazioni sopra sono equivalenze.
\end{proposition}

\section{Topologia di sottospazio}
\begin{definition}[Topologia di sottospazio]
Sia $X$ uno spazio topologico e $Y\subseteq X$. La \textbf{topologia di sottospazio} di $Y$ è la topologia meno fine che rende l'inclusione $i:Y\to X$ continua.\\
La definizione è ben posta in quanto la topologia discreta rende l'inclusione continua e l'intersezione di topologie è sempre una topologia.
\end{definition}
\begin{proposition}[Caratterizzazione della topologia di sottospazio]
La topologia di sottospazio di $Y$ è $\tau\res Y=\{B\subseteq Y\mid \exists A\in \tau_X\ t.c.\ B=Y\cap A\}$, cioè sono le intersezioni di aperti globali con l'insieme.
\end{proposition}

\noindent
Se non specifichiamo altrimenti considereremo ogni sottospazio dotato della topologia di sottospazio.

\begin{remark}
Dalla definizione segue che i chiusi della topologia di sottospazio sono della forma $Y\cap C$ con $C$ chiuso in $X$.
\end{remark}
\begin{remark}
Se $\Bc$ è una base di $\tau$, $\Bc'=\{A\cap Y\mid A\in \Bc\}$ è una base di $\tau\res Y$.
\end{remark}
\begin{proposition}[Aperto di un aperto e Chiuso di un chiuso]
$Y$ è aperto in $X$ se e solo se tutti gli aperti in $Y$ sono aperti in $X$. $Y$ è chiuso in $X$ se e solo se ogni chiuso in $Y$ è chiuso in $X$.
\end{proposition}

\begin{proposition}[Proprietà universale della topologia di sottospazio]
Siano $X,Z$ spazi topologici e sia $Y\subseteq Z$. Data una mappa $f:X\to Y$ e chiamando $i:Y\to Z$, $f$ è continua se e solo se  $i\circ f:X\to Z$ è continua.
\end{proposition}

\begin{proposition}[Restrizione di continua è continua]
Sia $f:X\to Y$ continua, $Z\subseteq X$. Allora $f\res Z:Z\to Y$ è continua
\end{proposition}

\begin{lemma}[Chiusura in sottospazi]\label{ChiusuraInSottospazi}
Siano $X$ uno spazio topologico e $A\subseteq Z\subseteq X$. Allora la chiusura di $A$ in $Z$ (che indicheremo con $\ol{A}^Z$) coincide con $\ol A\cap Z$, dove $\ol A$ è la chiusura di $A$ in $X$.
\end{lemma}
\begin{remark}
Questo è falsissimo per le parti interne, per esempio $\{0\}\subseteq \{0\}\subseteq \R$. Chiaramente $int_{\{0\}}(\{0\})=\{0\}$ ma $int_\R(\{0\})\cap \{0\}=\emptyset\cap \{0\}=\emptyset$.
\end{remark}
\vspace{0.5cm}

\noindent
Vediamo quali proprietà di numerabilità passano a sottospazi:
\begin{proposition}
Sia $X$ uno spazio topologico e $Y\subseteq X$.
\begin{enumerate}[noitemsep]
\item Se $X$ è II-numerabile allora $Y$ è II-numerabile
\item Se $X$ è I-numerabile allora $Y$ è I-numerabile
\item Se $X$ è separabile NON sempre $Y$ è separabile
\item Se $X$ è metrizzabile allora $Y$ è metrizzabile e la topologia di sottospazio è la topologia indotta dalla metrica di $X$ ristretta a $Y$
\item Se $X$ è metrizzabile e separabile allora $Y$ è separabile
\end{enumerate}
\end{proposition}


\section{Mappe aperte e chiuse}
\begin{definition}[Mappe aperte e chiuse]
Sia $f:X\to Y$ tra spazi topologici. $f$ è \textbf{aperta} se $f(A)$ è aperto in $Y$ per ogni $A$ aperto in $X$. Analogamente $f$ è \textbf{chiusa} se per ogni chiuso $C$ di $X$, $f(C)$ è chiuso in $Y$.
\end{definition}

\begin{lemma}[Funzione aperta se e solo se aperta su base]\label{FunzioneAPertaSSEApertaSuBase}
Data $f:X\to Y$ e $\Bc$ una base di $X$, $f$ è aperta se e solo se $f(B)$ è aperto in $Y$ per ogni $B\in \Bc$.
\end{lemma}


\begin{definition}[Immersione topologica]
Una funzione continua $f:X\to Y$ è una \textbf{immersione topologica} se è un omeomorfismo tra $X$ e $f(X)$.
\end{definition}

\begin{remark}
Ricordiamo che se $f:X\to Y$ è continua e bigettiva, $f$ non è sempre un omeomorfismo.
\end{remark}
\begin{remark}
Se $f\ii$ è continua allora per $A$ aperto in $X$, $(f\ii)\ii(A)=f(A)$ è aperto in $Y$. Analogamente per i chiusi.
\end{remark}

\begin{proposition}[Caratterizzazione delle immersioni topologiche in aperti / chiusi]\label{CaratterizzazioneImmersioniTopologicheInApertiOChiusi}
Se $f:X\to Y$ è continua e iniettiva allora
\begin{itemize}[noitemsep]
\item $f$ chiusa $\coimplies$ $f$ immersione topologica in un chiuso di $Y$
\item $f$ aperta $\coimplies$ $f$ immersione topologica in un aperto di $Y$
\end{itemize}
\end{proposition}
\begin{remark}
Esistono immersioni topologiche né aperte né chiuse, per esempio $i:[0,1)\to \R$.
\end{remark}


\section{Prodotti}
Consideriamo ora un modo per costruire spazi a partire da altri spazi.
\begin{definition}[Prodotto cartesiano]
Data $\{X_i\}_{i\in I}$ una famiglia di insiemi, il loro \textbf{prodotto (cartesiano)} è dato da
\[\prod_{i\in I}X_i=\left\{f:I\to\bigcup_{i\in I}X_i\mid \forall i\in I,\ f(i)\in X_i\right\}.\]
L'elemento $f\in \prod_{i\in I}X_i$ viene spesso denotato con $(f(i))_{i\in I}$, cioè una stringa di elementi $(x_i)_{i\in I}$ con $x_i\in X_i$ per ogni $i\in I$.\\
Chiamiamo $x_i=f(i)$ la \textbf{coordinata $i$-esima} di $f=(x_i)_{i\in I}$.\\
Sul prodotto $\prod_{i\in I}X_i$ è definita la \textbf{proiezione $i$-esima} per ogni $i\in I$, ovvero è definita la mappa
\[\pi_i:\funcDef{\displaystyle \prod_{j\in I}X_j}{X_i}{f}{f(i)}\]
o equivalentemente $\pi_i((x_j)_{j\in I})=x_i$.
\end{definition}

\begin{definition}[Diagonale]
La \textbf{diagonale} di $X$ è il seguente sottoinsieme di $X\times X$
\[\Delta_X=\{(x,x)\mid x\in X\}.\]
\end{definition}

Se gli $X_i$ sono spazi topologici, vorremmo definire una topologia sul prodotto:
\begin{definition}[Topologia prodotto]
Siano $(X_i,\tau_i)$ spazi topologici per ogni $i\in I$. La \textbf{topologia prodotto} su $\prod_{i\in I}X_i$ è la topologia meno fine che rende ogni proiezione I-esima continua.
\end{definition}
\begin{proposition}[Caratterizzazione della topologia prodotto]
La topologia prodotto $\tau$ di $\prod_{i\in I}X_i$ è ben definita ed ammette come prebase l'insieme
\[\{\pi_i\ii(A)\mid A\in \tau_i,i\in I\}=\{\{(x_j)_{j\in I}\mid x_i\in A\}\mid A\in \tau_i,i\in I\}\]
\end{proposition}
\begin{corollary}
Una base della topologia prodotto $\tau$ è data da
\[\left\{\bigcap_{j=1}^k\pi_{i_j}\ii(A_j)\mid k\in \N,\ i_1,\cdots,i_k\in I,\ A_j\in \tau_{i_j}\ \forall j\in I\right\}.\]
Inoltre se $\Bc_j$ è una base di $\tau_j$ allora una base di $\tau$ è data da
\[\left\{\bigcap_{j=1}^k\pi_{i_j}\ii(A_j)\mid k\in \N,\ i_1,\cdots,i_k\in I,\ A_j\in \Bc_j\ \forall j\in I\right\}.\]
\end{corollary}

\begin{remark}
Se $I$ è finito possiamo semplificare la scrittura della base di $\tau$ in
\[\Bc=\{A_1\times \cdots\times A_{\#I}\mid A_i\in \tau_i\}\]
Infatti $\bigcap_{i=1}^{\#I}\pi_{i}\ii(A_i)=A_1\times \cdots \times A_{\# I}$.\smallskip

\noindent
Se invece $I$ è infinito vediamo che un generico aperto nella base di $\tau$ standard descritta sopra è dato da $\prod_{i\in I}A_i$ con $A_i\in \tau_i$ ma $A_i=X_i$ eccetto che per un numero finito di entrate.
\end{remark}

\begin{definition}[Box topology]
La topologia su $\prod_{i\in I}X_i$ data da $\{\prod_{i\in I}A_i\mid A_i\in \tau_i\}$ è detta \textbf{box topology} sul prodotto. Se $I$ è finito la box topology e la topologia prodotto coincidono.
\end{definition}

\begin{proposition}[Prodotto di chiusi è chiuso]\label{ProdottoDiChiusiEChiuso}
Se $C_i\subseteq X_i$ sono chiusi allora $\prod_{i\in I}C_i$ è chiuso in $\prod_{i\in I}X_i$.
\end{proposition}

\begin{proposition}[Prodotto finito di metrici è metrico]
Dati $(X,d_X),\ (Y,d_Y)$ spazi metrici con topologie indotte $\tau_X,\tau_Y$, la topologia prodotto su $X\times Y$ (che denotiamo $\tau_X\times \tau_Y$) coincide con la topologia indotta su $X\times Y$ da
\[d_\infty((x,y),(x',y'))=\max\{d_X(x,x'),d_Y(y,y')\}.\]
In particolare il prodotto finito di spazi metrizzabili è metrizzabile.
\end{proposition}
\begin{remark}
Essendo le metriche $p$ equivalenti su $\R^n$\footnote{Lo mostreremo nel capitolo sulla compattezza. (\ref{EquivalenzaNormeRn})} (e in particolare $\R^2$) il risultato vale anche per $d_2$ e $d_1$. Segue immediatamente, per esempio, che $\R^n\times \R^m\cong \R^{n+m}$
\end{remark}

\begin{proposition}[Prodotto numerabile di metrici è metrico]\label{ProdottoNumerabileDiMetriciEMetrico}
Sia $\{X_i\}_{i\in \N}$ una famiglia numerabile di spazi metrici, allora
\[X=\prod_{i\in\N}X_i\]
è metrico.
\end{proposition}



\begin{remark}
Il prodotto più che numerabile di spazi metrici può non essere metrico (\ref{ProdottoDiINumerabileNonINumerabile}).
\end{remark}

\subsection{Proiezioni da un prodotto in un fattore}

\begin{theorem}[Proprietà universale del prodotto]
Se $X_i$ sono spazi topologici per ogni $i\in I$ e $Y$ è un altro spazio topologico, data $f:Y\to \prod_{i\in I}X_i$ si ha che
\[f\ \text{continua}\coimplies \pi_i\circ f:Y\to X_i\text{ è continua }\forall i\in I.\]
\end{theorem}

\begin{theorem}[Le proiezioni sono aperte]\label{ProiezioniSonoAperte}
Se $X_i$ sono spazi topologici per ogni $i\in I$, le proiezioni $\pi_i$ sono aperte per ogni $i\in I$.
\end{theorem}
\begin{remark}
Le proiezioni non sono sempre chiuse (\ref{ProiezioniNonSempreChiuse}).
\end{remark}

\subsection{Immersioni dei fattori nel prodotto}
\begin{proposition}[Immersioni dei fattori nei prodotti]\label{ImmersioniFattoriInProdottiSonoImmersioneTopologica}
Fissiamo $k\in I$ un indice e $x_i\in X_i$ per ogni $i\neq k$. Allora la funzione
\[j:\funcDef{X_k}{\displaystyle \prod_{i\in I}X_i}{x}{(y_i)_{i\in I}},\quad y_i=
\begin{cases}
x_i & se\ i\neq k\\
x & se\ i=k
\end{cases}\]
è una immersione topologica.
\end{proposition}

Con la stessa dimostrazione troviamo il seguente fatto più generale
\begin{proposition}
Se $I'\subseteq I$ e fissiamo $x_i\in X_i$ per $i\notin I'$ allora
\[j:\funcDef{\displaystyle \prod_{i\in I'}X_i}{\displaystyle \prod_{i\in I}X_i}{(x_h)_{h\in I'}}{(y_i)_{i\in I}},\quad y_i=\begin{cases}
x_i &se\ i\notin I'\\
x_h & se\ i=h\in I'
\end{cases}\]
è un'immersione topologica.
\end{proposition}

\subsection{Topologia della convergenza puntuale}
\begin{definition}
Sia $Y$ uno spazio topologico e $X$ un insieme. Definiamo
\[\{f:X\to Y\}=\prod_{x\in X}Y=Y^X\]
e la topologia prodotto su questo spazio è detta \textbf{topologia della convergenza puntuale}.
\end{definition}
\begin{proposition}
Sia $\{f_n\}_{n\in\N}$ una successione di funzioni in $Y^X$, allora $f_n\to f$ in questa topologia se e solo se per ogni $x\in X$ si ha $f_n(x)\to f(x)$ in $Y$.
\end{proposition}

\section{Assiomi di separazione}
Consideriamo adesso diversi modi in cui i punti del nostro spazio possono essere distinti gli uni dagli altri. Più assiomi di separazione vengono rispettati, più modi abbiamo per scegliere intorni dei nostri punti. Se non vengono rispettati degli assiomi di separazione lo spazio ha un aspetto più spigoloso o appiccicoso (ci sono punti vicini a tanti altri), mentre più assiomi di separazione vengono rispettati, più lo spazio topologico comincia a diventare simile agli spazi ai quali siamo comunemente abituati (per esempio, tutti gli spazi metrici sono almeno spazi di Hausdorff).
\begin{definition}[Assiomi di separazione]
Sia $X$ uno spazio topologico. Affermiamo che $X$ soddisfa l'assioma
\begin{itemize}[noitemsep]
\item $T_0$ se $\forall x\neq y\in X$ esiste $U$ aperto in $X$ tale che $x\in U$ e $y\notin U$ o viceversa.
\item $T_1$ se $\forall x\neq y\in X$ esistono $U,V$ aperti in $X$ tali che $x\in U$, $y\notin U$ e $x\notin V$, $y\in V$.
\item $T_2$ se $\forall x\neq y\in X$ esistono $U,V$ aperti in $X$ tali che $x\in U,\ y\in V$ e $U\cap V=\emptyset$.
\end{itemize}
Se $X$ soddisfa $T_2$ viene detto \textbf{spazio di Hausdorff}.
\end{definition}
\begin{remark}
Definizioni analoghe ed equivalenti si ottengono considerando intorni al posto di aperti.
\end{remark}
\begin{remark}
$T_2\implies T_1\implies T_0$ e queste sono implicazioni strette (\ref{T1NonT2-T0NonT1}).
\end{remark}
\begin{remark}
Non tutti gli spazi topologici sono $T_0$, per esempio se $\#X\geq 2$ allora la topologia indiscreta su $X$ non è $T_0$.
\end{remark}

\begin{proposition}[Gli spazi metrici sono Hausdorff]\label{MetriciSonoHausdorff}
Se $(X,d)$ è uno spazio metrico allora la topologia indotta da $d$ su $X$ è $T_2$.
\end{proposition}
\begin{proposition}[Caratterizzazione degli spazi $T_1$]\label{CaratterizzazioneT1}
Uno spazio topologico $X$ è $T_1$ se e solo se i punti sono chiusi, che succede se e solo se la topologia di $X$ è più fine della topologia cofinita.
\end{proposition}

\begin{proposition}[Caratterizzazione degli spazi $T_2$]\label{CaratterizzazioneT2}
$X$ è di Hausdorff se e solo se la diagonale $\Delta_X=\{(x,x)\mid x\in X\}$ è un chiuso di $X\times X$.
\end{proposition}
\begin{corollary}
Sia $f:X\to Y$ continua con $Y$ di Hausdorff, allora il grafico di $f$ ($\Gamma_f=\{(x,f(x))\mid x\in X\}\subseteq X\times Y$) è un chiuso.
\end{corollary}
\begin{corollary}
Siano $f,g:X\to Y$ continue e $Y$ di Hausdorff, allora $\{x\in X\mid f(x)=g(x)\}\subseteq X$ è un chiuso.
\end{corollary}
\begin{corollary}[Funzioni continue concordanti su un denso in $T_2$ coincidono]
Siano $f,g:X\to Y$ continue e $Y$ di Hausdorff, allora se $f=g$ su un denso di $X$, $f=g$ come funzioni da $X\to Y$, infatti un chiuso che contiene un denso è tutto lo spazio.
\end{corollary}
\begin{corollary}
Se $f:X\to X$ è continua e $X$ è $T_2$ allora $Fix(f)$ è un chiuso di $X$.
\end{corollary}

\begin{theorem}[Unicità del limite per Hausdorff]\label{LimiteUnicoSeHausdorff}
Se $X$ è $T_2$, $x_n$ è una successione a valori in $X$ e $x_n\to x$, $x_n\to y$ con $x,y\in X$, allora $x=y$.
\end{theorem}
\begin{proposition}[Primi assiomi di separazione sono stabili per sottospazi, prodotti e raffinamenti]
Per $i=0,1,2$ vediamo che
\begin{enumerate}[noitemsep]
\item Sottospazi di spazi $T_i$ sono $T_i$
\item Prodotti di spazi $T_i$ sono $T_i$
\item Raffinamenti di topologie $T_i$ sono $T_i$.
\end{enumerate}
\end{proposition}
\noindent Le proprietà sopra ci informano che $T_0,\ T_1$ e $T_2$ sono in un certo concetti di separazione fondamentali e stabili. Questi assiomi ci permettono di distinguere punti, ma non dicono niente sugli insiemi, è quindi possibile per esempio non trovare aperti che separano punti da chiusi disgiunti. Seguono allora i seguenti

\begin{definition}[Assiomi di separazione 3 e 4]
Uno spazio topologico $X$ si dice
\begin{itemize}[noitemsep]
\item $T_3$ se $\forall x\in X$ e per ogni $C$ chiuso in $X$ tale che $x\notin C$ esistono $U,V$ aperti in $X$ tali che $x\in U,\ C\subseteq V$ e $U\cap V=\emptyset$.
\item $T_4$ se per ogni $C,D$ chiusi in $X$ tali che $C\cap D=\emptyset$ esistono $U,V$ aperti in $X$ tali che $C\subseteq U,\ D\subseteq V$ e $U\cap V=\emptyset$.
\end{itemize}
\end{definition}
\begin{remark}
Se $X$ è $T_1$ allora $T_4\implies T_3\implies T_2$.
\end{remark}
\begin{remark}
Se $X$ non è $T_1$ allora è possibile che $X$ sia $T_4$ o $T_3$ senza essere $T_2,\ T_1$ o $T_0$ (\ref{SpazioT4NonT0}).
\end{remark}

\begin{definition}[Regolari e Normali]
Affermiamo che uno spazio è \textbf{regolare} se è $T_1$ e $T_3$. Affermiamo che uno spazio è \textbf{normale} se è $T_1$ e $T_4$.
\end{definition}
\begin{remark}
Normale $\implies$ Regolare $\implies$ Hausdorff e le implicazioni sono strette (per esempio il piano di Sorgenfrey (\ref{PianoDiSorgenfrey}) è regolare ma non normale e l'esempio (\ref{EsempioT2NonRegolare}) mostra uno spazio Hausdorff non regolare).
\end{remark}

\noindent Mostriamo che gli spazi metrici sono normali. Per fare ciò abbiamo bisogno di un paio di lemmi:
\begin{lemma}
La distanza punto-insieme è $1-$Lipschitziana, in particolare è continua.
\end{lemma}
\begin{lemma}[La chiusura in un metrico sono i punti a distanza nulla]
Se $X$ è uno spazio metrico e $A\subseteq X$ allora $d_A\ii(0)=\ol A$. In particolare se $C$ è un chiuso allora $C=d_C\ii(0)$.
\end{lemma}

\begin{proposition}[Spazi metrici sono normali]
Se $(X,d)$ è uno spazio metrico allora $X$ con la topologia indotta da $d$ è normale.
\end{proposition}
\noindent Nella dimostrazione sopra abbiamo ricavato il seguente risultato nel caso di spazi metrici
\begin{proposition}[Lemma di Urysohn]
Se $X$ è $T_4$, dati $C,D$ chiusi disgiunti esiste $f:X\to[0,1]$ continua tale che
\[f\ii(0)=D,\qquad f\ii(1)=C.\]
\end{proposition}
\vspace{0.5cm}

\noindent Consideriamo ora alcune proprietà degli spazi $T_3$ e $T_4$
%\noindent L'ereditarietà che avevamo di $T_0,T_1,T_2$ per sottospazi, prodotti e raffinamenti non vale per $T_3$ e $T_4$ in generale (il piano di Sorgenfrey \ref{PianoDiSorgenfrey} mostra in un colpo solo che il prodotto di $T_4$ non è $T_4$, il prodotto di normali non è normale e raffinamenti di $T_4$ non sono $T_4$). Abbiamo però le seguenti proprietà:
\begin{proposition}[Ereditarietà per sottospazi di $T_3$ e $T_4$]
Valgono le seguenti proprietà:
\begin{enumerate}[noitemsep]
\item Sottospazi di spazi $T_3$ sono $T_3$
\item Sottospazi chiusi di spazi $T_4$ sono $T_4$
\end{enumerate}
\end{proposition}
\begin{remark}
Sottospazi non chiusi di uno spazio $T_4$ possono non essere $T_4$.
\end{remark}

\begin{proposition}[Caratterizzazione di $T_3$ con intorni]\label{CaratterizzazioneT3ConIntorni}
Uno spazio topologico $X$ è $T_3$ se e solo se gli intorni chiusi dei punti formano sistemi fondamentali di intorni.
\end{proposition}

\begin{proposition}[Prodotti di $T_3$ sono $T_3$]
Dato un insieme di indici $I$, se $X_i$ è uno spazio $T_3$ per ogni $i\in I$ allora
\[\prod_{i\in I}X_i\text{ è uno spazio }T_3.\]
\end{proposition}
\begin{corollary}
Il prodotto di spazi regolari è regolare.
\end{corollary}

\begin{remark}
Prodotti di spazi $T_4$ non sono sempre $T_4$ (\ref{PianoDiSorgenfrey}). Lo stesso esempio mostra che prodotto di spazi normali non sempre è normale.
\end{remark}

\begin{remark}
Raffinamenti di topologie $T_3$ o $T_4$ possono non essere $T_3$ o $T_4$ (un esempio con $T_4$ è il piano di Sorgenfrey (\ref{PianoDiSorgenfrey})). Anche se aggiungere nuovi aperti non influisce su come potevamo distinguere i punti e i chiusi già presenti, raffinare introduce nuovi chiusi nella topologia e non è detto che si presentino gli aperti necessari per separarli.
\end{remark}



\section{Ricoprimenti fondamentali}
Cominciamo ora ad approfondire i ricoprimenti. Spesso vogliamo studiare proprietà di uno spazio ricostruendole a partire da proprietà locali e i ricoprimenti, specialmente i ricoprimenti aperti, sono uno dei modi più versatili di farlo. In questa sezione introduciamo i tipi di ricoprimenti più utili.
\begin{definition}[Ricoprimento]
Sia $X$ uno spazio topologico, un \textbf{ricoprimento} di $X$ è una famiglia $\{B_i\}_{i\in I}$ di sottoinsiemi di $X$ tale che \[X=\bigcup_{i\in I}B_i.\]
Un ricoprimento è \textbf{aperto} (rispettivamente \textbf{chiuso}) se ogni $B_i$ è aperto (rispettivamente chiuso).
\end{definition}
\begin{definition}[Ricoprimento fondamentale]
Un ricoprimento $\{B_i\}_{i\in I}$ di $X$ è \textbf{fondamentale} se $\forall A\subseteq X$, $A$ è aperto se e solo se $A\cap B_i$ è aperto in $B_i$ per ogni $i\in I$ (o equivalentemente $A$ è chiuso se e solo se $A\cap B_i$ è chiuso in $B_i$ per ogni $i\in I$).
\end{definition}
\begin{remark}
Osserviamo che se $A$ è aperto/chiuso allora $A\cap B_i$ è aperto/chiuso in $B_i$ per definizione di topologia di sottospazio, quindi l'unica implicazione rilevante è l'altra.
\end{remark}

\begin{theorem}[I ricoprimenti aperti sono fondamentali]\label{RicoprimentiApertiSonoFondamentali}
Ogni ricoprimento aperto è fondamentale.
\end{theorem}

\begin{remark}
I ricoprimenti chiusi non sono sempre fondamentali (\ref{RicoprimentoChiusoNonFondamentale})
\end{remark}

\begin{theorem}[Incollamento delle funzioni]\label{IncollamentoDelleFunzioniRicoprimentoFondamentale}
Sia $\{B_i\}_{i\in I}$ un ricoprimento fondamentale di $X$ e sia $f:X\to Y$ una funzione, allora
\[f\text{ continua }\coimplies f\res {B_i}:B_i\to Y \text{ continua }\forall i\in I\]
\end{theorem}

\begin{remark}[Funzioni definite a tratti]
Sia $X=A\cup B$ e definiamo $f:X\to Y$ come
\[f(x)=\begin{cases}
g(x) &se\ x\in A\\
h(x) &se\ x\in B
\end{cases},\quad \text{con }g:A\to Y,\ h:B\to Y.\]
Se $X=A\sqcup B$ ci sono poche speranze che il ricoprimento sia fondamentale (se lo fosse vedremo che questo implica che lo spazio è sconnesso), quindi per garantire una buona definizione poniamo che $g\res{A\cap B}=h\res{A\cap B}$. Vediamo che in questo caso, $g,h$ continue e $\{A,B\}$ fondamentale ci permette di concludere che $f$ è continua.
\end{remark}
\begin{remark}
Il metodo appena illustrato per definire funzioni continue definite a tratti è molto più efficiente e stabile rispetto al confrontare i limiti direzionali come si era abituati a fare dai corsi di analisi. In spazi topologici astratti quei limiti perdono quasi ogni significato (ricordiamo per esempio che il limite in spazi non $T_2$ può non essere unico (\ref{LimiteUnicoSeHausdorff})).
\end{remark}
\vspace{0.5cm}

\noindent
Usare solo aperti è un po' restrittivo, proviamo a costruire ricoprimenti fondamentali anche a partire da ricoprimenti chiusi. Vedremo che i ricoprimenti chiusi sono fondamentali se rispettano una particolare ipotesi di finitezza.

\begin{definition}[Famiglia localmente finita]
Sia $X$ uno spazio topologico e sia $\{B_i\}_{i\in I}$ una famiglia di suoi sottoinsiemi. Essa è \textbf{localmente finita} se $\forall x\in X$ esiste $U$ intorno di $x\in X$ tale che \[|\{i\in I\mid U\cap B_i\neq\emptyset\}|\in \N.\]
\end{definition}

\begin{lemma}[Chiusura e unione finita commutano]
Sia $X$ uno spazio topologico e siano $C_1,\cdots, C_k$ sottoinsiemi di $X$, allora
\[\ol{\bigcup_{i=1}^kC_i}=\bigcup_{i=1}^k\ol {C_i}.\]
\end{lemma}
\begin{remark}
Osserviamo che $\bigcup_{j\in I}\ol{C_j}\subseteq \ol{\bigcup_{i\in I}C_i}$ vale per unioni arbitrarie.
\end{remark}

\begin{lemma}[Chiusura e unione localmente finita commutano]\label{ChiusuraEUnioneLocalmenteFinitaCommutano}
Sia $\{C_i\}_{i\in I}$ una famiglia localmente finita di sottoinsiemi di $X$, allora
\[\ol{\bigcup_{i\in I}C_i}=\bigcup_{i\in I}\ol {C_i}.\]
\end{lemma}

\begin{corollary}
Un'unione localmente finita di chiusi è chiusa.
\end{corollary}

\begin{theorem}\label{RicoprimentoChiusoLocalmenteFinitoEFondamentale}
Un ricoprimento chiuso localmente finito è fondamentale.
\end{theorem}
\begin{corollary}
Un ricoprimento chiuso finito è fondamentale.
\end{corollary}

\section{Spazi connessi}

\begin{definition}[Connessione]
Uno spazio topologico è \textbf{connesso} se vale una delle seguenti condizioni equivalenti:
\begin{enumerate}[noitemsep]
\item $X$ non ammette partizione in aperti non banali, ovvero se $X=A\cup B,\ A\cap B=\emptyset$ con $A,B$ aperti allora $A=\emptyset$ o $B=\emptyset$.
\item $X$ non ammette partizione in chiusi non banali.
\item Se $A\subseteq X$ è sia aperto che chiuso allora $A=\emptyset$ o $A=X$.
\end{enumerate}
Uno spazio $X$ è \textbf{sconnesso} se non è connesso, ovvero ammette partizione aperta/chiusa non banale.
\end{definition}
\begin{remark}
Chiaramente $1$ e $2$ sono equivalenti dato che $A$ e $B$ sono complementari. Inoltre $1$ e $3$ sono equivalenti in quanto se $A$ è sia aperto che chiuso, $X\bs A$ è sia chiuso che aperto.
\end{remark}

\begin{theorem}
L'intervallo $[0,1]$ è connesso.
\end{theorem}

\noindent Mostreremo tra poco che le funzioni continue mandano spazi connessi in spazi connessi. Siamo quindi giustificati dall'intuizione visiva a dare la seguente definizione:
\begin{definition}[Cammino]
Sia $X$ uno spazio topologico. Un \textbf{cammino} in $X$ è una funzione continua $\gamma:[0,1]\to X$\footnote{Alcuni definiscono un cammino come una continua $\gamma:I\to X$ dove $I$ è un qualsiasi intervallo chiuso di $\R$}.
\end{definition}

\begin{definition}[Giunzione]
Siano $\gamma_1,\gamma_2:[0,1]\to X$ due cammini tali che $\gamma_1(1)=\gamma_2(0)$. Definiamo la loro \textbf{giunzione} come la mappa
\[\gamma:[0,1]\to X,\quad \gamma(t)=\begin{cases}
\gamma_1(2t) &se\ t\in [0,1/2]\\
\gamma_2(2t-1) &se\ t\in [1/2,1]
\end{cases}.\]
Di solito indichiamo la giunzione di $\gamma_1$ e $\gamma_2$ con $\gamma_1\ast \gamma_2$.
\end{definition}
\begin{remark}
$\gamma_1\ast \gamma_2$ è un cammino, infatti $\gamma_1(2\cdot 1/2)=\gamma_1(1)=\gamma_2(0)=\gamma_2((2\cdot 1/2) -1)$, $\{[0,1/2],[1/2,1]\}$ è un ricoprimento fondamentale di $[0,1]$ e $\gamma_1(2t):[0,1/2]\to X,\ \gamma_2(2t-1):[1/2,1]\to X$ sono continue in quanto composizioni di continue.
\end{remark}
\begin{remark}
Se non avessimo avuto i ricoprimenti fondamentali a disposizione, giustificare la continuità di $\gamma_1\ast \gamma_2$ in $1/2$ sarebbe stato quasi impossibile con i limiti $t\to 1/2^+,\ t\to 1/2^-$ in quanto $X$ potrebbe non essere $T_2$ (no unicità del limite (\ref{LimiteUnicoSeHausdorff})) o potrebbe non essere I-numerabile (continuità e continuità per successioni potrebbero non coincidere (\ref{ChiusiApertiContinuePerSuccessioni})).
\end{remark}

\begin{definition}[Connessione per archi]
Uno spazio topologico $X$ è \textbf{connesso per archi} se $\forall x_0,x_1\in X$ esiste un cammino $\gamma:[0,1]\to X$ tale che $\gamma(0)=x_0$ e $\gamma(1)=x_1$.
\end{definition}

\begin{theorem}[Spazio connesso per archi è connesso]\label{ConnessoPerArchiImplicaConnesso}
Se $X$ è connesso per archi allora $X$ è connesso
\end{theorem}

\begin{remark}
La definizione di connessione per archi è subordinata al fatto che $[0,1]$ è connesso e in questa dimostrazione lo abbiamo usato. Un ragionamento del tipo ``$[0,1]$ è connesso perché è connesso per archi" sarebbe insensato.
\end{remark}

\begin{definition}[Insieme convesso]
Un sottoinsieme $C$ di $\R^n$ è \textbf{convesso} se $\forall p,q\in C$, il segmento che li congiunge è tutto contenuto in $C$, ovvero
\[\{tp+(1-t)q\mid t\in [0,1]\}\subseteq C\quad \forall p,q\in C.\]
L'espressione $tp+(1-t)q$ con $t\in [0,1]$ è detta una \textbf{combinazione convessa} di $p$ e $q$.
\end{definition}
\begin{remark}
Uno spazio convesso è connesso per archi. Basta prendere come cammino il segmento che congiunge i punti.
\end{remark}

\begin{definition}[Intervallo]
Gli \textbf{intervalli} sono i convessi di $\R$.
\end{definition}

\begin{theorem}[Connessi su $\R$]\label{ConnessoConnessoPerArchiEConvessoCoincidonoSuR}
Sia $C\subseteq \R$. Le seguenti affermazioni sono equivalenti:
\begin{enumerate}[noitemsep]
\item $C$ è connesso.
\item $C$ è connesso per archi.
\item $C$ è convesso.
\end{enumerate}
\end{theorem}

\begin{proposition}[Se un denso è connesso, lo spazio è connesso]\label{SeDensoConnessoSpazioConnesso}
Siano $Z,Y$ sottospazi di $X$ tali che $Z\subseteq Y\subseteq \ol Z$. Se $Z$ è connesso allora anche $Y$ è connesso.
\end{proposition}
\begin{corollary}
Se $Z$ è connesso, $\ol Z$ è connesso.
\end{corollary}

\begin{proposition}[Continue preservano connessione]\label{ContinuePreservanoConnessione}
Sia $f:X\to Y$ continua. Allora abbiamo che $f$ manda connessi in connessi e connessi per archi in connessi per archi.
\end{proposition}

\begin{remark}
Uno spazio connesso non è necessariamente connesso per archi (\ref{ConnessoNonConnessoPerArchi}).
\end{remark}

\begin{theorem}[Prodotto finito di connessi è connesso]\label{ProdottoFinitoDiConnessiEConnesso}
Se $X$ e $Y$ sono connessi allora $X\times Y$ è connesso.
\end{theorem}

\begin{theorem}[Prodotto finito di connessi per archi è connesso per archi]\label{ProdottoFinitoDiConnessiPerArchiEConnessoPerArchi}
Se $X$ e $Y$ sono connessi per archi allora $X\times Y$ è connesso per archi.
\end{theorem}


\subsection{Componenti connesse}
\begin{proposition}[Unione di connessi che si intersecano è connessa]\label{UnioneDiConnessiCheSiIntersecanoEConnessa}
Dati $Y_i$ connessi tali che $\emptyset\neq \bigcap Y_i$, si ha che $Y=\bigcup_{i\in I}Y_i$ è connesso.
\end{proposition}

\begin{definition}[Componente connessa]
Sia $x_0\in X$, definiamo la \textbf{componente connessa} di $x_0$ ($C(x_0)$) come il più grande sottoinsieme connesso di $X$ che contiene $x_0$, cioè
\[C(x_0)=\bigcup_{x_0\in C,\ C\text{ conn.}}C.\]
La buona definizione segue dal fatto che $\{x_0\}$ è connesso e dalla proposizione precedente.
\end{definition}

\begin{proposition}\label{LeComponentiConnesseSonoChiuse}
Ogni componente connessa è chiusa
\end{proposition}

\begin{proposition}
Le componenti connesse danno una partizione di $X$.
\end{proposition}
\begin{corollary}
Se $X$ ha un numero finito di componenti connesse allora ognuna di queste è sia aperta che chiusa.
\end{corollary}

\begin{remark}
In generale le componenti connesse non sono aperte. (\ref{PartiConnesseAperte})
\end{remark}

\begin{definition}[Componenti connesse per archi]
Consideriamo la seguente relazione di equivalenza su $X$:
\[x_0\sim x_1\coimplies \text{ esiste un cammino da $x_0$ a $x_1$}.\]
La \textbf{componente connessa per archi} di $x_0$ (che denotiamo $A(x_0)$) è la classe di equivalenza di $x_0$ rispetto alla relazione.
\end{definition}

\begin{proposition}[Caratterizzazione delle componenti connesse per archi]\label{CaratterizzazioneDelleComponentiConnessePerArchi}
$A(x_0)$ è il più grande connesso per archi che contiene $x_0$.
\end{proposition}

\begin{remark}
Le componenti connesse per archi in generale non sono né aperte né chiuse. (\ref{PartiConnessePerArchiNeAperteNeChiuse})
\end{remark}

\begin{remark}
Poiché connesso per archi implica connesso, si ha che $A(x_0)\subseteq C(x_0)$.
\end{remark}

\begin{definition}[Zero-esimo gruppo di omotopia]
Dato $X$ spazio topologico definiamo lo \textbf{$0-$esimo gruppo di omotopia} come
\[\pi_0(X)=\{\text{componenti connesse per archi di }X\}.\]
\end{definition}

\begin{remark}
Se $f:X\to Y$ è continua, $f$ induce una funzione di insiemi tra $\pi_0(X)$ e $\pi_0(Y)$ (se $x$ e $y$ sono connessi da un cammino $\gamma$, $f(x)$ e $f(y)$ sono connessi dal cammino $f\circ \gamma$).
\end{remark}
\begin{remark}
$\pi_0(id_X)=id_{\pi_0(X)}$
e le mappe indotte preservano la composizione di funzioni continue.
\end{remark}

Possiamo riassumere quanto detto in
\begin{fact}\label{Pi0EFuntoreDaTopASet}
$\pi_0$ è un funtore covariante da $Top$ a $Set$
\end{fact}
\begin{remark}
Se $f:X\to Y$ è un omeomorfismo allora $\pi_0(f):\pi_0(X)\to\pi_0(Y)$ è una mappa biunivoca.
\end{remark}

\subsubsection{Locale connessione per archi}
\begin{definition}
$X$ è \textbf{localmente connesso (connesso per archi)} se ogni punto di $X$ ha un sistema fondamentale di intorni connessi  (connessi per archi).
\end{definition}
\begin{remark}
Esistono spazi connessi per archi che non sono localmente connessi per archi (\ref{PettineInfinito}).
\end{remark}

\begin{proposition}[Componenti connesse per archi in localmente connesso per archi sono aperte e chiuse]\label{ComponentiConnessePerArchiInLocalmenteConnessoPerArchiSonoAperteEChiuse}
Se $X$ è localmente connesso per archi allora le componenti connesse per archi sono sia aperte che chiuse.
\end{proposition}

\begin{theorem}[Connesso localmente connesso per archi è connesso per archi]\label{ConnessoLocalmenteConnessoPerArchiEConnessoPerArchi}
Se $X$ è connesso e  localmente connesso per archi allora è connesso per archi.
\end{theorem}

\begin{proposition}[Aperto in localmente connesso per archi è localmente connesso per archi]\label{ApertoInLocalmenteConnessoPerArchiEConnessoPerArchi}
Se $X$ è localmente connesso per archi allora ogni suo aperto è localmente connesso per archi.
\end{proposition}
\begin{corollary}[Aperto connesso in localmente connesso per archi è connesso per archi]
Se $X$ è localmente connesso per archi e $A\subseteq X$ è un aperto connesso allora $A$ è connesso per archi.
\end{corollary}

\begin{proposition}[Componenti connesse per archi di aperto in localmente connesso per archi sono aperte]\label{ComponentiConnessePerArchiDiApertoInLocalmenteConnessoPerArchiSonoAperte}
Le componenti connesse per archi di un aperto in uno spazio localmente connesso per archi sono aperte.
\end{proposition}
\begin{remark}
Se $X$ è localmente connesso per archi e $A\subseteq X$ è un aperto allora $A$ è connesso se e solo se è connesso per archi.
\end{remark}

\begin{remark}
Se $X$ è $Y$ sono omeomorfi allora hanno lo stesso numero di componenti connesse/connesse per archi.
\end{remark}


\section{Compattezza}
\begin{definition}[Spazio compatto]
Uno spazio topologico $X$ si dice \textbf{compatto} se ogni ricoprimento aperto di $X$ ammette un sottoricoprimento finito.
\end{definition}

\begin{theorem}[Alexander debole]
Se $\Bc$ è una base della topologia di $Z$ e da ogni ricoprimento di $Z$ costituito da aperti di base è possibile estrarre un sottoricoprimento finito allora $Z$ è compatto.
\end{theorem}

\begin{theorem}[Alexander]\label{TeoremaDiAlexander}
Se $X$ è uno spazio topologico e $\Dc$ è una sua prebase si ha che se da ogni ricoprimento di aperti in $\Dc$ si può estrarre un sottoricoprimento finito allora $X$ è compatto.
\end{theorem}

\begin{remark}
Se $X$ è finito allora è compatto dato che ammette un numero finito di aperti.
\end{remark}

\begin{remark}
Insiemi non compatti esistono (\ref{RnNonECompatto}).
\end{remark}

\begin{theorem}[Continue mandano compatti in compatti]\label{ContinuePreservanoCompattezza}
Sia $f:X\to Y$ continua. Se $X$ è compatto allora $f(X)$ è compatto.
\end{theorem}
\begin{corollary}
Se $X$ e $Y$ sono omeomorfi allora $X$ è compatto se e solo se $Y$ è compatto.
\end{corollary}

\begin{definition}[Proprietà dell'intersezione finita]
Una famiglia di sottoinsiemi $\{Y_i\}_{i\in I}$ di un insieme $X$ ha la \textbf{proprietà dell'intersezione finita} se per ogni $J\subseteq I$ finito si ha $\bigcap_{i\in J}Y_i\neq \emptyset$.
\end{definition}

\begin{remark}
Esistono famiglie di sottoinsiemi che godono della proprietà dell'intersezione finita (\ref{EsempioIntersezioneFinita})
\end{remark}

\begin{proposition}[Formulazione di compattezza con i chiusi]
Sia $X$ uno spazio topologico. Si ha che $X$ è compatto se e solo se per ogni famiglia $\{C_i\}_{i\in I}$ di chiusi che gode della proprietà dell'intersezione finita si ha che $\bigcap_{i\in I} C_i\neq \emptyset$.
\end{proposition}
\begin{corollary}\label{IntersezioneDiChiusiInscatolatiInCompatto}
Se $X$ è compatto e $C_n$ è una successione di chiusi non vuoti tali che $C_{n+1}\subseteq C_n$ allora $\bigcap_{n\in \N}C_n\neq \emptyset$.
\end{corollary}

\begin{remark}
Sia la compattezza dello spazio che la chiusura dei termini sono condizioni necessarie (\ref{EsempioIntersezioneFinita}).
\end{remark}

\subsection{Sottoinsiemi compatti}
\begin{remark}
$Y\subseteq X$ è compatto se e solo se per ogni $\{U_i\}_{i\in I}$ famiglia di aperti di $X$ tale che $Y\subseteq \bigcup_{i\in I}U_i$ esiste $J\subseteq I$ finito tale che $Y\subseteq \bigcup_{i\in J}U_i$.
\end{remark}

\begin{theorem}[Un chiuso di un compatto è compatto]\label{ChiusoInCompattoECompatto}
Se $X$ è compatto e $C\subseteq X$ è chiuso allora $C$ è compatto.
\end{theorem}

\begin{remark}
Sottoinsiemi compatti di uno spazio topologico non sono necessariamente chiusi (\ref{SottoinsiemeCompattoNonChiuso}), ma con il prossimo teorema vediamo che aggiungendo l'ipotesi \emph{Hausdorff} all'ipotesi di compattezza allora sottoinsiemi chiusi e sottoinsiemi compatti coincidono.
\end{remark}

\begin{theorem}[Compatti in Hausdorff sono chiusi]\label{CompattoInT2EChiuso}
Se $X$ è $T_2$ e $Y\subseteq X$ è compatto allora $Y$ è chiuso.
\end{theorem}
\noindent I compatti Hausdorff sono spazi molto interessanti, per esempio:

\begin{proposition}[Compatto Hausdorff è regolare]
Se $X$ è compatto e $T_2$ allora è regolare.
\end{proposition}

In realtà vale la condizione più forte

\begin{theorem}[Compatto Hausdorff è normale]\label{CompattoT2ENormale}
Se $X$ è compatto e $T_2$ allora $X$ è normale.
\end{theorem}

\begin{remark}
Per spazi compatti vale
\[\text{Normale $\coimplies$ Regolare $\coimplies$ Hausdorff},\]
mentre ricordiamo che per spazi generali le frecce vanno verso destra e sono implicazioni strette.
\end{remark}

\noindent L'assioma $T_2$ gioca bene anche con le mappe a dominio in un compatto. I seguenti risultati finiscono per essere tra i modi più comuni per costruire omeomorfismi a partire da spazi compatti:

\begin{theorem}[Continue da compatto a $T_2$ sono chiuse]\label{ContinueDaCompattoInT2SonoChiuse}
Se $X$ è compatto, $Y$ è $T_2$ e $f:X\to Y$ è continua allora $f$ è chiusa.
\end{theorem}
\begin{corollary}
Se $X$ è compatto, $Y$ è $T_2$ e $f:X\to Y$ è continua e bigettiva allora $f$ è un omeomorfismo.
\end{corollary}

\begin{definition}[Funzione propria]
Una funzione continua $f:X\to Y$ \`e \textbf{propria} se per ogni $K\subseteq Y$ compatto $f\ii(K)$ \`e compatto.
\end{definition}

\begin{proposition}[Proprie a immagine in loc.cpt $T_2$ sono chiuse]\label{ProprieInT2LocalmenteCompattoSonoChiuse}
Sia $f:X\to Y$ continua propria e supponiamo che $Y$ sia localmente compatto e Hausdorff. Allora $f$ \`e una mappa chiusa.
\end{proposition}

\subsection{Compattezza per prodotti}
\begin{theorem}[Tychonoff debole]
$X$ e $Y$ sono compatti se e solo se $X\times Y$ è compatto.
\end{theorem}

\begin{theorem}[Tychonoff]\label{TeoremaDiTychonoff}
Sia $I\neq \emptyset$ e sia $X_i$ uno spazio topologico per ogni $i\in I$, allora $\prod_{i\in I}X_i$ è compatto se e solo se $X_i$ è compatto per ogni $i\in I$.
\end{theorem}
\begin{corollary}
Per ogni insieme $X$, lo spazio $\{f:X\to [0,1]\}=[0,1]^X$ è compatto con la topologia della convergenza puntuale.
\end{corollary}

\begin{theorem}[Wallace]\label{TeoremaWallace}
Siano $X$ e $Y$ spazi topologici e siano $A\subseteq X,\ B\subseteq Y$ sottoinsiemi compatti. Se $N\subseteq X\times Y$ \`e un aperto tale che $A\times B\subseteq N$ allora esistono $U\subseteq X$ e $V\subseteq Y$ aperti tali che
\[A\times B\subseteq U\times V\subseteq N.\]
\end{theorem}


\subsection{Compattificazione di Alexandroff}
Visto quante proprietà hanno gli spazi compatti, ci domandiamo se esiste un modo per trasformare uno spazio dato in uno spazio compatto senza cambiarlo troppo. Questo tipo di metodo si chiama ``compattificazione" e in questa sezione studiamo in particolare il metodo di Alexandroff, il quale restituisce uno spazio compatto aggiungendo un solo punto.
\begin{definition}[Compattificazione]
Dato uno spazio topologico $X$, una \textbf{compattificazione} di $X$ è data da uno spazio $\hat X$ e una mappa continua $i:X\to \hat X$ tale che:
\begin{enumerate}[noitemsep]
\item $i:X\hookrightarrow \hat X$ è una immersione topologica
\item $i(X)$ è denso in $\hat X$
\item $\hat X$ è compatto.
\end{enumerate}
Se $|\hat X\bs i(X)|=1$ si dice $\hat X$ è una \textbf{compattificazione ad un punto} di $X$.
\end{definition}
\begin{remark}
Intuitivamente, la definizione dice che $\hat X$ è una compattificazione di $X$ se ha un sottospazio omeomorfo a $X$, questo sottospazio è quasi tutto $\hat X$ e $\hat X$ stesso è compatto.
\end{remark}
\noindent
Definiamo un procedimento per trovare una compattificazione ad un punto:

\begin{definition}[Compattificazione di Alexandroff]
Dato uno spazio topologico $X$ consideriamo $\hat X=X\cup\{\infty\}$, dove $\infty$ è un generico elemento non contenuto in $X$. L'insieme $\hat X$ dotato della seguente topologia è detto la \textbf{compattificazione di Alexandroff} di $X$:\\
$A\subseteq \hat X$ è aperto se
\begin{itemize}[noitemsep]
\item $\infty\notin A$ e  $A$ aperto in $X$, oppure
\item $\infty \in A$ e $X\bs A$ è chiuso \emph{compatto} di $X$.
\end{itemize}
\end{definition}
\begin{theorem}[La compattificazione di Alexandroff \`e una compattificazione]
La compattificazione di Alexandroff di uno spazio non compatto è una compattificazione ad un punto.
\end{theorem}
\begin{remark}
Se $X$ è uno spazio compatto e provassimo a costruirne la compattificazione di Alexandroff vedremmo che continua a valere tutto eccetto la condizione ``$X$ denso in $\hat X$", infatti in tal caso $\{\infty\}$ sarebbe un valido aperto di $\hat X$ e quindi $\infty\notin\ol X$ per la caratterizzazione (\ref{CaratterizzazioneChiusura}).
\end{remark}

\begin{theorem}[Unicità della compattificazione di Alexandroff]\label{UnicitaCompattificazioneAlexandroff}
Se $Y$ è uno spazio compatto $T_2$ e $P\in Y$ è tale che $Y\bs\{P\}$ non è compatto allora $Y$ è omeomorfo alla compattificazione di Alexandroff di $Y\bs\{P\}$.
\end{theorem}

\subsubsection{Proiezione stereografica}
Sappiamo che $\R^n$ non è compatto (\ref{RnNonECompatto}), possiamo quindi provare a studiarne la compattificazione di Alexandroff. Sarà il caso che una particolare mappa, detta ``proiezione stereografica" ci permetterà di vedere che la compattificazione è $S^n$.
\begin{definition}[Proiezione stereografica]
Data una dimensione $n$, la \textbf{proiezione stereografica} è la mappa $\pi:S^n\bs\{N\}\to\R^n$ che associa ad ogni punto $P$ della sfera (escluso appunto $N$ in polo nord) il punto di intersezione tra la retta passante per $N$ e $P$ con l'iperpiano $\R^n\times \{0\}$ identificato con $\R^n$.\\
Esplicitamente si ha che se $P=(x_1,\cdots,x_n,x_{n+1})=(x,x_{n+1})$ allora il punto corrisponde alla soluzione di
\[t(x,x_{n+1})+(1-t)(0,1)=(\pi(P),0)\implies t=\frac1{(1-x_{n+1})}\]
cioè
\[\pi(x_1,\cdots,x_{n+1})=\pa{\frac{x_1}{1-x_{n+1}},\cdots, \frac{x_n}{1-x_{n+1}}}.\]
\end{definition}
\begin{remark}
La proiezione stereografica è continua.
\end{remark}

\begin{theorem}
La proiezione stereografica è un omeomorfismo tra $S^n\bs\{N\}$ e $\R^n$.
\end{theorem}
\begin{corollary}
Per ogni $P\in S^{n}$, $S^n\bs\{P\}\cong \R^n$.
\end{corollary}

\begin{remark}
Sappiamo che $\R^n$ non è compatto, ma $S^n$ è compatto e per quanto appena detto aggiungere un punto ci porta da $\R^n$ a $S^n$, quindi $S^n$ è la compattificazione di Alexandroff di $\R^n$ (teorema \ref{UnicitaCompattificazioneAlexandroff}) dove l'immersione topologica può essere data dall'inversa della proiezione stereografica.
\end{remark}

\begin{remark}[Sfera di Riemann]
Osserviamo che topologicamente $\C$ è una copia di $\R^2$, quindi facendo questo stesso ragionamento su $\R^2$ e vedendolo come $\C$ troviamo una compattificazione dei complessi. Questa è la famosa \textbf{sfera di Riemann}.
\end{remark}




\subsection{Compattezza in spazi metrici}
Date le belle proprietà dei compatti e le belle proprietà degli spazi metrici, unire i due concetti non può che restituire una ricca teoria (per esempio vedremo che in spazzi metrici compattezza, compattezza per successione e totale limitatezza unita a completezza sono concetti equivalenti (\ref{CaratterizzazioneCompattiInMetrico})). L'approfondimento di questo tema è dominio dell'analisi ma in questa sede riportiamo i principali risultati e un breve studio sugli spazi completi e $\R^n$.

\subsubsection{Compattezza e assiomi di numerabilità}
Sappiamo che gli spazi metrici sono I-numerabili (\ref{MetricoEINumerabile}) e gli spazi metrici separabili sono II-numerabili (\ref{IINumerabileESeparabileEInMetricoCoincidono}). Studiamo dunque come si comportano le nozioni di compattezza con gli assiomi di numerabilità e le successioni.
\begin{definition}[Compattezza sequenziale]
Uno spazio topologico $X$ è \textbf{sequenzialmente compatto} o \textbf{compatto per successioni} se ogni successione a valori in $X$ ammette una sottosuccessione convergente.
\end{definition}

\begin{remark}
In generale compattezza e compattezza per successioni sono nozioni completamente distinte (esempi (\ref{CompattoNonCompattoPerSuccessioni}) e (\ref{CompattoPerSuccessioniNonCompatto})).
\end{remark}

\begin{definition}[Spazio Lindel\"of]
Uno spazio $X$ si dice di \textbf{Lindel\"of} se ogni ricoprimento aperto ammette un sottoricoprimento al più numerabile.
\end{definition}
\begin{remark}
Essere Lindel\"of è una condizione simile e più debole della compattezza.
\end{remark}

\begin{proposition}[I-numerabile compatto è sequenzialmente compatto]\label{INumerabileCompattoESequenzialmenteCompatto}
Se $X$ è I-numerabile ed è compatto allora è sequenzialmente compatto.
\end{proposition}
\begin{fact}
Il viceversa non è vero.
\end{fact}

\begin{proposition}[II-numerabile implica Lindel\"of]\label{IINumerabileImplicaLindelof}
Se uno spazio $X$ è II-numerabile allora è Lindel\"of.
\end{proposition}

\begin{proposition}[Compatto e sequenzialmente compatto coincidono in II-numerabile]\label{CompattoESequenzialmenteCompattoCoincidonoInIINumerabile}
Se $X$ è II-numerabile allora $X$ è compatto se e solo se $X$ è sequenzialmente compatto.
\end{proposition}

Possiamo riassumere quanto detto nella seguente
\begin{proposition}[Compattezza e Numerabilità]\label{CompattezzaEAssiomiDiNumerabilita}
In generale compatto e sequenzialmente compatto sono concetti distinti.\\
Se lo spazio è I-numerabile allora compatto implica sequenzialmente compatto.\\
Se lo spazio è II-numerabile allora compatto e sequenzialmente compatto coincidono.
\end{proposition}

\subsubsection{Limitatezza e Completezza}
\begin{proposition}[Compatti in metrico sono limitati]\label{CompattoInMetricoELimitato}
Se $(X,d)$ è metrico allora visto con la topologia indotta si ha che se $X$ è compatto allora $d$ è limitata.
\end{proposition}
\begin{remark}
Non vale il viceversa, infatti come sappiamo ogni topologia indotta da una metrica si può esprimere come indotta da una metrica limitata (\ref{OgniMetricoETopologicamenteEquivalenteALimitato}).
\end{remark}

\begin{definition}[Successione di Cauchy]
Una successione a valori in uno spazio metrico $X$ è \textbf{di Cauchy} se per ogni $\e>0$ esiste $n\in \N$ tale che $d(x_n,x_m)<\e$ per ogni $n,m>N$.
\end{definition}
\begin{remark}
Le successioni convergenti sono di Cauchy (si vede applicando la disuguaglianza triangolare a $x_n,\ x_m$ e il limite).
\end{remark}
\begin{definition}[Spazio completo]
Uno spazio metrico si dice \textbf{completo} se ogni successione di Cauchy converge.
\end{definition}
\begin{proposition}[Cauchy con sottosuccessione convergente è convergente]\label{CauchyConSottosuccessioneConvergenteEConvergente}
Se da una successione di Cauchy possiamo estrarre una sottosuccessione convergente allora la successione di partenza converge allo stesso limite.
\end{proposition}
\begin{corollary}[Metrico compatto per successioni è completo]\label{MetricoSequenzialmenteCompattoECompleto}
Se uno spazio metrico è sequenzialmente compatto allora è completo.
\end{corollary}
\begin{remark}
Uno spazio completo non è necessariamente sequenzialmente compatto (per esempio $\R^n$).
\end{remark}

\begin{remark}
Uno spazio (metrico) limitato e completo può comunque non essere compatto (\ref{LimiatoCompletoNonCompatto}).
\end{remark}


Cerchiamo di modificare leggermente la definizione di limitatezza in modo che, unita con la condizione di completezza, garantisca compattezza.

\begin{definition}[Spazio totalmente limitato]
Uno spazio metrico $X$ è \textbf{totalmente limitato} se per ogni $\e>0$ esiste un ricoprimento finito di $X$ costituito da palle aperte di raggio $\e$.
\end{definition}

\begin{proposition}[Totalmente limitato implica limitato]\label{TotalmenteLimitatoImplicaLimitato}
Se $X$ è totalmente limitato allora è limitato.
\end{proposition}
\begin{remark}
Non vale il viceversa.
\end{remark}

\begin{proposition}[Totalmente limitato implica II-numerabile]\label{TotalmenteLimitatoImplicaIINumerabile}
Se $X$ è totalmente limitato allora è II-numerabile.
\end{proposition}

\begin{theorem}[Caratterizzazione di compattezza per metrici]\label{CaratterizzazioneCompattiInMetrico}
Se $X$ è uno spazio metrico allora i seguenti sono fatti equivalenti:
\begin{enumerate}[noitemsep]
\item $X$ è compatto.
\item $X$ è sequenzialmente compatto.
\item $X$ è completo e totalmente limitato.
\end{enumerate}
\end{theorem}

% \noindent
% Possiamo riassumere le implicazioni che abbiamo usato in questa sezione con il seguente diagramma:

% \begin{scalebox}{0.8}{ % 0.8 is the scaling factor, adjust as needed
% \begin{tikzcd}
% 	&& {\begin{matrix}\text{succ. ammette}\\\text{sottosucc. di Cauchy}\end{matrix}} \\
% 	{\text{Compatto}} & {\text{Compatto per successioni}} & {\text{Completo}} \\
% 	&& {\text{Totalmente limitato}} \\
% 	& {\text{II-num.}}
% 	\arrow[from=2-2, to=2-3]
% 	\arrow[from=2-2, to=3-3]
% 	\arrow["{\text{I-num.}}", curve={height=-18pt}, from=2-1, to=2-2]
% 	\arrow[""{name=0, anchor=center, inner sep=0}, "{\text{Lindel\"of}}", curve={height=-18pt}, from=2-2, to=2-1]
% 	\arrow[shift right=5, curve={height=24pt}, from=3-3, to=1-3]
% 	\arrow[""{name=1, anchor=center, inner sep=0}, from=1-3, to=2-2]
% 	\arrow[from=3-3, to=4-2]
% 	\arrow["{+}"{description}, draw=none, from=2-3, to=3-3]
% 	\arrow[shorten >=3pt, from=2-3, to=1]
% 	\arrow[shorten >=11pt, from=4-2, to=0]
% \end{tikzcd}
% \end{scalebox}

\subsubsection{Numero di Lebesgue e Uniforme continuità}
\begin{definition}[Numero di Lebesgue]
Dato $X$ spazio metrico e $\Omega$ un suo ricoprimento si dice che $\Omega$ \textbf{ammette numero di Lebesgue} $\e>0$ se per ogni $x\in X$ esiste $U\in\Omega$ tale che $B(x,\e)\subseteq U$.
\end{definition}

\begin{remark}
Non tutti i ricoprimenti ammettono numero di Lebesgue (\ref{RicoprimentoSenzaNumeroDiLebesgue})
\end{remark}


\begin{theorem}[Ogni ricoprimento aperto in compatto ammette numero di Lebesgue]\label{InCompattoOgniRicoprimentoAmmetteNumeroDiLebesgue}
Se $X$ è uno spazio metrico compatto allora ogni ricoprimento aperto ammette un numero di Lebesgue.
\end{theorem}

\begin{definition}[Funzione uniformemente continua]
Una funzione $f:X\to Y$ è \textbf{uniformemente continua} se per ogni $\e>0$ esiste $\delta>0$ tale che \[d(x,y)<\delta\implies d(f(x),f(y))<\e.\]
\end{definition}
\begin{remark}
Le funzioni uniformemente continue sono continue.
\end{remark}
\begin{remark}
Non tutte le funzioni continue sono uniformemente continue (\ref{ContinuaNonUniformementeContinua})
\end{remark}

\begin{remark}
Le funzioni Lipschitziane sono uniformemente continue.
\end{remark}

\begin{theorem}[Heine-Cantor]\label{TeoremaHeineCantor}
Siano $X,Y$ spazi metrici con $X$ compatto. Se $f:X\to Y$ è continua allora è uniformemente continua.
\end{theorem}

\begin{remark}
Una funzione uniformemente continua porta successioni di Cauchy in successioni di Cauchy.
\end{remark}
\vspace{0.5cm}

\noindent Concludiamo studiando la seguente domanda (argomento trattato in anni accademici precedenti al 22/23):\\
siano $X,Y$ spazi metrici e siano $A\subseteq X$ e $f:A\to Y$ continua. Quando è possibile estendere $f$ a $f:\ol A\to Y$ in modo continuo?

\begin{remark}
Non è sempre possibile (\ref{ContinuaNonEstendibileAllaChiusura}).
\end{remark}

\begin{theorem}[Estensione di uniformemente continua alla chiusura del dominio]\label{UniformementeContinuaSiEstendeAllaChiusuraDelDominio}
Siano $X,Y$ metrici con $Y$ completo e $A\subseteq X$. Se $f:A\to Y$ è uniformemente continua allora $f$ si estende in modo unico a $\ol f:\ol A\to Y$ continua.
\end{theorem}


\subsubsection{Compattezza in $\R^n$}
Vediamo ora come si comportano gli insiemi compatti sullo spazio metrico più naturale per la nostra intuizione spaziale, $\R^n$.
\begin{theorem}\label{IntervalloChiusoECompatto}
L'intervallo $[0,1]$ è compatto.
\end{theorem}

\begin{application}[Compattezza degli intervalli]
$[a,b]$ è compatto mentre $(a,b),\ [a,b)$ e $(a,b]$ non sono compatti.
\end{application}

\begin{theorem}[Heine-Borel]\label{TeoremaHeineBorel}
Se $Y\subseteq \R^n$ allora $Y$ è compatto se e solo se $Y$ è chiuso e limitato.
\end{theorem}
\begin{application}\label{RnECompleto}
$\R^n$ è completo.
\end{application}

\begin{theorem}[Weierstrass]\label{TeoremaWeierstrass}
Se $X$ è compatto e $f:X\to \R$ è continua allora $f$ ammette massimo e minimo.
\end{theorem}

\begin{theorem}[Equivalenza delle norme su $\R^n$]\label{EquivalenzaNormeRn}
Tutte le norme su uno spazio vettoriale su $\R$ di dimensione finita inducono la stessa topologia.
\end{theorem}
\begin{remark}
Potevamo evitare di usare il fatto che le norme $1$ e $2$ sono equivalenti ragionando con le sfere rispetto alla norma $1$.
\end{remark}


\section{Topologia Quoziente}

\noindent Definiamo una topologia sui quozienti di spazi topologici

\begin{definition}[Spazio quoziente]
Sia $X$ uno spazio topologico e sia $\sim$ una relazione di equivalenza su $X$. Uno \textbf{spazio quoziente} $Y$ di $X$ rispetto alla relazione $\sim$ \`e uno spazio topologico tale che esiste una mappa continua $f:X\to Y$ che rispetta $\sim$ ($x_1\sim x_2\implies f(x_1)=f(x_2)$) e tale che per ogni $Z$ spazio topologico dotato di mappa continua $g:X\to Z$ che rispetta $\sim$ esiste un'unica mappa $h:Y\to Z$ continua che fa commutare il diagramma
\[\begin{tikzcd}
	X & Y \\
	Z
	\arrow["g"', from=1-1, to=2-1]
	\arrow["f", from=1-1, to=1-2]
	\arrow["h", from=1-2, to=2-1]
\end{tikzcd}\]
\end{definition}
\begin{remark}
$f$ è necessariamente surgettiva, altrimenti $h$ non potrebbe essere unica.
\end{remark}

\begin{proposition}[Esistenza e unicit\`a dello spazio quoziente]\label{EsistenzaUnicitaSpazioQuoziente}
Sia $X$ uno spazio topologico e sia $\sim$ una relazione di equivalenza su $X$, allora esiste uno spazio quoziente unico a meno di omeomorfismo canonico.
\end{proposition}

\noindent Riportiamo nuovamente la definizione del modello di spazio quoziente che abbiamo trovato:

\begin{definition}[Topologia quoziente]
Sia $X$ uno spazio topologico e sia $\sim$ una relazione di equivalenza su $X$. Se $\quot X\sim$ è l'insieme quoziente e $\pi:X\to \quot X\sim$ è la proiezione al quoziente allora la \textbf{topologia quoziente} su $\quot X\sim$ è data come segue:
\[A\subseteq \quot X\sim\text{ è aperto }\coimplies \ \pi\ii(A)\text{ è aperto in }X.\]
\end{definition}

\begin{theorem}[Caratterizzazione della topologia quoziente]\label{CaratterizzazioneTopologiaQuoziente}
La topologia quoziente su $\quot X\sim$ è la più fine che rende $\pi:X\to\quot X\sim$ continua.
\end{theorem}





\subsection{Passaggio a quoziente e Identificazioni}
\begin{definition}[Funzioni ottenute per passaggio a quoziente]
Se $f:X\to Y$ è una funzione continua e $x_1\sim x_2\implies f(x_1)=f(x_2)$ allora è ben definita
\[\ol f:\funcDef{\quot X\sim}{Y}{[x]}{f(x)},\]
la quale è continua per quanto detto sopra. Si dice che  $\ol f$ è stata ottenuta \textbf{per passaggio al quoziente} di $f$.
\end{definition}

\begin{proposition}\label{MappeIndotteAlQuozienteIniettiveSurgettive}
Se $\ol f$ è ottenuta per passaggio al quoziente come sopra si ha che
\begin{itemize}[noitemsep]
\item $\ol f$ è iniettiva se e solo se $f(x_1)=f(x_2)\coimplies x_1\sim x_2$
\item $\ol f$ è surgettiva se e solo se $f$ è surgettiva.
\end{itemize}
\end{proposition}
\begin{remark}
Se $f$ è surgettiva e $x_1\sim x_2\coimplies f(x_1)=f(x_2)$ allora $\ol f$ è una bigezione continua.
\end{remark}
\noindent In generale non esistono criteri semplici per verificare quando $\ol f$ è un omeomorfismo. Definiamo allora un tipo di funzione che induce un omeomorfismo se come relazione imponiamo $x_1\sim x_2\coimplies f(x_1)=f(x_2)$:

\begin{definition}[Identificazione]
Una funzione $f:X\to Y$ continua è detta \textbf{identificazione} se valgono le seguenti condizioni:
\begin{enumerate}[noitemsep]
\item $f$ è surgettiva
\item $A\subseteq Y$ è aperto se e solo se $f\ii(A)$ è aperto in $X$ \\(equivalentemente $C$ chiuso in $Y$ se e solo se $f\ii(C)$ chiuso in $X$).
\end{enumerate}
\end{definition}

\begin{theorem}[Identificazione induce omeomorfismo per quoziente]\label{IdentificazioniInduconoOmeomorfismiSulQuoziente}
Siano $f:X\to Y$ una identificazione e $\sim $ una relazione di equivalenza su $X$ data da $x_1\sim x_2\coimplies f(x_1)=f(x_2)$. Allora la mappa $\ol f:\quot X\sim \to Y$ ottenuta da $f$ passando al quoziente è un omeomorfismo
\end{theorem}
\begin{remark}
In realtà $\ol f$ definita come sopra è un omeomorfismo se e solo se $f$ \`e una identificazione.
\end{remark}

\begin{proposition}[Criterio sufficiente per definire identificazioni]\label{CriterioSufficientePerIdentificazioni}
Sia $f:X\to Y$ continua e surgettiva. Si ha che se $f$ è aperta o chiusa allora è una identificazione.
\end{proposition}
\begin{remark}
Esistono identificazioni che non sono né aperte né chiuse (\ref{IdentificazioneNeApertaNeChiusa})
\end{remark}


\subsection{Insiemi saturi}
\begin{definition}[Insieme saturo]
Data una funzione $f:X\to Y$, un insieme $A\subseteq X$ si dice \textbf{$f-$saturo} (o \textbf{saturo} se $f$ è chiara dal contesto
) se
\[f\ii(f(A))=A,\]
cioè se $x\in A$ e $f(x')=f(x)$ allora $x'\in A$.
\end{definition}
\begin{proposition}[Gli $f-$saturi sono le preimmagini tramite $f$]\label{CaratterizzazioneSaturi}
Data una funzione $f:X\to Y$ si ha che
\[A\subseteq X\text{ è saturo}\coimplies \exists B\subseteq Y\ t.c.\ A=f\ii(B)\]
\end{proposition}

\begin{remark}
Se $\pi:X\to \quot X\sim$ è la proiezione allora gli insiemi $\pi-$saturi di $X$ sono le unioni di classi di equivalenza.
\end{remark}
\begin{proposition}[Caratterizzazione di aperti e chiusi saturi]\label{CaratterizzazioneDiApertiEChiusiSaturi}
Gli aperti / chiusi saturi sono identificati dalle seguenti proprietà:
\begin{enumerate}[noitemsep]
\item $A\subseteq \quot X\sim$ è aperto se e solo se $A=\pi(B)$ con $B$ aperto saturo di $X$.
\item $C\subseteq \quot X\sim$ è chiuso se e solo se $C=\pi(D)$ con $D$ chiuso saturo di $X$.
\end{enumerate}
\end{proposition}

\begin{remark}
In generale abbiamo la seguente corrispondenza biunivoca:
\[\{\text{aperti di }\quot X\sim\}\overset{\pi}{\longleftrightarrow}\{\text{aperti saturi di }X\}\]
\end{remark}

\begin{remark}
Dato che $\pi:X\to \quot X\sim$ è continua e surgettiva si ha che quozienti di compatti sono compatti e quozienti di connessi sono connessi.
\end{remark}

\subsection{Collassamento, Unione disgiunta e Bouquet}
\begin{definition}[Collassamento]
Sia $X$ uno spazio topologico e sia $A\subseteq X$. Definiamo il \textbf{collassamento di $X$ su $A$} come lo spazio quoziente $\quot X\sim$ ottenuto definendo la seguente relazione di equivalenza:
\[x\sim y\coimplies x=y\text{ oppure }x,y\in A.\]
Come notazione scriviamo $\quot X\sim=\quot XA$. Si dice che $\quot XA$ è ottenuto da $X$ \textbf{collassando $A$ ad un punto}.
\end{definition}

\begin{application}
Detti $D^n=\{P\in \R^n\mid |P|\leq 1\}$ e $S^n=\{P\in\R^{n+1}\mid |P|=1\}$ si ha che
\[\quot{D^n}{S^{n-1}}\cong S^n.\]
\end{application}
\vspace{0.5cm}

\noindent Vediamo ora come possiamo ``attaccare" due spazi topologici tra loro. L'idea \`e considerarli come un unico spazio topologico formato dai due separati e poi imporre una relazione che collassa un insieme di due punti, uno per spazio.

\begin{definition}[Unione disgiunta]
Siano $X$ e $Y$ due spazi topologici. Su $X\sqcup Y$ imponiamo la seguente topologia:
\[A\subseteq X\sqcup Y\text{ aperto}\coimplies A\cap X\text{ aperto e }A\cap Y\text{ aperto}.\]
\end{definition}
\begin{remark}
$X$ e $Y$ sono aperti in $X\sqcup Y$, quindi $X\sqcup Y$ \`e sconnesso.
\end{remark}

\begin{definition}[Bouquet]
Siano $X$ e $Y$ spazi topologici e fissiamo $x_0\in X$ e $y_0\in Y$. Il \textbf{bouquet} o \textbf{wedge}\footnote{Il simbolo usato per wedge dai professori \`e ``$\vee$", che in \LaTeX~ si scrive con \texttt{\textbackslash vee}, non \texttt{\textbackslash wedge} come uno potrebbe sperare.} di $(X,x_0)$ e $(Y,y_0)$ \`e il quoziente
\[(X,x_0)\vee (Y,y_0)=\quot{X\sqcup Y}{\{x_0,y_0\}}.\]
\end{definition}
\begin{remark}
Cambiare i punti $x_0$ e $y_0$ pu\`o cambiare lo spazio bouquet che otteniamo. Ci\`o nonostante, se abbiamo fissato $x_0$ e $y_0$, $x_0$ e $y_0$ sono chiari da contesto oppure se $x_0$ e $y_0$ sono irrilevanti scriveremo
\[X\vee Y.\]
\end{remark}

\begin{proposition}[I fattori si immergono nel bouquet]\label{FattoriSiImmergonoInBouquetWedge}
Le mappe
\[i:X\to X\vee Y,\qquad j:Y\to X\vee Y\]
indotte dalle inclusioni di $X$ e $Y$ in $X\sqcup Y$ sono immersioni topologiche.
\end{proposition}

\begin{proposition}[$T_1$ passa al bouquet e immersioni sono chiuse]\label{T1PassaAlBouquetWedgeEImmersioniChiuse}
Se $X$ e $Y$ sono $T_1$ allora anche $X\vee Y$ \`e $T_1$ e le mappe $i:X\to X\vee Y,\ j:Y\to X\vee Y$ sono chiuse.
\end{proposition}

\begin{proposition}[$T_2$ passa al bouquet]\label{T2PassaAlBouquetWedge}
Se $X$ e $Y$ sono $T_2$, anche $X\vee Y$ \`e $T_2$.
\end{proposition}

\begin{proposition}[Bouquet \`e compatto se e solo se lo sono i fattori]\label{BouquetWedgeECompattoSeESoloSeFattoriCompatti}
$X\vee Y$ \`e compatto se e solo se $X$ e $Y$ sono compatti.
\end{proposition}

\begin{proposition}[Bouquet \`e connesso se e solo se lo sono i fattori]\label{BouquetEConnessoSeESoloSeLoSonoIFattori}
$X\vee Y$ \`e connesso se e solo se $X$ e $Y$ sono connessi.
\end{proposition}




\section{Quozienti per azioni di gruppi}
Ricordiamo le seguenti definizioni
\begin{definition}[Azione]
Dato un gruppo $G$ e un insieme $X$, una \textbf{azione} di $G$ su $X$ \`e un omomorfismo $G\to S(X)$, dove $S(X)$ sono le permutazioni di $X$. Come notazione scriviamo \[G\acts X\] e indichiamo l'immagine di $x\in X$ tramite la permutazione data da $g\in G$ come \[g\cdot x=gx=g(x)=\ell_g(x).\]
\end{definition}
\begin{definition}[Orbita e stabilizzatore]
Data una azione $G\acts X$ e fissato $x_0\in X$ definiamo l'\textbf{orbita} e lo \textbf{stabilizzatore} di $x_0$ rispettivamente come
\begin{align*}
&\orb_G(x_0)=G\cdot x_0=\{y\in X\mid \exists g\in G\ t.c.\ gx_0=y\}\\
&\stab_G(x_0)=\{g\in G\mid gx_0=x_0\}.
\end{align*}
\end{definition}
\begin{definition}
Una azione $G\acts X$ si dice
\begin{itemize}[noitemsep]
\item \textbf{fedele} se $\ell_g=id\coimplies g=1_G$,
\item \textbf{libera} se $\stab(x_0)=\{1_G\}$ per ogni $x_0\in X$,
\item \textbf{transitiva} se per ogni $x,y\in X$ esiste $g\in G$ tale che $gx=y$, cio\`e $G\cdot x_0=X$ per ogni $x_0\in X$.
\end{itemize}
\end{definition}
\begin{remark}
Data una azione $G\acts X$, la relazione
\[x\sim y\coimplies G\cdot x=G\cdot y\coimplies \exists g\in G\ t.c.\ gx=y\]
\`e di equivalenza. Chiamiamo questa la \textbf{relazione indotta} dall'azione.
\end{remark}
\vspace{0.5cm}

\noindent
Le azioni che riguardano il corso saranno solo le
\begin{definition}[Azione continua]
Dato un gruppo $G$ e uno spazio topologico $X$, una azione $G\acts X$ \`e detta \textbf{continua} se le mappe
\[\ell_g:\funcDef{X}{X}{x}{gx}\]
sono continue per ogni $g\in G$.
\end{definition}
\begin{remark}
Dato che $\ell_g\ii=\ell_{g\ii}$, si ha che in realt\`a le $\ell_g$ sono omeomorfismi di $X$ in s\'e.
\end{remark}
\noindent Da ora in poi assumeremo che tutte le azioni di un gruppo su uno spazio topologico siano continue.
\begin{notation}
Data una azione $G\acts X$ con $X$ spazio topologico poniamo
\[\quot XG=\quot X\sim,\]
dove $\sim$ \`e la relazione indotta dall'azione.
\end{notation}
\begin{remark}
Pi\`u propriamente dovremmo scrivere $G\backslash X$ perch\'e stiamo considerando una azione sinistra.
\end{remark}
\begin{remark}
Se $G\subseteq X$ (per esempio $X=\R$ e $G=\Z$) si presenta una ambiguit\`a rispetto a quale spazio intendiamo con
\[\quot XG,\]
pi\`u precisamente la relazione per la quale stiamo quozientando potrebbe essere quella indotta dall'azione di $G$ su $X$, ma anche ``$x=y$ oppure $x,y\in G$". Se non specifichiamo altrimenti o se non chiaro da contesto intenderemo il quoziente per azione.
\end{remark}

\begin{remark}[Caratterizzazione dei saturi per azione]\label{CaratterizzazioneSaturiPerAzione}
Dato $A\subseteq X$ si ha che se $\pi:X\to\quot XG$ \`e la proiezione allora
\[\pi\ii(\pi(A))=\bigcup_{g\in G}\{ga\mid a\in A\}=\bigcup_{g\in G}g\cdot A=G\cdot A.\]
Segue che $A$ \`e saturo se e solo se \`e $G-$invariante (dato che le classi sono le orbite).
\end{remark}

\begin{proposition}[Proiezioni per quozienti per azione]\label{ProiezioniSonoApertePerQuozientePerAzioneEAncheChiuseGruppoFinito}
Data una azione $G\acts X$, la proiezione $\pi:X\to\quot XG$ \`e una mappa aperta.
Inoltre, se $G$ \`e finito, allora $\pi$ \`e anche una mappa chiusa.
\end{proposition}
\begin{remark}
La proiezione non \`e sempre chiusa (\ref{ProiezioneQuozienteAzioneNonChiusa})
\end{remark}

\subsection{Assiomi di Separazione e Azioni}
Osserviamo che se $X$ gode di assiomi di separazione, $\quot XG$ pu\`o perderli quasi senza restrizione (presentiamo due esempi di quozienti di $\R^n$ non $T_1$: (\ref{RQuozienteQ}) (\ref{Matrici2x2QuozienteSimilitudine})).\\
Cerchiamo di capire quando $\quot XG$ \`e $T_2$.
\begin{definition}[Azioni vaganti, propriamente discontinue e proprie]
Data una azione $G\acts X$ con $X$ spazio topologico, affermiamo che questa \`e
\begin{itemize}[noitemsep]
\item \textbf{vagante} se per ogni $x\in X$ esiste $U$ intorno di $x$ tale che
\[\{g\in G\mid gU\cap U\neq \emptyset\}\text{ \`e finito.}\]
\item \textbf{propriamente discontinua} se per ogni $x\in X$ esiste $U$ intorno di $x$ tale che
\[\{g\in G\mid gU\cap U\neq \emptyset\}=\{1_G\}.\]
\item \textbf{propria} se per ogni compatto $K\subseteq X$ abbiamo che
\[\{g\in G\mid gK\cap K\neq \emptyset\}\text{ \`e finito.}\]
\end{itemize}
\end{definition}
\begin{remark}
Una azione propriamente discontinua \`e anche vagante.
\end{remark}
\begin{remark}
Osserviamo che
\[\stab_G(x)\subseteq \{g\in G\mid gU\cap U\neq \emptyset\},\]
dunque in particolare:
\begin{itemize}[noitemsep]
\item Se una azione \`e vagante allora gli stabilizzatori sono finiti.
\item Se una azione \`e propriamente discontinua allora \`e libera.
\end{itemize}
\end{remark}

\begin{theorem}[Caratterizzazione di azioni propriamente discontinue su $T_2$]\label{SuT2AzionePropriamenteDiscontinuaSeESoloSeLiberaEVagante}
Sia $X$ uno spazio $T_2$. Si ha che una azione $G\acts X$ \`e propriamente discontinua se e solo se \`e libera e vagante.
\end{theorem}

\begin{theorem}[Caratterizzazione azioni proprie su localmente compatti]\label{CaratterizzazioneAzionePropiasuLocalmenteCompatto}
Sia $X$ uno spazio localmente compatto. Si ha che $G\acts X$ \`e una azione propria se e solo se per ogni $x,y\in X$ esistono intorni $U$ e $V$ di $x$ e $y$ rispettivamente tali che
\[\{g\in G\mid gU\cap V\neq \emptyset\}\text{ \`e finito.}\]
\end{theorem}

\begin{remark}
Nel teorema l'implicazione $\impliedby$ non usa la locale compattezza.
\end{remark}

\noindent Siamo pronti per dare un criterio che garantisce che un quoziente di azione sia $T_2$:
\begin{theorem}[Criterio sufficiente per quoziente per azione $T_2$]\label{CriterioSufficientePerQuozientePerAzioneT2}
Sia $X$ localmente compatto e $T_2$. Si ha che se $G\acts X$ \`e propria allora $\quot XG$ \`e $T_2$.
\end{theorem}

\begin{remark}
L'unico motivo per cui abbiamo supposto localmente compatto e azione propria \`e per sfruttare la caratterizzazione data dal teorema (\ref{CaratterizzazioneAzionePropiasuLocalmenteCompatto}).
\end{remark}

\subsection{Domini fondamentali}
Proviamo a trovare dei sottoinsiemi di uno spazio che si proietti a quoziente come lo spazio intero.

\begin{definition}[Dominio fondamentale]
Fissiamo una azione $G\acts X$. Affermiamo che $D\subseteq X$ \`e un \textbf{dominio fondamentale} di $X$ per l'azione se
\begin{enumerate}[noitemsep]
\item $D$ \`e chiuso
\item $G\cdot D=\bigcup_{g\in G}gD=X$ ($\pi\res D$ \`e surgettiva)
\item la famiglia $\{gD\}_{g\in G}$ \`e localmente finita
\item Per ogni $g\in G\bs\{1_G\}$ abbiamo $g\rg D\cap \rg D=\emptyset$ ($\pi\res{\rg D}$ \`e iniettiva).
\end{enumerate}
\end{definition}
\begin{remark}
L'ultima propriet\`a viene inserita pi\`u che altro per ragioni storiche, nei teoremi successivi non viene usata molto.\\
La terza propriet\`a \`e molto forte, infatti vedremo che non tutte le azioni ammettono dominio fondamentale.
\end{remark}
\begin{remark}
In molti casi non solo si ha $D$ chiuso, ma $\ol{\rg D}=D$. Questo ci permette di pensare ai domini fondamentali come uno spazio la cui parte interna viene lasciata a s\'e e vengono effettuate delle identificazioni sul bordo.
\end{remark}

\begin{lemma}[Famiglie localmente finite incontrano compatti finite volte]\label{FamiglieLocalmenteFiniteIncontranoCompattiFiniteVolte}
Se $\{A_i\}_{i\in I}$ \`e una famiglia localmente finita e $K$ \`e compatto allora
\[\{i\in I\mid A_i\cap K\neq \emptyset\}\text{ \`e finito.}\]
\end{lemma}

\begin{lemma}[Azione con dominio fondamentale \`e propria]\label{AzioneConDominioFondamentaleEPropria}
Se una azione ammette dominio fondamentale allora \`e propria.
\end{lemma}


\begin{theorem}[Localmente compatto con dominio fondamentale]\label{LocalmenteCompattoConDominioFondamentale}
Sia $X$ localmente compatto e sia $D$ un dominio fondamentale per $G\acts X$. Valgono le seguenti:
\begin{itemize}[noitemsep]
\item se $X$ \`e $T_2$ allora $\quot XG$ \`e $T_2$
\item $\displaystyle \quot XG\cong \quot D\sim$, dove $\sim$ \`e la relazione indotta da $G\acts X$ ristretta a $D$.
\end{itemize}
\end{theorem}
\begin{remark}
Non ha senso dire che la relazione su $D$ che abbiamo considerato \`e la relazione indotta da $G\acts D$, perch\'e in generale $G$ non agisce su $D$, in quanto \`e possibile che $gD\not\subseteq D$.
\end{remark}

\begin{remark}
Trovare domini fondamentali compatti rende la vita pi\`u semplice, per esempio garantisce che il quoziente $\quot D\sim$ sia compatto, dunque una condizione sufficiente per trovare un omeomorfismo dal quoziente a un candidato spazio $Z$ \`e mostrare che la mappa \`e continua, bigettiva e che $Z$ \`e $T_2$ (vorremmo applicare (\ref{ContinueDaCompattoInT2SonoChiuse})).
\end{remark}

\section{Topologia dei Proiettivi}
Ricordiamo che
\[\Pj^n\K=\quot{\K^{n+1}\nz}\sim,\]
che possiamo interpretare come quoziente per la seguente azione:
\[\K^\times\acts \K^{n+1}\nz,\quad \la\cdot v=\la v.\]
Cerchiamo allora di dotare $\Pj^n\K$ di una topologia quoziente a partire dalla topologia di $\K^{n+1}\nz$. Gli unici campi che considereremo in questo capitolo sono $\R$ e $\C$ dato che conosciamo bene $\R^n$ e possiamo interpretare topologicamente $\C^n$ come $\R^{2n}$.
\subsection{Caso Reale}
Osserviamo che l'azione in esame non \`e propria, per esempio perch\'e $1+\e$ \`e arbitrariamente vicino all'identit\`a del gruppo, quindi non posso sperare di separare le orbite. Questo ci leva ogni speranza di trovare un dominio fondamentale (\ref{AzioneConDominioFondamentaleEPropria}).

Possiamo comunque cercare di restringere la nostra attenzione ad un opportuno sottoinsieme di $\R^{n+1}\nz$, che vedremo essere $S^n$.

\begin{remark}
$S^n$ contiene un rappresentante per ogni orbita dell'azione, infatti $v\sim \frac1{|v|}v\in S^n$.
\end{remark}
\begin{remark}
La proiezione ristretta alla sfera non \`e iniettiva, ma non \`e lontana: \`e $2$ a $1$.
\end{remark}
\noindent Date queste propriet\`a della sfera rispetto alla relazione in esame sospettiamo quanto segue:

\begin{theorem}[Proiettivi reali come identificazione antipodale di una sfera]\label{TopologiaProiettivoRealeDaSfera}
Consideriamo l'azione di $\znz2$ su $S^n$ data da $0\cdot p=p,\ 1\cdot p=-p$. Si ha che
\[\Pj^n\R\cong \quot{S^n}{\znz2}.\]
\end{theorem}
\begin{corollary}
I proiettivi reali sono compatti, connessi per archi e Hausdorff.
\end{corollary}

\begin{theorem}[Proiettivi reali come identificazione sul bordo di disco]\label{TopologiaProiettivoRealeDaDisco}
Poniamo su $D^n$ la relazione \[v\sim v'\coimplies v=v'\text{ oppure }|v|=|v'|=1\text{ e }v=-v'.\]
Si ha che $\quot {D^n}\sim\cong \Pj^n\R$.
\end{theorem}

\begin{remark}
Dal teorema (\ref{TopologiaProiettivoRealeDaDisco}) segue immediatamente che $\Pj^1\R\cong S^1$ (e NON nel modo che ci aspetteremmo pensando allo spazio delle direzioni di $\R^2$).\\
Dalla seguente successione di trasformazioni possiamo vedere che $\Pj^2\R$ \`e omeomorfo ad un nastro di M\"obius sul cui bordo \`e incollato un disco:
\begin{itemize}[noitemsep]
\item Consideriamo il disco $D^2$ con la relazione del teorema (\ref{TopologiaProiettivoRealeDaDisco})
\item Tagliamo un disco pi\`u piccolo in modo da ottenere un disco (questo) e una corona.
\item Tagliamo la corona secondo quello che sarebbe stato un diametro del disco originale
\item Raddrizziamo i due pezzi ricavati dalla corona in due rettangoli
\item Identifichiamo i due rettangoli lungo il lato che proveniva dal bordo del disco originale
\item Identifichiamo i lati del rettangolo ottenuto al passo precedente corrispondenti ai bordi creati quando abbiamo fatto il taglio lungo il diametro.
\end{itemize}
Avendo seguito questi passi dovremmo aver ottenuto un nastro di M\"obius (dal rettangolo) e un disco (quello tagliato da quello originale) che deve essere attaccato lungo il bordo del nastro di M\"obius. Vi chiedo sinceramente scusa perch\'e seguire questi passi anche con un disegno sotto non \`e semplicissimo, non oso immaginare a parole.
\end{remark}

\subsection{Caso Complesso}
Osserviamo nuovamente che l'azione che determina $\Pj^n\C$ non \`e propria, ma anche in questo caso si ha che
\[S^{2n+1}=\{v\in\C^{n+1}=\R^{2n+2}\mid |v|=1\}\]
incontra tutte le classi di equivalenza per la relazione su $\C^{n+1}\nz$ (per trovare un rappresentante sulla sfera basta dividere ogni vettore per la propria norma).

\begin{remark}
Per quanto appena detto, la mappa $\pi:\C^{n+1}\nz\to\Pj^n\C$ ristretta a $S^{2n+2}$ \`e surgettiva (e continua per definizione di topologia quoziente). Osserviamo quindi che $\Pj^n\C$ \`e compatto e connesso per archi.
\end{remark}

\noindent
Per studiare meglio i proiettivi osserviamo che grandi pezzi di $\Pj^n\K$ sono essenzialmente $\K^n$:
\begin{proposition}[Le carte affini sono omeomorfismi]\label{CarteAffiniSonoOmeomorfismi}
Se $\K$ \`e un campo dotato di una topologia (per quanto ci riguarda $\K=\R$ o $\K=\C$) allora le carte affini
\[J_i:\K^n\to U_i\]
sono omeomorfismi tra $\K^n$ e particolari aperti di $\Pj^n\K$.
\end{proposition}

\begin{proposition}
$\Pj^n\C$ \`e Hausdorff con la topologia quoziente.
\end{proposition}


\begin{corollary}
$\Pj^1\C\cong S^2$.
\end{corollary}

\subsection{Variet\`a topologiche}
\begin{definition}[Variet\`a topologica]
Una \textbf{variet\`a topologica} di dimensione $n$ \`e uno spazio topologico $X$ tale che
\begin{itemize}[noitemsep]
\item Per ogni $P\in X$ esiste un intorno aperto $U$ di $P$ omeomorfo ad un aperto di $\R^n$
\item $X$ \`e $T_2$
\item $X$ \`e II-numerabile.
\end{itemize}
\end{definition}
\begin{remark}
Se $X$ rispetta la prima propriet\`a si dice che $X$ \`e \textbf{localmente euclidea}. L'ultima condizione \`e omessa da alcuni autori.
\end{remark}

\begin{remark}
I proiettivi sono variet\`a topologiche, $\Pj^n\R$ di dimensione $n$ e $\Pj^n\C$ di dimensione $2n$.
\end{remark}

\begin{proposition}
Dato $X$ e dato $P\in X$ le seguenti condizioni su $U$ intorno aperto di $P$ sono equivalenti:
\begin{enumerate}[noitemsep]
\item $U$ omeomorfo ad un aperto di $\R^n$
\item $U$ omeomorfo a $\R^n$
\item $U$ omeomorfo a $B(0,\e)\subseteq \R^n$ per qualche $\e>0$.
\end{enumerate}
\end{proposition}

\begin{remark}
Uno spazio localmente euclideo non \`e necessariamente $T_2$ (\ref{LocalmenteEuclideoNonT2})
\end{remark}

\section{Appendice al capitolo 2}

\subsection{Esempi e controesempi}

\subsubsection{Spazi topologici}
\begin{example}[Distanze non topologicamente equivalenti]\label{DistanzeNonTopEquivalenti}
Le distanze $d_1$ e $d_\infty$ su $C([0,1])$ non sono topologicamente equivalenti.
\end{example}


\begin{example}[Punto aderente non di accumulazione]\label{AderenteNonAccumulazione}
Siano $X=\R$ e $Z=[0,1)\cup \{2\}$. Osserviamo che $2$ \`e aderente ma non di accumulazione. Sempre da questo esempio notiamo che $1$ \`e un punto di accumulazione che non appartiene a $Z$.
\end{example}

\subsubsection{Assiomi di numerabilit\`a}
\begin{example}[Insieme $I-$numerabile ma non $II-$numerabile]\label{INumerabileNonIINumerabile}
Sia $X$ un insieme  pi\`u che numerabile dotato della topologia discreta. Esso \`e chiaramente I-numerabile in quanto metrizzabile e un SFI per $x_0$ \`e $\{x_0\}$. Osserviamo per\`o che lo spazio non \`e secondo numerabile perch\'e ogni base deve contenere i singoletti (aperti scrivibili solo come unione di se stessi ed eventualmente $\emptyset$) e ce ne sono di una quantit\`a pi\`u che numerabile.
\end{example}

\subsubsection{Prodotti}
\begin{example}[Prodotto di $I-$numerabili non $I-$numerabile]\label{ProdottoDiINumerabileNonINumerabile}
Sia $I$ tale che $|I|>|\N|$ e per ogni $i\in I$ sia $X_i=\{0,1\}$ con la topologia discreta. Si ha che $X=\prod_{i\in I}X_i$ non \`e $I-$numerabile.
\end{example}
\begin{remark}
L'esempio precedente mostra anche che il prodotto pi\`u che numerabile di spazi metrizzabili pu\`o non essere metrizzabile.
\end{remark}


\begin{example}[Le proiezioni non sono sempre chiuse]\label{ProiezioniNonSempreChiuse}
Consideriamo $C=\{(x,y)\in\R^2\mid xy=1\}$: esso \`e preimmagine di $1$ tramite la mappa continua $(x,y)\mapsto xy$, quindi \`e un chiuso in quanto preimmagine di un chiuso tramite una mappa continua, eppure $\pi_1(C)=\R\nz$ che non \`e chiuso in $\R$.
\end{example}

\subsubsection{Assiomi di separazione}
\begin{proposition}\label{T1NonT2-T0NonT1}
Le implicazioni $T_2\implies T_1\implies T_0$ sono strette.
\end{proposition}

\begin{example}[Spazio $T_4$ non $T_0$]\label{SpazioT4NonT0}
La topologia indiscreta su $X$ di cardinalit\`a almeno $2$ rende lo spazio vuotamente sia $T_3$ che $T_4$, ma come sappiamo questo tipo di spazio non \`e $T_0$
\end{example}

\begin{definition}[Retta di Sorgenfrey]
La \textbf{retta di Sorgenfrey} \`e $\R$ dotato della topologia con la seguente base:
\[\{[a,b)\mid a<b,\ a,b\in\R\}.\]
\end{definition}
\begin{remark}
La topologia di Sorgenfrey \`e pi\`u fine della topologia euclidea, infatti posso scrivere $(a-\e,a+\e)$ come unione di aperti di Sorgenfrey come segue: sia $x_n\to a-\e$ una successione monotona contenuta in $(a-\e,a+\e)$. Segue che
\[(a-\e,a+\e)=\bigcup_{i}[x_i,a+\e).\]
\end{remark}

\begin{example}\label{RettaDiSorgenfreyENormale}
La retta di Sorgenfrey \`e normale.
\end{example}

\begin{lemma}\label{T4SeparabileAlloraSottoinsiemeDiscretoChiusoHaCardinalitaMinoreDelContinuo}
Sia $Z$ uno spazio $T_4$ e separabile. Se $D$ \`e un suo sottoinsieme chiuso e discreto allora $|D|<|\R|$.
\end{lemma}

\begin{example}[Piano di Sorgenfrey]\label{PianoDiSorgenfrey}
Il piano di Sorgenfrey, cio\`e il prodotto di due rette di Sorgenfrey, non \`e $T_4$.
\end{example}

\begin{example}[Spazio Hausdorff non regolare]\label{EsempioT2NonRegolare}
Consideriamo $\R$ dotato della topologia generata dagli aperti euclidei e $\R\bs\{\frac1n\mid n\in\N\nz\}$.
\end{example}

\subsubsection{Ricoprimenti}
\begin{example}[Ricoprimento chiuso non fondamentale]\label{RicoprimentoChiusoNonFondamentale}
Su $\R$ il ricoprimento $\{\{x\}\}_{x\in \R}$ non \`e fondamentale.
\end{example}

\subsubsection{Connessi}
\begin{example}[Connesso ma non connesso per archi]\label{ConnessoNonConnessoPerArchi}
L'insieme $Y=\{(0,0)\}\cup\{(x,\sin(1/x))\mid x>0,x\in \R\}\subseteq \R^2$ \`e connesso ma non connesso per archi
\end{example}

\begin{example}[Pettine infinito]\label{PettineInfinito}
Sia $X=(\R\times\{0\})\cup(\Q\times \R)\subseteq \R^2$. Dotato della topologia di sottospazio di $\R^2$, $X$ \`e uno spazio connesso per archi ma non localmente connesso per archi.
\end{example}

\begin{example}[Insieme con parti connesse non aperte]\label{PartiConnesseAperte}
$\Q$ \`e totalmente sconnesso ma i singoletti non sono aperti.
\end{example}

\begin{example}[Insieme con parti conn. per archi n\'e aperte n\'e chiuse]\label{PartiConnessePerArchiNeAperteNeChiuse}
$Y=\{(0,0)\}\cup\{(x,\sin(1/x)\mid x>0)\}\subseteq\R^2$. Sappiamo che i due pezzi specificati sono connessi per archi ma l'insieme non \`e connesso per archi, dunque questa \`e la partizione. Osserviamo che $\{(0,0)\}$ non \`e aperto e $\{(x,\sin(1/x)\mid x>0)\}$ non \`e chiuso.
\end{example}

\subsubsection{Compattezza}
\begin{example}[$\R^n$ non \`e compatto]\label{RnNonECompatto}
$\R^n$ non \`e compatto, infatti il ricoprimento $\{B(x_0,n)\}_{n\in \N}$ non ammette un sottoricoprimento finito (se $m$ fosse il massimo raggio allora tutti gli elementi fuori da $B(x_0,m)$ non sarebbero coperti).
\end{example}

\begin{example}[Famiglie che godono della propriet\`a dell'intersezione finita]\label{EsempioIntersezioneFinita}
Se $X=\R$ gli insiemi della forma $[n,+\infty)$ godono della propriet\`a dell'intersezione finita. Similmente per $X=[0,1]$ e gli insiemi della forma $(0,\frac1n)$. Inoltre in entrambi i casi l'intersezione di tutti i termini \`e vuota.
\end{example}

\begin{example}[Sottoinsiemi compatti non sono necessariamente chiusi]\label{SottoinsiemeCompattoNonChiuso}
Si consideri $\{1,2\}$ con la topologia $\{\emptyset, \{1,2\}\}$. Chiaramente $\{1\}$ \`e un sottospazio compatto ma non \`e un chiuso.


Un esempio meno particolare \`e dato dalla topologia cofinita su $\N$, infatti anche in questo spazio ogni sottoinsieme \`e compatto. Infatti se $A\subseteq \R$ e $\{U_i\}$ \`e un ricoprimento aperto, osservo che $|A\bs U_{i}|\in\N$ per ogni $i$, dunque fissato un primo insieme bastano finiti altri per finire di coprire $A$.
\end{example}

\begin{example}[Spazio compatto ma non compatto per successioni]\label{CompattoNonCompattoPerSuccessioni}
Lo spazio $X=[0,1]^{[0,1]}$ con la topologia della convergenza puntuale \`e compatto ma non compatto per successioni.
\end{example}
\begin{example}[Spazio compatto per successioni ma non compatto]\label{CompattoPerSuccessioniNonCompatto}
Sia $X=[0,1]^{[0,1]}$ e per ogni $f\in X$ sia $\supp(f)=\{x\in [0,1]\mid f(x)\neq0\}$ il supporto di $f$. Poniamo
\[Y=\{f\in X\mid \supp(f)\text{ \`e al pi\`u numerabile}\}.\]
Si ha che $Y$ \`e compatto per successioni ma non compatto con la topologia di sottospazio.
\end{example}

\begin{example}[Spazio limitato, completo ma non compatto]\label{LimiatoCompletoNonCompatto}
Sia $X$ un insieme infinito dotato della distanza discreta. Questo \`e uno spazio metrico limitato e completo ma non compatto.
\end{example}

\begin{example}[Ricoprimento senza numero di Lebesgue]\label{RicoprimentoSenzaNumeroDiLebesgue}
Consideriamo $X=\R$ e $\Omega=\{B(x,\frac1{x^2+1}\mid x\in \R)\}$. Chiaramente $\Omega$ \`e un ricoprimento aperto di $X$ perch\'e ho una palla per ogni punto, ma per $|x|\to\infty$ si ha che il raggio delle palle diminuisce quindi se $\e$ fosse un ipotetico numero di Lebesgue possiamo trovare un elemento del ricoprimento che non contiene alcuna palla di raggio $\e$ spostandoci abbastanza.
\end{example}

\begin{example}[Funzione continua non estendibile alla chiusura del dominio]\label{ContinuaNonEstendibileAllaChiusura}
Siano $X=Y=\R$, $A=(0,+\infty)$ e $f(x)=\frac1x$. Osserviamo che non possiamo estendere $f$ a $[0,+\infty)$ in modo continuo, infatti porre $f(0)\in \R$ creerebbe una discontinuit\`a in $0$.
\end{example}

\begin{example}[Funzione continua ma non uniformemente continua]\label{ContinuaNonUniformementeContinua}
Sia $X=(0,+\infty)$ e definiamo $f:X\to\R$ con $f(x)=x^2$. Affermiamo che $f$ è continua ma non uniformemente continua.
\end{example}

\subsubsection{Quozienti}
\begin{example}[Identificazione n\'e aperta n\'e chiusa]\label{IdentificazioneNeApertaNeChiusa}
Sia $X=\{(x,y)\mid x\geq 0\}\cup \{(x,0)\}\subseteq \R^2$ e consideriamo la mappa \[\pi:\funcDef{X}{\R}{(x,y)}{x}.\]
Osservo che $\pi$ non \`e n\'e aperta n\'e chiusa, per esempio $B((0,1),\frac12)$ \`e aperto in $X$ ma la sua immagine tramite $\pi$ \`e $[0,\frac12)$ che non \`e aperto. Similmente l'immagine di $\{(x,y)\mid xy=1\}$, che \`e un chiuso, \`e $(0,+\infty)$ che non \`e chiusa.

Ma $\pi$ \`e una identificazione: \`e chiaramente surgettiva e in quanto restrizione di una proiezione di $\R^2\to \R$ si ha che $\pi$ \`e continua. Se $C\subseteq \R$ \`e tale che $\pi\ii(C)$ \`e chiuso in $X$ osserviamo che $C$ \`e chiuso per successione, che in uno spazio metrico \`e equivalente ad essere chiuso. Infatti se $\{x_n\}$ \`e una successione a valori in $C$ convergente in $\R$ ($x_n\to \ol x$) considero $(x_n,0)\in \pi\ii(C)$. Dato che $\pi\ii(C)$ \`e chiuso e $(x_n,0)\to (\ol x,0)$ si deve avere che $(\ol x, 0)\in \pi\ii(C)$ e quindi $\ol x\in C$.
\end{example}

\begin{example}[Proiezione a quoziente per azione non chiusa]\label{ProiezioneQuozienteAzioneNonChiusa}
Considerando l'azione $\Z\acts \R$ data dalla traslazione si ha che
\[\pi:\R\to \quot\R\Z\]
non \`e una mappa chiusa.
\end{example}

\begin{example}[$\R$ quoziente $\Q$]\label{RQuozienteQ}
Consideriamo l'azione $\Q\acts \R$ di traslazione. Si ha che $\quot \R\Q$ non \`e $T_1$.
\end{example}

\begin{example}[$M(2,\R)$ quoziente similitudine]\label{Matrici2x2QuozienteSimilitudine}
Consideriamo lo spazio delle matrici $X=M(2,\R)\cong \R^4$ e definiamo l'azione di $GL(2,\R)\acts X$ come
\[\ell_P(A)=PAP\ii.\]
Si ha che $\quot X{GL(2,\R)}$ non \`e $T_1$.
\end{example}

\begin{example}[Spazio localmente euclideo ma non $T_2$]\label{LocalmenteEuclideoNonT2}
Lo spazio quoziente definito da $X=\R\times \{1,-1\}$ con la relazione $(x,\e)\sim(y,\e')\coimplies x=y\neq0$ oppure $(x,\e)=(y,\e')$, \`e localmente euclideo ma non $T_2$. Questo spazio si chiama \textbf{retta con due origini}.
\end{example}


\end{multicols*}

\section{Appendice al capitolo 2}

\begin{multicols*}{2}

\subsection{Esempi e controesempi}

\subsubsection{Spazi topologici}
\begin{example}[Distanze non topologicamente equivalenti]\label{DistanzeNonTopEquivalenti}
Le distanze $d_1$ e $d_\infty$ su $C([0,1])$ non sono topologicamente equivalenti.
\end{example}


\begin{example}[Punto aderente non di accumulazione]\label{AderenteNonAccumulazione}
Siano $X=\R$ e $Z=[0,1)\cup \{2\}$. Osserviamo che $2$ \`e aderente ma non di accumulazione. Sempre da questo esempio notiamo che $1$ \`e un punto di accumulazione che non appartiene a $Z$.
\end{example}

\subsubsection{Assiomi di numerabilit\`a}
\begin{example}[Insieme $I-$numerabile ma non $II-$numerabile]\label{INumerabileNonIINumerabile}
Sia $X$ un insieme  pi\`u che numerabile dotato della topologia discreta. Esso \`e chiaramente I-numerabile in quanto metrizzabile e un SFI per $x_0$ \`e $\{x_0\}$. Osserviamo per\`o che lo spazio non \`e secondo numerabile perch\'e ogni base deve contenere i singoletti (aperti scrivibili solo come unione di se stessi ed eventualmente $\emptyset$) e ce ne sono di una quantit\`a pi\`u che numerabile.
\end{example}

\subsubsection{Prodotti}
\begin{example}[Prodotto di $I-$numerabili non $I-$numerabile]\label{ProdottoDiINumerabileNonINumerabile}
Sia $I$ tale che $|I|>|\N|$ e per ogni $i\in I$ sia $X_i=\{0,1\}$ con la topologia discreta. Si ha che $X=\prod_{i\in I}X_i$ non \`e $I-$numerabile.
\end{example}
\begin{remark}
L'esempio precedente mostra anche che il prodotto pi\`u che numerabile di spazi metrizzabili pu\`o non essere metrizzabile.
\end{remark}


\begin{example}[Le proiezioni non sono sempre chiuse]\label{ProiezioniNonSempreChiuse}
Consideriamo $C=\{(x,y)\in\R^2\mid xy=1\}$: esso \`e preimmagine di $1$ tramite la mappa continua $(x,y)\mapsto xy$, quindi \`e un chiuso in quanto preimmagine di un chiuso tramite una mappa continua, eppure $\pi_1(C)=\R\nz$ che non \`e chiuso in $\R$.
\end{example}

\subsubsection{Assiomi di separazione}
\begin{proposition}\label{T1NonT2-T0NonT1}
Le implicazioni $T_2\implies T_1\implies T_0$ sono strette.
\end{proposition}

\begin{example}[Spazio $T_4$ non $T_0$]\label{SpazioT4NonT0}
La topologia indiscreta su $X$ di cardinalit\`a almeno $2$ rende lo spazio vuotamente sia $T_3$ che $T_4$, ma come sappiamo questo tipo di spazio non \`e $T_0$
\end{example}

\begin{definition}[Retta di Sorgenfrey]
La \textbf{retta di Sorgenfrey} \`e $\R$ dotato della topologia con la seguente base:
\[\{[a,b)\mid a<b,\ a,b\in\R\}.\]
\end{definition}
\begin{remark}
La topologia di Sorgenfrey \`e pi\`u fine della topologia euclidea, infatti posso scrivere $(a-\e,a+\e)$ come unione di aperti di Sorgenfrey come segue: sia $x_n\to a-\e$ una successione monotona contenuta in $(a-\e,a+\e)$. Segue che
\[(a-\e,a+\e)=\bigcup_{i}[x_i,a+\e).\]
\end{remark}

\begin{example}\label{RettaDiSorgenfreyENormale}
La retta di Sorgenfrey \`e normale.
\end{example}

\begin{lemma}\label{T4SeparabileAlloraSottoinsiemeDiscretoChiusoHaCardinalitaMinoreDelContinuo}
Sia $Z$ uno spazio $T_4$ e separabile. Se $D$ \`e un suo sottoinsieme chiuso e discreto allora $|D|<|\R|$.
\end{lemma}

\begin{example}[Piano di Sorgenfrey]\label{PianoDiSorgenfrey}
Il piano di Sorgenfrey, cio\`e il prodotto di due rette di Sorgenfrey, non \`e $T_4$.
\end{example}

\begin{example}[Spazio Hausdorff non regolare]\label{EsempioT2NonRegolare}
Consideriamo $\R$ dotato della topologia generata dagli aperti euclidei e $\R\bs\{\frac1n\mid n\in\N\nz\}$.
\end{example}

\subsubsection{Ricoprimenti}
\begin{example}[Ricoprimento chiuso non fondamentale]\label{RicoprimentoChiusoNonFondamentale}
Su $\R$ il ricoprimento $\{\{x\}\}_{x\in \R}$ non \`e fondamentale.
\end{example}

\subsubsection{Connessi}
\begin{example}[Connesso ma non connesso per archi]\label{ConnessoNonConnessoPerArchi}
L'insieme $Y=\{(0,0)\}\cup\{(x,\sin(1/x))\mid x>0,x\in \R\}\subseteq \R^2$ \`e connesso ma non connesso per archi
\end{example}

\begin{example}[Pettine infinito]\label{PettineInfinito}
Sia $X=(\R\times\{0\})\cup(\Q\times \R)\subseteq \R^2$. Dotato della topologia di sottospazio di $\R^2$, $X$ \`e uno spazio connesso per archi ma non localmente connesso per archi.
\end{example}

\begin{example}[Insieme con parti connesse non aperte]\label{PartiConnesseAperte}
$\Q$ \`e totalmente sconnesso ma i singoletti non sono aperti.
\end{example}

\begin{example}[Insieme con parti conn. per archi n\'e aperte n\'e chiuse]\label{PartiConnessePerArchiNeAperteNeChiuse}
$Y=\{(0,0)\}\cup\{(x,\sin(1/x)\mid x>0)\}\subseteq\R^2$. Sappiamo che i due pezzi specificati sono connessi per archi ma l'insieme non \`e connesso per archi, dunque questa \`e la partizione. Osserviamo che $\{(0,0)\}$ non \`e aperto e $\{(x,\sin(1/x)\mid x>0)\}$ non \`e chiuso.
\end{example}

\subsubsection{Compattezza}
\begin{example}[$\R^n$ non \`e compatto]\label{RnNonECompatto}
$\R^n$ non \`e compatto, infatti il ricoprimento $\{B(x_0,n)\}_{n\in \N}$ non ammette un sottoricoprimento finito (se $m$ fosse il massimo raggio allora tutti gli elementi fuori da $B(x_0,m)$ non sarebbero coperti).
\end{example}

\begin{example}[Famiglie che godono della propriet\`a dell'intersezione finita]\label{EsempioIntersezioneFinita}
Se $X=\R$ gli insiemi della forma $[n,+\infty)$ godono della propriet\`a dell'intersezione finita. Similmente per $X=[0,1]$ e gli insiemi della forma $(0,\frac1n)$. Inoltre in entrambi i casi l'intersezione di tutti i termini \`e vuota.
\end{example}

\begin{example}[Sottoinsiemi compatti non sono necessariamente chiusi]\label{SottoinsiemeCompattoNonChiuso}
Si consideri $\{1,2\}$ con la topologia $\{\emptyset, \{1,2\}\}$. Chiaramente $\{1\}$ \`e un sottospazio compatto ma non \`e un chiuso.


Un esempio meno particolare \`e dato dalla topologia cofinita su $\N$, infatti anche in questo spazio ogni sottoinsieme \`e compatto. Infatti se $A\subseteq \R$ e $\{U_i\}$ \`e un ricoprimento aperto, osservo che $|A\bs U_{i}|\in\N$ per ogni $i$, dunque fissato un primo insieme bastano finiti altri per finire di coprire $A$.
\end{example}

\begin{example}[Spazio compatto ma non compatto per successioni]\label{CompattoNonCompattoPerSuccessioni}
Lo spazio $X=[0,1]^{[0,1]}$ con la topologia della convergenza puntuale \`e compatto ma non compatto per successioni.
\end{example}
\begin{example}[Spazio compatto per successioni ma non compatto]\label{CompattoPerSuccessioniNonCompatto}
Sia $X=[0,1]^{[0,1]}$ e per ogni $f\in X$ sia $\supp(f)=\{x\in [0,1]\mid f(x)\neq0\}$ il supporto di $f$. Poniamo
\[Y=\{f\in X\mid \supp(f)\text{ \`e al pi\`u numerabile}\}.\]
Si ha che $Y$ \`e compatto per successioni ma non compatto con la topologia di sottospazio.
\end{example}

\begin{example}[Spazio limitato, completo ma non compatto]\label{LimiatoCompletoNonCompatto}
Sia $X$ un insieme infinito dotato della distanza discreta. Questo \`e uno spazio metrico limitato e completo ma non compatto.
\end{example}

\begin{example}[Ricoprimento senza numero di Lebesgue]\label{RicoprimentoSenzaNumeroDiLebesgue}
Consideriamo $X=\R$ e $\Omega=\{B(x,\frac1{x^2+1}\mid x\in \R)\}$. Chiaramente $\Omega$ \`e un ricoprimento aperto di $X$ perch\'e ho una palla per ogni punto, ma per $|x|\to\infty$ si ha che il raggio delle palle diminuisce quindi se $\e$ fosse un ipotetico numero di Lebesgue possiamo trovare un elemento del ricoprimento che non contiene alcuna palla di raggio $\e$ spostandoci abbastanza.
\end{example}

\begin{example}[Funzione continua non estendibile alla chiusura del dominio]\label{ContinuaNonEstendibileAllaChiusura}
Siano $X=Y=\R$, $A=(0,+\infty)$ e $f(x)=\frac1x$. Osserviamo che non possiamo estendere $f$ a $[0,+\infty)$ in modo continuo, infatti porre $f(0)\in \R$ creerebbe una discontinuit\`a in $0$.
\end{example}

\begin{example}[Funzione continua ma non uniformemente continua]\label{ContinuaNonUniformementeContinua}
Sia $X=(0,+\infty)$ e definiamo $f:X\to\R$ con $f(x)=x^2$. Affermiamo che $f$ è continua ma non uniformemente continua.
\end{example}

\subsubsection{Quozienti}
\begin{example}[Identificazione n\'e aperta n\'e chiusa]\label{IdentificazioneNeApertaNeChiusa}
Sia $X=\{(x,y)\mid x\geq 0\}\cup \{(x,0)\}\subseteq \R^2$ e consideriamo la mappa \[\pi:\funcDef{X}{\R}{(x,y)}{x}.\]
Osservo che $\pi$ non \`e n\'e aperta n\'e chiusa, per esempio $B((0,1),\frac12)$ \`e aperto in $X$ ma la sua immagine tramite $\pi$ \`e $[0,\frac12)$ che non \`e aperto. Similmente l'immagine di $\{(x,y)\mid xy=1\}$, che \`e un chiuso, \`e $(0,+\infty)$ che non \`e chiusa.

Ma $\pi$ \`e una identificazione: \`e chiaramente surgettiva e in quanto restrizione di una proiezione di $\R^2\to \R$ si ha che $\pi$ \`e continua. Se $C\subseteq \R$ \`e tale che $\pi\ii(C)$ \`e chiuso in $X$ osserviamo che $C$ \`e chiuso per successione, che in uno spazio metrico \`e equivalente ad essere chiuso. Infatti se $\{x_n\}$ \`e una successione a valori in $C$ convergente in $\R$ ($x_n\to \ol x$) considero $(x_n,0)\in \pi\ii(C)$. Dato che $\pi\ii(C)$ \`e chiuso e $(x_n,0)\to (\ol x,0)$ si deve avere che $(\ol x, 0)\in \pi\ii(C)$ e quindi $\ol x\in C$.
\end{example}

\begin{example}[Proiezione a quoziente per azione non chiusa]\label{ProiezioneQuozienteAzioneNonChiusa}
Considerando l'azione $\Z\acts \R$ data dalla traslazione si ha che
\[\pi:\R\to \quot\R\Z\]
non \`e una mappa chiusa.
\end{example}

\begin{example}[$\R$ quoziente $\Q$]\label{RQuozienteQ}
Consideriamo l'azione $\Q\acts \R$ di traslazione. Si ha che $\quot \R\Q$ non \`e $T_1$.
\end{example}

\begin{example}[$M(2,\R)$ quoziente similitudine]\label{Matrici2x2QuozienteSimilitudine}
Consideriamo lo spazio delle matrici $X=M(2,\R)\cong \R^4$ e definiamo l'azione di $GL(2,\R)\acts X$ come
\[\ell_P(A)=PAP\ii.\]
Si ha che $\quot X{GL(2,\R)}$ non \`e $T_1$.
\end{example}

\begin{example}[Spazio localmente euclideo ma non $T_2$]\label{LocalmenteEuclideoNonT2}
Lo spazio quoziente definito da $X=\R\times \{1,-1\}$ con la relazione $(x,\e)\sim(y,\e')\coimplies x=y\neq0$ oppure $(x,\e)=(y,\e')$, \`e localmente euclideo ma non $T_2$. Questo spazio si chiama \textbf{retta con due origini}.
\end{example}
\end{multicols*}

\chapter{Teoria dell'omotopia e Rivestimenti}
\setlength{\parindent}{2pt}

\begin{multicols*}{2}


\section{La categoria hTop}
\begin{definition}[Omotopia]
Siano $f,g:X\to Y$ funzioni continue. Una \textbf{omotopia} tra $f$ e $g$ è una funzione continua
\[H:X\times [0,1]\to Y\]
tale che
\begin{itemize}[noitemsep]
\item $H(x,0)=f(x)$
\item $H(x,1)=g(x)$
\end{itemize}
\end{definition}
\begin{remark}
Stiamo definendo una famiglia di funzioni $H_t=H(\cdot,t):X\to Y$ che possiamo interpretare come ``snapshot" di una trasformazione continua di $f$ in $g$.
\end{remark}

\begin{notation}
Se esiste una omotopia tra $f$ e $g$ scriviamo\footnote{I professori usano la notazione $f\sim g$, ma data l'abbondanza di relazioni di equivalenza in questo corso, e dato che la notazione $\simeq$ è usata in alcuni libri, l'ho preferita per evitare ambiguità nel lettore (e per gusto personale).} $f\simeq g$ e diciamo che $f$ e $g$ sono mappe \textbf{omotope}.
\end{notation}

\begin{proposition}[Omotopia è relazione di equivalenza]\label{OmotopiaTraMappeERelazioneEquivalenza}
La relazione $f\sim g\coimplies f\simeq g$ è una relazione di equivalenza sull'insieme delle funzioni continue da $X$ a $Y$.
\end{proposition}
\begin{notation}
Se $C(X,Y)=\{f:X\to Y\text{ continue}\}$ allora poniamo
\[[X,Y]=\quot{C(X,Y)}\simeq\]
e data $f\in C(X,Y)$ indichiamo la sua classe in $[X,Y]$ con $[f]$ o $[f]_\simeq$ se si presenta ambiguità.
\end{notation}

\begin{proposition}[Composizione passa alla relazione di omotopia]\label{ComposizioneDiOmotopeDaOmotope}
Siano $X,Y,Z$ spazi topologici. Se $f,f':X\to Y$ e $g,g':Y\to Z$ sono mappe continue tali che $f\simeq f'$ e $g\simeq g'$ allora
\[g\circ f\simeq g'\circ f'.\]
\end{proposition}

\noindent Grazie a queste proprietà possiamo dare la definizione degli isomorfismi tra spazi equivalenti per omotopia:
\begin{definition}[Inversa omotopica e Equivalenza omotopica]
Sia $f:X\to Y$ continua. Una \textbf{inversa omotopica} di $f$ è una mappa $g:Y\to X$ continua tale che
\[g\circ f\simeq id_X\qquad f\circ g\simeq id_Y.\]
Se $f$ ammette inversa omotopica diremo che $f$ è una \textbf{equivalenza omotopica} e che $X$ e $Y$ sono \textbf{omotopicamente equivalenti}.
\end{definition}
\begin{remark}
Spazi omeomorfi sono omotopicamente equivalenti.
\end{remark}
\begin{notation}
Se $X$ e $Y$ sono omotopicamente equivalenti scriviamo $X\simeq Y$
\end{notation}


\begin{proposition}[Equivalenza omotopica è una equivalenza]\label{EquivalenzaOmotopicaERelazioneEquivalenza}
La relazione $X\sim Y\coimplies X\simeq Y$ è una relazione di equivalenza.
\end{proposition}


\begin{proposition}[]
Sia $f:X\to Y$ un'identificazione e $Z$ uno spazio localmente compatto, allora la funzione
\[H:\funcDef{X\times Z}{Y\times Z}{(x,t)}{(f(x),t)}\]
\`e un'identificazione.
\end{proposition}
\begin{corollary}[Omotopie passano al quoziente]\label{OmotopiaPassaAlQuoziente}
Se $\sim$ \`e una relazione di equivalenza su $X$ e $H:X\times[0,1]\to Z$ \`e una omotopia tale che $x\sim y\implies H(x,t)=H(y,t)$ allora
\[\ol H:\funcDef{\quot X\sim\times[0,1]}{Z}{([x],t)}{H(x,t)}\]
\`e un'omotopia.
\end{corollary}

\subsection{Funtore delle componenti connesse per archi}
Ricordiamo che $\pi_0$ è un funtore dalla categoria degli spazi topologici $Top$ alla categoria degli insiemi $Set$ (\ref{Pi0EFuntoreDaTopASet}), sappiamo cioè che
\[\pi_0(X)=\{\text{componenti connesse per archi di $X$}\}\]
è un insieme e che $f:X\to Y$ continua induce una mappa $f_\ast=\pi_0(f):\pi_0(X)\to\pi_0(Y)$ ben definita. Vediamo che il funtore e le omotopie sono compatibili, cioè
\begin{theorem}[Mappe omotope inducono la stessa mappa nei $\pi_0$]\label{MappeOmotopeInduconoLaStessaMappaNeiPi0}
Se $f,g:X\to Y$ sono mappe omotope allora inducono la stessa mappa $f_\ast=g_\ast:\pi_0(X)\to\pi_0(Y)$.
\end{theorem}
\noindent
Il teorema pu\`o essere interpretato come la buona definizione del funtore $\pi_0:hTop\to Set$ per quanto riguarda le frecce. Una conseguenza immediata della funtorialit\`a \`e il seguente
\begin{corollary}\label{SpaziOmotopicamenteEquivalentiHannoPi0InBigezione}
Se $X\simeq Y$ allora $\pi_0(X)$ è in bigezione con $\pi_0(Y)$.
\end{corollary}




















\section{Gruppo fondamentale}
In questa sezione definiamo l'oggetto più importante del capitolo, cioè il gruppo fondamentale.
\subsection{Omotopia di cammini}
\begin{notation}
Indichiamo l'insieme dei cammini da $x_0$ a $x_1$ in $X$ come
\[\Omega(X,x_0,x_1).\]
Se $X$ è chiaro dal contesto potremo ometterlo. Se $x_0=x_1$ potremo scrivere $\Omega(X,x_0)$, o addirittura $\Omega(x_0)$.\\
Un cammino tale che $x_0=x_1$ è detto \textbf{laccio} o \textbf{loop}.
\end{notation}
\noindent Per lavorare bene coi cammini notiamo che non possiamo usare le omotopie ``libere" definite precedentemente, infatti
\begin{remark}
Un cammino $\gamma$ è omotopo alla funzione costante $x\mapsto \gamma(0)$. In particolare cammini con un estremo in comune sono sempre omotopi.
\end{remark}
\noindent Per aggirare questo inconveniente diamo la seguente

\begin{definition}[Omotopia di cammini]
Una \textbf{omotopia di cammini} (o \textbf{omotopia a estremi fissati}) fra $\gamma_0, \gamma_1\in\Omega(X,x_0,x_1)$ è una omotopia $H:[0,1]\times[0,1]\to X$ tale che
\begin{enumerate}[noitemsep]
\item $H(x,0)=\gamma_0(x)$ e $H(x,1)=\gamma_1(x)$
\item $H(0,t)=x_0$ e $H(1,t)=x_1$, cioè $H(\cdot,t)\in\Omega(X,x_0,x_1)$.
\end{enumerate}
\end{definition}

\begin{proposition}
Le omotopie di cammini inducono una relazione di equivalenza su $\Omega(X,x_0,x_1)$.
\end{proposition}

\begin{notation}
Se $\gamma_1$ e $\gamma_2$ sono omotope a estremi fissati scriviamo $\gamma_1\simeq \gamma_2$.
Dato che cammini sono sempre omotopi a costanti per omotopie libere, con la notazione sopra intenderemo che esiste una omotopia a estremi fissi che porta l'uno nell'altro se non altrimenti specificato. Se scriviamo che $\gamma_1\simeq \gamma_2$ in $\Omega(X,x_0,x_1)$ allora intendiamo che sono omotope a estremi fissati.
\end{notation}

\begin{theorem}
Se $\gamma_1\simeq\gamma_1'$ in $\Omega(x_0,x_1)$ e $\gamma_2\simeq \gamma_2'$ in $\Omega(x_1,x_2)$, allora $\gamma_1\ast \gamma_2\simeq \gamma_1'\ast \gamma_2'$ in $\Omega(x_0,x_2)$.
\end{theorem}

\subsection{Gruppo Fondamentale}

\begin{definition}[Gruppo Fondamentale]
Sia $X$ uno spazio topologico e sia $x_0\in X$. Il \textbf{gruppo fondamentale} (o \textbf{primo gruppo di omotopia}) di $X$ con \textbf{punto base} $x_0\in X$ è
\[\pi_1(X,x_0)=\quot{\Omega(X,x_0,x_0)}{\simeq},\]
cioè l'insieme delle classi di equivalenza di lacci in $X$ passanti per $x_0$ per la relazione di omotopia a estremi fissati.
\end{definition}

\begin{remark}
$\pi_1(X,x_0)$ dipende solo dalla componente connessa per archi che contiene $x_0$.
\end{remark}

\noindent Mostriamo che il gruppo fondamentale è effettivamente un gruppo con l'operazione di giunzione. Consideriamo prima il seguente lemma
\begin{lemma}[Riparametrizzazioni]\label{RiparametrizzazioniRestituisconoCamminiOmotopi}
Sia $j:[0,1]\to[0,1]$ continua tale che $j(0)=0$ e $j(1)=1$. Allora $\alpha\simeq \al\circ j$ per ogni $\al\in\Omega(X,x_0,x_1)$.
\end{lemma}

\begin{theorem}[Il gruppo fondamentale è un gruppo]
Se $\ast$ è la mappa indotta dalla giunzione in omotopia, si ha che
\[(\pi_1(X,x_0),\ast)\text{ è un gruppo.}\]
\end{theorem}

\begin{definition}[Semplicemente connesso]
Uno spazio topologico $X$ è \textbf{semplicemente connesso} se è connesso per archi e $\pi_1(X,x_0)=\{1\}$ per qualsiasi $x_0\in X$
\end{definition}

\subsection{Cammini chiusi come applicazioni dal cerchio}
Per questa sezione poniamo
\[p:\funcDef{[0,1]}{S^1}{t}{e^{2\pi i t}}\]
dove consideriamo $S^1=\{e^{i\theta}\mid \theta\in[0,2\pi)\}\subseteq \C\cong\R^2$.
\bigskip
\begin{remark}
Esiste una corrispondenza biunivoca tra $\Omega(X,x_0)$ e $\{f:S^1\to X\mid x_0\in \imm f\}$
\end{remark}

\noindent Dalla definizione del gruppo fondamentale sospettiamo che questo ci permetta di determinare quando nello spazio sono presenti ``buchi" (definizioni più precise saranno date dopo). La seguente proposizione comincia a far intravedere questo concetto in modo più formale:
\begin{proposition}[Continua su bordo si estende se e solo se classe di omotopia banale]\label{OmotopiaBanaleEEstensioneDalCerchioAlDisco}
Dato $\al\in\Omega(x_0)$, si ha che
\[[\al]=1\text{ in }\pi_1(X,x_0)\coimplies \wh\al \text{ si estende da $S^1$ a $D^2$.}\]
\end{proposition}

\begin{proposition}
La corrispondenza $\al\mapsto\wh\al$ induce una mappa $\psi:\pi_1(X,x_0)\to[S^1,X]$ ben definita.
\end{proposition}

\begin{lemma}[Cammini complementari sul bordo di convesso]\label{CamminiComplementariSuBordoConvesso}
Sia $D$ un sottoinsieme convesso di $\R^2$ chiuso e limitato. Siano $\gamma_1$ e $\gamma_2$ due archi complementari su $\partial D$ (che per le ipotesi è connesso per archi) che vanno da $p$ a $q$, cioè $\gamma_1(0)=\gamma_2(0)=p,\ \gamma_1(1)=\gamma_2(1)=q$, $\imm \gamma_1\cap \imm\gamma_2=\{p,q\}$ e hanno supporto nel bordo. Allora $\gamma_1$ e $\gamma_2$ sono omotope a estremi fissi.
\end{lemma}

\begin{theorem}[Propriet\`a della corrispondenza tra $\pi_1(X)$ e $\spa{S^1,X}$]
Valgono le seguenti affermazioni:
\begin{itemize}[noitemsep]
\item Se $X$ è connesso per archi la $\psi:\pi_1(X,x_0)\to[S^1,X]$ definita prima è surgettiva.
\item $\psi(g)=\psi(h)$ se e solo se $g$ e $h$ sono coniugati in $\pi_1(X,x_0)$.
\end{itemize}
\end{theorem}

\noindent Riassumiamo i risultati ottenuti in questa sezione nella seguente
\begin{proposition}[Corrispondenza tra omotopie di cammini in $\Omega(X,x_0)$ e omotopie libere in ${[S^1,X]}$]\label{CorrispondenzaTraOmotopieDiCamminiTraLoopEOmotopieLibereDiMappeDefiniteSullaSfere}
Valgono i seguenti fatti:
\begin{itemize}[noitemsep]
\item C'è una corrispondenza biunivoca tra $\Omega(X,x_0)$ e le funzioni continue da $S^1$ a $X$ tali che $x_0$ appartiene all'immagine.
\item Un loop corrisponde alla classe banale nel gruppo fondamentale se e solo se la sua mappa su $S^1$ corrispondente si può estendere in modo continuo a tutto $D^2$
\item La corrispondenza sopra induce $\psi:\pi_1(X,x_0)\to[S^1,X]$ tale che
\begin{itemize}[noitemsep]
\item è costante sulle classi di coniugio del gruppo fondamentale e su classi diverse ha immagine diversa
\item Se $X$ è connesso per archi allora è surgettiva.
\end{itemize}
\end{itemize}
\end{proposition}

\subsection{Funtorialità del gruppo fondamentale}
Abbiamo visto come gli spazi topologici considerati con le funzioni continue formano una categoria. Se consideriamo gli spazi topologici fissando anche un punto, detto \textbf{punto base}, e consideriamo le mappe continue che mandano il punto base nel punto base troviamo un'altra categoria molto simile, che indichiamo con $Top_\ast$. Più formalmente
\begin{definition}[Categoria degli spazi topologici puntati]
Definiamo $Top_\ast$ come la categoria i cui oggetti sono coppie della forma $(X,x_0)$ con $X$ spazio topologico e $x_0\in X$, e i cui morfismi sono mappe continue $f:X\to Y$ che rispettano i punti base, cioè se $f:(X,x_0)\to(Y,y_0)$ allora $f$ è continua e $f(x_0)=y_0$. Gli oggetti di $Top_\ast$ sono detti \textbf{spazi topologici puntati} e i morfismi di $Top_\ast$ sono detti \textbf{mappe continue puntate}.
\end{definition}

\begin{remark}
Se $f:X\to Y$ continua manda $x_0$ in $y_0$ allora se $\al\in\Omega(x_0)$ abbiamo che $f\circ \al\in\Omega(y_0)$. Inoltre, se $\al\simeq \beta$ come cammini allora $f\circ \al\simeq f\circ \beta$, sempre come cammini (basta comporre l'omotopia con $f$).
\end{remark}
\noindent Quanto appena osservato ci dice che $f:X\to Y$ induce una mappa
\[f_\ast:\funcDef{\pi_1(X,x_0)}{\pi_1(Y,f(x_0))}{[\al]}{[f\circ \al]}.\]

\begin{proposition}[Funtore da $Top_\ast$ a $Grp$]\label{FuntorialitaDaMappePuntateAOmomorfismiDeiGruppiFondamentali}
Consideriamo la seguente associazione:
\begin{align*}
(X,x_0)&\mapsto\pi_1(X,x_0)
f:X\to Y&\mapsto f_\ast:\pi_1(X,x_0)\to\pi_1(Y,y_0)
\end{align*}
dove $f_\ast$ indica la mappa indotta da $f$ come sopra.
Si ha che questa associazione è un funtore da $Top_\ast$ a $Grp$, cioè
\begin{itemize}[noitemsep]
\item $id_\ast=id_{\pi_1(X,x_0)}$
\item Se $f:X\to Y$ e $g:Y\to Z$ continue tali che $y_0=f(x_0)$ e $z_0=g(y_0)$, allora
\[(g\circ f)_\ast=g_\ast\circ f_\ast:\pi_1(X,x_0)\to\pi_1(Z,z_0)\]
\item $f_\ast$ è un omomorfismo di gruppi.
\end{itemize}
\end{proposition}
\begin{corollary}
Se $f:X\to Y$ è un omeomorfismo allora $f_\ast:\pi_1(X,x_0)\to\pi_1(Y,y_0)$ è un isomorfismo di gruppi.
\end{corollary}

\noindent Definiamo ora l'analogo delle omotopie su $Top_\ast$:

\begin{definition}[Omotopia puntata]
Se $f,g:(X,x_0)\to(Y,y_0)$ sono continue e puntate definiamo una \textbf{omotopia puntata} da $f$ a $g$ una omotopia $H$ da $f$ a $g$ tale che  $H_t(\cdot ,t)$ è una mappa continua puntata da $(X,x_0)$ a $(Y,y_0)$, cioè  $H(x_0,t)=y_0$ per ogni $t\in[0,1]$.
\end{definition}

\begin{proposition}[Mappe omotope puntate inducono la stessa mappa sui gruppi fondamentali]\label{MappeOmotopePuntateInduconoStessaMappaSuGruppiFondamentali}
Siano $f,g:(X,x_0)\to(Y,y_0)$ mappe continue puntate omotope tramite una omotopia puntata $H$. Allora $f_\ast=g_\ast$ come mappe da $\pi_1(X,x_0)\to\pi_1(Y,y_0)$.
\end{proposition}

\noindent
Abbiamo quindi mostrato che $\pi_1$ \`e un funtore da $hTop_\ast$ a $Grp$.

\subsection{Dipendenze del gruppo fondamentale}

\begin{theorem}[Il punto base determina $\pi_1(X)$ a meno di isomorfismo]\label{PuntoBaseDeterminaPi1GruppoFondamentaleAMenoDiIsomorfismo}
Sia $X$ connesso per archi e siano $x_0,x_1\in X$. Ogni cammino $\gamma:[0,1]\to X$  da $x_0$ a $x_1$ induce un isomorfismo di gruppi $\gamma_\sharp:\pi_1(X,x_1)\to\pi_1(X,x_0)$.
\end{theorem}

\begin{remark}
L'isomorfismo dipende dal cammino.
\end{remark}

\begin{proposition}[Mappa omotopa all'identità induce isomorfismo]
Sia $f:X\to X$ una mappa omotopa all'identità e sia $x_0\in X$. Allora $f_\ast:\pi_1(X,x_0)\to\pi_1(X,f(x_0))$ è un isomorfismo di gruppi.
\end{proposition}

\begin{corollary}[Invarianza omotopica]\label{InvarianzaOmotopicaDelGruppoFondamentale}
Sia $f:X\to Y$ una equivalenza omotopica e sia $x_0\in X$. Si ha che $f_\ast:\pi_1(X,x_0)\to\pi_1(Y,f(x_0))$ è un isomorfismo. In particolare spazi omotopi hanno $\pi_1$ isomorfo.
\end{corollary}


\section{Spazi contraibili e retratti}
\subsection{Spazi contraibili}
Studiamo la classe banale per equivalenza omotopica:
\begin{definition}[Spazio contraibile]
Uno spazio $X$ è \textbf{contraibile} se è omotopicamente equivalente ad un punto.
\end{definition}

\noindent Uno degli esempi più comuni di spazi contraibili sono i seguenti:
\begin{definition}[Insieme stellato]
Un insime $\Omega\subseteq \R^n$ è \textbf{stellato} rispetto a $x_0\in \Omega$ se
\[\forall x\in\Omega,\ [x,x_0]\subseteq \Omega,\]
dove $[x,x_0]=\{tx+(1-t)x_0\mid t\in[0,1]\}$ indica il segmento con estremi $x$ e $x_0$.
\end{definition}
\begin{definition}[Insieme Convesso]
Un insime $\Omega\subseteq \R^n$ è \textbf{convesso} se è stellato rispetto a ogni suo punto, cioè se
\[\forall x,y\in \Omega,\ [x,y]\subseteq \Omega.\]
\end{definition}
\begin{remark}
Ogni insieme convesso è stellato, ma non ogni insieme stellato è convesso.
\end{remark}
\begin{remark}
Un insieme stellato è connesso per archi, infatti ogni punto appartiene alla classe di $x_0$ nel $\pi_0$.
\end{remark}

\begin{proposition}[Mappe a immagine in stellato]\label{MappeAImmagineInStellato}
Se $\Omega\subseteq \R^n$ è stellato e $X$ è uno spazio topologico qualsiasi allora tutte le mappe continue da $X$ a $\Omega$ sono omotope.
\end{proposition}
\begin{corollary}\label{StellatiSOnoContraibili}
Gli insiemi stellati sono contraibili.
\end{corollary}

\begin{remark}
Dato che $\R^n$ è stellato rispetto a $0$, esso è contraibile.
\end{remark}

\subsection{Retratti di Deformazione}
Nel definire un ``retratto" di $X$ potremmo pensare alla seguente
\begin{definition}[Retratto]
Sia $X$ uno spazio topologico e sia $Y\subseteq X$. $Y$ è un \textbf{retratto} di $X$ se esiste una mappa $r:X\to Y$ continua (detta \textbf{retrazione}) tale che $r(y)=y$ per ogni $y\in Y$.
\end{definition}

\begin{proposition}[Proprietà dei retratti]
Sia $Y\subseteq X$ un retratto e sia $r:X\to Y$ una retrazione. Si ha che
\begin{itemize}[noitemsep]
\item Se $X$ è $T_2$ allora $Y$ è chiuso.
\item Chiamando $i:Y\inj X$ l'inclusione si ha che $i_\ast:\pi_1(Y,y_0)\to\pi_1(X,y_0)$ è iniettiva.
\end{itemize}
\end{proposition}
\noindent
Purtroppo questa definizione non è molto significativa da sola, come ci mostra la seguente
\begin{remark}
Ogni punto $x_0\in X$ è un retratto di $X$.
\end{remark}

\noindent Le omotopie ci permettono di catturare meglio il concetto che volevamo descrivere:
\begin{definition}[Retratto di deformazione]
Sia $X$ uno spazio topologico e $Y\subseteq X$. $Y$ è un \textbf{retratto di deformazione} di $X$ se esiste una mappa $H:X\times[0,1]\to X$ continua tale che
\begin{enumerate}[noitemsep]
\item $H(x,0)=x$ per ogni $x\in X$,
\item $H(x,1)\in Y$ per ogni $x\in X$
\item $H(y,t)=y$ per ogni $y\in Y$ e per ogni $t\in [0,1]$,
\end{enumerate}
chiediamo cioè che esista una omotopia tra $id_X$ e una retrazione $r:X\to Y$ che fissi $Y$ ad ogni istante.
$H(\cdot,1):X\to Y\subseteq X$ è detta \textbf{retrazione di deformazione}.
\end{definition}
\begin{remark}
Se $Y$ è un retratto di deformazione di $X$ allora $Y\simeq X$
\end{remark}
\begin{remark}
Non tutti i retratti sono retratti di deformazione, infatti se $\#\pi_0(X)>1$ allora un punto è un retratto di $X$ ma non un retratto di deformazione perché $\#\pi_0(X)\neq\#\pi_0(\{pt.\})=1$ (\ref{SpaziOmotopicamenteEquivalentiHannoPi0InBigezione}).
\end{remark}
\begin{remark}
$S^n$ è un retratto di deformazione di $\R^{n+1}\nz$.
\end{remark}


\section{Rivestimenti}
\subsection{Omeomorfismi locali}
\begin{definition}[Omeomorfismo Locale]
Una funzione $f:X\to Y$ è un \textbf{omeomorfismo locale} se per ogni $x\in X$ esiste un intorno aperto $U$ di $x$ tale che $f(U)$ è aperto in $Y$ e $f\res U:U\to f(U)$ è un omeomorfismo.
Le mappe $s:f(U)\to U$ definite come $f\res{U}\ii$ sono dette \textbf{sezioni} di $f$.
\end{definition}

\begin{remark}\label{ContinuaIniettivaEApertaImplicaOmeomorfismoLocale}
Una mappa continua, iniettiva e aperta è un omeomorfismo locale.
\end{remark}

\begin{proposition}[Omeomorfismo locale implica aperta]
Se $f:X\to Y$ è un omeomorfismo locale allora $f$ è una mappa aperta.
\end{proposition}

\begin{remark}
La restrizione di un omeomorfismo locale ad un aperto è un omeomorfismo locale.
\end{remark}


\subsection{Rivestimenti}
\begin{definition}[Rivestimento]
Una funzione continua $p:E\to X$ è un \textbf{rivestimento} se
\begin{enumerate}[noitemsep]
\item $X$ è connesso
\item Per ogni $x\in X$ esiste un intorno $U$ di $x$ aperto, detto intorno \textbf{ben rivestito}, tale che
\[p\ii(U)=\bigsqcup_{i\in I}W_i,\]
dove per ogni $i\in I$ abbiamo che $W_i$ è aperto in $E$ e che $p\res {W_i}:W_i\to U$ è un omeomorfismo.
\end{enumerate}
In questa definizione, $X$ è detto \textbf{spazio base}, mentre $E$ è detto \textbf{spazio totale}.
\end{definition}

\begin{proposition}[Rivestimento implica Omeomorfismo locale]\label{RivestimentoImplicaOmeomorfismoLocale}
Se $p:E\to X$ è un rivestimento allora è anche un omeomorfismo locale.
\end{proposition}

\begin{definition}[Fibra]
Se $p:E\to X$ è un rivestimento e $x_0\in X$ chiamiamo $p\ii(x_0)\subseteq E$ la \textbf{fibra} di $x_0$.
\end{definition}

\begin{theorem}[Teorema delle Fibre]\label{TeoremaFibre}
Sia $p:E\to X$ un rivestimento e siano $x,y\in X$. Si ha che $|p\ii(x)|=|p\ii(y)|$, cioè le fibre hanno cardinalità costante.
\end{theorem}

\noindent Il teorema ci permette di definire una quantità importate per i rivestimenti:
\begin{definition}[Grado di un rivestimento]
Dato un rivestimento $p:E\to X$, chiamiamo \textbf{grado} del rivestimento la cardinalità di una qualsiasi fibra.
\end{definition}

\begin{remark}
Il teorema delle fibre (\ref{TeoremaFibre}) può essere riformulato in ``Il grado è definito per ogni rivestimento".
\end{remark}

\begin{remark}[I rivestimenti sono surgettivi]\label{RivestimentiSonoSurgettivi}
Se $p:E\to X$ è un rivestimento e $E\neq \emptyset$ allora $p$ è surgettivo.
\end{remark}

\begin{definition}[Rivestimento banale]
Un rivestimento è \textbf{banale} se ha grado 1.
\end{definition}

\begin{remark}
Un rivestimento è un omeomorfismo locale surgettivo, ma non vale l'implicazione opposta.
\end{remark}
\begin{example}
La retta con doppia origine e $\funcDef{(0,5)}{S^1}{t}{e^{2\pi it}}$ sono omeomorfismi locali surgettivi ma non sono rivestimenti, per esempio perché il grado non è ben definito.
\end{example}

\noindent Concludiamo la sezione fornendo un modo per trovare rivestimenti
\begin{theorem}[Rivestimento da azione propriamente discontinua]\label{ProiezioneQuozienteAzionePropriamenteDiscontinuaERivestimento}
Sia $G\acts X$ una azione propriamente discontinua tale che $\quot XG$ sia connesso. Allora $\pi:X\to\quot XG$ è un rivestimento.
\end{theorem}

\begin{remark}[Grado rivestimenti derivanti da propriamente discontinue]\label{GradoRivestimentiDaAzioniPropriamenteDiscontinue}
Un'azione propriamente discontinua è libera, quindi le orbite sono in bigezione con $G$. Dato che le orbite corrispondono a fibre, il grado di un rivestimento ottenuto da una azione propriamente discontinua è $|G|$.
\end{remark}

\begin{remark}[Rivestimento di proiettivi reali]
Consideriamo la proiezione che identifica gli antipodali
\[\pi:S^n\to\quot{S^n}{\{\pm id\}}\cong \Pj^n\R.\]
L'azione $\znz2\acts S^n$ corrispondente è propriamente discontinua e come sappiamo $\Pj^n\R$ è connesso. Segue che l'identificazione antipodale è un rivestimento di grado 2 di $\Pj^n\R$.
\end{remark}


\subsection{Sollevamenti}
I rivestimenti ci permettono di prendere funzioni a valori nello spazio base e di ``sollevarle" a funzioni nello spazio totale. Questo ci permette di evidenziare differenze difficili da esplicitare nello spazio base.
\begin{definition}[Sollevamento]
Sia $p:E\to X$ un rivestimento e sia $f:Y\to X$ una funzione continua data. Un \textbf{sollevamento} di $f$ è una funzione continua $\wt f:Y\to E$ tale che $f=p\circ \wt f$, cioè fa commutare il diagramma
\[\begin{tikzcd}
	& E \\
	Y & X
	\arrow["f", from=2-1, to=2-2]
	\arrow["p", from=1-2, to=2-2]
	\arrow["{\wt f}", dashed, from=2-1, to=1-2]
\end{tikzcd}\]
\end{definition}

\begin{theorem}[Unicità del sollevamento]\label{UnicitaSollevamenti}
Siano $p:E\to X$ un rivestimento e $f:Y\to X$ funzione continua fissata con $Y$ connesso. Se $\wt f_1$ e $\wt f_2$ sono due sollevamenti di $f$ che coincidono in un punto allora $\wt f_1=\wt f_2$.
\end{theorem}


\begin{theorem}[Esistenza e unicità del sollevamento dei cammini]\label{EsistenzaUnicitaSollevamentoCammini}
Sia $\gamma:[0,1]\to X$ un cammino tale che $\gamma(0)=x_0$ e sia $p:E\to X$ un rivestimento. Allora per ogni $\wt x_0\in p\ii(x_0)$ esiste un unico sollevamento $\wt \gamma:[0,1]\to E$ di $\gamma$ tale che $\wt \gamma(0)=\wt x_0$.
\end{theorem}


\begin{notation}
Se $p:E\to X$ è un rivestimento, $\al:[0,1]\to X$ è un cammino e $\wt x_0$ è un punto della fibra di $\al(0)$ allora indichiamo con $\wt \al_{\wt x_0}$ l'unico sollevamento di $\al$ che parte da $\wt x_0$.
\end{notation}

\begin{theorem}[Sollevamento dell'omotopia]\label{TeoremaSollevamentoOmotopia}
Siano $p:E\to X$  un rivestimento, $f:Y\to X$ continua, $\wt f:Y\to E$ un sollevamento di $f$ e $H:Y\times [0,1]\to X$ una omotopia tale che $H(\cdot,0)=f$. Allora esiste un (unico) sollevamento $\wt H:Y\times[0,1]\to E$ di $H$ ($p\circ \wt H=H$) tale che $\wt H(\cdot, 0)=\wt f$.
\end{theorem}

\begin{theorem}[Sollevamento delle omotopie di cammini]\label{SollevamentoOmotopieDiCammini}
Sia $H:[0,1]\times[0,1]\to X$ un'omotopia di cammini da $\al$ a $\beta$ in $\Omega(X,x_0,x_1)$. Sia $p:E\to X$ un rivestimento e fissiamo $\wt x_0\in p\ii(x_0)$. Allora $H$ si solleva ad una omotopia di cammini da $\wt \al_{\wt x_0}$ a $\wt \beta_{\wt x_0}$. In particolare se $\al\simeq \beta$ come cammini allora $\wt\al_{\wt x_0}$ e $\wt \beta_{\wt x_0}$ hanno lo stesso punto finale.
\end{theorem}

\begin{corollary}[Mappa sui $\pi_1$ indotta da rivestimento]\label{MappaIndottaDaRivestimentoSuGruppoFondamentaleEIniettiva}
Se $p:E\to X$ è un rivestimento allora $p_\ast:\pi_1(E,\wt x_0)\to \pi_1(X,x_0)$ è iniettiva (abbiamo scelto $\wt x_0$ e $x_0$ tali che $p(\wt x_0)=x_0$).
\end{corollary}

\begin{remark}[Il cerchio non è semplicemente connesso]
$\pi_1(S^1)\neq 0$.
\end{remark}




\section{Azione di Monodromia}
In questa sezione trattiamo una azione del gruppo fondamentale. Questa sarà una azione destra, cioè rispetta i soliti assiomi di una azione eccetto la seguente differenza nell'associatività:
\[\under{\text{azione sinistra}}{(gh)\cdot x=g\cdot(h\cdot x)=g(h(x))}\qquad \under{\text{azione destra}}{x\cdot (gh)=(x\cdot g)\cdot h=h(g(x))}.\]

\begin{definition}[Azione di monodromia]
Sia $p:E\to X$ un rivestimento, $x_0\in X$ e sia $F=p\ii(x_0)$ la fibra di $x_0$. Definiamo \textbf{l'azione di monodromia} di $\pi_1(X,x_0)$ su $F$ come segue:
\[\funcDef{F\times\pi_1(X,x_0)}{F}{(\wt x,[\al])}{\wt x\cdot[\al]=\wt\al_{\wt x}(1)}\]
\end{definition}
\begin{proposition}
L'azione di monodromia è una azione destra.
\end{proposition}


\begin{theorem}[Proprietà dell'azione di monodromia]\label{ProprietaAzioneMonodromia}
Siano $p:E\to X$ un rivestimento, $x_0\in X$ e  $F=p\ii(x_0)$ la fibra di $x_0$. Si ha che
\begin{enumerate}[noitemsep]
\item Se $X$ \`e connesso per archi l'azione di monodromia $\pi_1(X,x_0)\acts F$ è transitiva se e solo se $E$ è connesso per archi.
\item Per ogni $\wt x\in E$ si ha che
\[\stab(\wt x)=\{[\al]\in\pi_1(X,x_0)\mid \wt x\cdot[\al]=\wt x\}=p_\ast(\pi_1(E,\wt x)).\]
\end{enumerate}
\end{theorem}

\begin{corollary}
Se $X$ ammette rivestimento connesso non banale allora $\pi_1(X,x_0)\neq \{1\}$ per ogni $x_0\in X$, cioè $X$ NON è semplicemente connesso.
\end{corollary}
\begin{corollary}
Ogni rivestimento connesso di uno spazio semplicemente connesso è banale, cioè è un omeomorfismo.
\end{corollary}

\begin{corollary}
$S^1$ e $\Pj^n\R$ non sono semplicemente connessi.
\end{corollary}

\subsubsection{Sollevamento di mappe qualsiasi}

Grazie alle propriet\`a dell'azione di monodromia possiamo dare una caratterizzazione esatta di quando le mappe ammettono sollevamento:
\begin{theorem}[Sollevamento di mappe qualsiasi]\label{SollevamentoMappeQualsiasi}
Sia $p:\wt X\to X$ un rivestimento e fissiamo $x_0\in X$ e $\wt x_0\in p\ii(x_0)$. Sia $f:Y\to X$ continua e $y_0\in Y$ tale che $f(y_0)=x_0$.\\
Supponiamo che $Y$ sia connesso e localmente connesso per archi, allora \[\exists !\wt f:Y\to \wt X\ t.c.\ \wt f(y_0)=\wt x_0,\ p\circ \wt f=f\coimplies f_\ast(\gf Y{y_0})\subseteq p_\ast(\gf{\wt X}{\wt x_0}).\]
\[\begin{tikzcd}
	& {(\wt X,\wt x_0)} \\
	{(Y,y_0)} & {(X,x_0)}
	\arrow["p", from=1-2, to=2-2]
	\arrow["f"', from=2-1, to=2-2]
	\arrow["{\wt f}", dashed, from=2-1, to=1-2]
\end{tikzcd}\]
\end{theorem}








\subsection{Applicazioni dell'azione di Monodromia}
Siamo (finalmente) pronti per calcolare il gruppo fondamentale di $S^1$:
\begin{theorem}[Gruppo fondamentale del cerchio]\label{GruppoFondamentaleCerchio}
$\pi_1(S^1)\cong \Z$ tramite l'isomorfismo
\[\psi:\funcDef{\pi_1(S^1,1)}{\Z}{[\al]}{0\cdot [\al]}\]
dove il punto indica l'azione di monodromia per il rivestimento $p:\R\to S^1$ dato da $p(t)=e^{2\pi it}$.
\end{theorem}
\begin{remark}\label{SeSpazioCompletoSemplicementeConnessoAlloraGruppoFondamentaleIsomorfoAIsomorfismi}
Questo è un caso particolare del seguente risultato, che mostreremo verso la fine del capitolo (\ref{AutomorfismiDelRivestimentoUniversaleSonoIlGruppoFondamentale}):
\begin{center}
Se $p:E\to X$ è un rivestimento con $E$ semplicemente connesso, allora $\pi_1(X)\cong \Aut(p)=\{\gamma:E\to E\mid \gamma\text{ omeomorfismo t.c. }p\circ\gamma=p\}.$
\end{center}
\end{remark}

\begin{corollary}
Se $D\subseteq\R^2$ è semplicemente connesso e $D\supseteq S^1$ allora $S^1$ non è un retratto di $D$.
\end{corollary}


\begin{theorem}[Teorema di Brower]\label{TeoremaBrower}
Sia $f:D^2\to D^2$ continua. Allora $f$ ha un punto fisso, cioè esiste $x\in D$ tale che $f(x)=x$.
\end{theorem}

\begin{remark}
Il teorema di Brower vale per $f:D\to D$ se $D\subseteq \R^n$ con $D$ convesso compatto. L'idea è più o meno la stessa sfruttando l'idea che $S^{n-1}$ non è un retratto di $D^n$, però il $\pi_1$ non basta per mostrare questo fatto.
\end{remark}

\begin{remark}
Esistono $f:\R^n\to \R^n$ continue senza punti fissi. Più in generale esiste $f:\Omega\to\Omega$ continua senza punti fissi anche con $\Omega$ aperto, limitato e contraibile.
\bigskip

\noindent
Esiste $f:C\to C$ continua senza punti fissi con $C$ chiuso illimitato contraibile.
\end{remark}

\section{Teorema di Seifert-Van Kampen}
In questa sezione e la prossima cerchiamo di rispondere alla domanda
\begin{center}
Come calcolo i gruppi fondamentali?
\end{center}
Chiaramente se $X\simeq Y$, per l'invarianza omotopica del gruppo fondamentale (\ref{InvarianzaOmotopicaDelGruppoFondamentale}), $\pi_1(X)\cong \pi_1(Y)$.

\begin{example}
$S^1$ è un retratto di $\R^2\nz$, quindi $\pi_1(\R^2\nz)\cong \pi_1(S^1)\cong \Z$.
\end{example}

\noindent Ma come calcolo il gruppo di $\R^2\bs\{0,1\}$\footnote{Sto indentificando $\R^2$ con $\C$.} per esempio?  Si vede che $\R^2\bs\{0,1\}$ è omotopicamente equivalente a $S^1\vee S^1$, ma come calcolo il gruppo fondamentale di questo Bouquet?
\medskip

\noindent
Un'idea pu\`o essere spezzare lo spazio che ci interessa in pezzi pi\`u gestibili. Questo approccio \`e incapsulato dal seguente


\begin{theorem}[Seifert-Van Kampen]\label{TeoremaVanKampen}
Sia $X$ uno spazio topologico e siano $A,B$ aperti di $X$ tali che $X=A\cup B$. Supponiamo che $A,\ B$ e $A\cap B\neq \emptyset$ siano connessi per archi. Indichiamo le inclusioni come segue
\begin{align*}
&i_A:A\inj X &j_A:A\cap B\inj A\\
&i_B:B\inj X &j_B:A\cap B\inj B
\end{align*}
Fissiamo $x_0\in A\cap B$.\\
Sia $G$ un gruppo e siano $\vp_A:\pi_1(A,x_0)\to G$ e $\vp_B:\pi_1(B,x_0)\to G$ tali che
\[\vp_A\circ (j_A)_\ast=\vp_B\circ (j_B)_\ast:\gf{A\cap B}{x_0}\to G.\]
Allora esiste un unico omomorfismo $\vp:\gf X{x_0}\to G$ tale che $\vp_A=\vp\circ (i_A)_\ast$ e $\vp_B=\vp\circ (i_B)_\ast$, cioè esiste un'unica $\vp$ che fa commutare il seguente diagramma\footnote{Il teorema afferma che $\gf{X}{x_0}$ con le mappe date \`e il coprodotto fibrato di $\gf A{x_0}$ e $\gf B{x_0}$ rispetto a $\gf{A\cap B}{x_0}$ nella categoria $Grp$. In termini pi\`u adatti al corso, vedremo che ne \`e il prodotto amalgamato perch\'e rispetta la propriet\`a universale (\ref{EsistenzaUnicitaProdottoAmalgamato}).}
\[\begin{tikzcd}
	& {\gf A{x_0}} \\
	{\gf{A\cap B}{x_0}} && {\gf X{x_0}} && G \\
	& {\gf B{x_0}}
	\arrow["{(j_A)_\ast}", from=2-1, to=1-2]
	\arrow["{(j_B)_\ast}"', from=2-1, to=3-2]
	\arrow["{(i_B)_\ast}"', from=3-2, to=2-3]
	\arrow["{(i_A)_\ast}", from=1-2, to=2-3]
	\arrow["{\vp_A}", curve={height=-18pt}, from=1-2, to=2-5]
	\arrow["{\vp_B}", curve={height=18pt}, from=3-2, to=2-5]
	\arrow["\vp", dashed, from=2-3, to=2-5]
\end{tikzcd}\]
Inoltre $\gf X{x_0}$ è generato da $(i_A)_\ast(\gf A{x_0})$ e $(i_B)_\ast(\gf B{x_0})$.
\end{theorem}

\begin{corollary}[Divisione in semplicemente connessi con intersezione connessa per archi implica semplicemente connesso]
Sia $X=A\cup B$ con $A,B$ semplicemente connessi e $A\cap B$ connesso per archi. Allora $X$ è semplicemente connesso.
\end{corollary}

\begin{corollary}
$S^n$ è semplicemente connesso per $n\geq 2$.
\end{corollary}




\section{Calcolo del Gruppo fondamentale}
\subsection{Gruppo fondamentale del prodotto}

\begin{theorem}[Gruppo fondamentale del prodotto]\label{GruppoFondamentaleProdotto}
Siano $X,Y$ spazi topologici connessi per archi e siano $x_0\in X,\ y_0\in Y$. Si ha che
\[\pi_1(X\times Y,(x_0,y_0))\cong \pi_1(X,x_0)\times\gf Y{y_0}.\]
\end{theorem}

\subsection{Prodotto libero e gruppi liberi}

\begin{definition}[Prodotto libero]
Dati due gruppi $G,H$, un \textbf{prodotto libero} di $G$ e $H$ \`e un gruppo $K$ con omomorfismi $i_G:G\to K$ e $i_H:H\to K$ tale che per ogni coppia di omomorfismi $\vp_G:G\to Z,\ \vp_H:H\to Z$ esiste un unico $\vp:K\to Z$ tale che $\vp_G=\vp\circ i_G$ e $\vp_H=\vp\circ i_H$.
\[\begin{tikzcd}
	G \\
	& K && Z \\
	H
	\arrow["{i_G}", from=1-1, to=2-2]
	\arrow["{i_H}"', from=3-1, to=2-2]
	\arrow["\vp", dashed, from=2-2, to=2-4]
	\arrow["{\vp_G}", curve={height=-12pt}, from=1-1, to=2-4]
	\arrow["{\vp_H}"', curve={height=12pt}, from=3-1, to=2-4]
\end{tikzcd}\]
\end{definition}

\begin{theorem}[Esistenza del prodotto libero]\label{EsistenzaUnicitaGruppoLibero}
Il prodotto libero $K$ esiste ed \`e unico a meno di isomorfismo. Si denota $K=G\ast H$.
\end{theorem}

\begin{remark}
(NON DATA DURANTE IL CORSO) Il prodotto libero di due gruppi è il loro coprodotto nella categoria $Grp$. Intuitivamente torna abbiamo unito i due gruppi e ne abbiamo considerato il generato imponendo unicamente le relazioni di $H$ e di $G$.
\end{remark}

\begin{fact}[Propriet\`a dei prodotti liberi]
Valgono le seguenti proposizioni
\begin{enumerate}[noitemsep]
\item $G\ast\{1\}\cong G$
\item $G\ast H\cong H\ast G$
\item $(G_1\ast G_2)\ast G_3\cong G_1\ast(G_2\ast G_3)$
\item Se $G\neq \{1\}$ e $H\neq \{1\}$ allora $G\ast H$ non \`e abeliano \footnote{Per esempio, se $g\in G\bs\{1_G\}$ e $h\in H\bs\{1_H\}$ allora $gh\neq hg$ perch\'e forme ridotte diverse.}
\end{enumerate}
\end{fact}

\begin{definition}[Gruppo libero]
Il \textbf{gruppo libero} di rango $n$ \`e il gruppo
\[F_n\cong \under{n\text{ volte}}{\Z\ast\Z\ast\cdots\ast\Z}.\]
Se $x_i$ \`e un generatore dell'$i-$esima copia di $\Z$ nel prodotto libero allora un generico elemento di $F_n$ \`e dato da
\[x_{i_1}^{n_1}x_{i_2}^{n_2}\cdots x_{i_k}^{n_k}\quad \text{con }k\in\N,\ i_k\in\{1,\cdots,n\},\ n_i\in\Z.\]
\end{definition}

\begin{theorem}[Propriet\`a universale del gruppo libero]\label{ProprietaUniversaleGruppoLibero}
Sia $X$ un insieme e sia $f:X\to Z$ una funzione con $Z$ gruppo. Sia $F(X)$ il gruppo libero su $X$, cio\`e
\[F(X)=\bigast_{x\in X}\Z_x,\]
dove $\Z_x$ \`e una copia di $\Z$ generata formalmente da $x$.\\
Allora esiste un unico $\vp:F(X)\to Z$ omomorfismo di gruppi che estende $f$, in particolare quello dato da $\vp(x)=f(x)$, dove interpreto l'argomento di $\vp$ come la stringa data da un solo carattere.
\end{theorem}

\subsection{Van Kampen per intersezioni semplicemente connesse}
\begin{lemma}[Van Kampen per intersezione semplicemente connessa]\label{VanKampenIntersezioneSemplicementeConnessa}
Sia $X=A\cup B$ con $A,B$ aperti connessi per archi, $A\cap B$ connesso per archi e semplicemente connesso. Sia $x_0\in A\cap B$. Allora
\[\gf X{x_0}\cong \gf A{x_0}\ast \gf B{x_0}.\]
\end{lemma}


\begin{theorem}[Gruppo fondamentale del Bouquet di $n$ circonferenze]
Sia $X_n=\under{n\text{volte}}{S^1\vee\cdots\vee S^1}$, allora
\[\pi_1(X_n)=F_n\]
\end{theorem}

\subsection{Prodotto amalgamato}
Vogliamo generalizzare il prodotto libero in modo da poterlo usare nel calcolo anche per $A\cap B$ non semplicemente connesso.
\begin{definition}[Prodotto amalgamato]
Siano $H,G_1,G_2$ gruppi e siano $j_1:H\to G_1$ e $j_2:H\to G_2$ omomorfismi. Il \textbf{prodotto amalgamato} di $G_1$ e $G_2$ lungo $H$ \`e un gruppo $K$ con omomorfismi $i_1:G_1\to K$ e $i_2:G_2\to K$ tali che $i_1\circ j_1=i_2\circ j_2$ tali che per ogni coppia di omomorfismi $\vp_1:G_1\to Z,\vp_2:G_2\to Z$ con $\vp_1\circ j_1=\vp_2\circ j_2$ esiste un unico omomorfismo $\vp:K\to Z$ con $\vp\circ i_1=\vp_1$ e $\vp\circ i_2=\vp_2$.
\[\begin{tikzcd}
	& {G_1} \\
	H && K && Z \\
	& {G_2}
	\arrow["{j_1}", from=2-1, to=1-2]
	\arrow["{j_2}"', from=2-1, to=3-2]
	\arrow["{i_2}"', from=3-2, to=2-3]
	\arrow["{i_1}", from=1-2, to=2-3]
	\arrow["{\vp_1}", curve={height=-12pt}, from=1-2, to=2-5]
	\arrow["{\vp_2}"', curve={height=12pt}, from=3-2, to=2-5]
	\arrow["\vp", dashed, from=2-3, to=2-5]
\end{tikzcd}\]
Spesso si scrive $K=G_1\ast_H G_2$ (attenzione perch\'e $K$ dipende anche da $j_1$ e $j_2$).
\end{definition}
\begin{remark}
Nel caso $H=\{1\}$ si ricade nella definizione del prodotto libero.
\end{remark}

\begin{notation}[Sottogruppo normale generato]
Dato un gruppo $G$ e un sottoinsieme $S$ poniamo
\[\ps{\ps S}=\bigcap_{\smat{Z\normal G\\Z\supseteq S}}Z.\]
\end{notation}

\begin{theorem}[Esistenza e unicit\`a del prodotto amalgamato]\label{EsistenzaUnicitaProdottoAmalgamato}
Il prodotto amalgamato esiste ed \`e unico a meno di isomorfismo.
\end{theorem}

\begin{fact}\label{PerNormaleProdottoAmalgamatoBastaImporreRelazioniSuGeneratori}
Con le notazioni del teorema, se $\Omega$ \`e un insieme di generatori di $H$ allora al posto di $S=\{j_1(h)j_2(h)\ii\mid h\in H\}$ si pu\`o considerare $S'=\{j_1(h)j_2(h)\ii\mid h\in \Omega\}$ e generare comunque $N$.
\end{fact}

\subsection{Presentazioni di gruppi}
\begin{definition}[Presentazione]
Dato un insieme di simboli $S$ e $R\subseteq F(S)$  (dette \textbf{relazioni}) allora
\[G=\ps{S\mid R}=\quot {F(S)}{\ps{\ps{R}}}.\]
La scrittura $\ps{S\mid R}$ \`e detta \textbf{presentazione} di $G$.
\end{definition}
\begin{remark}[Problema della parola]
Chiedersi se una parola $w\in F(S)$ rappresenta l'identit\`a di $\ps{S\mid R}$ \`e equivalente a chiedersi se esistono $j\in\N$, $w_i\in F(S)$ e $r_{k_i}\in R$ tali che
\[w=\prod_{i=1}^jw_ir_i^{\pm 1}w_i\ii.\]
Questo problema in generale non ammette soluzione algoritmica (non possiamo stimare quanti tentativi dovremmo fare).
\end{remark}

\begin{proposition}[Propriet\`a universale delle presentazioni]\label{ProprietaUniversalePresentazioni}
Siano $G=\ps{S\mid R}$, $H$ un gruppo e $f:S\to H$ una funzione. Si ha che esiste un unico omomorfismo $\vp:G\to H$ tale che per ogni $s\in S$ \[\vp(\ol s)=f(s),\] dove $\ol s$ \`e la classe della parola data dal solo carattere $s$ se e solo se per ogni $s_{i_1}^{\e_1}\cdots s_{i_j}^{\e_j}\in R$ si ha
\[f(s_{i_1})^{\e_1}\cdots f(s_{i_j})^{\e_j}=1_H.\]
\end{proposition}

\begin{example}[Esempio del calcolo di una presentazione]
$\ps{a,b\mid aba\ii b\ii}=\Z\oplus\Z$.
\end{example}

\begin{proposition}[Presentazione del prodotto amalgamato]\label{PresentazioneProdottoAmalgamato}
Siano $j_1:H\to G_1$ e $j_2:H\to G_2$ omomorfismi. Siano $h_1,\cdots, h_k$ dei generatori di $H$ e siano $G_1=\ps{S_1\mid R_1},\ G_2=\ps{S_2\mid R_2}$. Allora una presentazione di $G_1\ast_H G_2$ \`e data da
\[G_1\ast_HG_2=\ps{S_1\cup S_2\mid R_1\cup R_2\cup R_3},\]
dove $R_3=\{w_iz_i\ii \mid i\in \{1,\cdots, k\}\}$, dove $w_i$ \`e una parola in $F(S_1)$ che rappresenta $j_1(h_i)$ e $z_i$ \`e una parola in $F(S_2)$ che rappresenta $j_2(h_i)$.
\end{proposition}
\subsection{Rango}
\begin{definition}[Rango]
Dato un gruppo $G$, il suo \textbf{rango} (che indichiamo $\rnk G$) \`e il minimo numero di generatori di $G$ (evitiamo questioni di buona definizione).
\end{definition}
\begin{example}[Rango di $\Z^n$]
$\rnk \Z^n=n$.
\end{example}

\begin{remark}[Rango e omomorfismi surgettivi]\label{RangoEOmomorfismiSurgettivi}
Se $\vp:G\to H$ \`e un omomorfismo di gruppi surgettivo allora $\rnk G\geq \rnk H$.
\end{remark}

\begin{remark}[Rango di sottogruppi pu\`o crescere]\label{RangoDiSottogruppiPuoCrescere}
Esistono esempi di gruppi $G$ tali che $H\leq G$ e $\rnk H>\rnk G$.
\end{remark}

\begin{example}[Gruppo con sottogruppo di rango più grande]
Il gruppo libero $F_2=\ps{a,b}$ ha rango $2$. Per ogni $k\geq 1$ poniamo $y_k=x^kyx^{-k}$ e per $m\geq 3$ sia $G_m<F_2$ il sottogruppo generato da $y_1,\cdots, y_m$. Si verifica che $G_m=F(\{y_1,\cdots,y_m\})$, che ha rango $m>2$.
\end{example}

\subsection{Gruppi fondamentali di proiettivi}
\begin{theorem}[I proiettivi complessi sono semplicemente connessi]\label{ProiettiviComplessiSonoSemplicementeConnessi}
$\Pj^n\C$ \`e semplicemente connesso per ogni $n$.
\end{theorem}

\begin{theorem}[Gruppi fondamentali dei proiettivi reali]\label{GruppiFondamentaliProiettiviReali}
Valgono le seguenti identit\`a
\begin{align*}
&\pi_1(\Pj^1\R)=\Z,\\
&\pi_1(\Pj^n\R)=\znz2.
\end{align*}
\end{theorem}

\subsection{Gruppi fondamentali di superfici}
\subsubsection{Toro}
Potremmo usare il fatto che $T^2=S^1\times S^1$ e che i gruppi fondamentali rispettano il prodotto diretto, ma preferiamo dare un'altra dimostrazione che si generalizza ad una classe pi\`u ampia di superfici.

\begin{theorem}[Gruppo fondamentale del toro]\label{GruppoFondamentaleToro}
Si ha che
\[\pi_1(T^2)=\ps{a,b\mid aba\ii b\ii}\cong \Z\oplus \Z.\]
\end{theorem}

\subsubsection{Superfici con dato genere}
Per ogni $g\in\N$ indichiamo con $\Sigma_g$ la superficie compatta di genere $g$ (cio\`e con $g$ ``buchi"). Per $g=0$ e $g=1$ si ha che $\Sigma_g$ \`e $S^2$ e $S^1\times S^1$ rispettivamente.

\begin{fact}[Classificazione delle superfici compatte orientabili]
La famiglia delle $\Sigma_g$ contiene tutte e sole le superfici compatte orientabili.
\end{fact}

\noindent Come per il toro, possiamo scrivere ogni $\Sigma_g$ in termini di un poligono quozientato per una relazione che identifica dei lati:\\
\ul{\textit{Costruzione di $\Sigma_g$}}) Sia $Q$ un $4g-$agono e numeriamo i vertici con $v_1,\cdots, v_{4g}$. Diamo un nome ai lati come segue:
\[a_i=v_{4i-3}v_{4i-2},\ b_i=v_{4i-2}v_{4i-1},\ c_i=v_{4i-1}v_{4i},\ d_i=v_{4i}v_{4i+1}.\]
Scrivendo i lati in ordine troveremmo
\[a_1,b_1,c_1,d_1,a_2,\cdots, d_{g-1},a_g,b_g,c_g,d_g.\]
Vogliamo imporre la seguente identificazione: $a_i$ si identifica con $c_i$ in modo tale che $v_{4i-3}\sim v_{4i}$ e $v_{4i-2}\sim v_{4i-1}$ e similmente $b_i$ si identifica con $d_i$ invertendo l'ordine di percorrenza del bordo.
\bigskip

\noindent Per rendere pi\`u chiaro come si realizzano queste identificazioni tagliamo $Q$ con i lati $v_1v_{4i+1}$ in modo da ottenere $g$ pezzi.

Il primo e l'ultimo pezzo (come indice massimo del vertice presente) hanno 5 lati, mentre i pezzi centrali hanno 6 lati. In ogni pezzo, 4 lati adiacenti sono della forma $a_i, b_i, c_i, d_i$.\\
Identificando questi quattro lati come detto prima troviamo:
\begin{itemize}[noitemsep]
\item per il primo e l'ultimo pezzo abbiamo un toro con un buco che ha come bordo una circonferenza
\item per i pezzi in mezzo abbiamo un toro con un buco che ha come bordo un bouquet di due cerchi.
\end{itemize}
Per completare l'identificazione dobbiamo riattaccare i pezzi: il primo pezzo si attacca lungo la circonferenza che ne \`e il bordo al primo pezzo centrale colmando una delle due circonferenze del bouquet. Reiterando questi incollamenti attacchiamo a catena tutti i pezzi fino all'ultimo, che colma l'ultima circonferenza rimasta nel bouquet che definisce il bordo del penultimo pezzo.

Abbiamo cos\`i ottenuto una superficie senza bordo con $g$ buchi, cio\`e $\Sigma_g$.

\begin{theorem}[Gruppo fondamentale delle superfici di genere $g$]\label{GruppoFondamentaleSuperficiDiGenereg}
Si ha che
\[\pi_1(\Sigma_g)=\ps{a_1,b_1,\cdots, a_g,b_g\mid \spa{a_1,b_1}\cdots\spa{a_g,b_g}},\] dove le quadre indicano i commutatori\footnote{$[a,b]=aba\ii b\ii$.}.
\end{theorem}


\begin{proposition}[Genere determina univocamente il $\pi_1$]\label{RangoDeterminaGruppoFondamentaleDisuperficieConGenereg}
Poniamo $\Gamma_g=\pi_1(\Sigma_g)$. Si ha che
\[\Gamma_g\cong \Gamma_{g'}\coimplies g=g'.\]
\end{proposition}

\begin{theorem}[Genere, classe di Omotopia e $\pi_1$ sono invarianti completi]\label{GenereClassiOmotopiaEGruppoFondamentaleSonoInvariantiCompletiPerSuperficiCompatte}
Le seguenti affermazioni sono equivalenti:
\begin{enumerate}[noitemsep]
\item $g=g'$
\item $\Sigma_g\cong \Sigma_{g'}$ (omeomorfismo)
\item $\Sigma_g\simeq \Sigma_{g'}$ (equivalenza omotopica)
\item $\pi_1(\Sigma_g)\cong \pi_1(\Sigma_{g'})$
\end{enumerate}
\end{theorem}

\begin{fact}
\`E possibile dimostrare che ogni superficie compatta non orientabile ammette un rivestimento a due fogli dato da una superficie compatta orientabile.
\end{fact}

\section{Rivestimento Universale}
\begin{definition}[Rivestimento universale]
Un rivestimento $p:E\to X$ \`e \textbf{universale} se $E$ \`e semplicemente connesso.
\end{definition}

\begin{example}
I seguenti sono rivestimenti universali
\begin{itemize}[noitemsep]
\item $p:\R\to S^1$ data da $p(t)=e^{2\pi it}$
\item $p:\R^n\to (S^1)^n$ data da $p(t_1,\cdots, t_n)=(e^{2\pi it_1},\cdots, e^{2\pi it_n})$
\item Per $n\geq 2$, $p:S^n\to \Pj^n\R$ data da $p(v)=[v]$.
\end{itemize}
\end{example}

Vedremo che il rivestimento universale, se esiste, \`e unico a meno di isomorfismo ed \`e definito da una propriet\`a universale.

\begin{theorem}[Gruppo fondamentale e fibra nel punto sono in bigezione]\label{GruppoFondamentaleEFibraSonoInBigezione}
Sia $p:E\to X$ un rivestimento universale e sia $x_0\in X$. Allora
\[|\gf X{x_0}|=|p\ii(x_0)|.\]
\end{theorem}

\begin{example}[Gruppi fondamentali dei proiettivi]
Per $n\geq 2$ si ha che $\pi_1(\Pj^n\R)\cong \znz2$.
\end{example}

\noindent
Cerchiamo di capire quando esistono i rivestimenti universali:
\begin{definition}[Semilocalmente semplicemente connesso]
Uno spazio $X$ \`e \textbf{semilocalmente semplicemente connesso} se ogni $x_0\in X$ ammette un intorno $U\subseteq X$ tale che se $i:U\inj X$ \`e l'inclusione allora $i_\ast:\gf U{x_0}\to \gf X{x_0}$ \`e l'omomorfismo banale, cio\`e
\begin{center}
``ogni laccio contenuto in $U$ \`e omotopicamente equivalente a $c_{x_0}$ in $X$."
\end{center}
\end{definition}

\begin{example}[Spazio non semilocalmente semplicemente connesso]
Consideriamo gli orecchini hawaiiani
\[H=\bigcup_{n\geq 1}\under{\doteqdot C_n}{\cpa{(x,y)\sep x^2+\pa{y-\frac1n}^2=\frac1{n^2}}}\subseteq\R^2.\]
\end{example}
\begin{corollary}
L'orecchino hawaiiano NON \`e omeomorfo al bouquet di infinite circonferenze perch\'e questo \`E semilocalmente semplicemente connesso.
\end{corollary}

\begin{remark}
Localmente semplicemente connesso implica semilocalmente semplicemente connesso.
\end{remark}

\begin{example}
Un bouquet di un numero finito di circonferenze e le variet\`a topologiche sono semilocalmente semplicemente connesse.
\end{example}

\begin{theorem}[Esistenza dei rivestimenti universali]\label{EsistenzaRivestimentiUniversali}
$X$ ammette un rivestimento universale se e solo se $X$ \`e semilocalmente semplicemente connesso.
\end{theorem}




\subsection{Propriet\`a categoriche dei rivestimenti}
\begin{lemma}[Fattorizzazione di rivestimenti]\label{FattorizzazioneDiRivestimenti}
Dati $\wt X_1,\ \wt X_2,\ X$ connessi per archi e localmente connessi per archi, se il diagramma commuta
\[\begin{tikzcd}
	{\wt X_1} \\
	X & {\wt X_2}
	\arrow["\vp", from=1-1, to=2-2]
	\arrow["{p_2}", from=2-2, to=2-1]
	\arrow["{p_1}"', from=1-1, to=2-1]
\end{tikzcd}\]
e sia $p_1$ che $p_2$ sono rivestimenti allora anche $\vp$ lo \`e.
\end{lemma}

\begin{remark}
Un risultato analogo vale se $p_1$ e $\vp$ sono rivestimenti. Il caso per $p_2$ e $\vp$ rivestimenti invece vale supponendo $X$ semilocalmente semplicemente connesso, ma non in generale.
\end{remark}


\begin{proposition}[Propriet\`a universale del rivestimento universale]\label{ProprietaUniversaleRiverstimentoUniversale}
Sia $p:E\to X$ un rivestimento universale e sia $\pi:\wt X\to X$ un rivestimento con $\wt X$ connesso per archi. Fissiamo $x_0\in X$ e siano $\hat x_0\in E,\ \wt x_0\in \wt X$ tali che $p(\hat x_0)=x_0=\pi(\wt x_0)$. Allora esiste un unico $\wt p:E\to \wt X$ rivestimento tale che $p=\pi\circ \wt p$.
\[\begin{tikzcd}
	& {\wt X} \\
	E & X
	\arrow["p", from=2-1, to=2-2]
	\arrow["\pi", from=1-2, to=2-2]
	\arrow["{\wt p}", dashed, from=2-1, to=1-2]
\end{tikzcd}\]
\end{proposition}

\begin{corollary}[Rivestimenti universali inducono omeomorfismo tra gli spazi totali]\label{RivestimentiUniversaliInduconoOmeomorfismoTraSpaziTotali}
Se $p_1:E_1\to X$ e $p_2:E_2\to X$ sono rivestimenti universali allora esiste un omeomorfismo $\vp:E_1\to E_2$ tale che
\[\begin{tikzcd}
	{E_1} && {E_2} \\
	& X
	\arrow["\vp", from=1-1, to=1-3]
	\arrow["{p_1}"', from=1-1, to=2-2]
	\arrow["{p_2}", from=1-3, to=2-2]
\end{tikzcd}\]
\end{corollary}







\begin{definition}[Morfismo di rivestimenti]
Siano $p_1:E_1\to X$ e $p_2:E_2\to X$ rivestimenti. Un \textbf{morfismo} tra $p_1$ e $p_2$ \`e una mappa $\vp:E_1\to E_2$ continua tale che $p_2\circ \vp=p_1$.
\[\begin{tikzcd}
	{E_1} && {E_2} \\
	& X
	\arrow["{p_1}"', from=1-1, to=2-2]
	\arrow["{p_2}", from=1-3, to=2-2]
	\arrow["\vp", from=1-1, to=1-3]
\end{tikzcd}\]
Se esiste un morfismo $\psi:E_2\to E_1$ tale che $\psi\circ \vp=id_{E_1}$ e $\vp\circ \psi=id_{E_2}$ allora $\vp$ \`e un \textbf{isomorfismo} di rivestimenti.\\
Se $E_1=E_2$ e $p_1=p_2$, un isomorfismo di rivestimenti \`e detto \textbf{automorfismo} di rivestimenti.
\end{definition}
\begin{remark}
I morfismi di rivestimenti sono chiusi per composizione e l'identit\`a \`e un morfismo.
\end{remark}


\begin{definition}[Automorfismi di rivestimenti]
L'insieme degli automorfismi di $p:E\to X$ \`e un gruppo, detto \textbf{gruppo degli automorfismi} di $p$, che indichiamo
\[\Aut(p)=\Aut_X(E)=\Aut(E).\]
Osserviamo che $\vp\in \Aut(p)$ se e solo se $\vp:E\to E$ \`e un omeomorfismo e $p\circ \vp=p$.
\end{definition}
\begin{remark}
Un morfismo di rivestimenti manda fibre in fibre. In particolare gli automorfismi inducono permutazioni di una fibra in se stessa, infatti se $\wt x\in p\ii(x)$ allora $p(\vp(\wt x))=p(\wt x)=x$ implica $\vp(\wt x)\in p\ii(x)$.
\end{remark}

\begin{proposition}[Azione di $\Aut(p)$ e di monodromia commutano]\label{AzioneDiAutomorfismiEMonodromiaCommutano}
L'azione di monodromia e quella di $\Aut(p)$ commutano, cio\`e per $F=p\ii(x_0)$, per ogni $\vp\in \Aut(p),\ \forall [\gamma]\in\gf X{x_0},\ \forall \wt x\in F$
\[\vp(\wt x\cdot[\gamma])=(\vp(\wt x))\cdot [\gamma].\]
\end{proposition}


\subsubsection{Isomorfismi di rivestimenti}
Data l'utilit\`a che avr\`a $\Aut(p)$ cerchiamo di capire quando due rivestimenti sono isomorfi.

\begin{theorem}[Caratterizzazione di rivestimenti isomorfi fissato un punto]\label{CaratterizzazioneRivestimentiIsomorfiFissatoPunto}
Siano $p_1:E_1\to X$ e $p_2:E_2\to X$ rivestimenti e fissiamo $\wt x_1\in E_1,\ \wt x_2\in E_2$ tali che $p_1(\wt x_1)=x_0=p_2(\wt x_2)$. Allora esiste un isomorfismo $\vp:E_1\to E_2$ con $\vp(\wt x_1)=\wt x_2$ se e solo se ${p_1}_\ast(\gf{E_1}{\wt x_1})={p_2}_\ast(\gf{E_2}{\wt x_2})$ in $\pi_1(X,x_0)$.
\end{theorem}

\begin{corollary}[Criterio di esistenza per automorfismi]\label{CriterioEsistenzaAutomorfismo}
Sia $p:E\to X$ un rivestimento e consideriamo $x_0\in X,\ F=p\ii(x_0)$ e $\wt x_1,\wt x_2\in F$, allora esiste un automorfismo $\vp\in\Aut(p)$ tale che $\vp(\wt x_1)=\wt x_2$ se e solo se ${p}_\ast(\gf{E}{\wt x_1})={p}_\ast(\gf{E}{\wt x_2})$ in $\pi_1(X,x_0)$.
\end{corollary}



\begin{proposition}[Caratterizzazione di rivestimenti isomorfi]\label{CaratterizzazioneRivestimentiIsomorfi}
Siano $p_1:E_1\to X$ e $p_2:E_2\to X$ rivestimenti e scegliamo $\wt x_j\in E_j$ con $p_1(\wt x_1)=x_0=p_2(\wt x_2)$. Allora esiste un isomorfismo $\vp:E_1\to E_2$ se e solo se ${p_1}_\ast(\gf{E_1}{\wt x_1})$ e ${p_2}_\ast(\gf{E_2}{\wt x_2})$ sono coniugati in $\gf X{x_0}$.
\end{proposition}














\subsection{Rivestimenti regolari e corrispondenza di Galois}
Da questo momento supponiamo sempre che gli spazi siano connessi per archi e localmente connessi per archi.
\bigskip

\noindent
Studiamo ora in pi\`u dettaglio l'azione di $\Aut(p)$ e i quozienti che ne possono derivare.

\begin{proposition}[Azione di $\Aut(p)$ \`e propriamente discontinua]\label{AzioneAutomorfismiRivestimentiEPropriamenteDiscontinua}
$\Aut(p)$ agisce in modo propriamente discontinuo su $E$.
\end{proposition}

\begin{theorem}[Per rivestimento da azione propriamente discontinua gli automorfismi sono il gruppo]\label{PerRivestimentoAzionePropriamenteDiscontinuaAutomorfismiEGruppoCoincidono}
Sia $G$ un gruppo che agisce su $Y$ in modo propriamente discontinuo. Se  $p:Y\to \quot YG$ \`e il rivestimento indotto dal quoziente (\ref{ProiezioneQuozienteAzionePropriamenteDiscontinuaERivestimento}) si ha che $\Aut(p)=G$\footnote{stiamo identificando $G$ con gli omeomorfismi che induce su $Y$, che possiamo fare perch\'e l'azione \`e propriamente discontinua, infatti $\ell_g\ii\circ \ell_h=id_Y$ se e solo se $g\ii h=1_G$}.
\end{theorem}

\noindent
Potremmo chiederci se \textit{ogni} rivestimento si pu\`o scrivere come quoziente per azione propriamente discontinua di gruppo. Purtroppo non \`e vero perch\'e una scrittura del genere implicherebbe $\Aut(p)=G$ transitivo sulle fibre (per definizione di quoziente $p(x)=p(y)$ se e solo se esiste $g\in G$ tale che $x=g\cdot y$) ma non tutti i gruppi degli automorfismi sono transitivi sulle fibre.\\
Definiamo allora la classe dei rivestimenti che hanno quest'ultima propriet\`a:
\begin{definition}[Rivestimento regolare]
Un rivestimento $p:E\to X$ \`e \textbf{regolare} (o \textbf{normale} o \textbf{di Galois}) se l'azione di $\Aut(p)$ \`e transitiva su ogni fibra rispetto a $p$.
\end{definition}

\begin{proposition}[I rivestimenti universali sono regolari]\label{RivestimentiUniversaliSonoRegolari}
I rivestimenti universali sono regolari.
\end{proposition}

\begin{lemma}\label{LemmaConiugioImmaginiTramiteRivestimentoDiGruppiFondamentali}
Siano $p:E\to X$ un rivestimento, $x_0,\ x_1\in X$ e $\wt x_0\in p\ii(x_0),\ \wt x_1\in p\ii(x_1)$. Allora dato $\wt\gamma \in \Omega(E,\wt x_0,\wt x_1)$ e posto $\gamma=p\circ \wt \gamma$ si ha che
\[\Phi_\gamma:\funcDef{\gf X{x_0}}{\gf X{x_1}}{[\eta]}{[\ol \gamma\ast \eta\ast \gamma]}\]
\`e un isomorfismo e induce l'isomorfismo
\[\Phi_\gamma\res{p_\ast(\gf E{\wt x_0})}:p_\ast(\gf E{\wt x_0})\to p_\ast(\gf E{\wt x_1})\]
\end{lemma}


\begin{theorem}[Caratterizzazioni dei rivestimenti regolari]\label{CaratterizzazioniRivestimentiRegolari}
Sia $p:E\to X$ un rivestimento e fissiamo $x_0\in X$ e $F=p\ii(x_0)$. Le seguenti affermazioni sono equivalenti:
\begin{enumerate}[noitemsep]
\item $\Aut(p)$ \`e transitivo su $F$
\item Esiste $\wt x\in F$ tale che $p_\ast(\gf E{\wt x})\normal \gf X{x_0}$
\item Per ogni $\wt x\in F$, $p_\ast(\gf E{\wt x})\normal \gf X{x_0}$
\item $p$ \`e regolare.
\end{enumerate}
\end{theorem}
\bigskip

\noindent La caratterizzazione in termini di sottogruppi normali risulter\`a chiave per trovare la corrispondenza di Galois che cerchiamo. Infatti grazie al prossimo risultato potremo scrivere gli automorfismi di rivestimenti regolari come quoziente del gruppo fondamentale dello spazio base.


\begin{theorem}[$\Aut(p)$ in termini del gruppo fondamentale]\label{AutomorfismiDiRivestimentiInTerminiDelGruppoFondamentale}
Sia $p:E\to X$ un rivestimento e $\wt x_0\in E$. Allora
\[\Aut(p)\cong \quot{N_{\gf X{x_0}}(p_\ast(\gf E{\wt x_0}))}{p_\ast(\gf E{\wt x_0})},\]
dove $x_0=p(\wt x_0)$ e  $N_G(H)$ indica il \textbf{normalizzatore} di $H$ in $G$\footnote{Se $H\leq G$ allora $N_G(H)=\cpa{g\in G\sep g\ii Hg=H}$}.
\end{theorem}

\begin{proposition}[Automorfismi di rivestimenti regolari]\label{AutomorfismiRivestimentiRegolari}
Se $p$ \`e regolare allora $\Aut(p)\cong \quot {\gf X{x_0}}{p_\ast(\gf E{\wt x_0})}$ con $x_0=p(\wt x_0)$
\end{proposition}

\begin{corollary}[Automorfismi del rivestimento universale sono il gruppo fondamentale]\label{AutomorfismiDelRivestimentoUniversaleSonoIlGruppoFondamentale}
Se $p$ \`e un rivestimento universale allora $\Aut(p)\cong \gf X{x_0}$.
\end{corollary}






\begin{corollary}[Corrispondenza di Galois per rivestimenti regolari]\label{CorrispondenzaGaloisRivestimentiRegolari}
Esiste una corrispondenza biunivoca tra i rivestimenti regolari di $X$ a meno di isomorfismo e sottogruppi normali di $\pi_1(X,x_0)$.
Sotto questa corrispondenza il grado del rivestimento corrisponde all'indice del sottogruppo.
\end{corollary}

\subsection{Applicazioni della teoria dei rivestimenti}
\begin{theorem}[Borsuk-Ulam]\label{TeoremaBorsukUlam}
Non esistono funzioni $f:S^2\to S^1$ continue tali che $f(-x)=-f(x)$ per ogni $z\in S^2$.
\end{theorem}

\begin{theorem}
Non esistono funzioni $f:S^2\to \R^2$ continue e iniettive.
\end{theorem}

\end{multicols*}

\chapter{Analisi Complessa}

\begin{multicols*}{2}


\section{Richiami di calcolo in pi\`u variabili}
\begin{definition}[Differenziabilit\`a e differenziale]
Sia $U\subseteq \R^m$ un aperto e sia $f:U\to \R^n$ una funzione. $f$ \`e  \textbf{differenziabile} in $x_0\in U$ se esiste una funzione lineare $L:\R^m\to \R^n$ tale che
\[f(x_0+v)=f(x_0)+L(v)+o(|v|).\]
In tal caso $L$ \`e unica e si dice \textbf{differenziale} di $f$ in $x_0$ e si denota $df_{x_0}:\R^m\to \R^n$.\footnote{La notazione usata nel corso di analisi 2 \`e $\Dc f(x_0)$ al posto di $df_{x_0}$.}
\end{definition}
\begin{remark}
Se $f$ \`e $\R-$lineare allora coincide con il proprio differenziale.
\end{remark}
\begin{remark}
Il differenziale di una composizione \`e la composizione dei differenziali.
\end{remark}
\begin{definition}[Derivate parziali]
Se $\{e_i\}$ \`e la base canonica di $\R^m$ allora definiamo la \textbf{$i-$esima derivata parziale} di $f$ come
\[df_{x_0}(e_i)=\pdpi if(x_0).\]
\end{definition}

\noindent
Per il resto del capitolo identifichiamo $\R^2$ con $\C$ tramite l'isomorfismo $(x,y)\leftrightarrow x+iy$.

\begin{remark}
$\C$ \`e sia un $\C-$spazio vettoriale (di dimensione $1$ con base $\{1\}$) che un $\R-$spazio vettoriale (di dimensione $2$ con base $\{1,i\}$).
\end{remark}

\begin{definition}[Coniugio]
La seguente mappa
\[\funcDef\C\C {x+iy}{x-iy}\]
\`e detta \textbf{coniugio}. Se $x\in \C$ spesso indichiamo la sua immagine tramite la mappa di coniugio con $\ol z$ e chiamiamo questo il \textbf{coniugato (complesso)} di $z$.
\end{definition}
\begin{remark}
Il coniugio \`e $\R-$lineare ma non $\C-$lineare ($-i=\ol{i}\neq i\ol{1}=i$).
\end{remark}

\begin{proposition}[Il differenziale \`e $\C$-lineare]
Il differenziale \`e $\C$-lineare, cio\`e
\[d((a+bi)f)_P=(a+bi)df_P.\]
\end{proposition}


\begin{proposition}[Propriet\`a del differenziale]
Siano $f_1,f_2:U\to\C$ con $U\subseteq\R^m$ aperto mappe differenziabili in $P$, allora
\begin{enumerate}[noitemsep]
\item $(d\ol f_1)_P=(\ol{df_1})_P$
\item $d(f_1f_2)_P=(df_1)_Pf_2(P)+f_1(P)(df_2)_P$
\item Se $f_1(P)\neq 0$ allora $\frac 1{f_1}$ \`e definita in un intorno di $P$ e
\[d\pa{\frac1{f_1}}_P=-\frac{(df_1)_P}{f_1(P)^2}.\]
\end{enumerate}
\end{proposition}

\begin{notation}
Per semplicit\`a notazionale a volte scriveremo $z$ al posto di $id_\C$, $\ol z$ al posto della mappa coniugio, $x$ per $a+bi\mapsto a=\Real(a+bi)$ e $y$ per $a+bi\mapsto b=\Imag(a+bi)$. Cercher\`o di rendere la notazione meno ambigua possibile quando possibile.
\end{notation}

\begin{remark}
$z=x+iy$ e $\ol z=x-iy$ come funzioni. Osserviamo inoltre che
\[dz=dx+idy,\quad d\ol z=dx-idy\]
per linearit\`a. Nonostante $z=dz$ e $\ol z=d\ol z$ manterremo le $d$ quando questo render\`a pi\`u chiari i ragionamenti (vale a dire nel contesto delle 1-forme).\\
Osserviamo che con queste notazioni $dx=x$ e $dy=y$.
\end{remark}

\begin{remark}
Se $f:U\to\C$ \`e differenziabile in $P\in U$ si ha che
\[df_P=\pdp[x]f(P)dx+\pdp[y]f(P)dy.\]
\end{remark}
\begin{remark}
Dato che $dz=dx+idy$ e $d\ol z=dx-idy$ si ha che
\[dx=\frac{dz+d\ol z}2,\qquad dy=\frac{dz-d\ol z}{2i}.\]
Da questo segue che
\begin{align*}
df=&\pdp[x]fdx+\pdp[y]fdy =\\
=&\pdp[x]f\frac{dz+d\ol z}2+\pdp[y]f\frac{dz-d\ol z}{2i}=\\
=&\frac12\pa{\pdp[x]f-i\pdp[y]f}dz+\frac12\pa{\pdp[x]f+i\pdp[y]f}d\ol z.
\end{align*}
\end{remark}
\noindent
Motivati dall'espressione sopra per $df$ diamo le seguenti definizioni:
\begin{notation}
Poniamo
\[\pdp[z]f=\frac12\pa{\pdp[x]f-i\pdp[y]f},\qquad \pdp[\ol z]f=\frac12\pa{\pdp[x]f+i\pdp[y]f},\]
da cui
\[df=\pdp[z]fdz+\pdp[\ol z]fd\ol z.\]
\end{notation}
\begin{remark}
Si ha che $\pdp[z]fdz$ \`e $\C-$lineare e $\pdp[\ol z]fd\ol z$ \`e $\C-$antilineare.
\end{remark}

Abbiamo dunque decomposto $f$ in una componente $\C-$lineare e una componente $\C-$antilineare. \\
\begin{proposition}[Le $\R$ lineari sono la somma diretta delle $\C$ lineari e $\C$ antilineari]\label{RlinSonoSommaDirettaClinECantilin}

\footnotesize{$\cpa{f:\C\to \C\sep \R-lin.}=\cpa{f:\C\to \C\sep \C-lin.}\oplus\cpa{f:\C\to \C\sep \C-antilin.}$}
\end{proposition}

\section{Funzioni olomorfe}
\begin{definition}[Funzione olomorfa]
Siano $U\subseteq \C$ un aperto, $z_0\in U$ e $f:U\to \C$. La funzione $f$ si dice \textbf{olomorfa} (o \textbf{derivabile in senso complesso}) in $z_0$ se il seguente limite esiste
\[\lim_{z\to z_0}\frac{f(z)-f(z_0)}{z-z_0}=f'(z_0)\in\C.\]
$f$ \`e \textbf{olomorfa} se lo \`e in $z_0$ per ogni $z_0\in U$.\\
Poniamo
\[\Oc(U)=\cpa{f:U\to\C\sep f\text{ olomorfa}}\]
\end{definition}
\noindent
Osserviamo che essere olomorfa ed essere differenziabile hanno definizioni formalmente molto simili, ma sono propriet\`a diverse come mostra la seguente proposizione. La differenza sostanziale sta nel fatto che $z-z_0$ nella definizione \`e un numero complesso, non una distanza. Pi\`u esplicitamente, la differenziabilit\`a richiede solo l'esistenza di \[\lim_{z\to z_0}\frac{f(z)-f(z_0)}{|z-z_0|},\] che \`e una condizione pi\`u debole.

\begin{lemma}
Se $f:U\to \C$ \`e olomorfa in $z_0\in U$ e $f'(z_0)$ denota il limite della definizione di olomorfa allora
\[df_{z_0}(w)=f'(z_0)w.\]
\end{lemma}

\begin{proposition}[Caratterizzazioni delle funzioni olomorfe]\label{CaratterizzazioneOlomorfaInPunto}
Sia $f:U\to \C$ con $U$ aperto e $z_0\in U$. Le seguenti affermazioni sono equivalenti:
\begin{enumerate}[noitemsep]
\item $f$ \`e olomorfa in $z_0$
\item $f$ \`e differenziabile in $z_0$ e $df_{z_0}$ \`e $\C-$lineare
\item $f$ \`e differenziabile in $z_0$ e $\displaystyle \pdp[\ol z]f(z_0)=0$
\item $f$ \`e differenziabile e valgono le \textbf{equazioni di Cauchy-Riemann}, cio\`e
\[\pdp[x]{\Real(f)}=\pdp[y]{\Imag(f)},\quad \pdp[y]{\Real(f)}=-\pdp[x]{\Imag(f)}.\]
\item $df_{z_0}(w)=f'(z_0)w$.
\end{enumerate}
\end{proposition}

\begin{theorem}[Propriet\`a aritmetiche delle olomorfe]\label{ProprietaAritmeticheOlomorfe}
Siano $f:U\to \C$ e $g:V\to \C$ olomorfe. Allora
\begin{enumerate}[noitemsep]
\item se $U=V$ allora $f+g:U\to \C$ \`e olomorfa
\item se $U=V$ allora $fg:U\to \C$ \`e olomorfa e
\[(fg)'=f'g+fg'\]
\item se $f(U)\subseteq V$ allora $g\circ f:U\to \C$ \`e olomorfa e
\[(g\circ f)'(z_0)=g'(f(z_0))f'(z_0)\]
\item se $f'(z_0)\neq 0$ e $f\in C^1$ allora esistono un intorno $W$ di $f(z_0)$ in $\C$ e un intorno $Z$ di $z_0$ in $U$ tali che $f(Z)=W$, $f\res Z\to W$ \`e bigettiva e $(f\res Z)\ii :W\to Z$ \`e olomorfa con
\[((f\res Z)\ii)'(f(z_0))=\frac1{f'(z_0)}\]
\end{enumerate}
\end{theorem}



\section{Analitiche}
Studiamo una classe di funzioni apparentemente pi\`u piccola delle funzioni olomorfe. Mostreremo successivamente che le funzioni analitiche da $U\subseteq \C$ aperto a $\C$ sono esattamente le funzioni olomorfe. Per il momento limitiamoci a studiarle in quanto tali.

\subsection{Serie di potenze}
\begin{definition}[Serie di potenze]
Una serie della forma
\[\sum_{n=0}^\infty a_n(z-z_0)^n\]
\`e detta \textbf{serie di potenze} di \textbf{centro} $z_0$.
\end{definition}
\begin{remark}
Una serie di potenze ammette un \textbf{raggio di convergenza}, cio\`e esiste $R>0$ tale che la serie converge assolutamente in $B_R(z_0)$ e diverge in $\C\bs \ol{B_R(z_0)}$.
\end{remark}

\begin{definition}[Funzione analitica]
Una funzione $f:U\to \C$ \`e \textbf{analitica} se per ogni $z_0\in U$ esiste $R>0$ tale che $f$ sia esprimibile come serie di potenze centrata in $z_0$ per ogni $z\in B_R(z_0)$.
\end{definition}
\begin{proposition}[Le analitiche sono continue]\label{AnaliticaImplicaContinua}
Una funzione $f:U\to \C$ analitica \`e continua.
\end{proposition}

\begin{theorem}[Serie di potenze sono analitiche]\label{SeriePotenzeSonoAnalitiche}
Supponiamo che
\[f(z)=\sum_{n=0}^\infty a_n(z-z_0)^n\]
per ogni $z\in B_R(z_0)$ con $R>0$. Allora $f$ \`e analitica su $B_R(z_0)$.
\end{theorem}
\begin{remark}
Il teorema potrebbe sembrare banale ma ricordiamo che analitica significa che la funzione si esprime come serie di potenze centrata in un \textit{qualsiasi} punto del dominio. La scrittura con centro in $z_0$ non \`e sufficiente.
\end{remark}

\begin{theorem}[Serie derivata]\label{DerivataSerieDiPotenze}
Se $f=\sum a_n(z-z_0)^n$ su $B_R(z_0)$ allora $f$ \`e olomorfa su $B_R(z_0)$ e
\[f'(z)=\under{\text{\textbf{serie derivata}}}{\sum_{n=1}^\infty a_nn(z-z_0)^{n-1}}\quad \text{per ogni $z\in B_R(z_0)$}\]
\end{theorem}
\begin{corollary}\label{AnaliticaImplicaOlomorfa}
Una funzione analitica complessa \`e $C^\infty$ in senso complesso.
\end{corollary}
\begin{corollary}[Serie di Taylor]\label{TaylorSerieDiPotenze}
Se $f(z)=\sum_{n=0}^\infty a_n(z-z_0)^n$ in un intorno di $z_0$ allora
\[f^{(n)}(z_0)=n!\ a_n,\]
in particolare l'espressione di $f$ come serie di potenze \`e unica:
\[f(z)=\sum_{n=0}^\infty \frac{f^{(n)}(z_0)}{n!}(z-z_0)^n\]
\end{corollary}

\begin{remark}[Composizione di analitiche \`e analitica]\label{ComposizioneAnaliticheEAnalitica}
Siano $f:U\to V$ e $g:V\to \C$ analitiche, allora $g\circ f$ \`e analitica.
\end{remark}



\subsection{Ordine di annullamento}
Studiamo come si comportano gli zeri delle funzioni analitiche. Questo si riveler\`a molto utile nel dare principi di rigidit\`a per le olomorfe.

\begin{definition}[Ordine di annullamento]
Sia $f:U\to \C$ olomorfa. L'\textbf{ordine di annullamento} (o \textbf{di svanimento} o \textbf{di zero}) di $f$ in $z_0$ \`e dato da
\[\begin{cases}
\min\cpa{n\in\N\sep
f^{(n)}(z_0)\neq 0} & \text{se }\exists n\in\N\ t.c.\ f^{(n)}(z_0)\neq 0\\
\infty &\text{altrimenti.}
\end{cases}\]
\end{definition}
\begin{remark}
$f$ ha ordine di annullamento $0$ se $f(z_0)\neq 0$ e ha ordine $\geq 1$ se $f(z_0)=0$.
\end{remark}

\begin{lemma}[Caratterizzazione dell'ordine di annullamento]\label{CaratterizzazioneOrdineDiAnnullamento}
Sia $f$ analitica (e quindi olomorfa per (\ref{AnaliticaImplicaOlomorfa})). Le seguenti proposizioni sono equivalenti:
\begin{enumerate}[noitemsep]
\item $z_0$ ha ordine di annullamento $n_0$ per $f$
\item $f(z)=(z-z_0)^{n_0}g(z)$ in un intorno di $z_0$ con $g(z_0)\neq 0$ e $g$ analitica.
\end{enumerate}
\end{lemma}

\begin{remark}
Se $f$ \`e analitica, $z_0$ ha ordine di annullamento $\infty$ per $f$ se e solo se $f=0$ in un intorno di $z_0$.
\end{remark}


\begin{lemma}\label{ZeriAnaliticaSonoIsolatiOIntornoNullo}
Sia $f:U\to \C$ analitica e sia $Z=\{z_0\in U\mid f(z_0)=0\}$. Sia $z_0\in Z$. Si ha che $z_0$ \`e un punto isolato di $Z$ o $z_0\in int(Z)$.
\end{lemma}

\begin{theorem}[Zeri di analitica sono isolati o coprono la comp. connessa]\label{ZeriDiAnaliticaSonoIsolatiOCopronoLaComponenteConnessa}
Sia $f:U\to\C$ analitica con $U$ connesso. Supponiamo che $Z=f\ii(0)\subseteq U$ abbia un punto di accumulazione in $U$. Allora $f=0$ come funzione su $U$.
\end{theorem}
\noindent
Possiamo riformulare questo risultato nel potente
\begin{corollary}[Principio di identit\`a per analitiche]\label{LuogoDoveAnaliticheCoincidonoSonoIsolatiOCopronoLaComponenteConnessa}
Se $U$ \`e connesso e $f,g:U\to \C$ sono analitiche e $W\subseteq U$ contiene un punto non isolato allora $f=g$ su $U$ se e solo se $f=g$ su $W$.
\end{corollary}

\section{Esponenziale e logaritmo complessi}
\subsection{Esponenziale complesso}
\begin{definition}[Esponenziale complesso]
Definiamo
\[e^z=\sum_{n=0}^\infty \frac{z^n}{n!}.\]
\end{definition}
\begin{remark}
L'esponenziale ha raggio di convergenza infinito. In particolare \`e sempre assolutamente convergente.
\end{remark}

\begin{proposition}[Propriet\`a dell'esponenziale complesso]\label{ProprietaEsponenzialeComplesso}
Valgono i seguenti fatti
\begin{enumerate}[noitemsep]
\item $e^{z+w}=e^ze^w$ per ogni $z,w\in \C$. In particolare $(e^{z})\ii=e^{-z}$ quindi $exp:\C\to \C^\times$ \`e ben definita.
\item Se $z=x+iy$ con $x,y\in \R$ allora
\[e^z=e^xe^{iy}=e^x(\cos y+i\sin y).\]
\item $\abs{e^{iy}}=\abs{\cos y+i\sin y}=1$, in particolare
\[\abs{e^{x+iy}}=\abs{e^x}=e^{\Real(z)}.\]
\item $exp:\C\to \C^\times$ \`e surgettiva
\item Essendo analitica, $\exp$ \`e olomorfa ed evidentemente $\exp'(z)=\exp(z)$.
\item $e^z=e^w$ se e solo se $z-w=2k\pi i$
\end{enumerate}
\end{proposition}


\subsection{Logaritmo complesso}
Sappiamo che $\exp:\C\to \C^\times$ \`e surgettiva. Questa mappa non \`e iniettiva, dunque non ha senso cercare una inversa globale, ma possiamo provare perlomeno a cercare $L:\C^\times\to C$ tale che $\exp L=id_{\C^\times}$.\\
Insiemisticamente una tale mappa esiste\footnote{(Curiosit\`a) \`e una formulazione equivalente dell'assioma della scelta}, ma vorremmo cercarla continua. Purtroppo mostreremo che una tale funzione non esiste.

\begin{proposition}[L'esponenziale complesso \`e un rivestimento]\label{EsponenzialeComplessoERivestimento}
La mappa $\exp:\C\to \C^\times$ \`e un rivestimento.
\end{proposition}

\begin{corollary}
Non esiste $L:\C^\times\to \C$ continua tale che $e^{L(z)}=z$ per ogni $z\in\C^\times$.
\end{corollary}

\noindent
Cerchiamo allora di invertire l'esponenziale su un sottodominio di $\C^\times$ in modo continuo

\begin{theorem}[Branche del logaritmo]\label{BrancheDelLogaritmo}
Sia $U\subseteq \C^\times$ aperto connesso tale che l'inclusione $U\inj \C^\times$ induca l'omomorfismo banale $\pi_1(U)\to \pi_1(\C^\times)$\footnote{Per esempio possiamo prendere $U$ semplicemente connesso. L'idea geometrica \`e che non possiamo considerare un insieme che contenga lacci che si attorcigliano attorno l'origine.}. Allora esiste una mappa detta \textbf{branca del logaritmo} continua
\[L:U\to\C\quad t.c.\quad e^{L(z)}=z\quad \forall z\in U.\]
Due tali branche che coincidono in un punto coincidono su tutto $U$.
\end{theorem}

\noindent
Definiamo allora un logaritmo standard:
\begin{definition}[Branca principale del logaritmo complesso]
Sia \[U=\C\bs\{z\in\C\mid \Imag(z)=0\ e\ \Real(z)\leq 0\}.\]
Chiaramente $U\subseteq\C^\times$ \`e semplicemente connesso quindi il teorema (\ref{BrancheDelLogaritmo}) garantisce l'esistenza di un logaritmo $L:U\to \C$, il quale \`e unico per l'unicit\`a dei sollevamenti (\ref{UnicitaSollevamenti}) se fissiamo $L(1)=0$ (scelta lecita perch\'e $\exp\ii(1)=\{2k\pi\}_{k\in\Z}$).
\end{definition}
\begin{proposition}[Formula esplicita per la branca principale del logaritmo]
Se $L$ \`e la branca principale del logaritmo si ha che
\[L(z)=\log|z|+i\arg(z),\]
dove $\log$ \`e il logaritmo reale e $\arg(z)$ \`e l'argomento principale di $z$, definito per esempio da
\[\arg(x+iy)=\begin{cases}
\arccos\pa{\frac x{\sqrt{x^2+y^2}}} & y\geq 0\\
-\arccos\pa{\frac x{\sqrt{x^2+y^2}}} & y\leq 0
\end{cases}\]
\end{proposition}



\begin{proposition}[Le branche del logaritmo sono olomorfe]\label{BrancheDelLogaritmoSonoOlomorfe}
Le branche del logaritmo complesso sono olomorfe.
\end{proposition}

\begin{theorem}[Espansione in serie del logaritmo]\label{EspansioneInSerieDelLogaritmo}
Sia $L:\C\bs\{z\in\R\subseteq\C\mid z\leq 0\}\to \C$ la branca principale del logaritmo. Allora per ogni $z\in B_1(0)$ abbiamo
\[L(1+z)=\sum_{n=1}^\infty(-1)^{n+1}\frac{z^n}{n}\]
\end{theorem}

\section{1-Forme complesse}
\begin{definition}[1-forma continua]
Sia $U\subseteq\C$ un aperto. Una \textbf{1-Forma} continua su $U$\footnote{sottointenderemo ``continua" quasi sempre} \`e una funzione $\omega:U\times \C\to \C$ continua e $\R-$lineare nel secondo argomento.\\
Di solito al posto di $\omega(z_0,v)$ si scrive $\omega(z_0)(v)$ o $\omega_{z_0}(v)$ per rimarcare il fatto che $\omega(z_0,\cdot)$ \`e una funzione $\R-$lineare.
\end{definition}
\begin{remark}
Se $f:U\to \C$ \`e $C^1$ allora $df$ \`e una 1-forma, infatti
\[df(z,w)=df_z(w),\]
dove la continuit\`a rispetto a $z$ \`e la definizione di $C^1$.
\end{remark}

\begin{definition}[Forme esatte e chiuse]
Una 1-forma $\omega$ si dice \textbf{esatta} se esiste $f:U\to \C$ di classe $C^1$ tale che $\omega=df$. In tal caso $f$ si chiama \textbf{primitiva} di $\omega=df$.\\
Una 1-forma $\omega$ su $U$ si dice \textbf{chiusa} se \`e localmente esatta, cio\`e se per ogni $z_0\in U$ esiste un intorno $V$ di $z_0$ con $z_0\in V\subseteq U$ tale che $\omega\res V$ \`e esatta, cio\`e esiste $f:V\to \C$ di classe $C^1$ con $\omega\res V\doteqdot \omega\res{V\times \C}=df$.
\end{definition}
\begin{remark}
Per definizione ogni forma esatta \`e chiusa.
\end{remark}

\begin{remark}[1-forme in coordinate]
Poich\'e $\Hom_\R(\C,\C)$ \`e un $\C-$spazio vettoriale e sia $\{dx,dy\}$ che $\{dz,d\ol z\}$ ne sono basi, possiamo scrivere ogni 1-forma $\omega$ su $U$ nelle forme
\[\omega(x+iy)=a(x,y)dx+b(x,y)dy,\]
dove $a,b:U\to \C$ sono continue oppure
\[\omega(z)=f(z)dz+g(z)d\ol z\]
dove $f,g:U\to \C$ sono continue.
\end{remark}

\subsection{Integrazione di 1-forme}
\begin{definition}[Integrale di funzione da intervallo reale a $\C$]
Se $f(t)=g(t)+ih(t)$ con $g,h:[a,b]\to \R$ allora
\[\int_a^bf(t)dt=\int_a^bg(t)dt+i\int_a^bh(t)dt.\]
\end{definition}

\begin{definition}[Integrale lungo una curva $C^1$]
Sia $\omega$ una 1-forma su $U\subseteq \C$ e sia $\gamma:[a,b]\to U$ un cammino $C^1$. \textbf{L'integrale di $\omega$ lungo $\gamma$} \`e dato da
\[\int_\gamma\omega=\int_a^b\omega_{\gamma(t)}(\gamma'(t))dt.\]
\end{definition}

\begin{definition}[Curva $C^1$ a tratti]
Una curva $\gamma:[a,b]\to \C$ \`e \textbf{$C^1$ a tratti} se esiste una partizione $\{t_0,\cdots, t_n\}$\footnote{Ricordiamo da Analisi 1 che una partizione ha la forma
\[a=t_0<t_1<\cdots<t_{n-1}<t_n=b\]
} di $[a,b]$ tale che $\gamma\res{\spa{t_i,t_{i+1}}}$ \`e $C^1$ per ogni $i$.
\end{definition}

\begin{remark}[Caso per curve $C^1$ a tratti]
Se $\gamma:[a,b]\to \C$ \`e $C^1$ a tratti poniamo
\[\int_\gamma\omega=\sum_{i=0}^{n-1}\int_{t_i}^{t_{i+1}}\omega_{\gamma(t)}(\gamma'(t))dt\]
\end{remark}

\begin{proposition}[Invarianza dell'integrale per riparametrizzazione]\label{InvarianzaIntegralePerRiparametrizzazione}
Sia $\delta:[c,d]\to U$ una riparametrizzazione di $\gamma$, cio\`e $\delta=\gamma\circ \vp$ con $\vp:[c,d]\to [a,b]$ di classe $C^1$ con $\vp(c)=a$ e $\vp(d)=b$. Allora
\[\int_\delta\omega=\int_\gamma\omega\]
\end{proposition}
\begin{proposition}[Integrazione di 1-forme esatte]\label{IntegrazioneDi1formeEsatte}
Se $\omega$ \`e esatta con primitiva $f:U\to \C$ allora
\[\int_\gamma\omega=f(\gamma(b))-f(\gamma(a)).\]
\end{proposition}

\begin{corollary}
Se $\gamma:[a,b]\to U$ \`e una curva $C^1$ a tratti chiusa e $\omega$ \`e una forma esatta su $U$ allora $\int_\gamma\omega=0$.
\end{corollary}

\begin{example}[Forma chiusa non esatta]
Consideriamo la forma $\omega=\frac1zdz$ definita su $\C^\times$.\\
\ul{\textit{Chiusa}}) Segue dal fatto che il logaritmo \`e localmente olomorfo e la sua derivata \`e \[L'(z)=\frac1z.\]
\ul{\textit{Non Esatta}}) Sia $\gamma:[0,1]\to \C^\times$ data da $\gamma(t)=e^{2\pi it}$. Allora
\[\int_\gamma\omega=\int_0^1\frac1{\gamma(t)}\gamma'(t)dt=\int_0^1e^{-2\pi it}2\pi ie^{2\pi it}dt=2\pi i\neq 0.\]
\end{example}

\noindent
Poich\'e diremo spesso ``aperto connesso"diamo la seguente
\begin{definition}[Dominio]
Un insieme $D\subseteq \C$ \`e un \textbf{dominio} se \`e aperto e connesso.
\end{definition}
\begin{remark}
Poich\'e $\C$ \`e localmente connesso per archi si ha che un dominio   \`e anche connesso per archi e localmente connesso per archi (\ref{ApertoInLocalmenteConnessoPerArchiEConnessoPerArchi}).
\end{remark}



\begin{proposition}[Caratterizzazione esattezza con integrali su lacci]\label{CaratterizzazioneEsatezzaInTerminiDiIntegraliSuLacci}
Sia $D\subseteq \C$ dominio e $\omega$ una 1-forma su $D$. Allora $\omega$ \`e esatta se e solo se $\int_\gamma\omega=0$ per ogni $\gamma$ laccio su $D$ di classe $C^1$ a tratti.
\end{proposition}

\begin{corollary}[Caratterizzazione di forme esatte in disco]\label{CaratterizzazioneEsattezzaInDiscoConIntegraliSuBordiRettangolari}
Se $D$ \`e un disco allora $\omega$ \`e esatta in $D$ se e solo se $\int_\gamma\omega=0$ per ogni curva chiusa $\gamma$ bordo di un rettangolo parallelo agli assi.
\end{corollary}

\begin{corollary}
Se $D$ \`e un disco allora $\omega$ \`e esatta in $D$ se e solo se $\omega$ \`e chiusa in $D$.
\end{corollary}


\subsection{Primitive lungo curve e lungo omotopie}
\begin{lemma}[Numero di Lebesgue per forme chiuse]\label{NumeroLebesgueFormeChiuse}
Siano $\gamma:[a,b]\to D$ continua e $\omega$ una 1-forma chiusa su $D$. Allora esiste $\e>0$ tale che se $[c,d]\subseteq [a,b]$ e $d-c<\e$ allora esiste un disco $U\subseteq D$ tale che $\omega$ \`e esatta su $U$ e $\gamma([c,d])\subseteq U$. Per brevit\`a diremo che $\e$ \`e un \textbf{numero di Lebesgue} per $\omega$ sulla curva $\gamma$\footnote{Questa definizione NON \`e standard e NON \`e stata data durante il corso. L'ho data io per rendere pi\`u scorrevole il testo quando useremo questo lemma.}.
\end{lemma}

\begin{definition}[Primitiva lungo una curva]
Sia $\gamma:[a,b]\to D$ continua. Sia $\omega=Pdx+Qdy$ una 1-forma chiusa su $D$. Una \textbf{primitiva} per $\omega$ \textbf{lungo $\gamma$} \`e una funzione continua $f:[a,b]\to \C$ tale che per ogni $t_0\in[a,b]$ esistono $U\subseteq D$ intorno aperto di $\gamma(t_0)$ e $F:U\to \C$ primitiva di $\omega$ su $U$ tale che $F(\gamma(t))=f(t)$ per ogni $t\in\gamma\ii(U)$.\footnote{Ricordo che la notazione che uso in queste dispense assegna il simbolo $\ol \gamma$ al cammino inverso. Con $\gamma\ii(U)$ intendo la preimmagine di $U$ tramite la mappa $\gamma:[a,b]\to D$}
\end{definition}

\begin{theorem}[Esistenza e quasi unicit\`a delle primitive lungo curve]\label{EsistenzaQuasiUnicitaPrimitiveLungoCurve}
Per ogni $\gamma:[a,b]\to D$ continua e per ogni 1-forma $\omega$ chiusa su $D$ esiste una primitiva di $\omega$ lungo $\gamma$, che \`e unica a meno di costanti additive.
\end{theorem}


\begin{definition}[Primitiva lungo una omotopia]
Siano $H:[0,1]\times[a,b]\to D$ una omotopia fra $\gamma_0$ e $\gamma_1$ e $\omega$ una 1-forma chiusa su $D$. Una \textbf{primitiva di $\omega$ lungo $H$} \`e $f:[0,1]\times[a,b]\to D$ tale che per ogni $(s_0,t_0)\in [0,1]\times[a,b]$ esistono $U\subseteq D$ intorno di $H(s_0,t_0)$ e $F:U\to\C$ primitiva di $\omega$ tale che $F(H(s,t))=f(s,t)$ per ogni $(s,t)\in H\ii(U)$.
\end{definition}
\begin{remark}
La restrizione di $f$ primitiva lungo una omotopia ai segmenti orizzontali o verticali restituisce una primitiva di $\omega$ lungo la curva corrispondente.
\end{remark}

\begin{theorem}[Esistenza e quasi unicit\`a delle primitive lungo omotopie]\label{EsistenzaEQuasiUnicitaPrimitiveLungoOmotopie}
Se $\omega$ \`e una 1-forma chiusa allora ammette una primitiva lungo qualsiasi omotopia $H:[0,1]\times[a,b]\to D$.
\end{theorem}

\begin{corollary}[Integrazione di forme chiuse tramite primitiva lungo curve]\label{IntegrazioneFormeChiuseTramitePrimitivaLungoCurve}
Se $\omega$ \`e chiusa, $\gamma$ \`e $C^1$ a tratti e $f$ \`e primitiva di $\omega$ lungo $\gamma$ allora
\[\int_\gamma\omega=f(b)-f(a).\]
\end{corollary}

\noindent Se $\gamma$ \`e continua ma non $C^1$ a tratti diamo come definizione di integrale quella che mantiene vero questo risultato
\begin{definition}[Integrale di 1-forme chiuse su cammini continui]
Data $\omega$ una forma chiusa su $U$ e $\gamma:[a,b]\to U$, allora, detta $f$ una primitiva di $\omega$ lungo $\gamma$, definiamo
\[\int_\gamma\omega=f(b)-f(a).\]
\end{definition}


\begin{theorem}[Invarianza dell'integrale per cammini omotopi]\label{InvarianzaIntegraleCamminiOmotopi}
Siano $D\subseteq\C$ dominio, $\gamma_0,\gamma_1:[a,b]\to D$ curve omotope a estremi fissi e $\omega$ chiusa su $D$. Allora
\[\int_{\gamma_0}\omega=\int_{\gamma_1}\omega.\]
\end{theorem}


\subsection{Forme chiuse da funzioni olomorfe}
Da questo momento in poi nel corso la definizione di ``funzione olomorfa" diventer\`a implicitamente ``funzione olomorfa e $C^1$". Il motivo verr\`a reso chiaro tra qualche risultato.
\bigskip

\noindent Citiamo il
\begin{theorem}[Gauss-Green]\label{TeoremaGaussGreen}
Sia $\omega=Pdx+Qdy$ una 1-forma $C^1$ nell'intorno di un rettangolo $\ol R$ con lati paralleli agli assi. Allora
\[\iint_R\pa{\pdp Q-\pdp[y]P}dxdy=\int_{\partial R}\omega.\]
\end{theorem}
\begin{remark}
Se $\omega=Adz+Bd\ol z$ allora
\[\iint_R\pa{\pdp[z] B-\pdp[\ol z]A}dzd\ol z=\int_{\partial R}\omega.\]
\end{remark}
\begin{theorem}[Caratterizzazione delle forme chiuse $C^1$ con derivate parziali]\label{CaratterizzazioneFormeChiuseC1TramiteDerivateParziali}
Sia $\omega=Pdx+Qdy$ una 1-forma $C^1$ su $D$ dominio. Allora \[\omega  \text{ chiusa} \coimplies \pdp[y]P=\pdp Q.\]
\end{theorem}

\begin{corollary}[$fdz$ chiusa se e solo se $f$ olomorfa]\label{fdzchiusaSeESoloSefOlomorfaInC1}
Sia $f\in C^1(D)$. Allora $\omega=fdz$ \`e chiusa se e solo se $f$ \`e olomorfa.
\end{corollary}

\begin{corollary}
La forma $fdz$ \`e esatta in $D$ se e solo se esiste $F:D\to\C$ olomorfa tale che $F'=f$.
\end{corollary}

\noindent In realt\`a una implicazione della caratterizzazione (\ref{fdzchiusaSeESoloSefOlomorfaInC1}) vale anche senza supporre $f$ di classe $C^1$.

\begin{theorem}[Cauchy]\label{OlomorfaDefinisceFormaChiusa}
Se $f$ \`e olomorfa allora $fdz$ \`e chiusa.
\end{theorem}

\noindent Poich\'e l'implicazione (\ref{OlomorfaDefinisceFormaChiusa}) non \`e stata dimostrata, la trattazione che segue \`e rigorosa solo se sostituiamo la parola ``olomorfa" con ``olomorfa $C^1$". Per rendere pi\`u generali le dispense proceder\`o invocando quando necessario il teorema di Cauchy (\ref{OlomorfaDefinisceFormaChiusa}) quando il corso ha usato l'analogo per le $C^1$. Se preferite attenervi solo a ci\`o che \`e stato dimostrato considerate solo funzioni olomorfe di classe $C^1$ e sostituite ogni riferimento a (\ref{OlomorfaDefinisceFormaChiusa}) con un riferimento a (\ref{fdzchiusaSeESoloSefOlomorfaInC1}).
\bigskip

\noindent
Concludiamo la sezione fornendo un utile risultato sull'estensione dell'olomorfia.


\begin{theorem}[Continua olomorfa fuori un segmento d\`a forma chiusa]\label{ContinuaOlomorfaEccettoInSegmentoDaFormaChiusa}
Sia $D\subseteq \C$ aperto, $f$ continua in $D$ e olomorfa fuori da un segmento $L\subseteq D$. Allora $fdz$ \`e chiusa in $D$.
\end{theorem}
\begin{corollary}\label{ContinuaOlomorfaEccettoUnPuntoEOlomorfa}
Sia $D\subseteq \C$ aperto, $f$ continua in $D$ e olomorfa $C^1$ fuori da un segmento $L\subseteq D$. Allora $f$ \`e olomorfa in $D$.
\end{corollary}

\section{Indice di avvolgimento e Formula di Cauchy}
\begin{definition}[Indice di avvolgimento]
Sia $\gamma$ una curva chiusa in $\C$ e sia $z_0$ un punto che NON appartiene al supporto di $\gamma$ ($z_0\in \C\bs \gamma([0,1])$). Allora l'\textbf{indice di avvolgimento} di $\gamma$ attorno $z_0$ \`e dato da
\[\Ind(\gamma,z_0)=\frac1{2\pi i}\int_{\gamma}\frac{dz}{z-z_0}.\]
\end{definition}
\begin{remark}
La forma $\frac{dz}{z-z_0}$ \`e chiusa in $\C\bs\{z_0\}$, quindi ammette primitiva lungo $\gamma$ anche se $\gamma$ \`e continua ma non $C^1$ (\ref{EsistenzaQuasiUnicitaPrimitiveLungoCurve}). L'integrale nella definizione di $\Ind(\gamma,z_0)$ \`e quindi ben definito anche se $\gamma$ non \`e $C^1$ a tratti.
\end{remark}

\begin{proposition}[Indice di avvolgimento \`e intero]\label{IndiceAvvolgimentoEIntero}
Per ogni $z_0$ e per ogni curva continua $\gamma$ che non passa per $z_0$ si ha che
\[\Ind(\gamma,z_0)\in \Z.\]
\end{proposition}
\begin{remark}
Intuitivamente $\Ind(\gamma,z_0)$ conta quante volte $\gamma$ ``gira attorno a $z_0$".
\end{remark}





\begin{theorem}[Formula integrale di Cauchy]\label{FormulaIntegraleCauchy}
Siano $D\subseteq\C$ aperto, $f\in\Oc(D)$ e $\gamma:[0,1]\to D$ una curva chiusa omotopa ad una curva costante. Sia $z\in D\bs\gamma([0,1])$. Allora\footnote{Qui e in futuro useremo $\zeta$ quando $z$ \`e gi\`a occupato. Quindi per $d\zeta$ intendiamo comunque $dx+idy$.}
\[\Ind(\gamma,z)f(z)=\frac1{2\pi i}\int_\gamma \frac{f(\zeta)}{\zeta-z}d\zeta.\]
\end{theorem}
\begin{remark}
Dal teorema segue immediatamente che una funzione olomorfa definita su un disco \`e completamente determinata dal suo valore sul bordo del disco.
\end{remark}

\subsection{Olomorfa implica analitica}
\begin{theorem}[Continue sul bordo di un disco definiscono olomorfa nel disco]\label{ContinueSuBordoDiDiscoDefinisconoOlomorfaInDisco}
Sia $B=B(z_0,R)$ e $h:\partial B\to \C$ continua. Sia $\gamma:[0,2\pi]\to \C$ data da $\gamma(t)=z_0+Re^{it}$ e poniamo
\[f(z)=\frac1{2\pi i}\int_{\gamma}\frac{h(\zeta)}{\zeta-z}d\zeta.\]
Allora abbiamo che
\begin{enumerate}[noitemsep]
\item $f(z)=\ser na_n(z-z_0)^n$ serie di potenze con raggio di convergenza maggiore o uguale a $R$.
\item $a_n=\displaystyle\frac1{2\pi i}\int_\gamma\frac{h(\zeta)}{(\zeta-z_0)^{n+1}}d\zeta$.
\end{enumerate}
In particolare $f\in \Oc(B)$ e $a_n=\frac{f^{(n)}(z_0)}{n!}$.
\end{theorem}


\begin{corollary}[Una funzione olomorfa \`e analitica]\label{OlomorfaImplicaAnalitica}
Dati $D\subseteq \C$ aperto, $f\in \Oc(D)$ e $\ol B=\ol{B(z_0,R)}\subseteq D$ si ha che
\begin{enumerate}[noitemsep]
\item $\displaystyle f(z)=\ser na_n(z-z_0)^n$ con raggio di convergenza maggiore o uguale a $R$
\item Posto $\gamma(t)=z_0+Re^{it}$ si ha che \[a_n=\frac{f^{(n)}(z_0)}{n!}=\frac1{2\pi i}\int_{\gamma}\frac{f(\zeta)}{(\zeta-z_0)^{n+1}}d\zeta.\]
\end{enumerate}
\end{corollary}

\subsubsection{Propriet\`a delle olomorfe ereditate delle analitiche}
\begin{corollary}[Propriet\`a delle olomorfe ereditate dalle analitiche]\label{ProprietaCheOlomorfeEreditanoDaAnalitiche}
Sia $D\subseteq \C$ aperto, $f\in \Oc(D)$ e $\ol B=\ol{B(z_0,R)}\subseteq D$, allora
\begin{enumerate}[noitemsep]
\item $f\in C^\infty(D)$
\item Per ogni $z\in B$ e per ogni $n\in \N$
\[f^{(n)}(z)=n!\cauchyint[n+1]<-z>{\gamma},\]
dove $\gamma(t)=z_0+Re^{it}$.
\item Per ogni $n\in\N$ abbiamo $f^{(n)}\in \Oc(D)$
\end{enumerate}
\end{corollary}


\noindent Data l'importanza della seguente propriet\`a ereditata dalle analitiche la isoliamo in una proposizione:
\begin{proposition}[Principio di identit\`a]\label{PrincipioDiIdentita}
Sia $D$ un dominio e siano $f,g:D\to \C$ olomorfe. Se $f$ e $g$ coincidono su un aperto allora coincidono su tutto $D$.
\end{proposition}

\section{Applicazioni}
\subsection{Disuguaglianze di Cauchy e Teorema di Liouville}
\begin{notation}
Siano $D\subseteq \C$ aperto, $z_0\in D$, $\ol B=\ol{B(z_0,R)}\subseteq D$ e $f$ olomorfa su $D$. Per ogni $r\leq R$ poniamo
\[M(r)=\max\cpa{|f(z)|\sep \abs{z-z_0}=r}=\max_{z\in \partial B_r(z_0)}\{|f(z)|\}.\]
\end{notation}
\begin{proposition}[Disuguaglianze di Cauchy]\label{DisuguaglianzeCauchy}
Siano $D\subseteq \C$ aperto, $z_0\in D$, $\ol B=\ol{B(z_0,r)}\subseteq D$ e $f$ olomorfa su $D$.
Allora
\[\abs{f^{(n)}(z_0)}\leq \frac{n!M(r)}{r^n}.\]
\end{proposition}
\begin{remark}
Le disuguaglianze di Cauchy danno una maggiorazione di $\abs{f^{(n)}(z_0)}$ in termini di $n,\ r$ e il massimo che $f$ assume sul BORDO del disco di raggio $r$ centrato in $z_0$.
\end{remark}

\begin{corollary}[Teorema di Liouville]\label{TeoremaLiouville}
Ogni funzione limitata olomorfa su $\C$\footnote{Le funzioni olomorfe su $\C$ si dicono \textbf{intere}.} \`e costante.
\end{corollary}

\begin{corollary}[Teorema fondamentale dell'algebra]\label{TeoremaFondamentaleAlgebra}
Ogni polinomio $p\in\C[z]$ di grado $n\geq 1$ ha esattamente $n$ radici distinte contate con molteplicit\`a.
\end{corollary}

\subsection{Principio della media}
\begin{definition}[Propriet\`a della media]
Una funzione $f:D\to \C$ soddisfa la \textbf{propriet\`a della media} se per ogni $z_0\in D$ esiste $r_0>0$ tale che $\ol{B_{r_0}(z_0)}\subseteq D$ e per ogni $r\in(0,r_0)$ si ha che
\[f(z_0)=\frac1{2\pi}\int_0^{2\pi}f(z_0+re^{it})dt.\]
\end{definition}
\begin{proposition}[Propriet\`a della media implica continua]\label{ProprietaDellaMediaImplicaContinua}
Se $f:D\to \C$ soddisfa la propriet\`a della media allora \`e continua.
\end{proposition}
\begin{theorem}[Principio della media]\label{PrincipioDellaMedia}
Siano $D\subseteq\C$ aperto, $f\in\Oc(D)$ e $\ol B=\ol{B(z_0,r)}\subseteq D$. Allora
\[f(z_0)=\frac1{2\pi}\int_0^{2\pi} f(z_0+re^{it})dt,\]
in particolare $f$ soddisfa la propriet\`a della media.
\end{theorem}
\begin{remark}
Nelle ipotesi del teorema, anche $\Real(f)$ e $\Imag(f)$ soddisfano la propriet\`a della media.
\end{remark}
\begin{remark}[Propriet\`a della media su area]
Nelle ipotesi del teorema vale
\[f(z_0)=\frac1{\pi r^2}\int_Bf(z)dxdy.\]
\end{remark}



\subsection{Principio del massimo}
\begin{theorem}[Principio del massimo 1]\label{PrincipioMassimo1}
Sia $f:D\to \C$ con la propriet\`a della media. Supponiamo che esista $z_0\in D$ massimo locale per $|f|$ (oppure per $f$ se $f:D\to \R$). Allora $f$ \`e costante in un intorno di $z_0$.
\end{theorem}

\begin{corollary}[Olomorfa con massimo locale in modulo o per componente \`e costante]\label{OlomorfaConMassimoLocalInModuloOPerComponenteECostante}
Se $D\subseteq \C$ dominio e $f\in\Oc(D)$ \`e tale che esiste $z_0\in D$ punto di massimo locale per $|f|$ (oppure $\Real(f)$ oppure $\Imag(f)$) allora $f$ \`e costante.
\end{corollary}


\begin{corollary}[Principio del massimo 2]\label{PrincipioMassimo2}
Sia $D\subseteq \C$ un dominio limitato. Sia $f\in \Oc(D)\cap C^0\pa{\ol D}$. Poniamo
\[M=\max_{x\in \partial D}|f(z)|.\]
Allora
\begin{enumerate}[noitemsep]
\item $|f(z)|\leq M$ per ogni $z\in D$
\item se esiste $z_0\in D$ tale che $|f(z_0)|=M$ allora $f$ \`e costante.
\end{enumerate}
Valgono anche risultati analoghi per $\Real(f)$ e $\Imag(f)$ al posto di $|f|$.
\end{corollary}
\begin{remark}
L'ipotesi di limitatezza \`e necessaria. Consideriamo per esempio
\[D=\cpa{z\in\C\sep\Real(z)>0}\quad\text{e}\quad f(z)=e^z.\]
Evidentemente $e^z$ \`e olomorfa e continua su $D$ perch\'e restrizione di olomorfa su $\C$. Osserviamo che $|f|\res{\partial D}=1$ ma $\displaystyle\sup_{z\in D}|f(z)|=+\infty$.\\
Il problema \`e il comportamento nel punto all'infinito.
\end{remark}
\begin{remark}
Nel caso $\Real(f)$ e $\Imag(f)$ valgono anche dei principi del minimo. Basta applicare il principio del massimo a $-\Real(f)$ e $-\Imag(f)$.
\end{remark}

\begin{theorem}[Teorema dell'applicazione aperta]\label{TeoremaApplicazioneAperta}
Ogni $f\in \Oc(D)$ non costante \`e una mappa aperta.
\end{theorem}

\section{Singolarit\`a}
\begin{definition}[Singolarit\`a]
Se $f\in \Oc(B_r(z_0)\bs\{z_0\})$ allora $z_0$ \`e una \textbf{singolarit\`a} per $f$.
\end{definition}


\begin{definition}[Corona]
Dati $0\leq r<R\leq+\infty$, l'\textbf{anello} (o \textbf{corona}) definito da $r$ e $R$ \`e l'insieme
\[A(r,R)=\cpa{z\in\C\sep r<|z|<R}.\]
Se $r>R$ poniamo $A(r,R)=\emptyset$.
\end{definition}
\begin{remark}
Esempi particolari di anelli sono $A(0,R)=B(0,R)\nz$, $A(r,+\infty)=\C\bs\ol{B(0,r)}$ e $A(0,+\infty)=\C\nz$
\end{remark}
\begin{lemma}\label{IntegraleDiUnGiroAttornoAnelloNonDipendeDallaDistanzaDalCentro}
Se $f\in \Oc(A(r,R))$ allora dato $\rho\in(r,R)$, $\int_{\rho e^{2\pi it}}fdz$ non dipende da $\rho$.
\end{lemma}
\begin{lemma}\label{ValoriOlomorfaSuAnelloComeSottrazioneDiIntegrali}
Sia $f\in \Oc(A(r,R))$ e $z_0\in A(r,R)$. Siano $\rho_1<\rho_2$ e $r<\rho_1<|z_0|<\rho_2<R$. Allora
\[f(z_0)=\cauchyint{\rho_2e^{2\pi it}}-\cauchyint{\rho_1e^{2\pi it}}\]
\end{lemma}

\subsection{Serie di Laurent}
Poich\'e le serie di potenze convergono in un intero disco non possono essere lo strumento adatto per catturare le singolarit\`a. Consideriamo allora una generalizzazione delle serie di potenze che ammette potenze negative, ovvero le serie di Laurent
\begin{definition}[Serie di Laurent]
Una \textbf{serie di Laurent} centrata in $z_0$ \`e un'espressione della forma
\[f=\sum_{n=-\infty}^{+\infty}a_n(z-z_0)^n,\]
con $a_n\in\C$.\\
Diremo che $f$ \`e \textbf{convergente} se
\[f_+=\ser n a_n(z-z_0)^n\quad \text{e}\quad f_-=\ser m a_{-m}\pa{\frac1{z-z_0}}^m\]
sono convergenti.
\end{definition}
\begin{remark}[Le serie di Laurent convergono in anello]
Data $f$ serie di Laurent proviamo a capire dove converge assolutamente. Per ci\`o che sappiamo sulle serie di potenze si ha che $f_+$ converge assolutamente in un disco. Sia $R$ il raggio di questo disco. Possiamo interpretare $f_-$ come una serie di potenze nelle $\frac1{z-z_0}$ e questo restituisce un raggio di convergenza assoluta $\frac1r$, cio\`e $f_-$ converge assolutamente se
\[\frac1{|z-z_0|}<\frac1r\coimplies |z-z_0|>r.\]
Mettendo insieme queste informazioni abbiamo che $f$ converge assolutamente nell'anello
\[z_0+A(r,R).\]
Osserviamo che se $r>R$ allora $f$ non converge perch\'e in ogni punto una tra $f_+$ e $f_-$ non converge.
\end{remark}


\begin{proposition}[propriet\`a delle serie di Laurent]\label{ProprietaLaurent}
Se $f(z)=\sum_{n\in\Z}a_n(z-z_0)^n$ converge in $\Omega=z_0+A(r,R)$ allora
\begin{enumerate}[noitemsep]
\item $f$ \`e olomorfa in $\Omega$
\item per ogni $r<\rho<R$ vale $\displaystyle a_n=\cauchyint[n+1]{z_0+\rho e^{2\pi it}}$.
\end{enumerate}
\end{proposition}

\begin{theorem}[Olomorfa su Anello \`e serie di Laurent]\label{OlomorfaSuAnelloESerieDiLaurent}
Siano $0\leq r<R\leq +\infty$, $z_0\in \C$ e $\Omega=z_0+A(r,R)$. Se $f\in\Oc(\Omega)$ allora $f$ si esprime tramite un'unica serie di Laurent in $z_0$ convergente in $\Omega$.
\end{theorem}


\subsection{Tipi di singolarit\`a}
\begin{definition}[Tipi di singolarit\`a]
Sia $f\in \Oc(B_R(z_0)\bs\{z_0\})=\Oc(z_0+A(0,R))$ e scriviamo $f$ in serie di Laurent come segue
\[f(z)=\sum_{n\in\Z}a_n(z-z_0)^n.\]
Si dice che $z_0$ \`e
\begin{itemize}[noitemsep]
\item una \textbf{singolarit\`a eliminabile} se $a_n=0$ per ogni $n<0$.
\item un \textbf{polo} di \textbf{ordine} $k>0$ se $a_{-k}\neq 0$ e $a_n=0$ per $n<-k$.\\
Un polo si dice \textbf{semplice} se ha ordine $1$.
\item una \textbf{singolarit\`a essenziale} altrimenti.
\end{itemize}
\end{definition}
\begin{remark}[Singolarit\`a eliminabili possono essere ignorate]
Se $z_0$ \`e una singolarit\`a eliminabile di $f=\sum a_k(z-z_0)^k$ allora ponendo $f(z_0)=a_0$ si ha che $f\in\Oc(B_R(z_0))$.
\end{remark}
\begin{remark}[I poli si possono fattorizzare]
Se $z_0$ \`e un polo di ordine $k$ per $f=\sum_n a_n$ allora
\[f(z)=\frac1{(z-z_0)^k}h(z)\]
dove $h$ \`e una funzione olomorfa e $h(z_0)\neq 0$.
\end{remark}

\begin{theorem}[Estensione di Riemann]\label{TeoremaEstensioneRiemann}
Sia $f\in \Oc(B^\ast)$, dove $B^\ast=B_R(z_0)\bs\{z_0\}$. Allora se $|f|$ \`e limitato in $B^\ast$ si ha che $z_0$ \`e una singolarit\`a eliminabile.
\end{theorem}

\begin{theorem}[Casorati-Weierstass]\label{TeoremaCasoratiWeierstrass}
Sia $f\in \Oc(B^\ast)$, dove $B^\ast=B_R(z_0)\bs\{z_0\}$. Se $z_0$ \`e un singolarit\`a essenziale allora $f(B^\ast)$ \`e denso in $\C$.
\end{theorem}


\section{Funzioni meromorfe e Residui}
\begin{definition}[Funzione meromorfa]
Sia $U\subseteq \C$ aperto. Una funzione \textbf{meromorfa} su $U$ \`e una funzione olomorfa $f:U\bs S\to \C$ dove $S\subseteq U$ \`e discreto e per ogni $z_0\in S$ si ha che $z_0$ \`e un polo.\footnote{Intuitivamente una funzione \`e meromorfa se \`e olomorfa ovunque eccetto in qualche polo isolato.}
\end{definition}

\begin{remark}
Se $f,g:U\to\C$ sono olomorfe con $U$ connesso e $g$ non costantemente nulla, allora $\frac fg$ \`e meromorfa.
\end{remark}

\begin{remark}[Inversione mantiene l'ordine scambiando zeri e poli]
Se $f:U\to \C$ \`e olomorfa non costantemente nulla con $U$ dominio e $f(z_0)=0$ con ordine di annullamento $n_0$ allora $\frac1{f}$ ha un polo di ordine $n_0$ in $z_0$.
\end{remark}

\begin{definition}[Residuo]
Siano $U\subseteq \C$ aperto, $z_0\in U$ e $f:U\bs \{z_0\}\to \C$ olomorfa. Definiamo il \textbf{residuo} di $f$ in $z_0$ come
\[\Res(f,z_0)=a_{-1}=\cauchyint[-1+1]{\al}=\frac1{2\pi i}\int_\al f(z)dz,\]
dove $\sum_{n\in\Z}a_n(z-z_0)^n$ \`e lo sviluppo di Laurent di $f$ su $B(z_0,\e)\bs\{z_0\}$ per $\e$ abbastanza piccolo e $\al(t)=z_0+\e e^{it}$.
\end{definition}
\begin{remark}
Il residuo \`e una misura di   ``quanto la forma $fdz$ non \`e esatta".
\end{remark}



\begin{proposition}[Formula del residuo per rapporto di olomorfe]\label{FormulaResiduoPerRapportoDiOlomorfe}
Se $g,h:U\to \C$ olomorfe con $z_0$ uno zero di ordine $1$ per $h$ e $g(z_0)\neq 0$ allora posta $f=\frac gh$ abbiamo che $z_0$ \`e un polo di ordine $1$ per $f$ e
\[\Res(f,z_0)=\lim_{z\to z_0}(z-z_0)\frac{g(z)}{h(z)}=\frac{g(z_0)}{h'(z_0)}.\]
\end{proposition}

\begin{theorem}[Teorema dei Residui]\label{TeoremaResidui}
Sia $U\subseteq \C$ aperto, $S\subseteq U$ discreto chiuso\footnote{Per esempio: una successione che tende a un punto in $U$ non \`e ammessa, invece una che accumula a un punto sul bordo di $U$ va bene perch\'e come sottoinsieme di $U$ \`e chiusa.}, $f:U\bs S\to \C$ olomorfa. Sia $K\subseteq U$ una regione compatta omeomorfa a $D^2$. Sia $\al:[0,1]\to\partial K$ una parametrizzazione di $\partial K$ in senso antiorario. Assumiamo anche $S\cap \partial K=\emptyset$. Allora valgono le seguenti affermazioni:
\begin{enumerate}[noitemsep]
\item $S\cap K$ \`e finito
\item $\displaystyle\int_\al f(z)dz=2\pi i\sum_{z_0\in S\cap K}\Res(f,z_0)$.
\end{enumerate}
\end{theorem}

\subsection{Derivata Logaritmica}
\begin{definition}[Derivata logaritmica]
Data una funzione derivabile $f$ definiamo la sua \textbf{derivata logaritmica} nei punti dove non si annulla come
\[\frac{f'}f.\]
\end{definition}
\begin{remark}[La derivata logaritmica trasforma poli e zeri in poli semplici]
Se $z_0$ \`e uno zero o polo di $f$ tale che
\[f(z)=(z-z_0)^{n_0}g(z)\]
con $g(z_0)\neq 0$, $g(z)$ olomorfa vicina a $z_0$ e $n_0\neq 0$ allora vicino a $z_0$ si ha che
\begin{align*}
\frac{f'(z)}{f(z)}=&\frac{n_0(z-z_0)^{n_0-1}g(z)+(z-z_0)^{n_0}g'(z)}{(z-z_0)^{n_0}g(z)}=\\
=&\frac{n_0}{z-z_0}+\frac{g'(z)}{g(z)}.
\end{align*}
Osserviamo inoltre che $\frac{g'(z)}{g(z)}$ \`e olomorfa vicino a $z_0$ perch\'e $g(z_0)\neq 0$.
\end{remark}
\begin{remark}
Simbolicamente si ha che
\[\frac{f'}{f}=\log(f)',\]
da cui il nome.
\end{remark}


\begin{proposition}[Residuo in polo della derivata logaritmica \`e l'ordine del polo/zero]\label{ResiduoDellaDerivataLogaritmicaEOrdine}
Se $f:U\to \C$ \`e meromorfa anche $\frac{f'}{f}$ lo \`e e i poli di $\frac{f'}{f}$ coincidono esattamente con i poli e gli zeri di $f$, inoltre
\[\Res\pa{\frac{f'}f,z_0}=n_0\]
dove $n_0$ \`e l'ordine di $z_0$ come zero di $f$ se positivo o $-n_0$ \`e l'ordine di $z_0$ come polo di $f$.
\end{proposition}

\begin{theorem}[Teorema di Derivata logaritmica]\label{TeoremaDerivataLogaritmica}
Sia $f:U\to \C$ meromorfa e $K\subseteq U$ compatto omeomorfo a $D^2$. Se $\al$ \`e una parametrizzazione in senso antiorario di $\partial K$ e $\partial K$ non contiene n\'e zeri n\'e poli di $f$ allora
\[\int_\al\frac{f'(z)}{f(z)}dz=2\pi i(Z-P)\]
dove $Z$ \`e la somma degli ordini di tutti gli zeri di $f$ contenuti in $K$ e $P$ \`e la somma degli ordini di tutti i poli di $f$ in $K$.
\end{theorem}

\begin{corollary}[Teorema di Rouch\'e]\label{TeoremaDiRouche}
Siano $f,g:U\to\C$ olomorfe e  $K\subseteq U$ un compatto omeomorfo a $D^2$. Supponiamo che $|g(z)|<|f(z)|$ per ogni $z\in\partial K$ (in particolare $f(z)\neq 0$ per ogni $z\in\partial K$). Allora il numero di zeri contati con molteplicit\`a di $f$ e di $f-g$ in $K$ coincidono.
\end{corollary}

\end{multicols*}



\end{document}
